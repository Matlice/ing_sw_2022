\documentclass[11pt,a4paper]{report}
\usepackage{color}
\usepackage{ifthen}
\usepackage{makeidx}
\usepackage{ifpdf}
\usepackage[headings]{fullpage}
\usepackage{listings}
\lstset{language=Java,breaklines=true}
\ifpdf \usepackage[pdftex, pdfpagemode={UseOutlines},bookmarks,colorlinks,linkcolor={blue},plainpages=false,pdfpagelabels,citecolor={red},breaklinks=true]{hyperref}
  \usepackage[pdftex]{graphicx}
  \pdfcompresslevel=9
  \DeclareGraphicsRule{*}{mps}{*}{}
\else
  \usepackage[dvips]{graphicx}
\fi

\newcommand{\entityintro}[3]{%
  \hbox to \hsize{%
    \vbox{%
      \hbox to .2in{}%
    }%
    {\bf  #1}%
    \dotfill\pageref{#2}%
  }
  \makebox[\hsize]{%
    \parbox{.4in}{}%
    \parbox[l]{5in}{%
      \vspace{1mm}%
      #3%
      \vspace{1mm}%
    }%
  }%
}
\newcommand{\refdefined}[1]{
\expandafter\ifx\csname r@#1\endcsname\relax
\relax\else
{$($in \ref{#1}, page \pageref{#1}$)$}\fi}
\date{\today}
\chardef\textbackslash=`\\
\makeindex
\begin{document}
\sloppy
\addtocontents{toc}{\protect\markboth{Contents}{Contents}}
\tableofcontents
\chapter{Package it.matlice.ingsw}{
\label{it.matlice.ingsw}\hskip -.05in
\hbox to \hsize{\textit{ Package Contents\hfil Page}}
\vskip .13in
\hbox{{\bf  Classes}}
\entityintro{EntryPoint}{it.matlice.ingsw.EntryPoint}{Classe EntryPoint dell'applicazione, contiene il metodo main()}
\vskip .1in
\vskip .1in
\section{\label{it.matlice.ingsw.EntryPoint}\index{EntryPoint}Class EntryPoint}{
\vskip .1in 
Classe EntryPoint dell'applicazione, contiene il metodo main()\vskip .1in 
\subsection{Declaration}{
\begin{lstlisting}[frame=none]
public class EntryPoint
 extends java.lang.Object\end{lstlisting}
\subsection{Constructor summary}{
\begin{verse}
{\bf EntryPoint()} \\
\end{verse}
}
\subsection{Method summary}{
\begin{verse}
{\bf main(String\lbrack \rbrack )} Metodo main() dell'applicazione\\
\end{verse}
}
\subsection{Constructors}{
\vskip -2em
\begin{itemize}
\item{ 
\index{EntryPoint()}
{\bf  EntryPoint}\\
\begin{lstlisting}[frame=none]
public EntryPoint()\end{lstlisting} %end signature
}%end item
\end{itemize}
}
\subsection{Methods}{
\vskip -2em
\begin{itemize}
\item{ 
\index{main(String\lbrack \rbrack )}
{\bf  main}\\
\begin{lstlisting}[frame=none]
public static void main(java.lang.String[] args)\end{lstlisting} %end signature
\begin{itemize}
\item{
{\bf  Description}

Metodo main() dell'applicazione
}
\item{
{\bf  Parameters}
  \begin{itemize}
   \item{
\texttt{args} -- nessuno}
  \end{itemize}
}%end item
\end{itemize}
}%end item
\end{itemize}
}
}
}
\chapter{Package it.matlice.ingsw.controller}{
\label{it.matlice.ingsw.controller}\hskip -.05in
\hbox to \hsize{\textit{ Package Contents\hfil Page}}
\vskip .13in
\hbox{{\bf  Interfaces}}
\entityintro{ReturnAction}{it.matlice.ingsw.controller.ReturnAction}{}
\vskip .13in
\hbox{{\bf  Classes}}
\entityintro{Controller}{it.matlice.ingsw.controller.Controller}{}
\entityintro{MenuAction}{it.matlice.ingsw.controller.MenuAction}{}
\vskip .1in
\vskip .1in
\section{\label{it.matlice.ingsw.controller.ReturnAction}\index{ReturnAction@\textit{ ReturnAction}}Interface ReturnAction}{
\vskip .1in 
\subsection{Declaration}{
\begin{lstlisting}[frame=none]
public interface ReturnAction
\end{lstlisting}
\subsection{Method summary}{
\begin{verse}
{\bf run()} \\
\end{verse}
}
\subsection{Methods}{
\vskip -2em
\begin{itemize}
\item{ 
\index{run()}
{\bf  run}\\
\begin{lstlisting}[frame=none]
java.lang.Object run()\end{lstlisting} %end signature
}%end item
\end{itemize}
}
}
\section{\label{it.matlice.ingsw.controller.Controller}\index{Controller}Class Controller}{
\vskip .1in 
\subsection{Declaration}{
\begin{lstlisting}[frame=none]
public class Controller
 extends java.lang.Object\end{lstlisting}
\subsection{Constructor summary}{
\begin{verse}
{\bf Controller(View, Model)} Costruttore per Controller\\
\end{verse}
}
\subsection{Method summary}{
\begin{verse}
{\bf addDefaultConfigurator()} Permette l'aggiunta di un utente configuratore di default al primo avvio\\
{\bf mainloop()} Mainloop dell'applicazione\\
{\bf registerUser()} Permette ad un nuovo utente di registrarsi come fruitore\\
\end{verse}
}
\subsection{Constructors}{
\vskip -2em
\begin{itemize}
\item{ 
\index{Controller(View, Model)}
{\bf  Controller}\\
\begin{lstlisting}[frame=none]
public Controller(it.matlice.ingsw.view.View view,it.matlice.ingsw.model.Model model)\end{lstlisting} %end signature
\begin{itemize}
\item{
{\bf  Description}

Costruttore per Controller
}
\item{
{\bf  Parameters}
  \begin{itemize}
   \item{
\texttt{view} -- la view per l'interazione utente}
   \item{
\texttt{model} -- il model a cui richiedere i dati}
  \end{itemize}
}%end item
\end{itemize}
}%end item
\end{itemize}
}
\subsection{Methods}{
\vskip -2em
\begin{itemize}
\item{ 
\index{addDefaultConfigurator()}
{\bf  addDefaultConfigurator}\\
\begin{lstlisting}[frame=none]
public void addDefaultConfigurator()\end{lstlisting} %end signature
\begin{itemize}
\item{
{\bf  Description}

Permette l'aggiunta di un utente configuratore di default al primo avvio
}
\end{itemize}
}%end item
\item{ 
\index{mainloop()}
{\bf  mainloop}\\
\begin{lstlisting}[frame=none]
public boolean mainloop()\end{lstlisting} %end signature
\begin{itemize}
\item{
{\bf  Description}

Mainloop dell'applicazione
}
\item{{\bf  Returns} -- 
false se l'esecuzione deve essere interrotta 
}%end item
\end{itemize}
}%end item
\item{ 
\index{registerUser()}
{\bf  registerUser}\\
\begin{lstlisting}[frame=none]
public boolean registerUser()\end{lstlisting} %end signature
\begin{itemize}
\item{
{\bf  Description}

Permette ad un nuovo utente di registrarsi come fruitore
}
\item{{\bf  Returns} -- 
true 
}%end item
\end{itemize}
}%end item
\end{itemize}
}
}
\section{\label{it.matlice.ingsw.controller.MenuAction}\index{MenuAction}Class MenuAction}{
\vskip .1in 
\subsection{Declaration}{
\begin{lstlisting}[frame=none]
public class MenuAction
 extends java.lang.Object\end{lstlisting}
\subsection{Constructor summary}{
\begin{verse}
{\bf MenuAction(String, Class, ReturnAction)} \\
{\bf MenuAction(String, Class, ReturnAction, boolean)} \\
{\bf MenuAction(String, Class, ReturnAction, boolean, Integer, Integer)} \\
\end{verse}
}
\subsection{Method summary}{
\begin{verse}
{\bf getAction()} \\
{\bf getIndex()} \\
{\bf getName()} \\
{\bf getPosition()} \\
{\bf getRequestedUserType()} \\
{\bf isDisabled()} \\
{\bf isPermitted(User)} \\
\end{verse}
}
\subsection{Constructors}{
\vskip -2em
\begin{itemize}
\item{ 
\index{MenuAction(String, Class, ReturnAction)}
{\bf  MenuAction}\\
\begin{lstlisting}[frame=none]
public MenuAction(java.lang.String name,java.lang.Class requestedUserType,ReturnAction action)\end{lstlisting} %end signature
}%end item
\item{ 
\index{MenuAction(String, Class, ReturnAction, boolean)}
{\bf  MenuAction}\\
\begin{lstlisting}[frame=none]
public MenuAction(java.lang.String name,java.lang.Class requestedUserType,ReturnAction action,boolean disabled)\end{lstlisting} %end signature
}%end item
\item{ 
\index{MenuAction(String, Class, ReturnAction, boolean, Integer, Integer)}
{\bf  MenuAction}\\
\begin{lstlisting}[frame=none]
public MenuAction(java.lang.String name,java.lang.Class requestedUserType,ReturnAction action,boolean disabled,java.lang.Integer index,java.lang.Integer position)\end{lstlisting} %end signature
}%end item
\end{itemize}
}
\subsection{Methods}{
\vskip -2em
\begin{itemize}
\item{ 
\index{getAction()}
{\bf  getAction}\\
\begin{lstlisting}[frame=none]
public ReturnAction getAction()\end{lstlisting} %end signature
}%end item
\item{ 
\index{getIndex()}
{\bf  getIndex}\\
\begin{lstlisting}[frame=none]
public java.lang.Integer getIndex()\end{lstlisting} %end signature
}%end item
\item{ 
\index{getName()}
{\bf  getName}\\
\begin{lstlisting}[frame=none]
public java.lang.String getName()\end{lstlisting} %end signature
}%end item
\item{ 
\index{getPosition()}
{\bf  getPosition}\\
\begin{lstlisting}[frame=none]
public int getPosition()\end{lstlisting} %end signature
}%end item
\item{ 
\index{getRequestedUserType()}
{\bf  getRequestedUserType}\\
\begin{lstlisting}[frame=none]
public java.lang.Class getRequestedUserType()\end{lstlisting} %end signature
}%end item
\item{ 
\index{isDisabled()}
{\bf  isDisabled}\\
\begin{lstlisting}[frame=none]
public boolean isDisabled()\end{lstlisting} %end signature
}%end item
\item{ 
\index{isPermitted(User)}
{\bf  isPermitted}\\
\begin{lstlisting}[frame=none]
public boolean isPermitted(it.matlice.ingsw.model.data.User u)\end{lstlisting} %end signature
}%end item
\end{itemize}
}
}
}
\chapter{Package it.matlice.ingsw.model.exceptions}{
\label{it.matlice.ingsw.model.exceptions}\hskip -.05in
\hbox to \hsize{\textit{ Package Contents\hfil Page}}
\vskip .1in
\vskip .1in
\section{\label{it.matlice.ingsw.model.exceptions.CannotParseDayException}\index{CannotParseDayException}Exception CannotParseDayException}{
\vskip .1in 
\subsection{Declaration}{
\begin{lstlisting}[frame=none]
public class CannotParseDayException
 extends java.lang.Exception\end{lstlisting}
\subsection{Constructor summary}{
\begin{verse}
{\bf CannotParseDayException()} \\
\end{verse}
}
\subsection{Constructors}{
\vskip -2em
\begin{itemize}
\item{ 
\index{CannotParseDayException()}
{\bf  CannotParseDayException}\\
\begin{lstlisting}[frame=none]
public CannotParseDayException()\end{lstlisting} %end signature
}%end item
\end{itemize}
}
\subsection{Members inherited from class Throwable }{
\texttt{java.lang.Throwable} {\small 
\refdefined{java.lang.Throwable}}
{\small 

\vskip -2em
\begin{itemize}
\item{\vskip -1.5ex 
\texttt{public final synchronized void {\bf  addSuppressed}(\texttt{Throwable} {\bf  arg0})
}%end signature
}%end item
\item{\vskip -1.5ex 
\texttt{public synchronized Throwable {\bf  fillInStackTrace}()
}%end signature
}%end item
\item{\vskip -1.5ex 
\texttt{public synchronized Throwable {\bf  getCause}()
}%end signature
}%end item
\item{\vskip -1.5ex 
\texttt{public String {\bf  getLocalizedMessage}()
}%end signature
}%end item
\item{\vskip -1.5ex 
\texttt{public String {\bf  getMessage}()
}%end signature
}%end item
\item{\vskip -1.5ex 
\texttt{public StackTraceElement {\bf  getStackTrace}()
}%end signature
}%end item
\item{\vskip -1.5ex 
\texttt{public final synchronized Throwable {\bf  getSuppressed}()
}%end signature
}%end item
\item{\vskip -1.5ex 
\texttt{public synchronized Throwable {\bf  initCause}(\texttt{Throwable} {\bf  arg0})
}%end signature
}%end item
\item{\vskip -1.5ex 
\texttt{public void {\bf  printStackTrace}()
}%end signature
}%end item
\item{\vskip -1.5ex 
\texttt{public void {\bf  printStackTrace}(\texttt{java.io.PrintStream} {\bf  arg0})
}%end signature
}%end item
\item{\vskip -1.5ex 
\texttt{public void {\bf  printStackTrace}(\texttt{java.io.PrintWriter} {\bf  arg0})
}%end signature
}%end item
\item{\vskip -1.5ex 
\texttt{public void {\bf  setStackTrace}(\texttt{StackTraceElement\lbrack \rbrack } {\bf  arg0})
}%end signature
}%end item
\item{\vskip -1.5ex 
\texttt{public String {\bf  toString}()
}%end signature
}%end item
\end{itemize}
}
}
\section{\label{it.matlice.ingsw.model.exceptions.CannotParseIntervalException}\index{CannotParseIntervalException}Exception CannotParseIntervalException}{
\vskip .1in 
\subsection{Declaration}{
\begin{lstlisting}[frame=none]
public class CannotParseIntervalException
 extends java.lang.Exception\end{lstlisting}
\subsection{Constructor summary}{
\begin{verse}
{\bf CannotParseIntervalException()} \\
\end{verse}
}
\subsection{Constructors}{
\vskip -2em
\begin{itemize}
\item{ 
\index{CannotParseIntervalException()}
{\bf  CannotParseIntervalException}\\
\begin{lstlisting}[frame=none]
public CannotParseIntervalException()\end{lstlisting} %end signature
}%end item
\end{itemize}
}
\subsection{Members inherited from class Throwable }{
\texttt{java.lang.Throwable} {\small 
\refdefined{java.lang.Throwable}}
{\small 

\vskip -2em
\begin{itemize}
\item{\vskip -1.5ex 
\texttt{public final synchronized void {\bf  addSuppressed}(\texttt{Throwable} {\bf  arg0})
}%end signature
}%end item
\item{\vskip -1.5ex 
\texttt{public synchronized Throwable {\bf  fillInStackTrace}()
}%end signature
}%end item
\item{\vskip -1.5ex 
\texttt{public synchronized Throwable {\bf  getCause}()
}%end signature
}%end item
\item{\vskip -1.5ex 
\texttt{public String {\bf  getLocalizedMessage}()
}%end signature
}%end item
\item{\vskip -1.5ex 
\texttt{public String {\bf  getMessage}()
}%end signature
}%end item
\item{\vskip -1.5ex 
\texttt{public StackTraceElement {\bf  getStackTrace}()
}%end signature
}%end item
\item{\vskip -1.5ex 
\texttt{public final synchronized Throwable {\bf  getSuppressed}()
}%end signature
}%end item
\item{\vskip -1.5ex 
\texttt{public synchronized Throwable {\bf  initCause}(\texttt{Throwable} {\bf  arg0})
}%end signature
}%end item
\item{\vskip -1.5ex 
\texttt{public void {\bf  printStackTrace}()
}%end signature
}%end item
\item{\vskip -1.5ex 
\texttt{public void {\bf  printStackTrace}(\texttt{java.io.PrintStream} {\bf  arg0})
}%end signature
}%end item
\item{\vskip -1.5ex 
\texttt{public void {\bf  printStackTrace}(\texttt{java.io.PrintWriter} {\bf  arg0})
}%end signature
}%end item
\item{\vskip -1.5ex 
\texttt{public void {\bf  setStackTrace}(\texttt{StackTraceElement\lbrack \rbrack } {\bf  arg0})
}%end signature
}%end item
\item{\vskip -1.5ex 
\texttt{public String {\bf  toString}()
}%end signature
}%end item
\end{itemize}
}
}
\section{\label{it.matlice.ingsw.model.exceptions.CannotParseTimeException}\index{CannotParseTimeException}Exception CannotParseTimeException}{
\vskip .1in 
\subsection{Declaration}{
\begin{lstlisting}[frame=none]
public class CannotParseTimeException
 extends java.lang.Exception\end{lstlisting}
\subsection{Constructor summary}{
\begin{verse}
{\bf CannotParseTimeException()} \\
\end{verse}
}
\subsection{Constructors}{
\vskip -2em
\begin{itemize}
\item{ 
\index{CannotParseTimeException()}
{\bf  CannotParseTimeException}\\
\begin{lstlisting}[frame=none]
public CannotParseTimeException()\end{lstlisting} %end signature
}%end item
\end{itemize}
}
\subsection{Members inherited from class Throwable }{
\texttt{java.lang.Throwable} {\small 
\refdefined{java.lang.Throwable}}
{\small 

\vskip -2em
\begin{itemize}
\item{\vskip -1.5ex 
\texttt{public final synchronized void {\bf  addSuppressed}(\texttt{Throwable} {\bf  arg0})
}%end signature
}%end item
\item{\vskip -1.5ex 
\texttt{public synchronized Throwable {\bf  fillInStackTrace}()
}%end signature
}%end item
\item{\vskip -1.5ex 
\texttt{public synchronized Throwable {\bf  getCause}()
}%end signature
}%end item
\item{\vskip -1.5ex 
\texttt{public String {\bf  getLocalizedMessage}()
}%end signature
}%end item
\item{\vskip -1.5ex 
\texttt{public String {\bf  getMessage}()
}%end signature
}%end item
\item{\vskip -1.5ex 
\texttt{public StackTraceElement {\bf  getStackTrace}()
}%end signature
}%end item
\item{\vskip -1.5ex 
\texttt{public final synchronized Throwable {\bf  getSuppressed}()
}%end signature
}%end item
\item{\vskip -1.5ex 
\texttt{public synchronized Throwable {\bf  initCause}(\texttt{Throwable} {\bf  arg0})
}%end signature
}%end item
\item{\vskip -1.5ex 
\texttt{public void {\bf  printStackTrace}()
}%end signature
}%end item
\item{\vskip -1.5ex 
\texttt{public void {\bf  printStackTrace}(\texttt{java.io.PrintStream} {\bf  arg0})
}%end signature
}%end item
\item{\vskip -1.5ex 
\texttt{public void {\bf  printStackTrace}(\texttt{java.io.PrintWriter} {\bf  arg0})
}%end signature
}%end item
\item{\vskip -1.5ex 
\texttt{public void {\bf  setStackTrace}(\texttt{StackTraceElement\lbrack \rbrack } {\bf  arg0})
}%end signature
}%end item
\item{\vskip -1.5ex 
\texttt{public String {\bf  toString}()
}%end signature
}%end item
\end{itemize}
}
}
\section{\label{it.matlice.ingsw.model.exceptions.DuplicateCategoryException}\index{DuplicateCategoryException}Exception DuplicateCategoryException}{
\vskip .1in 
\subsection{Declaration}{
\begin{lstlisting}[frame=none]
public class DuplicateCategoryException
 extends java.lang.Exception\end{lstlisting}
\subsection{Constructor summary}{
\begin{verse}
{\bf DuplicateCategoryException()} \\
\end{verse}
}
\subsection{Constructors}{
\vskip -2em
\begin{itemize}
\item{ 
\index{DuplicateCategoryException()}
{\bf  DuplicateCategoryException}\\
\begin{lstlisting}[frame=none]
public DuplicateCategoryException()\end{lstlisting} %end signature
}%end item
\end{itemize}
}
\subsection{Members inherited from class Throwable }{
\texttt{java.lang.Throwable} {\small 
\refdefined{java.lang.Throwable}}
{\small 

\vskip -2em
\begin{itemize}
\item{\vskip -1.5ex 
\texttt{public final synchronized void {\bf  addSuppressed}(\texttt{Throwable} {\bf  arg0})
}%end signature
}%end item
\item{\vskip -1.5ex 
\texttt{public synchronized Throwable {\bf  fillInStackTrace}()
}%end signature
}%end item
\item{\vskip -1.5ex 
\texttt{public synchronized Throwable {\bf  getCause}()
}%end signature
}%end item
\item{\vskip -1.5ex 
\texttt{public String {\bf  getLocalizedMessage}()
}%end signature
}%end item
\item{\vskip -1.5ex 
\texttt{public String {\bf  getMessage}()
}%end signature
}%end item
\item{\vskip -1.5ex 
\texttt{public StackTraceElement {\bf  getStackTrace}()
}%end signature
}%end item
\item{\vskip -1.5ex 
\texttt{public final synchronized Throwable {\bf  getSuppressed}()
}%end signature
}%end item
\item{\vskip -1.5ex 
\texttt{public synchronized Throwable {\bf  initCause}(\texttt{Throwable} {\bf  arg0})
}%end signature
}%end item
\item{\vskip -1.5ex 
\texttt{public void {\bf  printStackTrace}()
}%end signature
}%end item
\item{\vskip -1.5ex 
\texttt{public void {\bf  printStackTrace}(\texttt{java.io.PrintStream} {\bf  arg0})
}%end signature
}%end item
\item{\vskip -1.5ex 
\texttt{public void {\bf  printStackTrace}(\texttt{java.io.PrintWriter} {\bf  arg0})
}%end signature
}%end item
\item{\vskip -1.5ex 
\texttt{public void {\bf  setStackTrace}(\texttt{StackTraceElement\lbrack \rbrack } {\bf  arg0})
}%end signature
}%end item
\item{\vskip -1.5ex 
\texttt{public String {\bf  toString}()
}%end signature
}%end item
\end{itemize}
}
}
\section{\label{it.matlice.ingsw.model.exceptions.DuplicateFieldException}\index{DuplicateFieldException}Exception DuplicateFieldException}{
\vskip .1in 
\subsection{Declaration}{
\begin{lstlisting}[frame=none]
public class DuplicateFieldException
 extends java.lang.Exception\end{lstlisting}
\subsection{Constructor summary}{
\begin{verse}
{\bf DuplicateFieldException()} \\
\end{verse}
}
\subsection{Constructors}{
\vskip -2em
\begin{itemize}
\item{ 
\index{DuplicateFieldException()}
{\bf  DuplicateFieldException}\\
\begin{lstlisting}[frame=none]
public DuplicateFieldException()\end{lstlisting} %end signature
}%end item
\end{itemize}
}
\subsection{Members inherited from class Throwable }{
\texttt{java.lang.Throwable} {\small 
\refdefined{java.lang.Throwable}}
{\small 

\vskip -2em
\begin{itemize}
\item{\vskip -1.5ex 
\texttt{public final synchronized void {\bf  addSuppressed}(\texttt{Throwable} {\bf  arg0})
}%end signature
}%end item
\item{\vskip -1.5ex 
\texttt{public synchronized Throwable {\bf  fillInStackTrace}()
}%end signature
}%end item
\item{\vskip -1.5ex 
\texttt{public synchronized Throwable {\bf  getCause}()
}%end signature
}%end item
\item{\vskip -1.5ex 
\texttt{public String {\bf  getLocalizedMessage}()
}%end signature
}%end item
\item{\vskip -1.5ex 
\texttt{public String {\bf  getMessage}()
}%end signature
}%end item
\item{\vskip -1.5ex 
\texttt{public StackTraceElement {\bf  getStackTrace}()
}%end signature
}%end item
\item{\vskip -1.5ex 
\texttt{public final synchronized Throwable {\bf  getSuppressed}()
}%end signature
}%end item
\item{\vskip -1.5ex 
\texttt{public synchronized Throwable {\bf  initCause}(\texttt{Throwable} {\bf  arg0})
}%end signature
}%end item
\item{\vskip -1.5ex 
\texttt{public void {\bf  printStackTrace}()
}%end signature
}%end item
\item{\vskip -1.5ex 
\texttt{public void {\bf  printStackTrace}(\texttt{java.io.PrintStream} {\bf  arg0})
}%end signature
}%end item
\item{\vskip -1.5ex 
\texttt{public void {\bf  printStackTrace}(\texttt{java.io.PrintWriter} {\bf  arg0})
}%end signature
}%end item
\item{\vskip -1.5ex 
\texttt{public void {\bf  setStackTrace}(\texttt{StackTraceElement\lbrack \rbrack } {\bf  arg0})
}%end signature
}%end item
\item{\vskip -1.5ex 
\texttt{public String {\bf  toString}()
}%end signature
}%end item
\end{itemize}
}
}
\section{\label{it.matlice.ingsw.model.exceptions.DuplicateUserException}\index{DuplicateUserException}Exception DuplicateUserException}{
\vskip .1in 
\subsection{Declaration}{
\begin{lstlisting}[frame=none]
public class DuplicateUserException
 extends java.lang.Exception\end{lstlisting}
\subsection{Constructor summary}{
\begin{verse}
{\bf DuplicateUserException()} \\
\end{verse}
}
\subsection{Constructors}{
\vskip -2em
\begin{itemize}
\item{ 
\index{DuplicateUserException()}
{\bf  DuplicateUserException}\\
\begin{lstlisting}[frame=none]
public DuplicateUserException()\end{lstlisting} %end signature
}%end item
\end{itemize}
}
\subsection{Members inherited from class Throwable }{
\texttt{java.lang.Throwable} {\small 
\refdefined{java.lang.Throwable}}
{\small 

\vskip -2em
\begin{itemize}
\item{\vskip -1.5ex 
\texttt{public final synchronized void {\bf  addSuppressed}(\texttt{Throwable} {\bf  arg0})
}%end signature
}%end item
\item{\vskip -1.5ex 
\texttt{public synchronized Throwable {\bf  fillInStackTrace}()
}%end signature
}%end item
\item{\vskip -1.5ex 
\texttt{public synchronized Throwable {\bf  getCause}()
}%end signature
}%end item
\item{\vskip -1.5ex 
\texttt{public String {\bf  getLocalizedMessage}()
}%end signature
}%end item
\item{\vskip -1.5ex 
\texttt{public String {\bf  getMessage}()
}%end signature
}%end item
\item{\vskip -1.5ex 
\texttt{public StackTraceElement {\bf  getStackTrace}()
}%end signature
}%end item
\item{\vskip -1.5ex 
\texttt{public final synchronized Throwable {\bf  getSuppressed}()
}%end signature
}%end item
\item{\vskip -1.5ex 
\texttt{public synchronized Throwable {\bf  initCause}(\texttt{Throwable} {\bf  arg0})
}%end signature
}%end item
\item{\vskip -1.5ex 
\texttt{public void {\bf  printStackTrace}()
}%end signature
}%end item
\item{\vskip -1.5ex 
\texttt{public void {\bf  printStackTrace}(\texttt{java.io.PrintStream} {\bf  arg0})
}%end signature
}%end item
\item{\vskip -1.5ex 
\texttt{public void {\bf  printStackTrace}(\texttt{java.io.PrintWriter} {\bf  arg0})
}%end signature
}%end item
\item{\vskip -1.5ex 
\texttt{public void {\bf  setStackTrace}(\texttt{StackTraceElement\lbrack \rbrack } {\bf  arg0})
}%end signature
}%end item
\item{\vskip -1.5ex 
\texttt{public String {\bf  toString}()
}%end signature
}%end item
\end{itemize}
}
}
\section{\label{it.matlice.ingsw.model.exceptions.InvalidCategoryException}\index{InvalidCategoryException}Exception InvalidCategoryException}{
\vskip .1in 
\subsection{Declaration}{
\begin{lstlisting}[frame=none]
public class InvalidCategoryException
 extends java.lang.Exception\end{lstlisting}
\subsection{Constructor summary}{
\begin{verse}
{\bf InvalidCategoryException()} \\
\end{verse}
}
\subsection{Constructors}{
\vskip -2em
\begin{itemize}
\item{ 
\index{InvalidCategoryException()}
{\bf  InvalidCategoryException}\\
\begin{lstlisting}[frame=none]
public InvalidCategoryException()\end{lstlisting} %end signature
}%end item
\end{itemize}
}
\subsection{Members inherited from class Throwable }{
\texttt{java.lang.Throwable} {\small 
\refdefined{java.lang.Throwable}}
{\small 

\vskip -2em
\begin{itemize}
\item{\vskip -1.5ex 
\texttt{public final synchronized void {\bf  addSuppressed}(\texttt{Throwable} {\bf  arg0})
}%end signature
}%end item
\item{\vskip -1.5ex 
\texttt{public synchronized Throwable {\bf  fillInStackTrace}()
}%end signature
}%end item
\item{\vskip -1.5ex 
\texttt{public synchronized Throwable {\bf  getCause}()
}%end signature
}%end item
\item{\vskip -1.5ex 
\texttt{public String {\bf  getLocalizedMessage}()
}%end signature
}%end item
\item{\vskip -1.5ex 
\texttt{public String {\bf  getMessage}()
}%end signature
}%end item
\item{\vskip -1.5ex 
\texttt{public StackTraceElement {\bf  getStackTrace}()
}%end signature
}%end item
\item{\vskip -1.5ex 
\texttt{public final synchronized Throwable {\bf  getSuppressed}()
}%end signature
}%end item
\item{\vskip -1.5ex 
\texttt{public synchronized Throwable {\bf  initCause}(\texttt{Throwable} {\bf  arg0})
}%end signature
}%end item
\item{\vskip -1.5ex 
\texttt{public void {\bf  printStackTrace}()
}%end signature
}%end item
\item{\vskip -1.5ex 
\texttt{public void {\bf  printStackTrace}(\texttt{java.io.PrintStream} {\bf  arg0})
}%end signature
}%end item
\item{\vskip -1.5ex 
\texttt{public void {\bf  printStackTrace}(\texttt{java.io.PrintWriter} {\bf  arg0})
}%end signature
}%end item
\item{\vskip -1.5ex 
\texttt{public void {\bf  setStackTrace}(\texttt{StackTraceElement\lbrack \rbrack } {\bf  arg0})
}%end signature
}%end item
\item{\vskip -1.5ex 
\texttt{public String {\bf  toString}()
}%end signature
}%end item
\end{itemize}
}
}
\section{\label{it.matlice.ingsw.model.exceptions.InvalidFieldException}\index{InvalidFieldException}Exception InvalidFieldException}{
\vskip .1in 
\subsection{Declaration}{
\begin{lstlisting}[frame=none]
public class InvalidFieldException
 extends java.lang.Exception\end{lstlisting}
\subsection{Constructor summary}{
\begin{verse}
{\bf InvalidFieldException()} \\
\end{verse}
}
\subsection{Constructors}{
\vskip -2em
\begin{itemize}
\item{ 
\index{InvalidFieldException()}
{\bf  InvalidFieldException}\\
\begin{lstlisting}[frame=none]
public InvalidFieldException()\end{lstlisting} %end signature
}%end item
\end{itemize}
}
\subsection{Members inherited from class Throwable }{
\texttt{java.lang.Throwable} {\small 
\refdefined{java.lang.Throwable}}
{\small 

\vskip -2em
\begin{itemize}
\item{\vskip -1.5ex 
\texttt{public final synchronized void {\bf  addSuppressed}(\texttt{Throwable} {\bf  arg0})
}%end signature
}%end item
\item{\vskip -1.5ex 
\texttt{public synchronized Throwable {\bf  fillInStackTrace}()
}%end signature
}%end item
\item{\vskip -1.5ex 
\texttt{public synchronized Throwable {\bf  getCause}()
}%end signature
}%end item
\item{\vskip -1.5ex 
\texttt{public String {\bf  getLocalizedMessage}()
}%end signature
}%end item
\item{\vskip -1.5ex 
\texttt{public String {\bf  getMessage}()
}%end signature
}%end item
\item{\vskip -1.5ex 
\texttt{public StackTraceElement {\bf  getStackTrace}()
}%end signature
}%end item
\item{\vskip -1.5ex 
\texttt{public final synchronized Throwable {\bf  getSuppressed}()
}%end signature
}%end item
\item{\vskip -1.5ex 
\texttt{public synchronized Throwable {\bf  initCause}(\texttt{Throwable} {\bf  arg0})
}%end signature
}%end item
\item{\vskip -1.5ex 
\texttt{public void {\bf  printStackTrace}()
}%end signature
}%end item
\item{\vskip -1.5ex 
\texttt{public void {\bf  printStackTrace}(\texttt{java.io.PrintStream} {\bf  arg0})
}%end signature
}%end item
\item{\vskip -1.5ex 
\texttt{public void {\bf  printStackTrace}(\texttt{java.io.PrintWriter} {\bf  arg0})
}%end signature
}%end item
\item{\vskip -1.5ex 
\texttt{public void {\bf  setStackTrace}(\texttt{StackTraceElement\lbrack \rbrack } {\bf  arg0})
}%end signature
}%end item
\item{\vskip -1.5ex 
\texttt{public String {\bf  toString}()
}%end signature
}%end item
\end{itemize}
}
}
\section{\label{it.matlice.ingsw.model.exceptions.InvalidIntervalException}\index{InvalidIntervalException}Exception InvalidIntervalException}{
\vskip .1in 
\subsection{Declaration}{
\begin{lstlisting}[frame=none]
public class InvalidIntervalException
 extends java.lang.Exception\end{lstlisting}
\subsection{Constructor summary}{
\begin{verse}
{\bf InvalidIntervalException()} \\
\end{verse}
}
\subsection{Constructors}{
\vskip -2em
\begin{itemize}
\item{ 
\index{InvalidIntervalException()}
{\bf  InvalidIntervalException}\\
\begin{lstlisting}[frame=none]
public InvalidIntervalException()\end{lstlisting} %end signature
}%end item
\end{itemize}
}
\subsection{Members inherited from class Throwable }{
\texttt{java.lang.Throwable} {\small 
\refdefined{java.lang.Throwable}}
{\small 

\vskip -2em
\begin{itemize}
\item{\vskip -1.5ex 
\texttt{public final synchronized void {\bf  addSuppressed}(\texttt{Throwable} {\bf  arg0})
}%end signature
}%end item
\item{\vskip -1.5ex 
\texttt{public synchronized Throwable {\bf  fillInStackTrace}()
}%end signature
}%end item
\item{\vskip -1.5ex 
\texttt{public synchronized Throwable {\bf  getCause}()
}%end signature
}%end item
\item{\vskip -1.5ex 
\texttt{public String {\bf  getLocalizedMessage}()
}%end signature
}%end item
\item{\vskip -1.5ex 
\texttt{public String {\bf  getMessage}()
}%end signature
}%end item
\item{\vskip -1.5ex 
\texttt{public StackTraceElement {\bf  getStackTrace}()
}%end signature
}%end item
\item{\vskip -1.5ex 
\texttt{public final synchronized Throwable {\bf  getSuppressed}()
}%end signature
}%end item
\item{\vskip -1.5ex 
\texttt{public synchronized Throwable {\bf  initCause}(\texttt{Throwable} {\bf  arg0})
}%end signature
}%end item
\item{\vskip -1.5ex 
\texttt{public void {\bf  printStackTrace}()
}%end signature
}%end item
\item{\vskip -1.5ex 
\texttt{public void {\bf  printStackTrace}(\texttt{java.io.PrintStream} {\bf  arg0})
}%end signature
}%end item
\item{\vskip -1.5ex 
\texttt{public void {\bf  printStackTrace}(\texttt{java.io.PrintWriter} {\bf  arg0})
}%end signature
}%end item
\item{\vskip -1.5ex 
\texttt{public void {\bf  setStackTrace}(\texttt{StackTraceElement\lbrack \rbrack } {\bf  arg0})
}%end signature
}%end item
\item{\vskip -1.5ex 
\texttt{public String {\bf  toString}()
}%end signature
}%end item
\end{itemize}
}
}
\section{\label{it.matlice.ingsw.model.exceptions.InvalidTimeException}\index{InvalidTimeException}Exception InvalidTimeException}{
\vskip .1in 
\subsection{Declaration}{
\begin{lstlisting}[frame=none]
public class InvalidTimeException
 extends java.lang.Exception\end{lstlisting}
\subsection{Constructor summary}{
\begin{verse}
{\bf InvalidTimeException()} \\
\end{verse}
}
\subsection{Constructors}{
\vskip -2em
\begin{itemize}
\item{ 
\index{InvalidTimeException()}
{\bf  InvalidTimeException}\\
\begin{lstlisting}[frame=none]
public InvalidTimeException()\end{lstlisting} %end signature
}%end item
\end{itemize}
}
\subsection{Members inherited from class Throwable }{
\texttt{java.lang.Throwable} {\small 
\refdefined{java.lang.Throwable}}
{\small 

\vskip -2em
\begin{itemize}
\item{\vskip -1.5ex 
\texttt{public final synchronized void {\bf  addSuppressed}(\texttt{Throwable} {\bf  arg0})
}%end signature
}%end item
\item{\vskip -1.5ex 
\texttt{public synchronized Throwable {\bf  fillInStackTrace}()
}%end signature
}%end item
\item{\vskip -1.5ex 
\texttt{public synchronized Throwable {\bf  getCause}()
}%end signature
}%end item
\item{\vskip -1.5ex 
\texttt{public String {\bf  getLocalizedMessage}()
}%end signature
}%end item
\item{\vskip -1.5ex 
\texttt{public String {\bf  getMessage}()
}%end signature
}%end item
\item{\vskip -1.5ex 
\texttt{public StackTraceElement {\bf  getStackTrace}()
}%end signature
}%end item
\item{\vskip -1.5ex 
\texttt{public final synchronized Throwable {\bf  getSuppressed}()
}%end signature
}%end item
\item{\vskip -1.5ex 
\texttt{public synchronized Throwable {\bf  initCause}(\texttt{Throwable} {\bf  arg0})
}%end signature
}%end item
\item{\vskip -1.5ex 
\texttt{public void {\bf  printStackTrace}()
}%end signature
}%end item
\item{\vskip -1.5ex 
\texttt{public void {\bf  printStackTrace}(\texttt{java.io.PrintStream} {\bf  arg0})
}%end signature
}%end item
\item{\vskip -1.5ex 
\texttt{public void {\bf  printStackTrace}(\texttt{java.io.PrintWriter} {\bf  arg0})
}%end signature
}%end item
\item{\vskip -1.5ex 
\texttt{public void {\bf  setStackTrace}(\texttt{StackTraceElement\lbrack \rbrack } {\bf  arg0})
}%end signature
}%end item
\item{\vskip -1.5ex 
\texttt{public String {\bf  toString}()
}%end signature
}%end item
\end{itemize}
}
}
\section{\label{it.matlice.ingsw.model.exceptions.InvalidTradeOfferException}\index{InvalidTradeOfferException}Exception InvalidTradeOfferException}{
\vskip .1in 
\subsection{Declaration}{
\begin{lstlisting}[frame=none]
public class InvalidTradeOfferException
 extends java.lang.Exception\end{lstlisting}
\subsection{Constructor summary}{
\begin{verse}
{\bf InvalidTradeOfferException()} \\
\end{verse}
}
\subsection{Constructors}{
\vskip -2em
\begin{itemize}
\item{ 
\index{InvalidTradeOfferException()}
{\bf  InvalidTradeOfferException}\\
\begin{lstlisting}[frame=none]
public InvalidTradeOfferException()\end{lstlisting} %end signature
}%end item
\end{itemize}
}
\subsection{Members inherited from class Throwable }{
\texttt{java.lang.Throwable} {\small 
\refdefined{java.lang.Throwable}}
{\small 

\vskip -2em
\begin{itemize}
\item{\vskip -1.5ex 
\texttt{public final synchronized void {\bf  addSuppressed}(\texttt{Throwable} {\bf  arg0})
}%end signature
}%end item
\item{\vskip -1.5ex 
\texttt{public synchronized Throwable {\bf  fillInStackTrace}()
}%end signature
}%end item
\item{\vskip -1.5ex 
\texttt{public synchronized Throwable {\bf  getCause}()
}%end signature
}%end item
\item{\vskip -1.5ex 
\texttt{public String {\bf  getLocalizedMessage}()
}%end signature
}%end item
\item{\vskip -1.5ex 
\texttt{public String {\bf  getMessage}()
}%end signature
}%end item
\item{\vskip -1.5ex 
\texttt{public StackTraceElement {\bf  getStackTrace}()
}%end signature
}%end item
\item{\vskip -1.5ex 
\texttt{public final synchronized Throwable {\bf  getSuppressed}()
}%end signature
}%end item
\item{\vskip -1.5ex 
\texttt{public synchronized Throwable {\bf  initCause}(\texttt{Throwable} {\bf  arg0})
}%end signature
}%end item
\item{\vskip -1.5ex 
\texttt{public void {\bf  printStackTrace}()
}%end signature
}%end item
\item{\vskip -1.5ex 
\texttt{public void {\bf  printStackTrace}(\texttt{java.io.PrintStream} {\bf  arg0})
}%end signature
}%end item
\item{\vskip -1.5ex 
\texttt{public void {\bf  printStackTrace}(\texttt{java.io.PrintWriter} {\bf  arg0})
}%end signature
}%end item
\item{\vskip -1.5ex 
\texttt{public void {\bf  setStackTrace}(\texttt{StackTraceElement\lbrack \rbrack } {\bf  arg0})
}%end signature
}%end item
\item{\vskip -1.5ex 
\texttt{public String {\bf  toString}()
}%end signature
}%end item
\end{itemize}
}
}
\section{\label{it.matlice.ingsw.model.exceptions.InvalidUserException}\index{InvalidUserException}Exception InvalidUserException}{
\vskip .1in 
\subsection{Declaration}{
\begin{lstlisting}[frame=none]
public class InvalidUserException
 extends java.lang.Exception\end{lstlisting}
\subsection{Constructor summary}{
\begin{verse}
{\bf InvalidUserException()} \\
\end{verse}
}
\subsection{Constructors}{
\vskip -2em
\begin{itemize}
\item{ 
\index{InvalidUserException()}
{\bf  InvalidUserException}\\
\begin{lstlisting}[frame=none]
public InvalidUserException()\end{lstlisting} %end signature
}%end item
\end{itemize}
}
\subsection{Members inherited from class Throwable }{
\texttt{java.lang.Throwable} {\small 
\refdefined{java.lang.Throwable}}
{\small 

\vskip -2em
\begin{itemize}
\item{\vskip -1.5ex 
\texttt{public final synchronized void {\bf  addSuppressed}(\texttt{Throwable} {\bf  arg0})
}%end signature
}%end item
\item{\vskip -1.5ex 
\texttt{public synchronized Throwable {\bf  fillInStackTrace}()
}%end signature
}%end item
\item{\vskip -1.5ex 
\texttt{public synchronized Throwable {\bf  getCause}()
}%end signature
}%end item
\item{\vskip -1.5ex 
\texttt{public String {\bf  getLocalizedMessage}()
}%end signature
}%end item
\item{\vskip -1.5ex 
\texttt{public String {\bf  getMessage}()
}%end signature
}%end item
\item{\vskip -1.5ex 
\texttt{public StackTraceElement {\bf  getStackTrace}()
}%end signature
}%end item
\item{\vskip -1.5ex 
\texttt{public final synchronized Throwable {\bf  getSuppressed}()
}%end signature
}%end item
\item{\vskip -1.5ex 
\texttt{public synchronized Throwable {\bf  initCause}(\texttt{Throwable} {\bf  arg0})
}%end signature
}%end item
\item{\vskip -1.5ex 
\texttt{public void {\bf  printStackTrace}()
}%end signature
}%end item
\item{\vskip -1.5ex 
\texttt{public void {\bf  printStackTrace}(\texttt{java.io.PrintStream} {\bf  arg0})
}%end signature
}%end item
\item{\vskip -1.5ex 
\texttt{public void {\bf  printStackTrace}(\texttt{java.io.PrintWriter} {\bf  arg0})
}%end signature
}%end item
\item{\vskip -1.5ex 
\texttt{public void {\bf  setStackTrace}(\texttt{StackTraceElement\lbrack \rbrack } {\bf  arg0})
}%end signature
}%end item
\item{\vskip -1.5ex 
\texttt{public String {\bf  toString}()
}%end signature
}%end item
\end{itemize}
}
}
\section{\label{it.matlice.ingsw.model.exceptions.InvalidUserTypeException}\index{InvalidUserTypeException}Exception InvalidUserTypeException}{
\vskip .1in 
\subsection{Declaration}{
\begin{lstlisting}[frame=none]
public class InvalidUserTypeException
 extends java.lang.Exception\end{lstlisting}
\subsection{Constructor summary}{
\begin{verse}
{\bf InvalidUserTypeException()} \\
\end{verse}
}
\subsection{Constructors}{
\vskip -2em
\begin{itemize}
\item{ 
\index{InvalidUserTypeException()}
{\bf  InvalidUserTypeException}\\
\begin{lstlisting}[frame=none]
public InvalidUserTypeException()\end{lstlisting} %end signature
}%end item
\end{itemize}
}
\subsection{Members inherited from class Throwable }{
\texttt{java.lang.Throwable} {\small 
\refdefined{java.lang.Throwable}}
{\small 

\vskip -2em
\begin{itemize}
\item{\vskip -1.5ex 
\texttt{public final synchronized void {\bf  addSuppressed}(\texttt{Throwable} {\bf  arg0})
}%end signature
}%end item
\item{\vskip -1.5ex 
\texttt{public synchronized Throwable {\bf  fillInStackTrace}()
}%end signature
}%end item
\item{\vskip -1.5ex 
\texttt{public synchronized Throwable {\bf  getCause}()
}%end signature
}%end item
\item{\vskip -1.5ex 
\texttt{public String {\bf  getLocalizedMessage}()
}%end signature
}%end item
\item{\vskip -1.5ex 
\texttt{public String {\bf  getMessage}()
}%end signature
}%end item
\item{\vskip -1.5ex 
\texttt{public StackTraceElement {\bf  getStackTrace}()
}%end signature
}%end item
\item{\vskip -1.5ex 
\texttt{public final synchronized Throwable {\bf  getSuppressed}()
}%end signature
}%end item
\item{\vskip -1.5ex 
\texttt{public synchronized Throwable {\bf  initCause}(\texttt{Throwable} {\bf  arg0})
}%end signature
}%end item
\item{\vskip -1.5ex 
\texttt{public void {\bf  printStackTrace}()
}%end signature
}%end item
\item{\vskip -1.5ex 
\texttt{public void {\bf  printStackTrace}(\texttt{java.io.PrintStream} {\bf  arg0})
}%end signature
}%end item
\item{\vskip -1.5ex 
\texttt{public void {\bf  printStackTrace}(\texttt{java.io.PrintWriter} {\bf  arg0})
}%end signature
}%end item
\item{\vskip -1.5ex 
\texttt{public void {\bf  setStackTrace}(\texttt{StackTraceElement\lbrack \rbrack } {\bf  arg0})
}%end signature
}%end item
\item{\vskip -1.5ex 
\texttt{public String {\bf  toString}()
}%end signature
}%end item
\end{itemize}
}
}
\section{\label{it.matlice.ingsw.model.exceptions.LoginInvalidException}\index{LoginInvalidException}Exception LoginInvalidException}{
\vskip .1in 
\subsection{Declaration}{
\begin{lstlisting}[frame=none]
public class LoginInvalidException
 extends java.lang.Exception\end{lstlisting}
\subsection{Constructor summary}{
\begin{verse}
{\bf LoginInvalidException()} \\
\end{verse}
}
\subsection{Constructors}{
\vskip -2em
\begin{itemize}
\item{ 
\index{LoginInvalidException()}
{\bf  LoginInvalidException}\\
\begin{lstlisting}[frame=none]
public LoginInvalidException()\end{lstlisting} %end signature
}%end item
\end{itemize}
}
\subsection{Members inherited from class Throwable }{
\texttt{java.lang.Throwable} {\small 
\refdefined{java.lang.Throwable}}
{\small 

\vskip -2em
\begin{itemize}
\item{\vskip -1.5ex 
\texttt{public final synchronized void {\bf  addSuppressed}(\texttt{Throwable} {\bf  arg0})
}%end signature
}%end item
\item{\vskip -1.5ex 
\texttt{public synchronized Throwable {\bf  fillInStackTrace}()
}%end signature
}%end item
\item{\vskip -1.5ex 
\texttt{public synchronized Throwable {\bf  getCause}()
}%end signature
}%end item
\item{\vskip -1.5ex 
\texttt{public String {\bf  getLocalizedMessage}()
}%end signature
}%end item
\item{\vskip -1.5ex 
\texttt{public String {\bf  getMessage}()
}%end signature
}%end item
\item{\vskip -1.5ex 
\texttt{public StackTraceElement {\bf  getStackTrace}()
}%end signature
}%end item
\item{\vskip -1.5ex 
\texttt{public final synchronized Throwable {\bf  getSuppressed}()
}%end signature
}%end item
\item{\vskip -1.5ex 
\texttt{public synchronized Throwable {\bf  initCause}(\texttt{Throwable} {\bf  arg0})
}%end signature
}%end item
\item{\vskip -1.5ex 
\texttt{public void {\bf  printStackTrace}()
}%end signature
}%end item
\item{\vskip -1.5ex 
\texttt{public void {\bf  printStackTrace}(\texttt{java.io.PrintStream} {\bf  arg0})
}%end signature
}%end item
\item{\vskip -1.5ex 
\texttt{public void {\bf  printStackTrace}(\texttt{java.io.PrintWriter} {\bf  arg0})
}%end signature
}%end item
\item{\vskip -1.5ex 
\texttt{public void {\bf  setStackTrace}(\texttt{StackTraceElement\lbrack \rbrack } {\bf  arg0})
}%end signature
}%end item
\item{\vskip -1.5ex 
\texttt{public String {\bf  toString}()
}%end signature
}%end item
\end{itemize}
}
}
\section{\label{it.matlice.ingsw.model.exceptions.RequiredFieldConstrainException}\index{RequiredFieldConstrainException}Exception RequiredFieldConstrainException}{
\vskip .1in 
\subsection{Declaration}{
\begin{lstlisting}[frame=none]
public class RequiredFieldConstrainException
 extends java.lang.Exception\end{lstlisting}
\subsection{Constructor summary}{
\begin{verse}
{\bf RequiredFieldConstrainException(String, LeafCategory)} \\
\end{verse}
}
\subsection{Method summary}{
\begin{verse}
{\bf getCategory()} \\
{\bf getField()} \\
\end{verse}
}
\subsection{Constructors}{
\vskip -2em
\begin{itemize}
\item{ 
\index{RequiredFieldConstrainException(String, LeafCategory)}
{\bf  RequiredFieldConstrainException}\\
\begin{lstlisting}[frame=none]
public RequiredFieldConstrainException(java.lang.String field,it.matlice.ingsw.model.data.LeafCategory category)\end{lstlisting} %end signature
}%end item
\end{itemize}
}
\subsection{Methods}{
\vskip -2em
\begin{itemize}
\item{ 
\index{getCategory()}
{\bf  getCategory}\\
\begin{lstlisting}[frame=none]
public it.matlice.ingsw.model.data.LeafCategory getCategory()\end{lstlisting} %end signature
}%end item
\item{ 
\index{getField()}
{\bf  getField}\\
\begin{lstlisting}[frame=none]
public java.lang.String getField()\end{lstlisting} %end signature
}%end item
\end{itemize}
}
\subsection{Members inherited from class Throwable }{
\texttt{java.lang.Throwable} {\small 
\refdefined{java.lang.Throwable}}
{\small 

\vskip -2em
\begin{itemize}
\item{\vskip -1.5ex 
\texttt{public final synchronized void {\bf  addSuppressed}(\texttt{Throwable} {\bf  arg0})
}%end signature
}%end item
\item{\vskip -1.5ex 
\texttt{public synchronized Throwable {\bf  fillInStackTrace}()
}%end signature
}%end item
\item{\vskip -1.5ex 
\texttt{public synchronized Throwable {\bf  getCause}()
}%end signature
}%end item
\item{\vskip -1.5ex 
\texttt{public String {\bf  getLocalizedMessage}()
}%end signature
}%end item
\item{\vskip -1.5ex 
\texttt{public String {\bf  getMessage}()
}%end signature
}%end item
\item{\vskip -1.5ex 
\texttt{public StackTraceElement {\bf  getStackTrace}()
}%end signature
}%end item
\item{\vskip -1.5ex 
\texttt{public final synchronized Throwable {\bf  getSuppressed}()
}%end signature
}%end item
\item{\vskip -1.5ex 
\texttt{public synchronized Throwable {\bf  initCause}(\texttt{Throwable} {\bf  arg0})
}%end signature
}%end item
\item{\vskip -1.5ex 
\texttt{public void {\bf  printStackTrace}()
}%end signature
}%end item
\item{\vskip -1.5ex 
\texttt{public void {\bf  printStackTrace}(\texttt{java.io.PrintStream} {\bf  arg0})
}%end signature
}%end item
\item{\vskip -1.5ex 
\texttt{public void {\bf  printStackTrace}(\texttt{java.io.PrintWriter} {\bf  arg0})
}%end signature
}%end item
\item{\vskip -1.5ex 
\texttt{public void {\bf  setStackTrace}(\texttt{StackTraceElement\lbrack \rbrack } {\bf  arg0})
}%end signature
}%end item
\item{\vskip -1.5ex 
\texttt{public String {\bf  toString}()
}%end signature
}%end item
\end{itemize}
}
}
}
\chapter{Package it.matlice.ingsw.model}{
\label{it.matlice.ingsw.model}\hskip -.05in
\hbox to \hsize{\textit{ Package Contents\hfil Page}}
\vskip .13in
\hbox{{\bf  Interfaces}}
\entityintro{Authentication}{it.matlice.ingsw.model.Authentication}{Rappresenta un token di accesso.}
\vskip .13in
\hbox{{\bf  Classes}}
\entityintro{Model}{it.matlice.ingsw.model.Model}{Model dell'applicazione, fornisce i metodi per accedere ai dati}
\entityintro{Settings}{it.matlice.ingsw.model.Settings}{}
\vskip .1in
\vskip .1in
\section{\label{it.matlice.ingsw.model.Authentication}\index{Authentication@\textit{ Authentication}}Interface Authentication}{
\vskip .1in 
Rappresenta un token di accesso. funzionalità simili a jwt\vskip .1in 
\subsection{Declaration}{
\begin{lstlisting}[frame=none]
public interface Authentication
\end{lstlisting}
\subsection{Method summary}{
\begin{verse}
{\bf getExpirationTime()} Ottiene la data entro cui il login è valido\\
{\bf getLoginTime()} Ottiene la data di login\\
{\bf getUser()} \\
{\bf isValid()} \\
{\bf loginProblem()} \\
\end{verse}
}
\subsection{Methods}{
\vskip -2em
\begin{itemize}
\item{ 
\index{getExpirationTime()}
{\bf  getExpirationTime}\\
\begin{lstlisting}[frame=none]
long getExpirationTime()\end{lstlisting} %end signature
\begin{itemize}
\item{
{\bf  Description}

Ottiene la data entro cui il login è valido
}
\item{{\bf  Returns} -- 
unix time stamp della data di scadenza della sessione 
}%end item
\end{itemize}
}%end item
\item{ 
\index{getLoginTime()}
{\bf  getLoginTime}\\
\begin{lstlisting}[frame=none]
long getLoginTime()\end{lstlisting} %end signature
\begin{itemize}
\item{
{\bf  Description}

Ottiene la data di login
}
\item{{\bf  Returns} -- 
unix time stamp del login 
}%end item
\end{itemize}
}%end item
\item{ 
\index{getUser()}
{\bf  getUser}\\
\begin{lstlisting}[frame=none]
data.User getUser()\end{lstlisting} %end signature
\begin{itemize}
\item{{\bf  Returns} -- 
Ritorna l'utente al quale il login è associato 
}%end item
\end{itemize}
}%end item
\item{ 
\index{isValid()}
{\bf  isValid}\\
\begin{lstlisting}[frame=none]
boolean isValid()\end{lstlisting} %end signature
\begin{itemize}
\item{{\bf  Returns} -- 
true se la sessione è valida 
}%end item
\end{itemize}
}%end item
\item{ 
\index{loginProblem()}
{\bf  loginProblem}\\
\begin{lstlisting}[frame=none]
java.lang.String loginProblem()\end{lstlisting} %end signature
\begin{itemize}
\item{{\bf  Returns} -- 
una stringa che identifica l'errore di sessione o null se la sessione è valida 
}%end item
\end{itemize}
}%end item
\end{itemize}
}
}
\section{\label{it.matlice.ingsw.model.Model}\index{Model}Class Model}{
\vskip .1in 
Model dell'applicazione, fornisce i metodi per accedere ai dati\vskip .1in 
\subsection{Declaration}{
\begin{lstlisting}[frame=none]
public class Model
 extends java.lang.Object\end{lstlisting}
\subsection{Constructor summary}{
\begin{verse}
{\bf Model(HierarchyFactory, CategoryFactory, UserFactory, SettingsFactory, OfferFactory, MessageFactory)} Costruttore del Model\\
\end{verse}
}
\subsection{Method summary}{
\begin{verse}
{\bf acceptTrade(Offer, String, Settings.Day, Interval.Time)} Permette di accettare una proposta di scambio, avanzando una prima proposta iniziale di giorno, data e ora\\
{\bf acceptTradeMessage(Message)} Accetta un luogo, giorno e ora per lo scambio, concludendo il processo di scambio\\
{\bf addConfiguratorUser(String, boolean)} Aggiunge un utente configuratore\\
{\bf authenticate(AuthMethod, AuthData)} Dati i parametri di autenticazione, ritorna il token di autenticazione\\
{\bf authenticationType(String)} Ritorna una lista di metodi disponibili per l'utente\\
{\bf changePassword(Authentication, String)} Permette il cambio password all'utente\\
{\bf configureSettings(String, int, List, List, List)} Imposta i parametri di configurazione\\
{\bf createCategory(String, String, Category)} Crea una nuova categoria foglia con i parametri passati\\
{\bf createHierarchy(Category)} Crea una nuova gerarchia con la categoria root specificata\\
{\bf createOffer(User, String, LeafCategory, Map)} Crea un nuovo articolo\\
{\bf createTradeOffer(Offer, Offer)} Aggiunge una proposta di scambio tra due offerte\\
{\bf finalizeLogin(Authentication)} Ultimo passaggio per completare il login Imposta l'ultimo accesso dell'utente\\
{\bf getHierarchies()} Ritorna le gerarchie\\
{\bf getLeafCategories()} Ritorna una lista con tutte le categorie foglia di ogni gerarchia\\
{\bf getOffersByCategory(LeafCategory)} Ritorna la lista delle offerte di una certa categoria\\
{\bf getOffersByUser(User)} Ritorna la lista delle offerte relative all'utente\\
{\bf getRetractableOffers(User)} Ritorna una lista di offerte che possono essere ritirate dall'utente Possono essere ritirate le offerte in stato non RETRACTED\\
{\bf getSelectedOffers(Authentication)} Ritorna la lista di offerte dell'utente che sono state selezionate allo scambio\\
{\bf getTradableOffers(Offer)} Ritorna una lisa di articoli scambiabili con l'articolo dato, ovvero tutte le offerte aperte aperte di altri utenti appartenenti alla stessa caategoria foglia dell'offerta data\\
{\bf getTradableOffers(User)} Ritorna una lista di aticoli dell'utente disponibili allo scambio, ovvero la lista delle sue offerte aperte\\
{\bf getUserMessages(Authentication)} Ritorna la lista di messaggi per l'utente, relativi a offerte in scambio\\
{\bf hasConfiguredSettings()} Ritorna true se sono stati configurati i parametri dell'applicazione (piazza, scadenza, giorni, orari, luoghi)\\
{\bf isValidRootCategoryName(String)} Controlla l'esistenza di una categoria radice il nome passato, non possono esserci due categorie root con lo stesso nome\\
{\bf readSettings()} Ritorna i parametri di configurazione attuali\\
{\bf registerUser(String, String)} Aggiunge un nuovo utente fruitore\\
{\bf replyToMessage(Message, String, Settings.Day, Interval.Time)} Permette di rispondere ad una proposta di scambio in un luogo, giorno e ora con una controproposta\\
{\bf retractOffer(Offer)} Ritira un'offerta\\
{\bf timeIteration()} Verifica tutte le condizioni legate al tempo, come ad esempio le scadenze delle offerte\\
\end{verse}
}
\subsection{Constructors}{
\vskip -2em
\begin{itemize}
\item{ 
\index{Model(HierarchyFactory, CategoryFactory, UserFactory, SettingsFactory, OfferFactory, MessageFactory)}
{\bf  Model}\\
\begin{lstlisting}[frame=none]
public Model(data.factories.HierarchyFactory hf,data.factories.CategoryFactory cf,data.factories.UserFactory uf,data.factories.SettingsFactory sf,data.factories.OfferFactory af,data.factories.MessageFactory mf)\end{lstlisting} %end signature
\begin{itemize}
\item{
{\bf  Description}

Costruttore del Model
}
\item{
{\bf  Parameters}
  \begin{itemize}
   \item{
\texttt{hf} -- la hierarchy factory che permette di interfacciarsi col DB per le gerarchie}
   \item{
\texttt{cf} -- la category factory che permette di interfacciarsi col DB per le categorie}
   \item{
\texttt{uf} -- la user factory che permette di interfacciarsi col DB per gli utenti}
   \item{
\texttt{sf} -- la settings factory che permette di interfacciarsi col DB per i parametri di configurazione}
   \item{
\texttt{af} -- la article factory che permette di interfacciarsi col DB per gli articoli}
  \end{itemize}
}%end item
\end{itemize}
}%end item
\end{itemize}
}
\subsection{Methods}{
\vskip -2em
\begin{itemize}
\item{ 
\index{acceptTrade(Offer, String, Settings.Day, Interval.Time)}
{\bf  acceptTrade}\\
\begin{lstlisting}[frame=none]
public java.util.Calendar acceptTrade(data.Offer offer,java.lang.String location,data.Settings.Day day,data.Interval.Time time)\end{lstlisting} %end signature
\begin{itemize}
\item{
{\bf  Description}

Permette di accettare una proposta di scambio, avanzando una prima proposta iniziale di giorno, data e ora
}
\item{
{\bf  Parameters}
  \begin{itemize}
   \item{
\texttt{offer} -- offerta da accettare}
   \item{
\texttt{location} -- proposta di luogo di scambio}
   \item{
\texttt{day} -- proposta di giorno di scambio}
   \item{
\texttt{time} -- proposta di ora di scambio}
  \end{itemize}
}%end item
\item{{\bf  Returns} -- 
momento dello scambio 
}%end item
\end{itemize}
}%end item
\item{ 
\index{acceptTradeMessage(Message)}
{\bf  acceptTradeMessage}\\
\begin{lstlisting}[frame=none]
public void acceptTradeMessage(data.Message m)\end{lstlisting} %end signature
\begin{itemize}
\item{
{\bf  Description}

Accetta un luogo, giorno e ora per lo scambio, concludendo il processo di scambio
}
\end{itemize}
}%end item
\item{ 
\index{addConfiguratorUser(String, boolean)}
{\bf  addConfiguratorUser}\\
\begin{lstlisting}[frame=none]
public java.lang.String addConfiguratorUser(java.lang.String username,boolean defaultPassword) throws java.sql.SQLException, it.matlice.ingsw.model.exceptions.DuplicateUserException, it.matlice.ingsw.model.exceptions.InvalidUserTypeException, it.matlice.ingsw.model.auth.exceptions.InvalidPasswordException\end{lstlisting} %end signature
\begin{itemize}
\item{
{\bf  Description}

Aggiunge un utente configuratore
}
\item{
{\bf  Parameters}
  \begin{itemize}
   \item{
\texttt{username} -- username dell'utente da creare}
   \item{
\texttt{defaultPassword} -- true per utilizzare una password default, altrimenti la genera casualmente}
  \end{itemize}
}%end item
\item{{\bf  Returns} -- 
password dell'utente appena creato 
}%end item
\item{{\bf  Throws}
  \begin{itemize}
   \item{\vskip -.6ex \texttt{java.sql.SQLException} -- errore di connessione col database}
   \item{\vskip -.6ex \texttt{it.matlice.ingsw.model.exceptions.DuplicateUserException} -- utente con username già esistente}
   \item{\vskip -.6ex \texttt{it.matlice.ingsw.model.auth.exceptions.InvalidPasswordException} -- }
   \item{\vskip -.6ex \texttt{it.matlice.ingsw.model.exceptions.InvalidUserTypeException} -- }
  \end{itemize}
}%end item
\end{itemize}
}%end item
\item{ 
\index{authenticate(AuthMethod, AuthData)}
{\bf  authenticate}\\
\begin{lstlisting}[frame=none]
public Authentication authenticate(auth.AuthMethod method,auth.AuthData data)\end{lstlisting} %end signature
\begin{itemize}
\item{
{\bf  Description}

Dati i parametri di autenticazione, ritorna il token di autenticazione
}
\item{
{\bf  Parameters}
  \begin{itemize}
   \item{
\texttt{method} -- metodo di autenticazione}
   \item{
\texttt{data} -- parametri di autenticazione}
  \end{itemize}
}%end item
\item{{\bf  Returns} -- 
token di autenticazione 
}%end item
\end{itemize}
}%end item
\item{ 
\index{authenticationType(String)}
{\bf  authenticationType}\\
\begin{lstlisting}[frame=none]
public java.util.List authenticationType(java.lang.String username) throws it.matlice.ingsw.model.exceptions.InvalidUserException\end{lstlisting} %end signature
\begin{itemize}
\item{
{\bf  Description}

Ritorna una lista di metodi disponibili per l'utente
}
\item{
{\bf  Parameters}
  \begin{itemize}
   \item{
\texttt{username} -- username dell'utente}
  \end{itemize}
}%end item
\item{{\bf  Returns} -- 
lista di metodi disponibili 
}%end item
\end{itemize}
}%end item
\item{ 
\index{changePassword(Authentication, String)}
{\bf  changePassword}\\
\begin{lstlisting}[frame=none]
public void changePassword(Authentication auth,java.lang.String newPassword) throws java.sql.SQLException, it.matlice.ingsw.model.auth.exceptions.InvalidPasswordException, it.matlice.ingsw.model.exceptions.LoginInvalidException\end{lstlisting} %end signature
\begin{itemize}
\item{
{\bf  Description}

Permette il cambio password all'utente
}
\item{
{\bf  Parameters}
  \begin{itemize}
   \item{
\texttt{auth} -- autenticazione, permette di identificare l'utente e verificare sia loggato}
   \item{
\texttt{newPassword} -- nuova password che si vuole impostare}
  \end{itemize}
}%end item
\item{{\bf  Throws}
  \begin{itemize}
   \item{\vskip -.6ex \texttt{java.sql.SQLException} -- errore di connessione col database}
   \item{\vskip -.6ex \texttt{it.matlice.ingsw.model.auth.exceptions.InvalidPasswordException} -- password non rispettante i requisiti di sicurezza}
   \item{\vskip -.6ex \texttt{it.matlice.ingsw.model.exceptions.LoginInvalidException} -- utente non loggato}
  \end{itemize}
}%end item
\end{itemize}
}%end item
\item{ 
\index{configureSettings(String, int, List, List, List)}
{\bf  configureSettings}\\
\begin{lstlisting}[frame=none]
public boolean configureSettings(java.lang.String city,int daysDue,java.util.List locations,java.util.List days,java.util.List intervals)\end{lstlisting} %end signature
\begin{itemize}
\item{
{\bf  Description}

Imposta i parametri di configurazione
}
\item{
{\bf  Parameters}
  \begin{itemize}
   \item{
\texttt{city} -- piazza di scambio}
   \item{
\texttt{daysDue} -- giorni di scadenza}
   \item{
\texttt{locations} -- luoghi}
   \item{
\texttt{days} -- giorni}
   \item{
\texttt{intervals} -- intervalli}
  \end{itemize}
}%end item
\item{{\bf  Returns} -- 
true se è stato effettuato un tentativo di sovrascrivere la città in memoria, false altrimenti 
}%end item
\end{itemize}
}%end item
\item{ 
\index{createCategory(String, String, Category)}
{\bf  createCategory}\\
\begin{lstlisting}[frame=none]
public data.Category createCategory(java.lang.String name,java.lang.String description,data.Category father)\end{lstlisting} %end signature
\begin{itemize}
\item{
{\bf  Description}

Crea una nuova categoria foglia con i parametri passati
}
\item{
{\bf  Parameters}
  \begin{itemize}
   \item{
\texttt{name} -- nome della categoria}
   \item{
\texttt{description} -- descrizione della categoria}
   \item{
\texttt{father} -- padre della categoria (null se è categoria root)}
  \end{itemize}
}%end item
\item{{\bf  Returns} -- 
 
}%end item
\end{itemize}
}%end item
\item{ 
\index{createHierarchy(Category)}
{\bf  createHierarchy}\\
\begin{lstlisting}[frame=none]
public void createHierarchy(data.Category root) throws java.sql.SQLException\end{lstlisting} %end signature
\begin{itemize}
\item{
{\bf  Description}

Crea una nuova gerarchia con la categoria root specificata
}
\item{
{\bf  Parameters}
  \begin{itemize}
   \item{
\texttt{root} -- categoria root della gerarchia}
  \end{itemize}
}%end item
\item{{\bf  Throws}
  \begin{itemize}
   \item{\vskip -.6ex \texttt{java.sql.SQLException} -- errore di connessione col database}
  \end{itemize}
}%end item
\end{itemize}
}%end item
\item{ 
\index{createOffer(User, String, LeafCategory, Map)}
{\bf  createOffer}\\
\begin{lstlisting}[frame=none]
public data.Offer createOffer(data.User u,java.lang.String name,data.LeafCategory e,java.util.Map fields) throws it.matlice.ingsw.model.exceptions.RequiredFieldConstrainException\end{lstlisting} %end signature
\begin{itemize}
\item{
{\bf  Description}

Crea un nuovo articolo
}
\item{
{\bf  Parameters}
  \begin{itemize}
   \item{
\texttt{e} -- categoria a cui appartiene l'articolo da creare}
  \end{itemize}
}%end item
\item{{\bf  Returns} -- 
articolo creato 
}%end item
\end{itemize}
}%end item
\item{ 
\index{createTradeOffer(Offer, Offer)}
{\bf  createTradeOffer}\\
\begin{lstlisting}[frame=none]
public void createTradeOffer(data.Offer offerToTrade,data.Offer offerToAccept) throws it.matlice.ingsw.model.exceptions.InvalidTradeOfferException\end{lstlisting} %end signature
\begin{itemize}
\item{
{\bf  Description}

Aggiunge una proposta di scambio tra due offerte
}
\item{
{\bf  Parameters}
  \begin{itemize}
   \item{
\texttt{offerToTrade} -- articolo dell'utente che propone lo scambio}
   \item{
\texttt{offerToAccept} -- articolo richiesto in cambio}
  \end{itemize}
}%end item
\end{itemize}
}%end item
\item{ 
\index{finalizeLogin(Authentication)}
{\bf  finalizeLogin}\\
\begin{lstlisting}[frame=none]
public void finalizeLogin(Authentication auth) throws java.sql.SQLException\end{lstlisting} %end signature
\begin{itemize}
\item{
{\bf  Description}

Ultimo passaggio per completare il login Imposta l'ultimo accesso dell'utente
}
\item{
{\bf  Parameters}
  \begin{itemize}
   \item{
\texttt{auth} -- autenticazione, permette di identificare l'utente e verificare sia loggato}
  \end{itemize}
}%end item
\item{{\bf  Throws}
  \begin{itemize}
   \item{\vskip -.6ex \texttt{java.sql.SQLException} -- errore di connessione col database}
  \end{itemize}
}%end item
\end{itemize}
}%end item
\item{ 
\index{getHierarchies()}
{\bf  getHierarchies}\\
\begin{lstlisting}[frame=none]
public java.util.List getHierarchies()\end{lstlisting} %end signature
\begin{itemize}
\item{
{\bf  Description}

Ritorna le gerarchie
}
\item{{\bf  Returns} -- 
lista di gerarchie a sistema 
}%end item
\end{itemize}
}%end item
\item{ 
\index{getLeafCategories()}
{\bf  getLeafCategories}\\
\begin{lstlisting}[frame=none]
public java.util.List getLeafCategories()\end{lstlisting} %end signature
\begin{itemize}
\item{
{\bf  Description}

Ritorna una lista con tutte le categorie foglia di ogni gerarchia
}
\item{{\bf  Returns} -- 
lista di categorie 
}%end item
\end{itemize}
}%end item
\item{ 
\index{getOffersByCategory(LeafCategory)}
{\bf  getOffersByCategory}\\
\begin{lstlisting}[frame=none]
public java.util.List getOffersByCategory(data.LeafCategory cat)\end{lstlisting} %end signature
\begin{itemize}
\item{
{\bf  Description}

Ritorna la lista delle offerte di una certa categoria
}
\item{
{\bf  Parameters}
  \begin{itemize}
   \item{
\texttt{cat} -- categoria}
  \end{itemize}
}%end item
\item{{\bf  Returns} -- 
liste di offerte dell'utente 
}%end item
\end{itemize}
}%end item
\item{ 
\index{getOffersByUser(User)}
{\bf  getOffersByUser}\\
\begin{lstlisting}[frame=none]
public java.util.List getOffersByUser(data.User user)\end{lstlisting} %end signature
\begin{itemize}
\item{
{\bf  Description}

Ritorna la lista delle offerte relative all'utente
}
\item{
{\bf  Parameters}
  \begin{itemize}
   \item{
\texttt{user} -- utente}
  \end{itemize}
}%end item
\item{{\bf  Returns} -- 
liste di offerte dell'utente 
}%end item
\end{itemize}
}%end item
\item{ 
\index{getRetractableOffers(User)}
{\bf  getRetractableOffers}\\
\begin{lstlisting}[frame=none]
public java.util.List getRetractableOffers(data.User user)\end{lstlisting} %end signature
\begin{itemize}
\item{
{\bf  Description}

Ritorna una lista di offerte che possono essere ritirate dall'utente Possono essere ritirate le offerte in stato non RETRACTED
}
\item{
{\bf  Parameters}
  \begin{itemize}
   \item{
\texttt{user} -- utente}
  \end{itemize}
}%end item
\item{{\bf  Returns} -- 
lista di offerte ritirabili 
}%end item
\end{itemize}
}%end item
\item{ 
\index{getSelectedOffers(Authentication)}
{\bf  getSelectedOffers}\\
\begin{lstlisting}[frame=none]
public java.util.List getSelectedOffers(Authentication auth)\end{lstlisting} %end signature
\begin{itemize}
\item{
{\bf  Description}

Ritorna la lista di offerte dell'utente che sono state selezionate allo scambio
}
\item{
{\bf  Parameters}
  \begin{itemize}
   \item{
\texttt{auth} -- token di autenticazione dell'utente a cui le offerte son riferite}
  \end{itemize}
}%end item
\item{{\bf  Returns} -- 
lista di offerte selezionate 
}%end item
\end{itemize}
}%end item
\item{ 
\index{getTradableOffers(Offer)}
{\bf  getTradableOffers}\\
\begin{lstlisting}[frame=none]
public java.util.List getTradableOffers(data.Offer offerToTrade)\end{lstlisting} %end signature
\begin{itemize}
\item{
{\bf  Description}

Ritorna una lisa di articoli scambiabili con l'articolo dato, ovvero tutte le offerte aperte aperte di altri utenti appartenenti alla stessa caategoria foglia dell'offerta data
}
\item{
{\bf  Parameters}
  \begin{itemize}
   \item{
\texttt{offerToTrade} -- offerta da scambiaare}
  \end{itemize}
}%end item
\item{{\bf  Returns} -- 
lista di offerte scambiabili 
}%end item
\end{itemize}
}%end item
\item{ 
\index{getTradableOffers(User)}
{\bf  getTradableOffers}\\
\begin{lstlisting}[frame=none]
public java.util.List getTradableOffers(data.User owner)\end{lstlisting} %end signature
\begin{itemize}
\item{
{\bf  Description}

Ritorna una lista di aticoli dell'utente disponibili allo scambio, ovvero la lista delle sue offerte aperte
}
\item{
{\bf  Parameters}
  \begin{itemize}
   \item{
\texttt{owner} -- proprietario}
  \end{itemize}
}%end item
\item{{\bf  Returns} -- 
lista di offerte scambiabili 
}%end item
\end{itemize}
}%end item
\item{ 
\index{getUserMessages(Authentication)}
{\bf  getUserMessages}\\
\begin{lstlisting}[frame=none]
public java.util.List getUserMessages(Authentication auth)\end{lstlisting} %end signature
\begin{itemize}
\item{
{\bf  Description}

Ritorna la lista di messaggi per l'utente, relativi a offerte in scambio
}
\item{
{\bf  Parameters}
  \begin{itemize}
   \item{
\texttt{auth} -- token di autenticazione dell'utente a cui le offerte son riferite}
  \end{itemize}
}%end item
\item{{\bf  Returns} -- 
lista di messaggi 
}%end item
\end{itemize}
}%end item
\item{ 
\index{hasConfiguredSettings()}
{\bf  hasConfiguredSettings}\\
\begin{lstlisting}[frame=none]
public boolean hasConfiguredSettings()\end{lstlisting} %end signature
\begin{itemize}
\item{
{\bf  Description}

Ritorna true se sono stati configurati i parametri dell'applicazione (piazza, scadenza, giorni, orari, luoghi)
}
\item{{\bf  Returns} -- 
boolean 
}%end item
\end{itemize}
}%end item
\item{ 
\index{isValidRootCategoryName(String)}
{\bf  isValidRootCategoryName}\\
\begin{lstlisting}[frame=none]
public boolean isValidRootCategoryName(java.lang.String name)\end{lstlisting} %end signature
\begin{itemize}
\item{
{\bf  Description}

Controlla l'esistenza di una categoria radice il nome passato, non possono esserci due categorie root con lo stesso nome
}
\item{
{\bf  Parameters}
  \begin{itemize}
   \item{
\texttt{name} -- nome della categoria radice}
  \end{itemize}
}%end item
\item{{\bf  Returns} -- 
true se il nome della categoria radice non è già esistente 
}%end item
\end{itemize}
}%end item
\item{ 
\index{readSettings()}
{\bf  readSettings}\\
\begin{lstlisting}[frame=none]
public data.Settings readSettings()\end{lstlisting} %end signature
\begin{itemize}
\item{
{\bf  Description}

Ritorna i parametri di configurazione attuali
}
\end{itemize}
}%end item
\item{ 
\index{registerUser(String, String)}
{\bf  registerUser}\\
\begin{lstlisting}[frame=none]
public void registerUser(java.lang.String username,java.lang.String password) throws java.sql.SQLException, it.matlice.ingsw.model.exceptions.DuplicateUserException, it.matlice.ingsw.model.exceptions.InvalidUserTypeException, it.matlice.ingsw.model.auth.exceptions.InvalidPasswordException\end{lstlisting} %end signature
\begin{itemize}
\item{
{\bf  Description}

Aggiunge un nuovo utente fruitore
}
\item{
{\bf  Parameters}
  \begin{itemize}
   \item{
\texttt{username} -- username dell'utente da creare}
   \item{
\texttt{password} -- password associata all'utente da creare}
  \end{itemize}
}%end item
\item{{\bf  Throws}
  \begin{itemize}
   \item{\vskip -.6ex \texttt{java.sql.SQLException} -- errore di connessione col database}
   \item{\vskip -.6ex \texttt{it.matlice.ingsw.model.exceptions.DuplicateUserException} -- utente con username già esistente}
   \item{\vskip -.6ex \texttt{it.matlice.ingsw.model.exceptions.InvalidUserTypeException} -- }
   \item{\vskip -.6ex \texttt{it.matlice.ingsw.model.auth.exceptions.InvalidPasswordException} -- }
  \end{itemize}
}%end item
\end{itemize}
}%end item
\item{ 
\index{replyToMessage(Message, String, Settings.Day, Interval.Time)}
{\bf  replyToMessage}\\
\begin{lstlisting}[frame=none]
public java.util.Calendar replyToMessage(data.Message replyto,java.lang.String place,data.Settings.Day day,data.Interval.Time time)\end{lstlisting} %end signature
\begin{itemize}
\item{
{\bf  Description}

Permette di rispondere ad una proposta di scambio in un luogo, giorno e ora con una controproposta
}
\item{
{\bf  Parameters}
  \begin{itemize}
   \item{
\texttt{replyto} -- messaggio a cui rispondere}
   \item{
\texttt{place} -- luogo della controproposta}
   \item{
\texttt{day} -- giorno della controproposta}
   \item{
\texttt{time} -- orario della controproposta}
  \end{itemize}
}%end item
\item{{\bf  Returns} -- 
momento dello scambio 
}%end item
\end{itemize}
}%end item
\item{ 
\index{retractOffer(Offer)}
{\bf  retractOffer}\\
\begin{lstlisting}[frame=none]
public void retractOffer(data.Offer offerToRetract)\end{lstlisting} %end signature
\begin{itemize}
\item{
{\bf  Description}

Ritira un'offerta
}
\item{
{\bf  Parameters}
  \begin{itemize}
   \item{
\texttt{offerToRetract} -- offerta da ritirare}
  \end{itemize}
}%end item
\end{itemize}
}%end item
\item{ 
\index{timeIteration()}
{\bf  timeIteration}\\
\begin{lstlisting}[frame=none]
public void timeIteration()\end{lstlisting} %end signature
\begin{itemize}
\item{
{\bf  Description}

Verifica tutte le condizioni legate al tempo, come ad esempio le scadenze delle offerte
}
\end{itemize}
}%end item
\end{itemize}
}
}
\section{\label{it.matlice.ingsw.model.Settings}\index{Settings}Class Settings}{
\vskip .1in 
\subsection{Declaration}{
\begin{lstlisting}[frame=none]
public class Settings
 extends java.lang.Object\end{lstlisting}
\subsection{Field summary}{
\begin{verse}
{\bf LOGIN\_EXPIRATION\_TIME} Tempo in secondi di validità della sessione\\
\end{verse}
}
\subsection{Constructor summary}{
\begin{verse}
{\bf Settings()} \\
\end{verse}
}
\subsection{Fields}{
\begin{itemize}
\item{
\index{LOGIN\_EXPIRATION\_TIME}
\label{it.matlice.ingsw.model.Settings.LOGIN_EXPIRATION_TIME}\texttt{public static final long\ {\bf  LOGIN\_EXPIRATION\_TIME}}
\begin{itemize}
\item{\vskip -.9ex 
Tempo in secondi di validità della sessione}
\end{itemize}
}
\end{itemize}
}
\subsection{Constructors}{
\vskip -2em
\begin{itemize}
\item{ 
\index{Settings()}
{\bf  Settings}\\
\begin{lstlisting}[frame=none]
public Settings()\end{lstlisting} %end signature
}%end item
\end{itemize}
}
}
}
\chapter{Package it.matlice.ingsw.model.auth}{
\label{it.matlice.ingsw.model.auth}\hskip -.05in
\hbox to \hsize{\textit{ Package Contents\hfil Page}}
\vskip .13in
\hbox{{\bf  Interfaces}}
\entityintro{AuthData}{it.matlice.ingsw.model.auth.AuthData}{Questa classe rappresenta i dati necessari per l'autenticazione con un AuthMethod.}
\entityintro{Authenticable}{it.matlice.ingsw.model.auth.Authenticable}{L'interfaccia rappresenta una classe in grado di fornire una lista di AuthMethod in grado di fornire l'autenticazione.}
\entityintro{AuthMethod}{it.matlice.ingsw.model.auth.AuthMethod}{Raoppresenta un metodo di autenticazione e compie le azioni affini alle procedure di autenticazione, quali il cambio password, gestione delle scadenze ecc...}
\vskip .1in
\vskip .1in
\section{\label{it.matlice.ingsw.model.auth.AuthData}\index{AuthData@\textit{ AuthData}}Interface AuthData}{
\vskip .1in 
Questa classe rappresenta i dati necessari per l'autenticazione con un AuthMethod.\vskip .1in 
\subsection{Declaration}{
\begin{lstlisting}[frame=none]
public interface AuthData
\end{lstlisting}
\subsection{All known subinterfaces}{PasswordAuthData\small{\refdefined{it.matlice.ingsw.model.auth.password.PasswordAuthData}}}
\subsection{All classes known to implement interface}{PasswordAuthData\small{\refdefined{it.matlice.ingsw.model.auth.password.PasswordAuthData}}}
}
\section{\label{it.matlice.ingsw.model.auth.Authenticable}\index{Authenticable@\textit{ Authenticable}}Interface Authenticable}{
\vskip .1in 
L'interfaccia rappresenta una classe in grado di fornire una lista di AuthMethod in grado di fornire l'autenticazione.\vskip .1in 
\subsection{Declaration}{
\begin{lstlisting}[frame=none]
public interface Authenticable
\end{lstlisting}
\subsection{All known subinterfaces}{PasswordAuthenticable\small{\refdefined{it.matlice.ingsw.model.auth.password.PasswordAuthenticable}}, User\small{\refdefined{it.matlice.ingsw.model.data.User}}, CustomerUser\small{\refdefined{it.matlice.ingsw.model.data.CustomerUser}}, ConfiguratorUser\small{\refdefined{it.matlice.ingsw.model.data.ConfiguratorUser}}, CustomerUserImpl\small{\refdefined{it.matlice.ingsw.model.data.impl.jdbc.types.CustomerUserImpl}}, ConfiguratorUserImpl\small{\refdefined{it.matlice.ingsw.model.data.impl.jdbc.types.ConfiguratorUserImpl}}}
\subsection{All classes known to implement interface}{User\small{\refdefined{it.matlice.ingsw.model.data.User}}}
\subsection{Method summary}{
\begin{verse}
{\bf getAuthMethods()} Fornisce una lista di metodi di autenticazione in grado di autenticare la classe.\\
\end{verse}
}
\subsection{Methods}{
\vskip -2em
\begin{itemize}
\item{ 
\index{getAuthMethods()}
{\bf  getAuthMethods}\\
\begin{lstlisting}[frame=none]
java.util.List getAuthMethods()\end{lstlisting} %end signature
\begin{itemize}
\item{
{\bf  Description}

Fornisce una lista di metodi di autenticazione in grado di autenticare la classe.
}
\item{{\bf  Returns} -- 
una lista di istanze di metodi di autenticazione. 
}%end item
\end{itemize}
}%end item
\end{itemize}
}
}
\section{\label{it.matlice.ingsw.model.auth.AuthMethod}\index{AuthMethod@\textit{ AuthMethod}}Interface AuthMethod}{
\vskip .1in 
Raoppresenta un metodo di autenticazione e compie le azioni affini alle procedure di autenticazione, quali il cambio password, gestione delle scadenze ecc...\vskip .1in 
\subsection{Declaration}{
\begin{lstlisting}[frame=none]
public interface AuthMethod
\end{lstlisting}
\subsection{All known subinterfaces}{PasswordAuthMethod\small{\refdefined{it.matlice.ingsw.model.auth.password.PasswordAuthMethod}}}
\subsection{All classes known to implement interface}{PasswordAuthMethod\small{\refdefined{it.matlice.ingsw.model.auth.password.PasswordAuthMethod}}}
\subsection{Method summary}{
\begin{verse}
{\bf getUser()} \\
{\bf performAuthentication(AuthData)} la funzione effettua il procedimento di autenticazione in funzione della sua istanza\\
\end{verse}
}
\subsection{Methods}{
\vskip -2em
\begin{itemize}
\item{ 
\index{getUser()}
{\bf  getUser}\\
\begin{lstlisting}[frame=none]
Authenticable getUser()\end{lstlisting} %end signature
}%end item
\item{ 
\index{performAuthentication(AuthData)}
{\bf  performAuthentication}\\
\begin{lstlisting}[frame=none]
boolean performAuthentication(AuthData data)\end{lstlisting} %end signature
\begin{itemize}
\item{
{\bf  Description}

la funzione effettua il procedimento di autenticazione in funzione della sua istanza
}
\item{
{\bf  Parameters}
  \begin{itemize}
   \item{
\texttt{data} -- dati necessari all'autenticazione}
  \end{itemize}
}%end item
\item{{\bf  Returns} -- 
true se l'autenticazione ha successo. 
}%end item
\end{itemize}
}%end item
\end{itemize}
}
}
}
\chapter{Package it.matlice.ingsw.model.auth.exceptions}{
\label{it.matlice.ingsw.model.auth.exceptions}\hskip -.05in
\hbox to \hsize{\textit{ Package Contents\hfil Page}}
\vskip .1in
\vskip .1in
\section{\label{it.matlice.ingsw.model.auth.exceptions.InvalidPasswordException}\index{InvalidPasswordException}Exception InvalidPasswordException}{
\vskip .1in 
\subsection{Declaration}{
\begin{lstlisting}[frame=none]
public class InvalidPasswordException
 extends java.lang.Exception\end{lstlisting}
\subsection{Constructor summary}{
\begin{verse}
{\bf InvalidPasswordException()} \\
\end{verse}
}
\subsection{Constructors}{
\vskip -2em
\begin{itemize}
\item{ 
\index{InvalidPasswordException()}
{\bf  InvalidPasswordException}\\
\begin{lstlisting}[frame=none]
public InvalidPasswordException()\end{lstlisting} %end signature
}%end item
\end{itemize}
}
\subsection{Members inherited from class Throwable }{
\texttt{java.lang.Throwable} {\small 
\refdefined{java.lang.Throwable}}
{\small 

\vskip -2em
\begin{itemize}
\item{\vskip -1.5ex 
\texttt{public final synchronized void {\bf  addSuppressed}(\texttt{Throwable} {\bf  arg0})
}%end signature
}%end item
\item{\vskip -1.5ex 
\texttt{public synchronized Throwable {\bf  fillInStackTrace}()
}%end signature
}%end item
\item{\vskip -1.5ex 
\texttt{public synchronized Throwable {\bf  getCause}()
}%end signature
}%end item
\item{\vskip -1.5ex 
\texttt{public String {\bf  getLocalizedMessage}()
}%end signature
}%end item
\item{\vskip -1.5ex 
\texttt{public String {\bf  getMessage}()
}%end signature
}%end item
\item{\vskip -1.5ex 
\texttt{public StackTraceElement {\bf  getStackTrace}()
}%end signature
}%end item
\item{\vskip -1.5ex 
\texttt{public final synchronized Throwable {\bf  getSuppressed}()
}%end signature
}%end item
\item{\vskip -1.5ex 
\texttt{public synchronized Throwable {\bf  initCause}(\texttt{Throwable} {\bf  arg0})
}%end signature
}%end item
\item{\vskip -1.5ex 
\texttt{public void {\bf  printStackTrace}()
}%end signature
}%end item
\item{\vskip -1.5ex 
\texttt{public void {\bf  printStackTrace}(\texttt{java.io.PrintStream} {\bf  arg0})
}%end signature
}%end item
\item{\vskip -1.5ex 
\texttt{public void {\bf  printStackTrace}(\texttt{java.io.PrintWriter} {\bf  arg0})
}%end signature
}%end item
\item{\vskip -1.5ex 
\texttt{public void {\bf  setStackTrace}(\texttt{StackTraceElement\lbrack \rbrack } {\bf  arg0})
}%end signature
}%end item
\item{\vskip -1.5ex 
\texttt{public String {\bf  toString}()
}%end signature
}%end item
\end{itemize}
}
}
}
\chapter{Package it.matlice.ingsw.model.auth.password}{
\label{it.matlice.ingsw.model.auth.password}\hskip -.05in
\hbox to \hsize{\textit{ Package Contents\hfil Page}}
\vskip .13in
\hbox{{\bf  Interfaces}}
\entityintro{PasswordAuthenticable}{it.matlice.ingsw.model.auth.password.PasswordAuthenticable}{L'interfaccia rappresenta una classe che può essere autenticata mediante il metodo PasswordAuthMethod.}
\vskip .13in
\hbox{{\bf  Classes}}
\entityintro{PasswordAuthData}{it.matlice.ingsw.model.auth.password.PasswordAuthData}{Questa classe rappresenta i dati necessari per l'autenticazione con password.}
\entityintro{PasswordAuthMethod}{it.matlice.ingsw.model.auth.password.PasswordAuthMethod}{Rappresenta il metodo di autenticazione mediante password.}
\vskip .1in
\vskip .1in
\section{\label{it.matlice.ingsw.model.auth.password.PasswordAuthenticable}\index{PasswordAuthenticable@\textit{ PasswordAuthenticable}}Interface PasswordAuthenticable}{
\vskip .1in 
L'interfaccia rappresenta una classe che può essere autenticata mediante il metodo PasswordAuthMethod.

La gestione della password avvene mediante la verifica dell'hash (salted) della password, rendendo così necessario il salvataggio si hash e salt a livello di struttura dati.\vskip .1in 
\subsection{Declaration}{
\begin{lstlisting}[frame=none]
public interface PasswordAuthenticable
 extends it.matlice.ingsw.model.auth.Authenticable\end{lstlisting}
\subsection{All known subinterfaces}{CustomerUserImpl\small{\refdefined{it.matlice.ingsw.model.data.impl.jdbc.types.CustomerUserImpl}}, ConfiguratorUserImpl\small{\refdefined{it.matlice.ingsw.model.data.impl.jdbc.types.ConfiguratorUserImpl}}}
\subsection{All classes known to implement interface}{CustomerUserImpl\small{\refdefined{it.matlice.ingsw.model.data.impl.jdbc.types.CustomerUserImpl}}, ConfiguratorUserImpl\small{\refdefined{it.matlice.ingsw.model.data.impl.jdbc.types.ConfiguratorUserImpl}}}
\subsection{Method summary}{
\begin{verse}
{\bf getPasswordHash()} Il metodo ritorna l'hash della password salvato precedentemente.\\
{\bf getPasswordSalt()} \\
{\bf setPassword(byte\lbrack \rbrack )} Aggiorna la password associata all'utente \_\_rigenerando\_\_\ il salt\\
{\bf setSalt(byte\lbrack \rbrack )} \\
\end{verse}
}
\subsection{Methods}{
\vskip -2em
\begin{itemize}
\item{ 
\index{getPasswordHash()}
{\bf  getPasswordHash}\\
\begin{lstlisting}[frame=none]
byte[] getPasswordHash()\end{lstlisting} %end signature
\begin{itemize}
\item{
{\bf  Description}

Il metodo ritorna l'hash della password salvato precedentemente. l'hash è computato secondo la seguente espressione: \textbackslash \lbrack\ passwordHash = base64enc(hmac\_\{sha256\}(key=password, data=salt)) \textbackslash \rbrack 
}
\item{{\bf  Returns} -- 
l'hash della password salvato in precedenza 
}%end item
\end{itemize}
}%end item
\item{ 
\index{getPasswordSalt()}
{\bf  getPasswordSalt}\\
\begin{lstlisting}[frame=none]
byte[] getPasswordSalt()\end{lstlisting} %end signature
\begin{itemize}
\item{{\bf  Returns} -- 
ritorna il salt, generato casualmente, utilizzato per "salare la password" 
}%end item
\end{itemize}
}%end item
\item{ 
\index{setPassword(byte\lbrack \rbrack )}
{\bf  setPassword}\\
\begin{lstlisting}[frame=none]
void setPassword(byte[] password) throws it.matlice.ingsw.model.auth.exceptions.InvalidPasswordException\end{lstlisting} %end signature
\begin{itemize}
\item{
{\bf  Description}

Aggiorna la password associata all'utente \_\_rigenerando\_\_\ il salt
}
\item{
{\bf  Parameters}
  \begin{itemize}
   \item{
\texttt{password} -- nuova password}
  \end{itemize}
}%end item
\end{itemize}
}%end item
\item{ 
\index{setSalt(byte\lbrack \rbrack )}
{\bf  setSalt}\\
\begin{lstlisting}[frame=none]
void setSalt(byte[] salt) throws it.matlice.ingsw.model.auth.exceptions.InvalidPasswordException\end{lstlisting} %end signature
}%end item
\end{itemize}
}
\subsection{Members inherited from class Authenticable }{
\texttt{it.matlice.ingsw.model.auth.Authenticable} {\small 
\refdefined{it.matlice.ingsw.model.auth.Authenticable}}
{\small 

\vskip -2em
\begin{itemize}
\item{\vskip -1.5ex 
\texttt{public List {\bf  getAuthMethods}()
}%end signature
}%end item
\end{itemize}
}
}
\section{\label{it.matlice.ingsw.model.auth.password.PasswordAuthData}\index{PasswordAuthData}Class PasswordAuthData}{
\vskip .1in 
Questa classe rappresenta i dati necessari per l'autenticazione con password.\vskip .1in 
\subsection{Declaration}{
\begin{lstlisting}[frame=none]
public final class PasswordAuthData
 extends java.lang.Object implements it.matlice.ingsw.model.auth.AuthData\end{lstlisting}
\subsection{Constructor summary}{
\begin{verse}
{\bf PasswordAuthData(String)} \\
\end{verse}
}
\subsection{Method summary}{
\begin{verse}
{\bf equals(Object)} \\
{\bf hashCode()} \\
{\bf password()} \\
{\bf toString()} \\
\end{verse}
}
\subsection{Constructors}{
\vskip -2em
\begin{itemize}
\item{ 
\index{PasswordAuthData(String)}
{\bf  PasswordAuthData}\\
\begin{lstlisting}[frame=none]
public PasswordAuthData(java.lang.String password)\end{lstlisting} %end signature
}%end item
\end{itemize}
}
\subsection{Methods}{
\vskip -2em
\begin{itemize}
\item{ 
\index{equals(Object)}
{\bf  equals}\\
\begin{lstlisting}[frame=none]
public boolean equals(java.lang.Object arg0)\end{lstlisting} %end signature
}%end item
\item{ 
\index{hashCode()}
{\bf  hashCode}\\
\begin{lstlisting}[frame=none]
public native int hashCode()\end{lstlisting} %end signature
}%end item
\item{ 
\index{password()}
{\bf  password}\\
\begin{lstlisting}[frame=none]
public java.lang.String password()\end{lstlisting} %end signature
}%end item
\item{ 
\index{toString()}
{\bf  toString}\\
\begin{lstlisting}[frame=none]
public java.lang.String toString()\end{lstlisting} %end signature
}%end item
\end{itemize}
}
}
\section{\label{it.matlice.ingsw.model.auth.password.PasswordAuthMethod}\index{PasswordAuthMethod}Class PasswordAuthMethod}{
\vskip .1in 
Rappresenta il metodo di autenticazione mediante password. La classe è un wrapper di una classe PasswordAuthenticable.\vskip .1in 
\subsection{Declaration}{
\begin{lstlisting}[frame=none]
public class PasswordAuthMethod
 extends java.lang.Object implements it.matlice.ingsw.model.auth.AuthMethod\end{lstlisting}
\subsection{Field summary}{
\begin{verse}
{\bf MAC\_ALGO} \\
{\bf PASSWORD\_SEC\_REGEX} \\
{\bf SALT\_LENGTH} \\
\end{verse}
}
\subsection{Constructor summary}{
\begin{verse}
{\bf PasswordAuthMethod(PasswordAuthenticable)} \\
\end{verse}
}
\subsection{Method summary}{
\begin{verse}
{\bf getAuthData(String)} Metodo ausiliario che ritorna un istanza valida di authdata per la classe.\\
{\bf getUser()} \\
{\bf isPasswordValid(String)} \\
{\bf performAuthentication(AuthData)} la funzione effettua il procedimento di autenticazione in funzione della sua istanza.\\
{\bf setPassword(String)} \\
{\bf setPassword(String, boolean)} questo metodo permette di impostare la password dell'utente\\
\end{verse}
}
\subsection{Fields}{
\begin{itemize}
\item{
\index{MAC\_ALGO}
\label{it.matlice.ingsw.model.auth.password.PasswordAuthMethod.MAC_ALGO}\texttt{public static final java.lang.String\ {\bf  MAC\_ALGO}}
}
\item{
\index{PASSWORD\_SEC\_REGEX}
\label{it.matlice.ingsw.model.auth.password.PasswordAuthMethod.PASSWORD_SEC_REGEX}\texttt{public static final java.util.regex.Pattern\ {\bf  PASSWORD\_SEC\_REGEX}}
}
\item{
\index{SALT\_LENGTH}
\label{it.matlice.ingsw.model.auth.password.PasswordAuthMethod.SALT_LENGTH}\texttt{public static final int\ {\bf  SALT\_LENGTH}}
}
\end{itemize}
}
\subsection{Constructors}{
\vskip -2em
\begin{itemize}
\item{ 
\index{PasswordAuthMethod(PasswordAuthenticable)}
{\bf  PasswordAuthMethod}\\
\begin{lstlisting}[frame=none]
public PasswordAuthMethod(PasswordAuthenticable user)\end{lstlisting} %end signature
}%end item
\end{itemize}
}
\subsection{Methods}{
\vskip -2em
\begin{itemize}
\item{ 
\index{getAuthData(String)}
{\bf  getAuthData}\\
\begin{lstlisting}[frame=none]
public static it.matlice.ingsw.model.auth.AuthData getAuthData(java.lang.String password)\end{lstlisting} %end signature
\begin{itemize}
\item{
{\bf  Description}

Metodo ausiliario che ritorna un istanza valida di authdata per la classe.
}
\item{
{\bf  Parameters}
  \begin{itemize}
   \item{
\texttt{password} -- password di login}
  \end{itemize}
}%end item
\item{{\bf  Returns} -- 
un'istanza di AuthData valida. 
}%end item
\end{itemize}
}%end item
\item{ 
\index{getUser()}
{\bf  getUser}\\
\begin{lstlisting}[frame=none]
it.matlice.ingsw.model.auth.Authenticable getUser()\end{lstlisting} %end signature
}%end item
\item{ 
\index{isPasswordValid(String)}
{\bf  isPasswordValid}\\
\begin{lstlisting}[frame=none]
public static boolean isPasswordValid(java.lang.String password)\end{lstlisting} %end signature
}%end item
\item{ 
\index{performAuthentication(AuthData)}
{\bf  performAuthentication}\\
\begin{lstlisting}[frame=none]
public boolean performAuthentication(it.matlice.ingsw.model.auth.AuthData data)\end{lstlisting} %end signature
\begin{itemize}
\item{
{\bf  Description}

la funzione effettua il procedimento di autenticazione in funzione della sua istanza. È necessario che venga asserito che data sia di un tipo valido all'autenticazione con password
}
\item{
{\bf  Parameters}
  \begin{itemize}
   \item{
\texttt{data} -- dati necessari all'autenticazione}
  \end{itemize}
}%end item
\item{{\bf  Returns} -- 
true se l'autenticazione ha successo. 
}%end item
\end{itemize}
}%end item
\item{ 
\index{setPassword(String)}
{\bf  setPassword}\\
\begin{lstlisting}[frame=none]
public void setPassword(java.lang.String password) throws it.matlice.ingsw.model.auth.exceptions.InvalidPasswordException\end{lstlisting} %end signature
}%end item
\item{ 
\index{setPassword(String, boolean)}
{\bf  setPassword}\\
\begin{lstlisting}[frame=none]
public void setPassword(java.lang.String password,boolean skipComplexityCheck) throws it.matlice.ingsw.model.auth.exceptions.InvalidPasswordException\end{lstlisting} %end signature
\begin{itemize}
\item{
{\bf  Description}

questo metodo permette di impostare la password dell'utente
}
\item{
{\bf  Parameters}
  \begin{itemize}
   \item{
\texttt{password} -- nuova password che rispetti i parametri impostati nella configurazione}
  \end{itemize}
}%end item
\item{{\bf  Throws}
  \begin{itemize}
   \item{\vskip -.6ex \texttt{it.matlice.ingsw.model.auth.exceptions.InvalidPasswordException} -- nel caso in cui la password non rispetti la complessità richiesta}
  \end{itemize}
}%end item
\end{itemize}
}%end item
\end{itemize}
}
}
}
\chapter{Package it.matlice.ingsw.model.data}{
\label{it.matlice.ingsw.model.data}\hskip -.05in
\hbox to \hsize{\textit{ Package Contents\hfil Page}}
\vskip .13in
\hbox{{\bf  Classes}}
\entityintro{Category}{it.matlice.ingsw.model.data.Category}{La classe rappresenta una categoria.}
\entityintro{ConfiguratorUser}{it.matlice.ingsw.model.data.ConfiguratorUser}{la classe rappresenta un utente con privilegi di configurazione}
\entityintro{CustomerUser}{it.matlice.ingsw.model.data.CustomerUser}{Rappresenta un utente fruitore}
\entityintro{Hierarchy}{it.matlice.ingsw.model.data.Hierarchy}{la classe rappresenta una gerarchia.}
\entityintro{Interval}{it.matlice.ingsw.model.data.Interval}{}
\entityintro{Interval.Time}{it.matlice.ingsw.model.data.Interval.Time}{}
\entityintro{LeafCategory}{it.matlice.ingsw.model.data.LeafCategory}{rappresenta una categoria di tipo foglia.}
\entityintro{Message}{it.matlice.ingsw.model.data.Message}{}
\entityintro{NodeCategory}{it.matlice.ingsw.model.data.NodeCategory}{rappresenta una categoria di tipo nodo.}
\entityintro{Offer}{it.matlice.ingsw.model.data.Offer}{}
\entityintro{Offer.OfferStatus}{it.matlice.ingsw.model.data.Offer.OfferStatus}{}
\entityintro{Settings}{it.matlice.ingsw.model.data.Settings}{}
\entityintro{Settings.Day}{it.matlice.ingsw.model.data.Settings.Day}{Classe che rappresenta i giorni della settimana}
\entityintro{TypeDefinition}{it.matlice.ingsw.model.data.TypeDefinition}{rappresenta un tipo di dato, salvandone il tipo e se esso è obbligatorio}
\entityintro{TypeDefinition.TypeAssociation}{it.matlice.ingsw.model.data.TypeDefinition.TypeAssociation}{Tipi di campo disponibili, nel caso specifico si possono aggiungere solo stringhe}
\entityintro{User}{it.matlice.ingsw.model.data.User}{rappresenta un utente del sistema}
\entityintro{User.UserTypes}{it.matlice.ingsw.model.data.User.UserTypes}{}
\vskip .1in
\vskip .1in
\section{\label{it.matlice.ingsw.model.data.Category}\index{Category}Class Category}{
\vskip .1in 
La classe rappresenta una categoria. Essendo una hashmap, internamente ad essa avremo le associazioni dei campi nativi.

Notare che la struttura dati Category è una mappa distribuita in un albero: - L'aggiunta di campi non ancora definiti avviene sul nodo dove put() viene chiamato - La modifica di campi avviene in uno dei nodi padre (se dispobibile) o nel nodo dove put viene chiamato - size() ritorna la somma dei membri del nodo e quella dei parenti prima di lui - isEmpty() ritorna true se tutti i predecessori e il nodo su cui è chiamato sono vuoti - get() ritorna il membro dal nodo, e se questo non è presente prova a recuperarlo da un antenato - funzionamento analogo per remove() e containsKey() e clear()\vskip .1in 
\subsection{Declaration}{
\begin{lstlisting}[frame=none]
public abstract class Category
 extends java.util.HashMap\end{lstlisting}
\subsection{All known subclasses}{NodeCategory\small{\refdefined{it.matlice.ingsw.model.data.NodeCategory}}, LeafCategory\small{\refdefined{it.matlice.ingsw.model.data.LeafCategory}}, NodeCategoryImpl\small{\refdefined{it.matlice.ingsw.model.data.impl.jdbc.types.NodeCategoryImpl}}, LeafCategoryImpl\small{\refdefined{it.matlice.ingsw.model.data.impl.jdbc.types.LeafCategoryImpl}}}
\subsection{Constructor summary}{
\begin{verse}
{\bf Category()} \\
\end{verse}
}
\subsection{Method summary}{
\begin{verse}
{\bf clear()} \\
{\bf containsKey(Object)} \\
{\bf containsValue(Object)} \\
{\bf fullEntrySet()} Ritorna l'unione degli entry set del nodo e di tutti i suoi antenati\\
{\bf fullToString()} Ritorna una stringa che rappresenta la categoria a partire dalla categoria radice (es.\\
{\bf get(Object)} \\
{\bf getChildLeafs()} \\
{\bf getDescription()} \\
{\bf getFather()} \\
{\bf getName()} \\
{\bf isCategoryValid()} \\
{\bf isEmpty()} \\
{\bf isRequired(String)} \\
{\bf isRoot()} identifica se una categoria non ha padre\\
{\bf isValidChildCategoryName(String)} \\
{\bf put(String, TypeDefinition)} \\
{\bf remove(Object)} \\
{\bf setFather(NodeCategory)} \\
{\bf size()} \\
{\bf toString()} Formatta la categoria in una stringa secondo una struttura ad albero\\
\end{verse}
}
\subsection{Constructors}{
\vskip -2em
\begin{itemize}
\item{ 
\index{Category()}
{\bf  Category}\\
\begin{lstlisting}[frame=none]
public Category()\end{lstlisting} %end signature
}%end item
\end{itemize}
}
\subsection{Methods}{
\vskip -2em
\begin{itemize}
\item{ 
\index{clear()}
{\bf  clear}\\
\begin{lstlisting}[frame=none]
void clear()\end{lstlisting} %end signature
}%end item
\item{ 
\index{containsKey(Object)}
{\bf  containsKey}\\
\begin{lstlisting}[frame=none]
boolean containsKey(java.lang.Object arg0)\end{lstlisting} %end signature
}%end item
\item{ 
\index{containsValue(Object)}
{\bf  containsValue}\\
\begin{lstlisting}[frame=none]
boolean containsValue(java.lang.Object arg0)\end{lstlisting} %end signature
}%end item
\item{ 
\index{fullEntrySet()}
{\bf  fullEntrySet}\\
\begin{lstlisting}[frame=none]
public java.util.Set fullEntrySet()\end{lstlisting} %end signature
\begin{itemize}
\item{
{\bf  Description}

Ritorna l'unione degli entry set del nodo e di tutti i suoi antenati
}
\item{{\bf  Returns} -- 
l'entriset dell'albero fino al nodo corrente 
}%end item
\end{itemize}
}%end item
\item{ 
\index{fullToString()}
{\bf  fullToString}\\
\begin{lstlisting}[frame=none]
public java.lang.String fullToString()\end{lstlisting} %end signature
\begin{itemize}
\item{
{\bf  Description}

Ritorna una stringa che rappresenta la categoria a partire dalla categoria radice (es. "Libro \textgreater  Romanzo \textgreater  Romanzo Giallo")
}
\item{{\bf  Returns} -- 
stringa 
}%end item
\end{itemize}
}%end item
\item{ 
\index{get(Object)}
{\bf  get}\\
\begin{lstlisting}[frame=none]
java.lang.Object get(java.lang.Object arg0)\end{lstlisting} %end signature
}%end item
\item{ 
\index{getChildLeafs()}
{\bf  getChildLeafs}\\
\begin{lstlisting}[frame=none]
public abstract java.util.List getChildLeafs()\end{lstlisting} %end signature
\begin{itemize}
\item{{\bf  Returns} -- 
tutte le categorie foglia discendenti (o se stessa se è foglia) 
}%end item
\end{itemize}
}%end item
\item{ 
\index{getDescription()}
{\bf  getDescription}\\
\begin{lstlisting}[frame=none]
public abstract java.lang.String getDescription()\end{lstlisting} %end signature
\begin{itemize}
\item{{\bf  Returns} -- 
ritorna la descrizione della categoria 
}%end item
\end{itemize}
}%end item
\item{ 
\index{getFather()}
{\bf  getFather}\\
\begin{lstlisting}[frame=none]
public NodeCategory getFather()\end{lstlisting} %end signature
}%end item
\item{ 
\index{getName()}
{\bf  getName}\\
\begin{lstlisting}[frame=none]
public abstract java.lang.String getName()\end{lstlisting} %end signature
\begin{itemize}
\item{{\bf  Returns} -- 
ritorna il nome della categoria 
}%end item
\end{itemize}
}%end item
\item{ 
\index{isCategoryValid()}
{\bf  isCategoryValid}\\
\begin{lstlisting}[frame=none]
public boolean isCategoryValid()\end{lstlisting} %end signature
}%end item
\item{ 
\index{isEmpty()}
{\bf  isEmpty}\\
\begin{lstlisting}[frame=none]
boolean isEmpty()\end{lstlisting} %end signature
}%end item
\item{ 
\index{isRequired(String)}
{\bf  isRequired}\\
\begin{lstlisting}[frame=none]
public boolean isRequired(java.lang.String name)\end{lstlisting} %end signature
}%end item
\item{ 
\index{isRoot()}
{\bf  isRoot}\\
\begin{lstlisting}[frame=none]
public boolean isRoot()\end{lstlisting} %end signature
\begin{itemize}
\item{
{\bf  Description}

identifica se una categoria non ha padre
}
\item{{\bf  Returns} -- 
true se è una categoria root 
}%end item
\end{itemize}
}%end item
\item{ 
\index{isValidChildCategoryName(String)}
{\bf  isValidChildCategoryName}\\
\begin{lstlisting}[frame=none]
public abstract boolean isValidChildCategoryName(java.lang.String name)\end{lstlisting} %end signature
}%end item
\item{ 
\index{put(String, TypeDefinition)}
{\bf  put}\\
\begin{lstlisting}[frame=none]
public TypeDefinition put(java.lang.String key,TypeDefinition value)\end{lstlisting} %end signature
}%end item
\item{ 
\index{remove(Object)}
{\bf  remove}\\
\begin{lstlisting}[frame=none]
java.lang.Object remove(java.lang.Object arg0)\end{lstlisting} %end signature
}%end item
\item{ 
\index{setFather(NodeCategory)}
{\bf  setFather}\\
\begin{lstlisting}[frame=none]
public void setFather(NodeCategory father)\end{lstlisting} %end signature
}%end item
\item{ 
\index{size()}
{\bf  size}\\
\begin{lstlisting}[frame=none]
int size()\end{lstlisting} %end signature
}%end item
\item{ 
\index{toString()}
{\bf  toString}\\
\begin{lstlisting}[frame=none]
public java.lang.String toString()\end{lstlisting} %end signature
\begin{itemize}
\item{
{\bf  Description}

Formatta la categoria in una stringa secondo una struttura ad albero
}
\item{{\bf  Returns} -- 
stringa rappresentante la categoria 
}%end item
\end{itemize}
}%end item
\end{itemize}
}
\subsection{Members inherited from class HashMap }{
\texttt{java.util.HashMap} {\small 
\refdefined{java.util.HashMap}}
{\small 

\vskip -2em
\begin{itemize}
\item{\vskip -1.5ex 
\texttt{public void {\bf  clear}()
}%end signature
}%end item
\item{\vskip -1.5ex 
\texttt{public Object {\bf  clone}()
}%end signature
}%end item
\item{\vskip -1.5ex 
\texttt{public Object {\bf  compute}(\texttt{java.lang.Object} {\bf  arg0},
\texttt{function.BiFunction} {\bf  arg1})
}%end signature
}%end item
\item{\vskip -1.5ex 
\texttt{public Object {\bf  computeIfAbsent}(\texttt{java.lang.Object} {\bf  arg0},
\texttt{function.Function} {\bf  arg1})
}%end signature
}%end item
\item{\vskip -1.5ex 
\texttt{public Object {\bf  computeIfPresent}(\texttt{java.lang.Object} {\bf  arg0},
\texttt{function.BiFunction} {\bf  arg1})
}%end signature
}%end item
\item{\vskip -1.5ex 
\texttt{public boolean {\bf  containsKey}(\texttt{java.lang.Object} {\bf  arg0})
}%end signature
}%end item
\item{\vskip -1.5ex 
\texttt{public boolean {\bf  containsValue}(\texttt{java.lang.Object} {\bf  arg0})
}%end signature
}%end item
\item{\vskip -1.5ex 
\texttt{public Set {\bf  entrySet}()
}%end signature
}%end item
\item{\vskip -1.5ex 
\texttt{public void {\bf  forEach}(\texttt{function.BiConsumer} {\bf  arg0})
}%end signature
}%end item
\item{\vskip -1.5ex 
\texttt{public Object {\bf  get}(\texttt{java.lang.Object} {\bf  arg0})
}%end signature
}%end item
\item{\vskip -1.5ex 
\texttt{public Object {\bf  getOrDefault}(\texttt{java.lang.Object} {\bf  arg0},
\texttt{java.lang.Object} {\bf  arg1})
}%end signature
}%end item
\item{\vskip -1.5ex 
\texttt{public boolean {\bf  isEmpty}()
}%end signature
}%end item
\item{\vskip -1.5ex 
\texttt{public Set {\bf  keySet}()
}%end signature
}%end item
\item{\vskip -1.5ex 
\texttt{public Object {\bf  merge}(\texttt{java.lang.Object} {\bf  arg0},
\texttt{java.lang.Object} {\bf  arg1},
\texttt{function.BiFunction} {\bf  arg2})
}%end signature
}%end item
\item{\vskip -1.5ex 
\texttt{public Object {\bf  put}(\texttt{java.lang.Object} {\bf  arg0},
\texttt{java.lang.Object} {\bf  arg1})
}%end signature
}%end item
\item{\vskip -1.5ex 
\texttt{public void {\bf  putAll}(\texttt{Map} {\bf  arg0})
}%end signature
}%end item
\item{\vskip -1.5ex 
\texttt{public Object {\bf  putIfAbsent}(\texttt{java.lang.Object} {\bf  arg0},
\texttt{java.lang.Object} {\bf  arg1})
}%end signature
}%end item
\item{\vskip -1.5ex 
\texttt{public Object {\bf  remove}(\texttt{java.lang.Object} {\bf  arg0})
}%end signature
}%end item
\item{\vskip -1.5ex 
\texttt{public boolean {\bf  remove}(\texttt{java.lang.Object} {\bf  arg0},
\texttt{java.lang.Object} {\bf  arg1})
}%end signature
}%end item
\item{\vskip -1.5ex 
\texttt{public Object {\bf  replace}(\texttt{java.lang.Object} {\bf  arg0},
\texttt{java.lang.Object} {\bf  arg1})
}%end signature
}%end item
\item{\vskip -1.5ex 
\texttt{public boolean {\bf  replace}(\texttt{java.lang.Object} {\bf  arg0},
\texttt{java.lang.Object} {\bf  arg1},
\texttt{java.lang.Object} {\bf  arg2})
}%end signature
}%end item
\item{\vskip -1.5ex 
\texttt{public void {\bf  replaceAll}(\texttt{function.BiFunction} {\bf  arg0})
}%end signature
}%end item
\item{\vskip -1.5ex 
\texttt{public int {\bf  size}()
}%end signature
}%end item
\item{\vskip -1.5ex 
\texttt{public Collection {\bf  values}()
}%end signature
}%end item
\end{itemize}
}
\subsection{Members inherited from class AbstractMap }{
\texttt{java.util.AbstractMap} {\small 
\refdefined{java.util.AbstractMap}}
{\small 

\vskip -2em
\begin{itemize}
\item{\vskip -1.5ex 
\texttt{public void {\bf  clear}()
}%end signature
}%end item
\item{\vskip -1.5ex 
\texttt{protected Object {\bf  clone}() throws java.lang.CloneNotSupportedException
}%end signature
}%end item
\item{\vskip -1.5ex 
\texttt{public boolean {\bf  containsKey}(\texttt{java.lang.Object} {\bf  arg0})
}%end signature
}%end item
\item{\vskip -1.5ex 
\texttt{public boolean {\bf  containsValue}(\texttt{java.lang.Object} {\bf  arg0})
}%end signature
}%end item
\item{\vskip -1.5ex 
\texttt{public abstract Set {\bf  entrySet}()
}%end signature
}%end item
\item{\vskip -1.5ex 
\texttt{public boolean {\bf  equals}(\texttt{java.lang.Object} {\bf  arg0})
}%end signature
}%end item
\item{\vskip -1.5ex 
\texttt{public Object {\bf  get}(\texttt{java.lang.Object} {\bf  arg0})
}%end signature
}%end item
\item{\vskip -1.5ex 
\texttt{public int {\bf  hashCode}()
}%end signature
}%end item
\item{\vskip -1.5ex 
\texttt{public boolean {\bf  isEmpty}()
}%end signature
}%end item
\item{\vskip -1.5ex 
\texttt{public Set {\bf  keySet}()
}%end signature
}%end item
\item{\vskip -1.5ex 
\texttt{public Object {\bf  put}(\texttt{java.lang.Object} {\bf  arg0},
\texttt{java.lang.Object} {\bf  arg1})
}%end signature
}%end item
\item{\vskip -1.5ex 
\texttt{public void {\bf  putAll}(\texttt{Map} {\bf  arg0})
}%end signature
}%end item
\item{\vskip -1.5ex 
\texttt{public Object {\bf  remove}(\texttt{java.lang.Object} {\bf  arg0})
}%end signature
}%end item
\item{\vskip -1.5ex 
\texttt{public int {\bf  size}()
}%end signature
}%end item
\item{\vskip -1.5ex 
\texttt{public String {\bf  toString}()
}%end signature
}%end item
\item{\vskip -1.5ex 
\texttt{public Collection {\bf  values}()
}%end signature
}%end item
\end{itemize}
}
}
\section{\label{it.matlice.ingsw.model.data.ConfiguratorUser}\index{ConfiguratorUser}Class ConfiguratorUser}{
\vskip .1in 
la classe rappresenta un utente con privilegi di configurazione\vskip .1in 
\subsection{Declaration}{
\begin{lstlisting}[frame=none]
public abstract class ConfiguratorUser
 extends it.matlice.ingsw.model.data.User\end{lstlisting}
\subsection{All known subclasses}{ConfiguratorUserImpl\small{\refdefined{it.matlice.ingsw.model.data.impl.jdbc.types.ConfiguratorUserImpl}}}
\subsection{Constructor summary}{
\begin{verse}
{\bf ConfiguratorUser()} \\
\end{verse}
}
\subsection{Constructors}{
\vskip -2em
\begin{itemize}
\item{ 
\index{ConfiguratorUser()}
{\bf  ConfiguratorUser}\\
\begin{lstlisting}[frame=none]
public ConfiguratorUser()\end{lstlisting} %end signature
}%end item
\end{itemize}
}
\subsection{Members inherited from class User }{
\texttt{it.matlice.ingsw.model.data.User} {\small 
\refdefined{it.matlice.ingsw.model.data.User}}
{\small 

\vskip -2em
\begin{itemize}
\item{\vskip -1.5ex 
\texttt{public abstract List {\bf  getAuthMethods}()
}%end signature
}%end item
\item{\vskip -1.5ex 
\texttt{public abstract Long {\bf  getLastLoginTime}()
}%end signature
}%end item
\item{\vskip -1.5ex 
\texttt{public abstract String {\bf  getUsername}()
}%end signature
}%end item
\item{\vskip -1.5ex 
\texttt{public abstract void {\bf  setLastLoginTime}(\texttt{long} {\bf  time})
}%end signature
}%end item
\end{itemize}
}
}
\section{\label{it.matlice.ingsw.model.data.CustomerUser}\index{CustomerUser}Class CustomerUser}{
\vskip .1in 
Rappresenta un utente fruitore\vskip .1in 
\subsection{Declaration}{
\begin{lstlisting}[frame=none]
public abstract class CustomerUser
 extends it.matlice.ingsw.model.data.User\end{lstlisting}
\subsection{All known subclasses}{CustomerUserImpl\small{\refdefined{it.matlice.ingsw.model.data.impl.jdbc.types.CustomerUserImpl}}}
\subsection{Constructor summary}{
\begin{verse}
{\bf CustomerUser()} \\
\end{verse}
}
\subsection{Constructors}{
\vskip -2em
\begin{itemize}
\item{ 
\index{CustomerUser()}
{\bf  CustomerUser}\\
\begin{lstlisting}[frame=none]
public CustomerUser()\end{lstlisting} %end signature
}%end item
\end{itemize}
}
\subsection{Members inherited from class User }{
\texttt{it.matlice.ingsw.model.data.User} {\small 
\refdefined{it.matlice.ingsw.model.data.User}}
{\small 

\vskip -2em
\begin{itemize}
\item{\vskip -1.5ex 
\texttt{public abstract List {\bf  getAuthMethods}()
}%end signature
}%end item
\item{\vskip -1.5ex 
\texttt{public abstract Long {\bf  getLastLoginTime}()
}%end signature
}%end item
\item{\vskip -1.5ex 
\texttt{public abstract String {\bf  getUsername}()
}%end signature
}%end item
\item{\vskip -1.5ex 
\texttt{public abstract void {\bf  setLastLoginTime}(\texttt{long} {\bf  time})
}%end signature
}%end item
\end{itemize}
}
}
\section{\label{it.matlice.ingsw.model.data.Hierarchy}\index{Hierarchy}Class Hierarchy}{
\vskip .1in 
la classe rappresenta una gerarchia. il nome della gerarchia è dato dal nome della sua categoria radice.\vskip .1in 
\subsection{Declaration}{
\begin{lstlisting}[frame=none]
public abstract class Hierarchy
 extends java.lang.Object\end{lstlisting}
\subsection{All known subclasses}{HierarchyImpl\small{\refdefined{it.matlice.ingsw.model.data.impl.jdbc.types.HierarchyImpl}}}
\subsection{Constructor summary}{
\begin{verse}
{\bf Hierarchy(Category)} \\
\end{verse}
}
\subsection{Method summary}{
\begin{verse}
{\bf getRootCategory()} \\
\end{verse}
}
\subsection{Constructors}{
\vskip -2em
\begin{itemize}
\item{ 
\index{Hierarchy(Category)}
{\bf  Hierarchy}\\
\begin{lstlisting}[frame=none]
public Hierarchy(Category rootCategory)\end{lstlisting} %end signature
}%end item
\end{itemize}
}
\subsection{Methods}{
\vskip -2em
\begin{itemize}
\item{ 
\index{getRootCategory()}
{\bf  getRootCategory}\\
\begin{lstlisting}[frame=none]
public Category getRootCategory()\end{lstlisting} %end signature
}%end item
\end{itemize}
}
}
\section{\label{it.matlice.ingsw.model.data.Interval}\index{Interval}Class Interval}{
\vskip .1in 
\subsection{Declaration}{
\begin{lstlisting}[frame=none]
public class Interval
 extends java.lang.Object implements java.lang.Comparable\end{lstlisting}
\subsection{Constructor summary}{
\begin{verse}
{\bf Interval(int, int)} \\
\end{verse}
}
\subsection{Method summary}{
\begin{verse}
{\bf compareTo(Interval)} \\
{\bf fromString(String)} Dato una stringa rappresentante un intervallo, ritorna l'istanza dell'intervallo\\
{\bf getEndingMinute()} \\
{\bf getStartingMinute()} \\
{\bf includes(Interval.Time)} Ritorna true se l'orario specificato è incluso nell'intervallo\\
{\bf mergeIntervals(List)} Data una lista di intervalli, li riduce nel minimo numero di intervalli\\
{\bf toString()} \\
\end{verse}
}
\subsection{Constructors}{
\vskip -2em
\begin{itemize}
\item{ 
\index{Interval(int, int)}
{\bf  Interval}\\
\begin{lstlisting}[frame=none]
public Interval(int start,int end)\end{lstlisting} %end signature
}%end item
\end{itemize}
}
\subsection{Methods}{
\vskip -2em
\begin{itemize}
\item{ 
\index{compareTo(Interval)}
{\bf  compareTo}\\
\begin{lstlisting}[frame=none]
public int compareTo(Interval o)\end{lstlisting} %end signature
}%end item
\item{ 
\index{fromString(String)}
{\bf  fromString}\\
\begin{lstlisting}[frame=none]
public static Interval fromString(java.lang.String intervalString) throws it.matlice.ingsw.model.exceptions.CannotParseIntervalException, it.matlice.ingsw.model.exceptions.InvalidIntervalException\end{lstlisting} %end signature
\begin{itemize}
\item{
{\bf  Description}

Dato una stringa rappresentante un intervallo, ritorna l'istanza dell'intervallo
}
\item{
{\bf  Parameters}
  \begin{itemize}
   \item{
\texttt{intervalString} -- stringa dell'intervallo}
  \end{itemize}
}%end item
\item{{\bf  Returns} -- 
istanza di intervallo 
}%end item
\item{{\bf  Throws}
  \begin{itemize}
   \item{\vskip -.6ex \texttt{it.matlice.ingsw.model.exceptions.CannotParseIntervalException} -- errore durante il parsing della stringa}
   \item{\vskip -.6ex \texttt{it.matlice.ingsw.model.exceptions.InvalidIntervalException} -- intervallo invalido}
  \end{itemize}
}%end item
\end{itemize}
}%end item
\item{ 
\index{getEndingMinute()}
{\bf  getEndingMinute}\\
\begin{lstlisting}[frame=none]
public int getEndingMinute()\end{lstlisting} %end signature
}%end item
\item{ 
\index{getStartingMinute()}
{\bf  getStartingMinute}\\
\begin{lstlisting}[frame=none]
public int getStartingMinute()\end{lstlisting} %end signature
}%end item
\item{ 
\index{includes(Interval.Time)}
{\bf  includes}\\
\begin{lstlisting}[frame=none]
public boolean includes(Interval.Time t)\end{lstlisting} %end signature
\begin{itemize}
\item{
{\bf  Description}

Ritorna true se l'orario specificato è incluso nell'intervallo
}
\item{
{\bf  Parameters}
  \begin{itemize}
   \item{
\texttt{t} -- orario da verificare}
  \end{itemize}
}%end item
\item{{\bf  Returns} -- 
bool 
}%end item
\end{itemize}
}%end item
\item{ 
\index{mergeIntervals(List)}
{\bf  mergeIntervals}\\
\begin{lstlisting}[frame=none]
public static java.util.List mergeIntervals(java.util.List intervals)\end{lstlisting} %end signature
\begin{itemize}
\item{
{\bf  Description}

Data una lista di intervalli, li riduce nel minimo numero di intervalli
}
\item{
{\bf  Parameters}
  \begin{itemize}
   \item{
\texttt{intervals} -- intervalli in ingresso}
  \end{itemize}
}%end item
\item{{\bf  Returns} -- 
la lista ridotta di intervalli 
}%end item
\end{itemize}
}%end item
\item{ 
\index{toString()}
{\bf  toString}\\
\begin{lstlisting}[frame=none]
public java.lang.String toString()\end{lstlisting} %end signature
}%end item
\end{itemize}
}
}
\section{\label{it.matlice.ingsw.model.data.Interval.Time}\index{Interval.Time}Class Interval.Time}{
\vskip .1in 
\subsection{Declaration}{
\begin{lstlisting}[frame=none]
public static class Interval.Time
 extends java.lang.Object implements java.lang.Comparable\end{lstlisting}
\subsection{Constructor summary}{
\begin{verse}
{\bf Time(int)} \\
\end{verse}
}
\subsection{Method summary}{
\begin{verse}
{\bf compareTo(Interval.Time)} \\
{\bf fromString(String)} Dato una stringa rappresentante un orario, ritorna l'istanza dell'orario\\
{\bf getHour()} \\
{\bf getMinute()} \\
{\bf getTime()} \\
{\bf toString()} \\
\end{verse}
}
\subsection{Constructors}{
\vskip -2em
\begin{itemize}
\item{ 
\index{Time(int)}
{\bf  Time}\\
\begin{lstlisting}[frame=none]
public Time(int time)\end{lstlisting} %end signature
}%end item
\end{itemize}
}
\subsection{Methods}{
\vskip -2em
\begin{itemize}
\item{ 
\index{compareTo(Interval.Time)}
{\bf  compareTo}\\
\begin{lstlisting}[frame=none]
public int compareTo(Interval.Time o)\end{lstlisting} %end signature
}%end item
\item{ 
\index{fromString(String)}
{\bf  fromString}\\
\begin{lstlisting}[frame=none]
public static Interval.Time fromString(java.lang.String timeString) throws it.matlice.ingsw.model.exceptions.CannotParseTimeException, it.matlice.ingsw.model.exceptions.InvalidTimeException\end{lstlisting} %end signature
\begin{itemize}
\item{
{\bf  Description}

Dato una stringa rappresentante un orario, ritorna l'istanza dell'orario
}
\item{
{\bf  Parameters}
  \begin{itemize}
   \item{
\texttt{timeString} -- stringa dell'orario}
  \end{itemize}
}%end item
\item{{\bf  Returns} -- 
istanza di orario 
}%end item
\item{{\bf  Throws}
  \begin{itemize}
   \item{\vskip -.6ex \texttt{it.matlice.ingsw.model.exceptions.CannotParseTimeException} -- errore durante il parsing della stringa}
   \item{\vskip -.6ex \texttt{it.matlice.ingsw.model.exceptions.InvalidTimeException} -- orario invalido}
  \end{itemize}
}%end item
\end{itemize}
}%end item
\item{ 
\index{getHour()}
{\bf  getHour}\\
\begin{lstlisting}[frame=none]
public int getHour()\end{lstlisting} %end signature
}%end item
\item{ 
\index{getMinute()}
{\bf  getMinute}\\
\begin{lstlisting}[frame=none]
public int getMinute()\end{lstlisting} %end signature
}%end item
\item{ 
\index{getTime()}
{\bf  getTime}\\
\begin{lstlisting}[frame=none]
public int getTime()\end{lstlisting} %end signature
}%end item
\item{ 
\index{toString()}
{\bf  toString}\\
\begin{lstlisting}[frame=none]
public java.lang.String toString()\end{lstlisting} %end signature
}%end item
\end{itemize}
}
}
\section{\label{it.matlice.ingsw.model.data.LeafCategory}\index{LeafCategory}Class LeafCategory}{
\vskip .1in 
rappresenta una categoria di tipo foglia.\vskip .1in 
\subsection{Declaration}{
\begin{lstlisting}[frame=none]
public abstract class LeafCategory
 extends it.matlice.ingsw.model.data.Category\end{lstlisting}
\subsection{All known subclasses}{LeafCategoryImpl\small{\refdefined{it.matlice.ingsw.model.data.impl.jdbc.types.LeafCategoryImpl}}}
\subsection{Constructor summary}{
\begin{verse}
{\bf LeafCategory()} \\
\end{verse}
}
\subsection{Method summary}{
\begin{verse}
{\bf convertToNode()} \\
\end{verse}
}
\subsection{Constructors}{
\vskip -2em
\begin{itemize}
\item{ 
\index{LeafCategory()}
{\bf  LeafCategory}\\
\begin{lstlisting}[frame=none]
public LeafCategory()\end{lstlisting} %end signature
}%end item
\end{itemize}
}
\subsection{Methods}{
\vskip -2em
\begin{itemize}
\item{ 
\index{convertToNode()}
{\bf  convertToNode}\\
\begin{lstlisting}[frame=none]
public abstract NodeCategory convertToNode()\end{lstlisting} %end signature
}%end item
\end{itemize}
}
\subsection{Members inherited from class Category }{
\texttt{it.matlice.ingsw.model.data.Category} {\small 
\refdefined{it.matlice.ingsw.model.data.Category}}
{\small 

\vskip -2em
\begin{itemize}
\item{\vskip -1.5ex 
\texttt{public void {\bf  clear}()
}%end signature
}%end item
\item{\vskip -1.5ex 
\texttt{public boolean {\bf  containsKey}(\texttt{java.lang.Object} {\bf  key})
}%end signature
}%end item
\item{\vskip -1.5ex 
\texttt{public boolean {\bf  containsValue}(\texttt{java.lang.Object} {\bf  value})
}%end signature
}%end item
\item{\vskip -1.5ex 
\texttt{public Set {\bf  fullEntrySet}()
}%end signature
}%end item
\item{\vskip -1.5ex 
\texttt{public String {\bf  fullToString}()
}%end signature
}%end item
\item{\vskip -1.5ex 
\texttt{public TypeDefinition {\bf  get}(\texttt{java.lang.Object} {\bf  key})
}%end signature
}%end item
\item{\vskip -1.5ex 
\texttt{public abstract List {\bf  getChildLeafs}()
}%end signature
}%end item
\item{\vskip -1.5ex 
\texttt{public abstract String {\bf  getDescription}()
}%end signature
}%end item
\item{\vskip -1.5ex 
\texttt{public NodeCategory {\bf  getFather}()
}%end signature
}%end item
\item{\vskip -1.5ex 
\texttt{public abstract String {\bf  getName}()
}%end signature
}%end item
\item{\vskip -1.5ex 
\texttt{public boolean {\bf  isCategoryValid}()
}%end signature
}%end item
\item{\vskip -1.5ex 
\texttt{public boolean {\bf  isEmpty}()
}%end signature
}%end item
\item{\vskip -1.5ex 
\texttt{public boolean {\bf  isRequired}(\texttt{java.lang.String} {\bf  name})
}%end signature
}%end item
\item{\vskip -1.5ex 
\texttt{public boolean {\bf  isRoot}()
}%end signature
}%end item
\item{\vskip -1.5ex 
\texttt{public abstract boolean {\bf  isValidChildCategoryName}(\texttt{java.lang.String} {\bf  name})
}%end signature
}%end item
\item{\vskip -1.5ex 
\texttt{public TypeDefinition {\bf  put}(\texttt{java.lang.String} {\bf  key},
\texttt{TypeDefinition} {\bf  value})
}%end signature
}%end item
\item{\vskip -1.5ex 
\texttt{public TypeDefinition {\bf  remove}(\texttt{java.lang.Object} {\bf  key})
}%end signature
}%end item
\item{\vskip -1.5ex 
\texttt{public void {\bf  setFather}(\texttt{NodeCategory} {\bf  father})
}%end signature
}%end item
\item{\vskip -1.5ex 
\texttt{public int {\bf  size}()
}%end signature
}%end item
\item{\vskip -1.5ex 
\texttt{public String {\bf  toString}()
}%end signature
}%end item
\end{itemize}
}
\subsection{Members inherited from class HashMap }{
\texttt{java.util.HashMap} {\small 
\refdefined{java.util.HashMap}}
{\small 

\vskip -2em
\begin{itemize}
\item{\vskip -1.5ex 
\texttt{public void {\bf  clear}()
}%end signature
}%end item
\item{\vskip -1.5ex 
\texttt{public Object {\bf  clone}()
}%end signature
}%end item
\item{\vskip -1.5ex 
\texttt{public Object {\bf  compute}(\texttt{java.lang.Object} {\bf  arg0},
\texttt{function.BiFunction} {\bf  arg1})
}%end signature
}%end item
\item{\vskip -1.5ex 
\texttt{public Object {\bf  computeIfAbsent}(\texttt{java.lang.Object} {\bf  arg0},
\texttt{function.Function} {\bf  arg1})
}%end signature
}%end item
\item{\vskip -1.5ex 
\texttt{public Object {\bf  computeIfPresent}(\texttt{java.lang.Object} {\bf  arg0},
\texttt{function.BiFunction} {\bf  arg1})
}%end signature
}%end item
\item{\vskip -1.5ex 
\texttt{public boolean {\bf  containsKey}(\texttt{java.lang.Object} {\bf  arg0})
}%end signature
}%end item
\item{\vskip -1.5ex 
\texttt{public boolean {\bf  containsValue}(\texttt{java.lang.Object} {\bf  arg0})
}%end signature
}%end item
\item{\vskip -1.5ex 
\texttt{public Set {\bf  entrySet}()
}%end signature
}%end item
\item{\vskip -1.5ex 
\texttt{public void {\bf  forEach}(\texttt{function.BiConsumer} {\bf  arg0})
}%end signature
}%end item
\item{\vskip -1.5ex 
\texttt{public Object {\bf  get}(\texttt{java.lang.Object} {\bf  arg0})
}%end signature
}%end item
\item{\vskip -1.5ex 
\texttt{public Object {\bf  getOrDefault}(\texttt{java.lang.Object} {\bf  arg0},
\texttt{java.lang.Object} {\bf  arg1})
}%end signature
}%end item
\item{\vskip -1.5ex 
\texttt{public boolean {\bf  isEmpty}()
}%end signature
}%end item
\item{\vskip -1.5ex 
\texttt{public Set {\bf  keySet}()
}%end signature
}%end item
\item{\vskip -1.5ex 
\texttt{public Object {\bf  merge}(\texttt{java.lang.Object} {\bf  arg0},
\texttt{java.lang.Object} {\bf  arg1},
\texttt{function.BiFunction} {\bf  arg2})
}%end signature
}%end item
\item{\vskip -1.5ex 
\texttt{public Object {\bf  put}(\texttt{java.lang.Object} {\bf  arg0},
\texttt{java.lang.Object} {\bf  arg1})
}%end signature
}%end item
\item{\vskip -1.5ex 
\texttt{public void {\bf  putAll}(\texttt{Map} {\bf  arg0})
}%end signature
}%end item
\item{\vskip -1.5ex 
\texttt{public Object {\bf  putIfAbsent}(\texttt{java.lang.Object} {\bf  arg0},
\texttt{java.lang.Object} {\bf  arg1})
}%end signature
}%end item
\item{\vskip -1.5ex 
\texttt{public Object {\bf  remove}(\texttt{java.lang.Object} {\bf  arg0})
}%end signature
}%end item
\item{\vskip -1.5ex 
\texttt{public boolean {\bf  remove}(\texttt{java.lang.Object} {\bf  arg0},
\texttt{java.lang.Object} {\bf  arg1})
}%end signature
}%end item
\item{\vskip -1.5ex 
\texttt{public Object {\bf  replace}(\texttt{java.lang.Object} {\bf  arg0},
\texttt{java.lang.Object} {\bf  arg1})
}%end signature
}%end item
\item{\vskip -1.5ex 
\texttt{public boolean {\bf  replace}(\texttt{java.lang.Object} {\bf  arg0},
\texttt{java.lang.Object} {\bf  arg1},
\texttt{java.lang.Object} {\bf  arg2})
}%end signature
}%end item
\item{\vskip -1.5ex 
\texttt{public void {\bf  replaceAll}(\texttt{function.BiFunction} {\bf  arg0})
}%end signature
}%end item
\item{\vskip -1.5ex 
\texttt{public int {\bf  size}()
}%end signature
}%end item
\item{\vskip -1.5ex 
\texttt{public Collection {\bf  values}()
}%end signature
}%end item
\end{itemize}
}
\subsection{Members inherited from class AbstractMap }{
\texttt{java.util.AbstractMap} {\small 
\refdefined{java.util.AbstractMap}}
{\small 

\vskip -2em
\begin{itemize}
\item{\vskip -1.5ex 
\texttt{public void {\bf  clear}()
}%end signature
}%end item
\item{\vskip -1.5ex 
\texttt{protected Object {\bf  clone}() throws java.lang.CloneNotSupportedException
}%end signature
}%end item
\item{\vskip -1.5ex 
\texttt{public boolean {\bf  containsKey}(\texttt{java.lang.Object} {\bf  arg0})
}%end signature
}%end item
\item{\vskip -1.5ex 
\texttt{public boolean {\bf  containsValue}(\texttt{java.lang.Object} {\bf  arg0})
}%end signature
}%end item
\item{\vskip -1.5ex 
\texttt{public abstract Set {\bf  entrySet}()
}%end signature
}%end item
\item{\vskip -1.5ex 
\texttt{public boolean {\bf  equals}(\texttt{java.lang.Object} {\bf  arg0})
}%end signature
}%end item
\item{\vskip -1.5ex 
\texttt{public Object {\bf  get}(\texttt{java.lang.Object} {\bf  arg0})
}%end signature
}%end item
\item{\vskip -1.5ex 
\texttt{public int {\bf  hashCode}()
}%end signature
}%end item
\item{\vskip -1.5ex 
\texttt{public boolean {\bf  isEmpty}()
}%end signature
}%end item
\item{\vskip -1.5ex 
\texttt{public Set {\bf  keySet}()
}%end signature
}%end item
\item{\vskip -1.5ex 
\texttt{public Object {\bf  put}(\texttt{java.lang.Object} {\bf  arg0},
\texttt{java.lang.Object} {\bf  arg1})
}%end signature
}%end item
\item{\vskip -1.5ex 
\texttt{public void {\bf  putAll}(\texttt{Map} {\bf  arg0})
}%end signature
}%end item
\item{\vskip -1.5ex 
\texttt{public Object {\bf  remove}(\texttt{java.lang.Object} {\bf  arg0})
}%end signature
}%end item
\item{\vskip -1.5ex 
\texttt{public int {\bf  size}()
}%end signature
}%end item
\item{\vskip -1.5ex 
\texttt{public String {\bf  toString}()
}%end signature
}%end item
\item{\vskip -1.5ex 
\texttt{public Collection {\bf  values}()
}%end signature
}%end item
\end{itemize}
}
}
\section{\label{it.matlice.ingsw.model.data.Message}\index{Message}Class Message}{
\vskip .1in 
\subsection{Declaration}{
\begin{lstlisting}[frame=none]
public abstract class Message
 extends java.lang.Object\end{lstlisting}
\subsection{All known subclasses}{MessageImpl\small{\refdefined{it.matlice.ingsw.model.data.impl.jdbc.types.MessageImpl}}}
\subsection{Constructor summary}{
\begin{verse}
{\bf Message()} \\
\end{verse}
}
\subsection{Method summary}{
\begin{verse}
{\bf getDate()} \\
{\bf getLocation()} \\
{\bf getReferencedOffer()} \\
{\bf getTime()} \\
{\bf toString()} \\
\end{verse}
}
\subsection{Constructors}{
\vskip -2em
\begin{itemize}
\item{ 
\index{Message()}
{\bf  Message}\\
\begin{lstlisting}[frame=none]
public Message()\end{lstlisting} %end signature
}%end item
\end{itemize}
}
\subsection{Methods}{
\vskip -2em
\begin{itemize}
\item{ 
\index{getDate()}
{\bf  getDate}\\
\begin{lstlisting}[frame=none]
public abstract java.util.Calendar getDate()\end{lstlisting} %end signature
}%end item
\item{ 
\index{getLocation()}
{\bf  getLocation}\\
\begin{lstlisting}[frame=none]
public abstract java.lang.String getLocation()\end{lstlisting} %end signature
}%end item
\item{ 
\index{getReferencedOffer()}
{\bf  getReferencedOffer}\\
\begin{lstlisting}[frame=none]
public abstract Offer getReferencedOffer()\end{lstlisting} %end signature
}%end item
\item{ 
\index{getTime()}
{\bf  getTime}\\
\begin{lstlisting}[frame=none]
public abstract java.lang.Long getTime()\end{lstlisting} %end signature
}%end item
\item{ 
\index{toString()}
{\bf  toString}\\
\begin{lstlisting}[frame=none]
public java.lang.String toString()\end{lstlisting} %end signature
}%end item
\end{itemize}
}
}
\section{\label{it.matlice.ingsw.model.data.NodeCategory}\index{NodeCategory}Class NodeCategory}{
\vskip .1in 
rappresenta una categoria di tipo nodo.\vskip .1in 
\subsection{Declaration}{
\begin{lstlisting}[frame=none]
public abstract class NodeCategory
 extends it.matlice.ingsw.model.data.Category\end{lstlisting}
\subsection{All known subclasses}{NodeCategoryImpl\small{\refdefined{it.matlice.ingsw.model.data.impl.jdbc.types.NodeCategoryImpl}}}
\subsection{Constructor summary}{
\begin{verse}
{\bf NodeCategory()} \\
\end{verse}
}
\subsection{Method summary}{
\begin{verse}
{\bf addChild(Category)} aggiunge una categoria tra i propri figli.\\
{\bf clone()} Returns a shallow copy of this \texttt{\small HashMap} instance: the keys and values themselves are not cloned.\\
{\bf getChildren()} \\
{\bf removeChild(Category)} rimuove un figlio\\
\end{verse}
}
\subsection{Constructors}{
\vskip -2em
\begin{itemize}
\item{ 
\index{NodeCategory()}
{\bf  NodeCategory}\\
\begin{lstlisting}[frame=none]
public NodeCategory()\end{lstlisting} %end signature
}%end item
\end{itemize}
}
\subsection{Methods}{
\vskip -2em
\begin{itemize}
\item{ 
\index{addChild(Category)}
{\bf  addChild}\\
\begin{lstlisting}[frame=none]
public Category addChild(Category child)\end{lstlisting} %end signature
\begin{itemize}
\item{
{\bf  Description}

aggiunge una categoria tra i propri figli.
}
\item{
{\bf  Parameters}
  \begin{itemize}
   \item{
\texttt{child} -- categoria figlio}
  \end{itemize}
}%end item
\item{{\bf  Returns} -- 
la categoria aggiunta 
}%end item
\end{itemize}
}%end item
\item{ 
\index{clone()}
{\bf  clone}\\
\begin{lstlisting}[frame=none]
public abstract java.lang.Object clone()\end{lstlisting} %end signature
\begin{itemize}
\item{
{\bf  Description}

Returns a shallow copy of this \texttt{\small HashMap} instance: the keys and values themselves are not cloned.
}
\item{{\bf  Returns} -- 
a shallow copy of this map 
}%end item
\end{itemize}
}%end item
\item{ 
\index{getChildren()}
{\bf  getChildren}\\
\begin{lstlisting}[frame=none]
public Category[] getChildren()\end{lstlisting} %end signature
}%end item
\item{ 
\index{removeChild(Category)}
{\bf  removeChild}\\
\begin{lstlisting}[frame=none]
public Category removeChild(Category child)\end{lstlisting} %end signature
\begin{itemize}
\item{
{\bf  Description}

rimuove un figlio
}
\item{
{\bf  Parameters}
  \begin{itemize}
   \item{
\texttt{child} -- figlio da rimuovere}
  \end{itemize}
}%end item
\item{{\bf  Returns} -- 
il figlio rimosso 
}%end item
\end{itemize}
}%end item
\end{itemize}
}
\subsection{Members inherited from class Category }{
\texttt{it.matlice.ingsw.model.data.Category} {\small 
\refdefined{it.matlice.ingsw.model.data.Category}}
{\small 

\vskip -2em
\begin{itemize}
\item{\vskip -1.5ex 
\texttt{public void {\bf  clear}()
}%end signature
}%end item
\item{\vskip -1.5ex 
\texttt{public boolean {\bf  containsKey}(\texttt{java.lang.Object} {\bf  key})
}%end signature
}%end item
\item{\vskip -1.5ex 
\texttt{public boolean {\bf  containsValue}(\texttt{java.lang.Object} {\bf  value})
}%end signature
}%end item
\item{\vskip -1.5ex 
\texttt{public Set {\bf  fullEntrySet}()
}%end signature
}%end item
\item{\vskip -1.5ex 
\texttt{public String {\bf  fullToString}()
}%end signature
}%end item
\item{\vskip -1.5ex 
\texttt{public TypeDefinition {\bf  get}(\texttt{java.lang.Object} {\bf  key})
}%end signature
}%end item
\item{\vskip -1.5ex 
\texttt{public abstract List {\bf  getChildLeafs}()
}%end signature
}%end item
\item{\vskip -1.5ex 
\texttt{public abstract String {\bf  getDescription}()
}%end signature
}%end item
\item{\vskip -1.5ex 
\texttt{public NodeCategory {\bf  getFather}()
}%end signature
}%end item
\item{\vskip -1.5ex 
\texttt{public abstract String {\bf  getName}()
}%end signature
}%end item
\item{\vskip -1.5ex 
\texttt{public boolean {\bf  isCategoryValid}()
}%end signature
}%end item
\item{\vskip -1.5ex 
\texttt{public boolean {\bf  isEmpty}()
}%end signature
}%end item
\item{\vskip -1.5ex 
\texttt{public boolean {\bf  isRequired}(\texttt{java.lang.String} {\bf  name})
}%end signature
}%end item
\item{\vskip -1.5ex 
\texttt{public boolean {\bf  isRoot}()
}%end signature
}%end item
\item{\vskip -1.5ex 
\texttt{public abstract boolean {\bf  isValidChildCategoryName}(\texttt{java.lang.String} {\bf  name})
}%end signature
}%end item
\item{\vskip -1.5ex 
\texttt{public TypeDefinition {\bf  put}(\texttt{java.lang.String} {\bf  key},
\texttt{TypeDefinition} {\bf  value})
}%end signature
}%end item
\item{\vskip -1.5ex 
\texttt{public TypeDefinition {\bf  remove}(\texttt{java.lang.Object} {\bf  key})
}%end signature
}%end item
\item{\vskip -1.5ex 
\texttt{public void {\bf  setFather}(\texttt{NodeCategory} {\bf  father})
}%end signature
}%end item
\item{\vskip -1.5ex 
\texttt{public int {\bf  size}()
}%end signature
}%end item
\item{\vskip -1.5ex 
\texttt{public String {\bf  toString}()
}%end signature
}%end item
\end{itemize}
}
\subsection{Members inherited from class HashMap }{
\texttt{java.util.HashMap} {\small 
\refdefined{java.util.HashMap}}
{\small 

\vskip -2em
\begin{itemize}
\item{\vskip -1.5ex 
\texttt{public void {\bf  clear}()
}%end signature
}%end item
\item{\vskip -1.5ex 
\texttt{public Object {\bf  clone}()
}%end signature
}%end item
\item{\vskip -1.5ex 
\texttt{public Object {\bf  compute}(\texttt{java.lang.Object} {\bf  arg0},
\texttt{function.BiFunction} {\bf  arg1})
}%end signature
}%end item
\item{\vskip -1.5ex 
\texttt{public Object {\bf  computeIfAbsent}(\texttt{java.lang.Object} {\bf  arg0},
\texttt{function.Function} {\bf  arg1})
}%end signature
}%end item
\item{\vskip -1.5ex 
\texttt{public Object {\bf  computeIfPresent}(\texttt{java.lang.Object} {\bf  arg0},
\texttt{function.BiFunction} {\bf  arg1})
}%end signature
}%end item
\item{\vskip -1.5ex 
\texttt{public boolean {\bf  containsKey}(\texttt{java.lang.Object} {\bf  arg0})
}%end signature
}%end item
\item{\vskip -1.5ex 
\texttt{public boolean {\bf  containsValue}(\texttt{java.lang.Object} {\bf  arg0})
}%end signature
}%end item
\item{\vskip -1.5ex 
\texttt{public Set {\bf  entrySet}()
}%end signature
}%end item
\item{\vskip -1.5ex 
\texttt{public void {\bf  forEach}(\texttt{function.BiConsumer} {\bf  arg0})
}%end signature
}%end item
\item{\vskip -1.5ex 
\texttt{public Object {\bf  get}(\texttt{java.lang.Object} {\bf  arg0})
}%end signature
}%end item
\item{\vskip -1.5ex 
\texttt{public Object {\bf  getOrDefault}(\texttt{java.lang.Object} {\bf  arg0},
\texttt{java.lang.Object} {\bf  arg1})
}%end signature
}%end item
\item{\vskip -1.5ex 
\texttt{public boolean {\bf  isEmpty}()
}%end signature
}%end item
\item{\vskip -1.5ex 
\texttt{public Set {\bf  keySet}()
}%end signature
}%end item
\item{\vskip -1.5ex 
\texttt{public Object {\bf  merge}(\texttt{java.lang.Object} {\bf  arg0},
\texttt{java.lang.Object} {\bf  arg1},
\texttt{function.BiFunction} {\bf  arg2})
}%end signature
}%end item
\item{\vskip -1.5ex 
\texttt{public Object {\bf  put}(\texttt{java.lang.Object} {\bf  arg0},
\texttt{java.lang.Object} {\bf  arg1})
}%end signature
}%end item
\item{\vskip -1.5ex 
\texttt{public void {\bf  putAll}(\texttt{Map} {\bf  arg0})
}%end signature
}%end item
\item{\vskip -1.5ex 
\texttt{public Object {\bf  putIfAbsent}(\texttt{java.lang.Object} {\bf  arg0},
\texttt{java.lang.Object} {\bf  arg1})
}%end signature
}%end item
\item{\vskip -1.5ex 
\texttt{public Object {\bf  remove}(\texttt{java.lang.Object} {\bf  arg0})
}%end signature
}%end item
\item{\vskip -1.5ex 
\texttt{public boolean {\bf  remove}(\texttt{java.lang.Object} {\bf  arg0},
\texttt{java.lang.Object} {\bf  arg1})
}%end signature
}%end item
\item{\vskip -1.5ex 
\texttt{public Object {\bf  replace}(\texttt{java.lang.Object} {\bf  arg0},
\texttt{java.lang.Object} {\bf  arg1})
}%end signature
}%end item
\item{\vskip -1.5ex 
\texttt{public boolean {\bf  replace}(\texttt{java.lang.Object} {\bf  arg0},
\texttt{java.lang.Object} {\bf  arg1},
\texttt{java.lang.Object} {\bf  arg2})
}%end signature
}%end item
\item{\vskip -1.5ex 
\texttt{public void {\bf  replaceAll}(\texttt{function.BiFunction} {\bf  arg0})
}%end signature
}%end item
\item{\vskip -1.5ex 
\texttt{public int {\bf  size}()
}%end signature
}%end item
\item{\vskip -1.5ex 
\texttt{public Collection {\bf  values}()
}%end signature
}%end item
\end{itemize}
}
\subsection{Members inherited from class AbstractMap }{
\texttt{java.util.AbstractMap} {\small 
\refdefined{java.util.AbstractMap}}
{\small 

\vskip -2em
\begin{itemize}
\item{\vskip -1.5ex 
\texttt{public void {\bf  clear}()
}%end signature
}%end item
\item{\vskip -1.5ex 
\texttt{protected Object {\bf  clone}() throws java.lang.CloneNotSupportedException
}%end signature
}%end item
\item{\vskip -1.5ex 
\texttt{public boolean {\bf  containsKey}(\texttt{java.lang.Object} {\bf  arg0})
}%end signature
}%end item
\item{\vskip -1.5ex 
\texttt{public boolean {\bf  containsValue}(\texttt{java.lang.Object} {\bf  arg0})
}%end signature
}%end item
\item{\vskip -1.5ex 
\texttt{public abstract Set {\bf  entrySet}()
}%end signature
}%end item
\item{\vskip -1.5ex 
\texttt{public boolean {\bf  equals}(\texttt{java.lang.Object} {\bf  arg0})
}%end signature
}%end item
\item{\vskip -1.5ex 
\texttt{public Object {\bf  get}(\texttt{java.lang.Object} {\bf  arg0})
}%end signature
}%end item
\item{\vskip -1.5ex 
\texttt{public int {\bf  hashCode}()
}%end signature
}%end item
\item{\vskip -1.5ex 
\texttt{public boolean {\bf  isEmpty}()
}%end signature
}%end item
\item{\vskip -1.5ex 
\texttt{public Set {\bf  keySet}()
}%end signature
}%end item
\item{\vskip -1.5ex 
\texttt{public Object {\bf  put}(\texttt{java.lang.Object} {\bf  arg0},
\texttt{java.lang.Object} {\bf  arg1})
}%end signature
}%end item
\item{\vskip -1.5ex 
\texttt{public void {\bf  putAll}(\texttt{Map} {\bf  arg0})
}%end signature
}%end item
\item{\vskip -1.5ex 
\texttt{public Object {\bf  remove}(\texttt{java.lang.Object} {\bf  arg0})
}%end signature
}%end item
\item{\vskip -1.5ex 
\texttt{public int {\bf  size}()
}%end signature
}%end item
\item{\vskip -1.5ex 
\texttt{public String {\bf  toString}()
}%end signature
}%end item
\item{\vskip -1.5ex 
\texttt{public Collection {\bf  values}()
}%end signature
}%end item
\end{itemize}
}
}
\section{\label{it.matlice.ingsw.model.data.Offer}\index{Offer}Class Offer}{
\vskip .1in 
\subsection{Declaration}{
\begin{lstlisting}[frame=none]
public abstract class Offer
 extends java.util.HashMap\end{lstlisting}
\subsection{All known subclasses}{OfferImpl\small{\refdefined{it.matlice.ingsw.model.data.impl.jdbc.types.OfferImpl}}}
\subsection{Constructor summary}{
\begin{verse}
{\bf Offer()} \\
\end{verse}
}
\subsection{Method summary}{
\begin{verse}
{\bf getCategory()} \\
{\bf getLinkedOffer()} \\
{\bf getName()} \\
{\bf getOwner()} \\
{\bf getProposedTime()} \\
{\bf getStatus()} \\
{\bf toString()} \\
\end{verse}
}
\subsection{Constructors}{
\vskip -2em
\begin{itemize}
\item{ 
\index{Offer()}
{\bf  Offer}\\
\begin{lstlisting}[frame=none]
public Offer()\end{lstlisting} %end signature
}%end item
\end{itemize}
}
\subsection{Methods}{
\vskip -2em
\begin{itemize}
\item{ 
\index{getCategory()}
{\bf  getCategory}\\
\begin{lstlisting}[frame=none]
public abstract LeafCategory getCategory()\end{lstlisting} %end signature
}%end item
\item{ 
\index{getLinkedOffer()}
{\bf  getLinkedOffer}\\
\begin{lstlisting}[frame=none]
public abstract Offer getLinkedOffer()\end{lstlisting} %end signature
}%end item
\item{ 
\index{getName()}
{\bf  getName}\\
\begin{lstlisting}[frame=none]
public abstract java.lang.String getName()\end{lstlisting} %end signature
}%end item
\item{ 
\index{getOwner()}
{\bf  getOwner}\\
\begin{lstlisting}[frame=none]
public abstract User getOwner()\end{lstlisting} %end signature
}%end item
\item{ 
\index{getProposedTime()}
{\bf  getProposedTime}\\
\begin{lstlisting}[frame=none]
public abstract java.lang.Long getProposedTime()\end{lstlisting} %end signature
}%end item
\item{ 
\index{getStatus()}
{\bf  getStatus}\\
\begin{lstlisting}[frame=none]
public abstract Offer.OfferStatus getStatus()\end{lstlisting} %end signature
}%end item
\item{ 
\index{toString()}
{\bf  toString}\\
\begin{lstlisting}[frame=none]
public java.lang.String toString()\end{lstlisting} %end signature
}%end item
\end{itemize}
}
\subsection{Members inherited from class HashMap }{
\texttt{java.util.HashMap} {\small 
\refdefined{java.util.HashMap}}
{\small 

\vskip -2em
\begin{itemize}
\item{\vskip -1.5ex 
\texttt{public void {\bf  clear}()
}%end signature
}%end item
\item{\vskip -1.5ex 
\texttt{public Object {\bf  clone}()
}%end signature
}%end item
\item{\vskip -1.5ex 
\texttt{public Object {\bf  compute}(\texttt{java.lang.Object} {\bf  arg0},
\texttt{function.BiFunction} {\bf  arg1})
}%end signature
}%end item
\item{\vskip -1.5ex 
\texttt{public Object {\bf  computeIfAbsent}(\texttt{java.lang.Object} {\bf  arg0},
\texttt{function.Function} {\bf  arg1})
}%end signature
}%end item
\item{\vskip -1.5ex 
\texttt{public Object {\bf  computeIfPresent}(\texttt{java.lang.Object} {\bf  arg0},
\texttt{function.BiFunction} {\bf  arg1})
}%end signature
}%end item
\item{\vskip -1.5ex 
\texttt{public boolean {\bf  containsKey}(\texttt{java.lang.Object} {\bf  arg0})
}%end signature
}%end item
\item{\vskip -1.5ex 
\texttt{public boolean {\bf  containsValue}(\texttt{java.lang.Object} {\bf  arg0})
}%end signature
}%end item
\item{\vskip -1.5ex 
\texttt{public Set {\bf  entrySet}()
}%end signature
}%end item
\item{\vskip -1.5ex 
\texttt{public void {\bf  forEach}(\texttt{function.BiConsumer} {\bf  arg0})
}%end signature
}%end item
\item{\vskip -1.5ex 
\texttt{public Object {\bf  get}(\texttt{java.lang.Object} {\bf  arg0})
}%end signature
}%end item
\item{\vskip -1.5ex 
\texttt{public Object {\bf  getOrDefault}(\texttt{java.lang.Object} {\bf  arg0},
\texttt{java.lang.Object} {\bf  arg1})
}%end signature
}%end item
\item{\vskip -1.5ex 
\texttt{public boolean {\bf  isEmpty}()
}%end signature
}%end item
\item{\vskip -1.5ex 
\texttt{public Set {\bf  keySet}()
}%end signature
}%end item
\item{\vskip -1.5ex 
\texttt{public Object {\bf  merge}(\texttt{java.lang.Object} {\bf  arg0},
\texttt{java.lang.Object} {\bf  arg1},
\texttt{function.BiFunction} {\bf  arg2})
}%end signature
}%end item
\item{\vskip -1.5ex 
\texttt{public Object {\bf  put}(\texttt{java.lang.Object} {\bf  arg0},
\texttt{java.lang.Object} {\bf  arg1})
}%end signature
}%end item
\item{\vskip -1.5ex 
\texttt{public void {\bf  putAll}(\texttt{Map} {\bf  arg0})
}%end signature
}%end item
\item{\vskip -1.5ex 
\texttt{public Object {\bf  putIfAbsent}(\texttt{java.lang.Object} {\bf  arg0},
\texttt{java.lang.Object} {\bf  arg1})
}%end signature
}%end item
\item{\vskip -1.5ex 
\texttt{public Object {\bf  remove}(\texttt{java.lang.Object} {\bf  arg0})
}%end signature
}%end item
\item{\vskip -1.5ex 
\texttt{public boolean {\bf  remove}(\texttt{java.lang.Object} {\bf  arg0},
\texttt{java.lang.Object} {\bf  arg1})
}%end signature
}%end item
\item{\vskip -1.5ex 
\texttt{public Object {\bf  replace}(\texttt{java.lang.Object} {\bf  arg0},
\texttt{java.lang.Object} {\bf  arg1})
}%end signature
}%end item
\item{\vskip -1.5ex 
\texttt{public boolean {\bf  replace}(\texttt{java.lang.Object} {\bf  arg0},
\texttt{java.lang.Object} {\bf  arg1},
\texttt{java.lang.Object} {\bf  arg2})
}%end signature
}%end item
\item{\vskip -1.5ex 
\texttt{public void {\bf  replaceAll}(\texttt{function.BiFunction} {\bf  arg0})
}%end signature
}%end item
\item{\vskip -1.5ex 
\texttt{public int {\bf  size}()
}%end signature
}%end item
\item{\vskip -1.5ex 
\texttt{public Collection {\bf  values}()
}%end signature
}%end item
\end{itemize}
}
\subsection{Members inherited from class AbstractMap }{
\texttt{java.util.AbstractMap} {\small 
\refdefined{java.util.AbstractMap}}
{\small 

\vskip -2em
\begin{itemize}
\item{\vskip -1.5ex 
\texttt{public void {\bf  clear}()
}%end signature
}%end item
\item{\vskip -1.5ex 
\texttt{protected Object {\bf  clone}() throws java.lang.CloneNotSupportedException
}%end signature
}%end item
\item{\vskip -1.5ex 
\texttt{public boolean {\bf  containsKey}(\texttt{java.lang.Object} {\bf  arg0})
}%end signature
}%end item
\item{\vskip -1.5ex 
\texttt{public boolean {\bf  containsValue}(\texttt{java.lang.Object} {\bf  arg0})
}%end signature
}%end item
\item{\vskip -1.5ex 
\texttt{public abstract Set {\bf  entrySet}()
}%end signature
}%end item
\item{\vskip -1.5ex 
\texttt{public boolean {\bf  equals}(\texttt{java.lang.Object} {\bf  arg0})
}%end signature
}%end item
\item{\vskip -1.5ex 
\texttt{public Object {\bf  get}(\texttt{java.lang.Object} {\bf  arg0})
}%end signature
}%end item
\item{\vskip -1.5ex 
\texttt{public int {\bf  hashCode}()
}%end signature
}%end item
\item{\vskip -1.5ex 
\texttt{public boolean {\bf  isEmpty}()
}%end signature
}%end item
\item{\vskip -1.5ex 
\texttt{public Set {\bf  keySet}()
}%end signature
}%end item
\item{\vskip -1.5ex 
\texttt{public Object {\bf  put}(\texttt{java.lang.Object} {\bf  arg0},
\texttt{java.lang.Object} {\bf  arg1})
}%end signature
}%end item
\item{\vskip -1.5ex 
\texttt{public void {\bf  putAll}(\texttt{Map} {\bf  arg0})
}%end signature
}%end item
\item{\vskip -1.5ex 
\texttt{public Object {\bf  remove}(\texttt{java.lang.Object} {\bf  arg0})
}%end signature
}%end item
\item{\vskip -1.5ex 
\texttt{public int {\bf  size}()
}%end signature
}%end item
\item{\vskip -1.5ex 
\texttt{public String {\bf  toString}()
}%end signature
}%end item
\item{\vskip -1.5ex 
\texttt{public Collection {\bf  values}()
}%end signature
}%end item
\end{itemize}
}
}
\section{\label{it.matlice.ingsw.model.data.Offer.OfferStatus}\index{Offer.OfferStatus}Class Offer.OfferStatus}{
\vskip .1in 
\subsection{Declaration}{
\begin{lstlisting}[frame=none]
public static final class Offer.OfferStatus
 extends java.lang.Enum\end{lstlisting}
\subsection{Field summary}{
\begin{verse}
{\bf CLOSED} \\
{\bf COUPLED} \\
{\bf EXCHANGE} \\
{\bf name} \\
{\bf OPEN} \\
{\bf RETRACTED} \\
{\bf SELECTED} \\
\end{verse}
}
\subsection{Method summary}{
\begin{verse}
{\bf getName()} \\
{\bf valueOf(String)} \\
{\bf values()} \\
\end{verse}
}
\subsection{Fields}{
\begin{itemize}
\item{
\index{OPEN}
\label{it.matlice.ingsw.model.data.Offer.OfferStatus.OPEN}\texttt{public static final Offer.OfferStatus\ {\bf  OPEN}}
}
\item{
\index{RETRACTED}
\label{it.matlice.ingsw.model.data.Offer.OfferStatus.RETRACTED}\texttt{public static final Offer.OfferStatus\ {\bf  RETRACTED}}
}
\item{
\index{COUPLED}
\label{it.matlice.ingsw.model.data.Offer.OfferStatus.COUPLED}\texttt{public static final Offer.OfferStatus\ {\bf  COUPLED}}
}
\item{
\index{SELECTED}
\label{it.matlice.ingsw.model.data.Offer.OfferStatus.SELECTED}\texttt{public static final Offer.OfferStatus\ {\bf  SELECTED}}
}
\item{
\index{EXCHANGE}
\label{it.matlice.ingsw.model.data.Offer.OfferStatus.EXCHANGE}\texttt{public static final Offer.OfferStatus\ {\bf  EXCHANGE}}
}
\item{
\index{CLOSED}
\label{it.matlice.ingsw.model.data.Offer.OfferStatus.CLOSED}\texttt{public static final Offer.OfferStatus\ {\bf  CLOSED}}
}
\item{
\index{name}
\label{it.matlice.ingsw.model.data.Offer.OfferStatus.name}\texttt{public final java.lang.String\ {\bf  name}}
}
\end{itemize}
}
\subsection{Methods}{
\vskip -2em
\begin{itemize}
\item{ 
\index{getName()}
{\bf  getName}\\
\begin{lstlisting}[frame=none]
public java.lang.String getName()\end{lstlisting} %end signature
}%end item
\item{ 
\index{valueOf(String)}
{\bf  valueOf}\\
\begin{lstlisting}[frame=none]
public static Offer.OfferStatus valueOf(java.lang.String name)\end{lstlisting} %end signature
}%end item
\item{ 
\index{values()}
{\bf  values}\\
\begin{lstlisting}[frame=none]
public static Offer.OfferStatus[] values()\end{lstlisting} %end signature
}%end item
\end{itemize}
}
\subsection{Members inherited from class Enum }{
\texttt{java.lang.Enum} {\small 
\refdefined{java.lang.Enum}}
{\small 

\vskip -2em
\begin{itemize}
\item{\vskip -1.5ex 
\texttt{protected final Object {\bf  clone}() throws CloneNotSupportedException
}%end signature
}%end item
\item{\vskip -1.5ex 
\texttt{public final int {\bf  compareTo}(\texttt{Enum} {\bf  arg0})
}%end signature
}%end item
\item{\vskip -1.5ex 
\texttt{public final boolean {\bf  equals}(\texttt{Object} {\bf  arg0})
}%end signature
}%end item
\item{\vskip -1.5ex 
\texttt{protected final void {\bf  finalize}()
}%end signature
}%end item
\item{\vskip -1.5ex 
\texttt{public final Class {\bf  getDeclaringClass}()
}%end signature
}%end item
\item{\vskip -1.5ex 
\texttt{public final int {\bf  hashCode}()
}%end signature
}%end item
\item{\vskip -1.5ex 
\texttt{public final String {\bf  name}()
}%end signature
}%end item
\item{\vskip -1.5ex 
\texttt{public final int {\bf  ordinal}()
}%end signature
}%end item
\item{\vskip -1.5ex 
\texttt{public String {\bf  toString}()
}%end signature
}%end item
\item{\vskip -1.5ex 
\texttt{public static Enum {\bf  valueOf}(\texttt{Class} {\bf  arg0},
\texttt{String} {\bf  arg1})
}%end signature
}%end item
\end{itemize}
}
}
\section{\label{it.matlice.ingsw.model.data.Settings}\index{Settings}Class Settings}{
\vskip .1in 
\subsection{Declaration}{
\begin{lstlisting}[frame=none]
public abstract class Settings
 extends java.lang.Object\end{lstlisting}
\subsection{All known subclasses}{SettingsImpl\small{\refdefined{it.matlice.ingsw.model.data.impl.jdbc.types.SettingsImpl}}}
\subsection{Constructor summary}{
\begin{verse}
{\bf Settings()} \\
\end{verse}
}
\subsection{Method summary}{
\begin{verse}
{\bf getCity()} \\
{\bf getDays()} \\
{\bf getDue()} \\
{\bf getIntervals()} \\
{\bf getLocations()} \\
\end{verse}
}
\subsection{Constructors}{
\vskip -2em
\begin{itemize}
\item{ 
\index{Settings()}
{\bf  Settings}\\
\begin{lstlisting}[frame=none]
public Settings()\end{lstlisting} %end signature
}%end item
\end{itemize}
}
\subsection{Methods}{
\vskip -2em
\begin{itemize}
\item{ 
\index{getCity()}
{\bf  getCity}\\
\begin{lstlisting}[frame=none]
public abstract java.lang.String getCity()\end{lstlisting} %end signature
}%end item
\item{ 
\index{getDays()}
{\bf  getDays}\\
\begin{lstlisting}[frame=none]
public abstract java.util.List getDays()\end{lstlisting} %end signature
}%end item
\item{ 
\index{getDue()}
{\bf  getDue}\\
\begin{lstlisting}[frame=none]
public abstract int getDue()\end{lstlisting} %end signature
}%end item
\item{ 
\index{getIntervals()}
{\bf  getIntervals}\\
\begin{lstlisting}[frame=none]
public abstract java.util.List getIntervals()\end{lstlisting} %end signature
}%end item
\item{ 
\index{getLocations()}
{\bf  getLocations}\\
\begin{lstlisting}[frame=none]
public abstract java.util.List getLocations()\end{lstlisting} %end signature
}%end item
\end{itemize}
}
}
\section{\label{it.matlice.ingsw.model.data.Settings.Day}\index{Settings.Day}Class Settings.Day}{
\vskip .1in 
Classe che rappresenta i giorni della settimana\vskip .1in 
\subsection{Declaration}{
\begin{lstlisting}[frame=none]
public static final class Settings.Day
 extends java.lang.Enum\end{lstlisting}
\subsection{Field summary}{
\begin{verse}
{\bf dayMap} \\
{\bf FRI} \\
{\bf MON} \\
{\bf SAT} \\
{\bf SUN} \\
{\bf THU} \\
{\bf TUE} \\
{\bf WED} \\
\end{verse}
}
\subsection{Method summary}{
\begin{verse}
{\bf fromString(String)} Permette di parsare un giorno a partire da una stringa\\
{\bf getCalendarDay()} Ritorna l'intero corrispondente al giorno per la libreria java.util.Calendar\\
{\bf getName()} Ritorna la rappresentazione (in italiano) del giorno\\
{\bf valueOf(String)} \\
{\bf values()} \\
\end{verse}
}
\subsection{Fields}{
\begin{itemize}
\item{
\index{MON}
\label{it.matlice.ingsw.model.data.Settings.Day.MON}\texttt{public static final Settings.Day\ {\bf  MON}}
}
\item{
\index{TUE}
\label{it.matlice.ingsw.model.data.Settings.Day.TUE}\texttt{public static final Settings.Day\ {\bf  TUE}}
}
\item{
\index{WED}
\label{it.matlice.ingsw.model.data.Settings.Day.WED}\texttt{public static final Settings.Day\ {\bf  WED}}
}
\item{
\index{THU}
\label{it.matlice.ingsw.model.data.Settings.Day.THU}\texttt{public static final Settings.Day\ {\bf  THU}}
}
\item{
\index{FRI}
\label{it.matlice.ingsw.model.data.Settings.Day.FRI}\texttt{public static final Settings.Day\ {\bf  FRI}}
}
\item{
\index{SAT}
\label{it.matlice.ingsw.model.data.Settings.Day.SAT}\texttt{public static final Settings.Day\ {\bf  SAT}}
}
\item{
\index{SUN}
\label{it.matlice.ingsw.model.data.Settings.Day.SUN}\texttt{public static final Settings.Day\ {\bf  SUN}}
}
\item{
\index{dayMap}
\label{it.matlice.ingsw.model.data.Settings.Day.dayMap}\texttt{public static final java.util.Map\ {\bf  dayMap}}
}
\end{itemize}
}
\subsection{Methods}{
\vskip -2em
\begin{itemize}
\item{ 
\index{fromString(String)}
{\bf  fromString}\\
\begin{lstlisting}[frame=none]
public static Settings.Day fromString(java.lang.String day) throws it.matlice.ingsw.model.exceptions.CannotParseDayException\end{lstlisting} %end signature
\begin{itemize}
\item{
{\bf  Description}

Permette di parsare un giorno a partire da una stringa
}
\item{
{\bf  Parameters}
  \begin{itemize}
   \item{
\texttt{day} -- rappresentazione come stringa del giorno}
  \end{itemize}
}%end item
\item{{\bf  Returns} -- 
giorno parsato 
}%end item
\item{{\bf  Throws}
  \begin{itemize}
   \item{\vskip -.6ex \texttt{it.matlice.ingsw.model.exceptions.CannotParseDayException} -- errore durante il parsing}
  \end{itemize}
}%end item
\end{itemize}
}%end item
\item{ 
\index{getCalendarDay()}
{\bf  getCalendarDay}\\
\begin{lstlisting}[frame=none]
public int getCalendarDay()\end{lstlisting} %end signature
\begin{itemize}
\item{
{\bf  Description}

Ritorna l'intero corrispondente al giorno per la libreria java.util.Calendar
}
\item{{\bf  Returns} -- 
 
}%end item
\end{itemize}
}%end item
\item{ 
\index{getName()}
{\bf  getName}\\
\begin{lstlisting}[frame=none]
public java.lang.String getName()\end{lstlisting} %end signature
\begin{itemize}
\item{
{\bf  Description}

Ritorna la rappresentazione (in italiano) del giorno
}
\item{{\bf  Returns} -- 
 
}%end item
\end{itemize}
}%end item
\item{ 
\index{valueOf(String)}
{\bf  valueOf}\\
\begin{lstlisting}[frame=none]
public static Settings.Day valueOf(java.lang.String name)\end{lstlisting} %end signature
}%end item
\item{ 
\index{values()}
{\bf  values}\\
\begin{lstlisting}[frame=none]
public static Settings.Day[] values()\end{lstlisting} %end signature
}%end item
\end{itemize}
}
\subsection{Members inherited from class Enum }{
\texttt{java.lang.Enum} {\small 
\refdefined{java.lang.Enum}}
{\small 

\vskip -2em
\begin{itemize}
\item{\vskip -1.5ex 
\texttt{protected final Object {\bf  clone}() throws CloneNotSupportedException
}%end signature
}%end item
\item{\vskip -1.5ex 
\texttt{public final int {\bf  compareTo}(\texttt{Enum} {\bf  arg0})
}%end signature
}%end item
\item{\vskip -1.5ex 
\texttt{public final boolean {\bf  equals}(\texttt{Object} {\bf  arg0})
}%end signature
}%end item
\item{\vskip -1.5ex 
\texttt{protected final void {\bf  finalize}()
}%end signature
}%end item
\item{\vskip -1.5ex 
\texttt{public final Class {\bf  getDeclaringClass}()
}%end signature
}%end item
\item{\vskip -1.5ex 
\texttt{public final int {\bf  hashCode}()
}%end signature
}%end item
\item{\vskip -1.5ex 
\texttt{public final String {\bf  name}()
}%end signature
}%end item
\item{\vskip -1.5ex 
\texttt{public final int {\bf  ordinal}()
}%end signature
}%end item
\item{\vskip -1.5ex 
\texttt{public String {\bf  toString}()
}%end signature
}%end item
\item{\vskip -1.5ex 
\texttt{public static Enum {\bf  valueOf}(\texttt{Class} {\bf  arg0},
\texttt{String} {\bf  arg1})
}%end signature
}%end item
\end{itemize}
}
}
\section{\label{it.matlice.ingsw.model.data.TypeDefinition}\index{TypeDefinition}Class TypeDefinition}{
\vskip .1in 
rappresenta un tipo di dato, salvandone il tipo e se esso è obbligatorio\vskip .1in 
\subsection{Declaration}{
\begin{lstlisting}[frame=none]
public final class TypeDefinition
 extends java.lang.Object\end{lstlisting}
\subsection{Constructor summary}{
\begin{verse}
{\bf TypeDefinition(boolean)} \\
{\bf TypeDefinition(TypeDefinition.TypeAssociation, boolean)} \\
\end{verse}
}
\subsection{Method summary}{
\begin{verse}
{\bf equals(Object)} \\
{\bf hashCode()} \\
{\bf required()} \\
{\bf toString()} \\
{\bf type()} \\
\end{verse}
}
\subsection{Constructors}{
\vskip -2em
\begin{itemize}
\item{ 
\index{TypeDefinition(boolean)}
{\bf  TypeDefinition}\\
\begin{lstlisting}[frame=none]
public TypeDefinition(boolean required)\end{lstlisting} %end signature
}%end item
\item{ 
\index{TypeDefinition(TypeDefinition.TypeAssociation, boolean)}
{\bf  TypeDefinition}\\
\begin{lstlisting}[frame=none]
public TypeDefinition(TypeDefinition.TypeAssociation type,boolean required)\end{lstlisting} %end signature
}%end item
\end{itemize}
}
\subsection{Methods}{
\vskip -2em
\begin{itemize}
\item{ 
\index{equals(Object)}
{\bf  equals}\\
\begin{lstlisting}[frame=none]
public boolean equals(java.lang.Object arg0)\end{lstlisting} %end signature
}%end item
\item{ 
\index{hashCode()}
{\bf  hashCode}\\
\begin{lstlisting}[frame=none]
public native int hashCode()\end{lstlisting} %end signature
}%end item
\item{ 
\index{required()}
{\bf  required}\\
\begin{lstlisting}[frame=none]
public boolean required()\end{lstlisting} %end signature
}%end item
\item{ 
\index{toString()}
{\bf  toString}\\
\begin{lstlisting}[frame=none]
public java.lang.String toString()\end{lstlisting} %end signature
}%end item
\item{ 
\index{type()}
{\bf  type}\\
\begin{lstlisting}[frame=none]
public TypeDefinition.TypeAssociation type()\end{lstlisting} %end signature
}%end item
\end{itemize}
}
}
\section{\label{it.matlice.ingsw.model.data.TypeDefinition.TypeAssociation}\index{TypeDefinition.TypeAssociation}Class TypeDefinition.TypeAssociation}{
\vskip .1in 
Tipi di campo disponibili, nel caso specifico si possono aggiungere solo stringhe\vskip .1in 
\subsection{Declaration}{
\begin{lstlisting}[frame=none]
public static final class TypeDefinition.TypeAssociation
 extends java.lang.Enum\end{lstlisting}
\subsection{Field summary}{
\begin{verse}
{\bf STRING} \\
\end{verse}
}
\subsection{Method summary}{
\begin{verse}
{\bf deserialize(String)} \\
{\bf getType()} \\
{\bf serialize(Object)} \\
{\bf valueOf(String)} \\
{\bf values()} \\
\end{verse}
}
\subsection{Fields}{
\begin{itemize}
\item{
\index{STRING}
\label{it.matlice.ingsw.model.data.TypeDefinition.TypeAssociation.STRING}\texttt{public static final TypeDefinition.TypeAssociation\ {\bf  STRING}}
}
\end{itemize}
}
\subsection{Methods}{
\vskip -2em
\begin{itemize}
\item{ 
\index{deserialize(String)}
{\bf  deserialize}\\
\begin{lstlisting}[frame=none]
public java.lang.Object deserialize(java.lang.String o)\end{lstlisting} %end signature
}%end item
\item{ 
\index{getType()}
{\bf  getType}\\
\begin{lstlisting}[frame=none]
public java.lang.Class getType()\end{lstlisting} %end signature
}%end item
\item{ 
\index{serialize(Object)}
{\bf  serialize}\\
\begin{lstlisting}[frame=none]
public java.lang.String serialize(java.lang.Object o)\end{lstlisting} %end signature
}%end item
\item{ 
\index{valueOf(String)}
{\bf  valueOf}\\
\begin{lstlisting}[frame=none]
public static TypeDefinition.TypeAssociation valueOf(java.lang.String name)\end{lstlisting} %end signature
}%end item
\item{ 
\index{values()}
{\bf  values}\\
\begin{lstlisting}[frame=none]
public static TypeDefinition.TypeAssociation[] values()\end{lstlisting} %end signature
}%end item
\end{itemize}
}
\subsection{Members inherited from class Enum }{
\texttt{java.lang.Enum} {\small 
\refdefined{java.lang.Enum}}
{\small 

\vskip -2em
\begin{itemize}
\item{\vskip -1.5ex 
\texttt{protected final Object {\bf  clone}() throws CloneNotSupportedException
}%end signature
}%end item
\item{\vskip -1.5ex 
\texttt{public final int {\bf  compareTo}(\texttt{Enum} {\bf  arg0})
}%end signature
}%end item
\item{\vskip -1.5ex 
\texttt{public final boolean {\bf  equals}(\texttt{Object} {\bf  arg0})
}%end signature
}%end item
\item{\vskip -1.5ex 
\texttt{protected final void {\bf  finalize}()
}%end signature
}%end item
\item{\vskip -1.5ex 
\texttt{public final Class {\bf  getDeclaringClass}()
}%end signature
}%end item
\item{\vskip -1.5ex 
\texttt{public final int {\bf  hashCode}()
}%end signature
}%end item
\item{\vskip -1.5ex 
\texttt{public final String {\bf  name}()
}%end signature
}%end item
\item{\vskip -1.5ex 
\texttt{public final int {\bf  ordinal}()
}%end signature
}%end item
\item{\vskip -1.5ex 
\texttt{public String {\bf  toString}()
}%end signature
}%end item
\item{\vskip -1.5ex 
\texttt{public static Enum {\bf  valueOf}(\texttt{Class} {\bf  arg0},
\texttt{String} {\bf  arg1})
}%end signature
}%end item
\end{itemize}
}
}
\section{\label{it.matlice.ingsw.model.data.User}\index{User}Class User}{
\vskip .1in 
rappresenta un utente del sistema\vskip .1in 
\subsection{Declaration}{
\begin{lstlisting}[frame=none]
public abstract class User
 extends java.lang.Object implements it.matlice.ingsw.model.auth.Authenticable\end{lstlisting}
\subsection{All known subclasses}{CustomerUser\small{\refdefined{it.matlice.ingsw.model.data.CustomerUser}}, ConfiguratorUser\small{\refdefined{it.matlice.ingsw.model.data.ConfiguratorUser}}, CustomerUserImpl\small{\refdefined{it.matlice.ingsw.model.data.impl.jdbc.types.CustomerUserImpl}}, ConfiguratorUserImpl\small{\refdefined{it.matlice.ingsw.model.data.impl.jdbc.types.ConfiguratorUserImpl}}}
\subsection{Constructor summary}{
\begin{verse}
{\bf User()} \\
\end{verse}
}
\subsection{Method summary}{
\begin{verse}
{\bf getAuthMethods()} \\
{\bf getLastLoginTime()} \\
{\bf getUsername()} \\
{\bf setLastLoginTime(long)} \\
\end{verse}
}
\subsection{Constructors}{
\vskip -2em
\begin{itemize}
\item{ 
\index{User()}
{\bf  User}\\
\begin{lstlisting}[frame=none]
public User()\end{lstlisting} %end signature
}%end item
\end{itemize}
}
\subsection{Methods}{
\vskip -2em
\begin{itemize}
\item{ 
\index{getAuthMethods()}
{\bf  getAuthMethods}\\
\begin{lstlisting}[frame=none]
java.util.List getAuthMethods()\end{lstlisting} %end signature
\begin{itemize}
\item{
{\bf  Description copied from it.matlice.ingsw.model.auth.Authenticable{\small \refdefined{it.matlice.ingsw.model.auth.Authenticable}} }

Fornisce una lista di metodi di autenticazione in grado di autenticare la classe.
}
\item{{\bf  Returns} -- 
una lista di istanze di metodi di autenticazione. 
}%end item
\end{itemize}
}%end item
\item{ 
\index{getLastLoginTime()}
{\bf  getLastLoginTime}\\
\begin{lstlisting}[frame=none]
public abstract java.lang.Long getLastLoginTime()\end{lstlisting} %end signature
}%end item
\item{ 
\index{getUsername()}
{\bf  getUsername}\\
\begin{lstlisting}[frame=none]
public abstract java.lang.String getUsername()\end{lstlisting} %end signature
}%end item
\item{ 
\index{setLastLoginTime(long)}
{\bf  setLastLoginTime}\\
\begin{lstlisting}[frame=none]
public abstract void setLastLoginTime(long time)\end{lstlisting} %end signature
}%end item
\end{itemize}
}
}
\section{\label{it.matlice.ingsw.model.data.User.UserTypes}\index{User.UserTypes}Class User.UserTypes}{
\vskip .1in 
\subsection{Declaration}{
\begin{lstlisting}[frame=none]
public static final class User.UserTypes
 extends java.lang.Enum\end{lstlisting}
\subsection{Field summary}{
\begin{verse}
{\bf CONFIGURATOR} \\
{\bf CUSTOMER} \\
\end{verse}
}
\subsection{Method summary}{
\begin{verse}
{\bf getTypeRepresentation()} \\
{\bf valueOf(String)} \\
{\bf values()} \\
\end{verse}
}
\subsection{Fields}{
\begin{itemize}
\item{
\index{CONFIGURATOR}
\label{it.matlice.ingsw.model.data.User.UserTypes.CONFIGURATOR}\texttt{public static final User.UserTypes\ {\bf  CONFIGURATOR}}
}
\item{
\index{CUSTOMER}
\label{it.matlice.ingsw.model.data.User.UserTypes.CUSTOMER}\texttt{public static final User.UserTypes\ {\bf  CUSTOMER}}
}
\end{itemize}
}
\subsection{Methods}{
\vskip -2em
\begin{itemize}
\item{ 
\index{getTypeRepresentation()}
{\bf  getTypeRepresentation}\\
\begin{lstlisting}[frame=none]
public java.lang.String getTypeRepresentation()\end{lstlisting} %end signature
}%end item
\item{ 
\index{valueOf(String)}
{\bf  valueOf}\\
\begin{lstlisting}[frame=none]
public static User.UserTypes valueOf(java.lang.String name)\end{lstlisting} %end signature
}%end item
\item{ 
\index{values()}
{\bf  values}\\
\begin{lstlisting}[frame=none]
public static User.UserTypes[] values()\end{lstlisting} %end signature
}%end item
\end{itemize}
}
\subsection{Members inherited from class Enum }{
\texttt{java.lang.Enum} {\small 
\refdefined{java.lang.Enum}}
{\small 

\vskip -2em
\begin{itemize}
\item{\vskip -1.5ex 
\texttt{protected final Object {\bf  clone}() throws CloneNotSupportedException
}%end signature
}%end item
\item{\vskip -1.5ex 
\texttt{public final int {\bf  compareTo}(\texttt{Enum} {\bf  arg0})
}%end signature
}%end item
\item{\vskip -1.5ex 
\texttt{public final boolean {\bf  equals}(\texttt{Object} {\bf  arg0})
}%end signature
}%end item
\item{\vskip -1.5ex 
\texttt{protected final void {\bf  finalize}()
}%end signature
}%end item
\item{\vskip -1.5ex 
\texttt{public final Class {\bf  getDeclaringClass}()
}%end signature
}%end item
\item{\vskip -1.5ex 
\texttt{public final int {\bf  hashCode}()
}%end signature
}%end item
\item{\vskip -1.5ex 
\texttt{public final String {\bf  name}()
}%end signature
}%end item
\item{\vskip -1.5ex 
\texttt{public final int {\bf  ordinal}()
}%end signature
}%end item
\item{\vskip -1.5ex 
\texttt{public String {\bf  toString}()
}%end signature
}%end item
\item{\vskip -1.5ex 
\texttt{public static Enum {\bf  valueOf}(\texttt{Class} {\bf  arg0},
\texttt{String} {\bf  arg1})
}%end signature
}%end item
\end{itemize}
}
}
}
\chapter{Package it.matlice.ingsw.model.data.impl.jdbc}{
\label{it.matlice.ingsw.model.data.impl.jdbc}\hskip -.05in
\hbox to \hsize{\textit{ Package Contents\hfil Page}}
\vskip .13in
\hbox{{\bf  Classes}}
\entityintro{CategoryFactoryImpl}{it.matlice.ingsw.model.data.impl.jdbc.CategoryFactoryImpl}{Classe che si occuperà di istanziare implementazioni di categorie, correttamente identificate da NodeCategory o LeafCategory, complete di struttura di (eventuali) figli a partire da una base di dati mediante Jdbc}
\entityintro{HierarchyFactoryImpl}{it.matlice.ingsw.model.data.impl.jdbc.HierarchyFactoryImpl}{Classe che si occupa di istanziare elementi di tipo Hierarchy una volta caricati da una base di dati Jdbc}
\entityintro{JdbcConnection}{it.matlice.ingsw.model.data.impl.jdbc.JdbcConnection}{}
\entityintro{MessageFactoryImpl}{it.matlice.ingsw.model.data.impl.jdbc.MessageFactoryImpl}{}
\entityintro{OfferFactoryImpl}{it.matlice.ingsw.model.data.impl.jdbc.OfferFactoryImpl}{}
\entityintro{SettingsFactoryImpl}{it.matlice.ingsw.model.data.impl.jdbc.SettingsFactoryImpl}{}
\entityintro{UserFactoryImpl}{it.matlice.ingsw.model.data.impl.jdbc.UserFactoryImpl}{classe in grado di istanziare User nella giusta declinazione a partire da una base di dati Jdbc}
\entityintro{XMLImport}{it.matlice.ingsw.model.data.impl.jdbc.XMLImport}{}
\entityintro{XMLImport.CategoryXML}{it.matlice.ingsw.model.data.impl.jdbc.XMLImport.CategoryXML}{}
\entityintro{XMLImport.ConfigurationXML}{it.matlice.ingsw.model.data.impl.jdbc.XMLImport.ConfigurationXML}{}
\entityintro{XMLImport.FieldXML}{it.matlice.ingsw.model.data.impl.jdbc.XMLImport.FieldXML}{}
\entityintro{XMLImport.HierarchyXML}{it.matlice.ingsw.model.data.impl.jdbc.XMLImport.HierarchyXML}{}
\entityintro{XMLImport.SettingsXML}{it.matlice.ingsw.model.data.impl.jdbc.XMLImport.SettingsXML}{}
\vskip .1in
\vskip .1in
\section{\label{it.matlice.ingsw.model.data.impl.jdbc.CategoryFactoryImpl}\index{CategoryFactoryImpl}Class CategoryFactoryImpl}{
\vskip .1in 
Classe che si occuperà di istanziare implementazioni di categorie, correttamente identificate da NodeCategory o LeafCategory, complete di struttura di (eventuali) figli a partire da una base di dati mediante Jdbc\vskip .1in 
\subsection{Declaration}{
\begin{lstlisting}[frame=none]
public class CategoryFactoryImpl
 extends java.lang.Object implements it.matlice.ingsw.model.data.factories.CategoryFactory\end{lstlisting}
\subsection{Constructor summary}{
\begin{verse}
{\bf CategoryFactoryImpl()} \\
\end{verse}
}
\subsection{Method summary}{
\begin{verse}
{\bf createCategory(String, String, Category, boolean)} \\
{\bf getCategory(int)} \\
{\bf getCategoryDAO()} \\
{\bf saveCategory(Category)} \\
\end{verse}
}
\subsection{Constructors}{
\vskip -2em
\begin{itemize}
\item{ 
\index{CategoryFactoryImpl()}
{\bf  CategoryFactoryImpl}\\
\begin{lstlisting}[frame=none]
public CategoryFactoryImpl() throws java.sql.SQLException\end{lstlisting} %end signature
}%end item
\end{itemize}
}
\subsection{Methods}{
\vskip -2em
\begin{itemize}
\item{ 
\index{createCategory(String, String, Category, boolean)}
{\bf  createCategory}\\
\begin{lstlisting}[frame=none]
it.matlice.ingsw.model.data.Category createCategory(java.lang.String nome,java.lang.String description,it.matlice.ingsw.model.data.Category father,boolean isLeaf)\end{lstlisting} %end signature
\begin{itemize}
\item{
{\bf  Description copied from it.matlice.ingsw.model.data.factories.CategoryFactory{\small \refdefined{it.matlice.ingsw.model.data.factories.CategoryFactory}} }

crea e salva una nuova categoria
}
\item{
{\bf  Parameters}
  \begin{itemize}
   \item{
\texttt{nome} -- nome della categoria}
   \item{
\texttt{father} -- categoria padre (null se si vuole creare una root category}
   \item{
\texttt{isLeaf} -- indica se la categoria creata potrà avere figli o se è eldiana nel finale alternativo di aot}
  \end{itemize}
}%end item
\item{{\bf  Returns} -- 
la categoria creata 
}%end item
\end{itemize}
}%end item
\item{ 
\index{getCategory(int)}
{\bf  getCategory}\\
\begin{lstlisting}[frame=none]
it.matlice.ingsw.model.data.Category getCategory(int id) throws java.sql.SQLException\end{lstlisting} %end signature
\begin{itemize}
\item{
{\bf  Description copied from it.matlice.ingsw.model.data.factories.CategoryFactory{\small \refdefined{it.matlice.ingsw.model.data.factories.CategoryFactory}} }

Ottiene una categoria tramite id. notare che questo metodo non dovrebbe essere usato direttamente dal controller, ma deve essere utilizzato per ottenere la categoria a partire da una gerarchia. la gestione dell'id è lasciata all'implementazione e non deve incidere nello sviluppo del controller.
}
\item{
{\bf  Parameters}
  \begin{itemize}
   \item{
\texttt{id} -- numero incrementale identificativo univoco della categoria}
  \end{itemize}
}%end item
\item{{\bf  Returns} -- 
una categoria se esiste 
}%end item
\item{{\bf  Throws}
  \begin{itemize}
   \item{\vskip -.6ex \texttt{java.sql.SQLException} -- }
  \end{itemize}
}%end item
\end{itemize}
}%end item
\item{ 
\index{getCategoryDAO()}
{\bf  getCategoryDAO}\\
\begin{lstlisting}[frame=none]
public <any> getCategoryDAO()\end{lstlisting} %end signature
}%end item
\item{ 
\index{saveCategory(Category)}
{\bf  saveCategory}\\
\begin{lstlisting}[frame=none]
it.matlice.ingsw.model.data.Category saveCategory(it.matlice.ingsw.model.data.Category category) throws java.sql.SQLException\end{lstlisting} %end signature
\begin{itemize}
\item{
{\bf  Description copied from it.matlice.ingsw.model.data.factories.CategoryFactory{\small \refdefined{it.matlice.ingsw.model.data.factories.CategoryFactory}} }

salva la categoria nel database aggiornandola, inoltre salva i campi se non esistono. Si noti che non è necessario cancellare i campi rimossi dato che da specifica, le categorie sono immutabili. È necessario quindi solo aggiungere i nuovi campi al momento della creazione.
}
\item{
{\bf  Parameters}
  \begin{itemize}
   \item{
\texttt{category} -- categoria da salvare}
  \end{itemize}
}%end item
\item{{\bf  Returns} -- 
la categoria aggiornata 
}%end item
\item{{\bf  Throws}
  \begin{itemize}
   \item{\vskip -.6ex \texttt{java.sql.SQLException} -- }
  \end{itemize}
}%end item
\end{itemize}
}%end item
\end{itemize}
}
}
\section{\label{it.matlice.ingsw.model.data.impl.jdbc.HierarchyFactoryImpl}\index{HierarchyFactoryImpl}Class HierarchyFactoryImpl}{
\vskip .1in 
Classe che si occupa di istanziare elementi di tipo Hierarchy una volta caricati da una base di dati Jdbc\vskip .1in 
\subsection{Declaration}{
\begin{lstlisting}[frame=none]
public class HierarchyFactoryImpl
 extends java.lang.Object implements it.matlice.ingsw.model.data.factories.HierarchyFactory\end{lstlisting}
\subsection{Constructor summary}{
\begin{verse}
{\bf HierarchyFactoryImpl()} \\
\end{verse}
}
\subsection{Method summary}{
\begin{verse}
{\bf createHierarchy(Category)} \\
{\bf deleteHierarchy(Hierarchy)} \\
{\bf getHierarchies()} \\
\end{verse}
}
\subsection{Constructors}{
\vskip -2em
\begin{itemize}
\item{ 
\index{HierarchyFactoryImpl()}
{\bf  HierarchyFactoryImpl}\\
\begin{lstlisting}[frame=none]
public HierarchyFactoryImpl() throws java.sql.SQLException\end{lstlisting} %end signature
}%end item
\end{itemize}
}
\subsection{Methods}{
\vskip -2em
\begin{itemize}
\item{ 
\index{createHierarchy(Category)}
{\bf  createHierarchy}\\
\begin{lstlisting}[frame=none]
it.matlice.ingsw.model.data.Hierarchy createHierarchy(it.matlice.ingsw.model.data.Category rootCategory) throws java.sql.SQLException\end{lstlisting} %end signature
\begin{itemize}
\item{
{\bf  Description copied from it.matlice.ingsw.model.data.factories.HierarchyFactory{\small \refdefined{it.matlice.ingsw.model.data.factories.HierarchyFactory}} }

crea una nuova gerarchia e la salva sulla base di dati
}
\item{
{\bf  Parameters}
  \begin{itemize}
   \item{
\texttt{rootCategory} -- la categoria root}
  \end{itemize}
}%end item
\item{{\bf  Returns} -- 
la nuova gerarchia 
}%end item
\item{{\bf  Throws}
  \begin{itemize}
   \item{\vskip -.6ex \texttt{java.sql.SQLException} -- }
  \end{itemize}
}%end item
\end{itemize}
}%end item
\item{ 
\index{deleteHierarchy(Hierarchy)}
{\bf  deleteHierarchy}\\
\begin{lstlisting}[frame=none]
void deleteHierarchy(it.matlice.ingsw.model.data.Hierarchy h) throws java.sql.SQLException\end{lstlisting} %end signature
\begin{itemize}
\item{
{\bf  Description copied from it.matlice.ingsw.model.data.factories.HierarchyFactory{\small \refdefined{it.matlice.ingsw.model.data.factories.HierarchyFactory}} }

rimuove una gerarchia dalla bd
}
\item{
{\bf  Parameters}
  \begin{itemize}
   \item{
\texttt{h} -- gerarchia da rimuovere}
  \end{itemize}
}%end item
\item{{\bf  Throws}
  \begin{itemize}
   \item{\vskip -.6ex \texttt{java.sql.SQLException} -- }
  \end{itemize}
}%end item
\end{itemize}
}%end item
\item{ 
\index{getHierarchies()}
{\bf  getHierarchies}\\
\begin{lstlisting}[frame=none]
java.util.List getHierarchies() throws java.sql.SQLException\end{lstlisting} %end signature
\begin{itemize}
\item{{\bf  Returns} -- 
Ottiene la lista di getrarchie presenti nel programma 
}%end item
\item{{\bf  Throws}
  \begin{itemize}
   \item{\vskip -.6ex \texttt{java.sql.SQLException} -- }
  \end{itemize}
}%end item
\end{itemize}
}%end item
\end{itemize}
}
}
\section{\label{it.matlice.ingsw.model.data.impl.jdbc.JdbcConnection}\index{JdbcConnection}Class JdbcConnection}{
\vskip .1in 
\subsection{Declaration}{
\begin{lstlisting}[frame=none]
public class JdbcConnection
 extends java.lang.Object\end{lstlisting}
\subsection{Method summary}{
\begin{verse}
{\bf close()} \\
{\bf getConnectionSource()} \\
{\bf getInstance()} \\
{\bf startInstance(String)} \\
\end{verse}
}
\subsection{Methods}{
\vskip -2em
\begin{itemize}
\item{ 
\index{close()}
{\bf  close}\\
\begin{lstlisting}[frame=none]
public void close() throws java.lang.Exception\end{lstlisting} %end signature
}%end item
\item{ 
\index{getConnectionSource()}
{\bf  getConnectionSource}\\
\begin{lstlisting}[frame=none]
public ConnectionSource getConnectionSource()\end{lstlisting} %end signature
}%end item
\item{ 
\index{getInstance()}
{\bf  getInstance}\\
\begin{lstlisting}[frame=none]
public static JdbcConnection getInstance()\end{lstlisting} %end signature
}%end item
\item{ 
\index{startInstance(String)}
{\bf  startInstance}\\
\begin{lstlisting}[frame=none]
public static JdbcConnection startInstance(java.lang.String url) throws java.sql.SQLException\end{lstlisting} %end signature
}%end item
\end{itemize}
}
}
\section{\label{it.matlice.ingsw.model.data.impl.jdbc.MessageFactoryImpl}\index{MessageFactoryImpl}Class MessageFactoryImpl}{
\vskip .1in 
\subsection{Declaration}{
\begin{lstlisting}[frame=none]
public class MessageFactoryImpl
 extends java.lang.Object implements it.matlice.ingsw.model.data.factories.MessageFactory\end{lstlisting}
\subsection{Constructor summary}{
\begin{verse}
{\bf MessageFactoryImpl()} \\
\end{verse}
}
\subsection{Method summary}{
\begin{verse}
{\bf answer(Message, Offer, String, Calendar)} \\
{\bf getUserMessages(User)} \\
{\bf send(Offer, String, Calendar, long)} \\
\end{verse}
}
\subsection{Constructors}{
\vskip -2em
\begin{itemize}
\item{ 
\index{MessageFactoryImpl()}
{\bf  MessageFactoryImpl}\\
\begin{lstlisting}[frame=none]
public MessageFactoryImpl() throws java.sql.SQLException\end{lstlisting} %end signature
}%end item
\end{itemize}
}
\subsection{Methods}{
\vskip -2em
\begin{itemize}
\item{ 
\index{answer(Message, Offer, String, Calendar)}
{\bf  answer}\\
\begin{lstlisting}[frame=none]
it.matlice.ingsw.model.data.Message answer(it.matlice.ingsw.model.data.Message msg,it.matlice.ingsw.model.data.Offer offer,java.lang.String location,java.util.Calendar date) throws java.sql.SQLException\end{lstlisting} %end signature
\begin{itemize}
\item{
{\bf  Description copied from it.matlice.ingsw.model.data.factories.MessageFactory{\small \refdefined{it.matlice.ingsw.model.data.factories.MessageFactory}} }

Permette di aggiungere un nuovo messaggio in risposta ad una proposta di uno scambio
}
\item{
{\bf  Parameters}
  \begin{itemize}
   \item{
\texttt{msg} -- messaggio a cui rispondere}
   \item{
\texttt{offer} -- offerta a cui si riferisce il messaggio (del ricevitore)}
   \item{
\texttt{location} -- luogo proposto}
   \item{
\texttt{date} -- data proposta}
  \end{itemize}
}%end item
\item{{\bf  Returns} -- 
messaggio inviato 
}%end item
\item{{\bf  Throws}
  \begin{itemize}
   \item{\vskip -.6ex \texttt{java.sql.SQLException} -- errore di database durante la creazione del messaggio}
  \end{itemize}
}%end item
\end{itemize}
}%end item
\item{ 
\index{getUserMessages(User)}
{\bf  getUserMessages}\\
\begin{lstlisting}[frame=none]
java.util.List getUserMessages(it.matlice.ingsw.model.data.User u) throws java.sql.SQLException\end{lstlisting} %end signature
\begin{itemize}
\item{
{\bf  Description copied from it.matlice.ingsw.model.data.factories.MessageFactory{\small \refdefined{it.matlice.ingsw.model.data.factories.MessageFactory}} }

Ottiene tutti i messaggi a cui l'utente deve fornire una risposta
}
\item{
{\bf  Parameters}
  \begin{itemize}
   \item{
\texttt{u} -- utente ricevitore dei messaggi}
  \end{itemize}
}%end item
\item{{\bf  Returns} -- 
lista di messaggi per l'utente 
}%end item
\item{{\bf  Throws}
  \begin{itemize}
   \item{\vskip -.6ex \texttt{java.sql.SQLException} -- errore di database durante l'ottenimento dei messaggi}
  \end{itemize}
}%end item
\end{itemize}
}%end item
\item{ 
\index{send(Offer, String, Calendar, long)}
{\bf  send}\\
\begin{lstlisting}[frame=none]
it.matlice.ingsw.model.data.Message send(it.matlice.ingsw.model.data.Offer offer,java.lang.String location,java.util.Calendar date,long timestamp) throws java.sql.SQLException\end{lstlisting} %end signature
\begin{itemize}
\item{
{\bf  Description copied from it.matlice.ingsw.model.data.factories.MessageFactory{\small \refdefined{it.matlice.ingsw.model.data.factories.MessageFactory}} }

Permette di aggiungere un nuovo messaggio durante la fase di proposta di uno scambio
}
\item{
{\bf  Parameters}
  \begin{itemize}
   \item{
\texttt{offer} -- offerta a cui si riferisce il messaggio (del ricevitore)}
   \item{
\texttt{location} -- luogo proposto}
   \item{
\texttt{date} -- data proposta}
   \item{
\texttt{timestamp} -- momento di creazione del messaggio}
  \end{itemize}
}%end item
\item{{\bf  Returns} -- 
messaggio inviato 
}%end item
\item{{\bf  Throws}
  \begin{itemize}
   \item{\vskip -.6ex \texttt{java.sql.SQLException} -- errore di database durante la creazione del messaggio}
  \end{itemize}
}%end item
\end{itemize}
}%end item
\end{itemize}
}
}
\section{\label{it.matlice.ingsw.model.data.impl.jdbc.OfferFactoryImpl}\index{OfferFactoryImpl}Class OfferFactoryImpl}{
\vskip .1in 
\subsection{Declaration}{
\begin{lstlisting}[frame=none]
public class OfferFactoryImpl
 extends java.lang.Object implements it.matlice.ingsw.model.data.factories.OfferFactory\end{lstlisting}
\subsection{Constructor summary}{
\begin{verse}
{\bf OfferFactoryImpl(SettingsFactory)} \\
\end{verse}
}
\subsection{Method summary}{
\begin{verse}
{\bf acceptTradeOffer(Offer, MessageFactory, String, Calendar)} \\
{\bf checkForDueDate()} \\
{\bf closeTradeOffer(Message)} \\
{\bf createTradeOffer(Offer, Offer)} \\
{\bf getOffers(LeafCategory)} \\
{\bf getOffers(User)} \\
{\bf getSelectedOffers(User)} \\
{\bf makeOffer(String, User, LeafCategory, Map)} \\
{\bf setOfferStatus(Offer, Offer.OfferStatus)} \\
{\bf updateDate(Offer, Calendar)} \\
\end{verse}
}
\subsection{Constructors}{
\vskip -2em
\begin{itemize}
\item{ 
\index{OfferFactoryImpl(SettingsFactory)}
{\bf  OfferFactoryImpl}\\
\begin{lstlisting}[frame=none]
public OfferFactoryImpl(it.matlice.ingsw.model.data.factories.SettingsFactory sf) throws java.sql.SQLException\end{lstlisting} %end signature
}%end item
\end{itemize}
}
\subsection{Methods}{
\vskip -2em
\begin{itemize}
\item{ 
\index{acceptTradeOffer(Offer, MessageFactory, String, Calendar)}
{\bf  acceptTradeOffer}\\
\begin{lstlisting}[frame=none]
void acceptTradeOffer(it.matlice.ingsw.model.data.Offer offer,it.matlice.ingsw.model.data.factories.MessageFactory mf,java.lang.String location,java.util.Calendar date) throws java.sql.SQLException\end{lstlisting} %end signature
}%end item
\item{ 
\index{checkForDueDate()}
{\bf  checkForDueDate}\\
\begin{lstlisting}[frame=none]
void checkForDueDate() throws java.sql.SQLException\end{lstlisting} %end signature
}%end item
\item{ 
\index{closeTradeOffer(Message)}
{\bf  closeTradeOffer}\\
\begin{lstlisting}[frame=none]
void closeTradeOffer(it.matlice.ingsw.model.data.Message m) throws java.sql.SQLException\end{lstlisting} %end signature
}%end item
\item{ 
\index{createTradeOffer(Offer, Offer)}
{\bf  createTradeOffer}\\
\begin{lstlisting}[frame=none]
void createTradeOffer(it.matlice.ingsw.model.data.Offer offerToTrade,it.matlice.ingsw.model.data.Offer offerToAccept) throws java.sql.SQLException\end{lstlisting} %end signature
\begin{itemize}
\item{
{\bf  Description copied from it.matlice.ingsw.model.data.factories.OfferFactory{\small \refdefined{it.matlice.ingsw.model.data.factories.OfferFactory}} }

Permette di accoppiare due offerte, la prima è dell'utente che propone lo scambio
}
\item{
{\bf  Parameters}
  \begin{itemize}
   \item{
\texttt{offerToTrade} -- offerta accoppiata}
   \item{
\texttt{offerToAccept} -- offerta selezionata}
  \end{itemize}
}%end item
\item{{\bf  Throws}
  \begin{itemize}
   \item{\vskip -.6ex \texttt{java.sql.SQLException} -- }
  \end{itemize}
}%end item
\end{itemize}
}%end item
\item{ 
\index{getOffers(LeafCategory)}
{\bf  getOffers}\\
\begin{lstlisting}[frame=none]
java.util.List getOffers(it.matlice.ingsw.model.data.LeafCategory cat) throws java.sql.SQLException\end{lstlisting} %end signature
\begin{itemize}
\item{
{\bf  Description copied from it.matlice.ingsw.model.data.factories.OfferFactory{\small \refdefined{it.matlice.ingsw.model.data.factories.OfferFactory}} }

Ottiene la lista di offerte associate a una categoria. Notare che l'istanza di categoria deve essere stata precedentemente creata e deve fare parte dell'albero delle categorie per far si che vengano rilevati tutti i campi delle categorie padre sopra di essa.
}
\item{
{\bf  Parameters}
  \begin{itemize}
   \item{
\texttt{cat} -- categoria}
  \end{itemize}
}%end item
\item{{\bf  Returns} -- 
offerte associate alla categoria 
}%end item
\item{{\bf  Throws}
  \begin{itemize}
   \item{\vskip -.6ex \texttt{java.sql.SQLException} -- .}
  \end{itemize}
}%end item
\end{itemize}
}%end item
\item{ 
\index{getOffers(User)}
{\bf  getOffers}\\
\begin{lstlisting}[frame=none]
java.util.List getOffers(it.matlice.ingsw.model.data.User owner) throws java.sql.SQLException\end{lstlisting} %end signature
\begin{itemize}
\item{
{\bf  Parameters}
  \begin{itemize}
   \item{
\texttt{owner} -- utente}
  \end{itemize}
}%end item
\item{{\bf  Returns} -- 
Ritorna tutte le offerte associate all'utente 
}%end item
\item{{\bf  Throws}
  \begin{itemize}
   \item{\vskip -.6ex \texttt{java.sql.SQLException} -- .}
  \end{itemize}
}%end item
\end{itemize}
}%end item
\item{ 
\index{getSelectedOffers(User)}
{\bf  getSelectedOffers}\\
\begin{lstlisting}[frame=none]
java.util.List getSelectedOffers(it.matlice.ingsw.model.data.User owner) throws java.sql.SQLException\end{lstlisting} %end signature
}%end item
\item{ 
\index{makeOffer(String, User, LeafCategory, Map)}
{\bf  makeOffer}\\
\begin{lstlisting}[frame=none]
it.matlice.ingsw.model.data.Offer makeOffer(java.lang.String name,it.matlice.ingsw.model.data.User owner,it.matlice.ingsw.model.data.LeafCategory category,java.util.Map field_values) throws it.matlice.ingsw.model.exceptions.RequiredFieldConstrainException, java.sql.SQLException\end{lstlisting} %end signature
\begin{itemize}
\item{
{\bf  Description copied from it.matlice.ingsw.model.data.factories.OfferFactory{\small \refdefined{it.matlice.ingsw.model.data.factories.OfferFactory}} }

Crea un'offerta per l'utente nella categoria. Notare che l'istanza di categoria deve essere stata precedentemente creata e deve fare parte dell'albero delle categorie per far si che vengano rilevati tutti i campi delle categorie padre sopra di essa.
}
\item{
{\bf  Parameters}
  \begin{itemize}
   \item{
\texttt{name} -- nome dell'articolo offerto}
   \item{
\texttt{owner} -- istanza dell'utente proprietario}
   \item{
\texttt{category} -- categoria dove posizionare l'offerta}
   \item{
\texttt{field\_values} -- mappa dei valori dei campi}
  \end{itemize}
}%end item
\item{{\bf  Returns} -- 
l'offerta istanziata 
}%end item
\item{{\bf  Throws}
  \begin{itemize}
   \item{\vskip -.6ex \texttt{it.matlice.ingsw.model.exceptions.RequiredFieldConstrainException} -- Se un campo obbligatorio non é stato compilato}
   \item{\vskip -.6ex \texttt{java.sql.SQLException} -- .}
  \end{itemize}
}%end item
\end{itemize}
}%end item
\item{ 
\index{setOfferStatus(Offer, Offer.OfferStatus)}
{\bf  setOfferStatus}\\
\begin{lstlisting}[frame=none]
void setOfferStatus(it.matlice.ingsw.model.data.Offer offer,it.matlice.ingsw.model.data.Offer.OfferStatus status) throws java.sql.SQLException\end{lstlisting} %end signature
\begin{itemize}
\item{
{\bf  Description copied from it.matlice.ingsw.model.data.factories.OfferFactory{\small \refdefined{it.matlice.ingsw.model.data.factories.OfferFactory}} }

Imposta lo stato di un'offerta
}
\item{
{\bf  Parameters}
  \begin{itemize}
   \item{
\texttt{offer} -- istanza dell'offerta da creare}
   \item{
\texttt{status} -- stato dell'offerta}
  \end{itemize}
}%end item
\item{{\bf  Throws}
  \begin{itemize}
   \item{\vskip -.6ex \texttt{java.sql.SQLException} -- .}
  \end{itemize}
}%end item
\end{itemize}
}%end item
\item{ 
\index{updateDate(Offer, Calendar)}
{\bf  updateDate}\\
\begin{lstlisting}[frame=none]
void updateDate(it.matlice.ingsw.model.data.Offer offer,java.util.Calendar date) throws java.sql.SQLException\end{lstlisting} %end signature
}%end item
\end{itemize}
}
}
\section{\label{it.matlice.ingsw.model.data.impl.jdbc.SettingsFactoryImpl}\index{SettingsFactoryImpl}Class SettingsFactoryImpl}{
\vskip .1in 
\subsection{Declaration}{
\begin{lstlisting}[frame=none]
public class SettingsFactoryImpl
 extends java.lang.Object implements it.matlice.ingsw.model.data.factories.SettingsFactory\end{lstlisting}
\subsection{Constructor summary}{
\begin{verse}
{\bf SettingsFactoryImpl()} \\
\end{verse}
}
\subsection{Method summary}{
\begin{verse}
{\bf addDay(Settings, Settings.Day)} \\
{\bf addInterval(Settings, Interval)} \\
{\bf addLocation(Settings, String)} \\
{\bf makeSettings(String, int, List, List, List)} \\
{\bf readSettings()} \\
{\bf removeDays(Settings)} \\
{\bf removeIntervals(Settings)} \\
{\bf removeLocations(Settings)} \\
{\bf setDue(Settings, int)} \\
\end{verse}
}
\subsection{Constructors}{
\vskip -2em
\begin{itemize}
\item{ 
\index{SettingsFactoryImpl()}
{\bf  SettingsFactoryImpl}\\
\begin{lstlisting}[frame=none]
public SettingsFactoryImpl() throws java.sql.SQLException\end{lstlisting} %end signature
}%end item
\end{itemize}
}
\subsection{Methods}{
\vskip -2em
\begin{itemize}
\item{ 
\index{addDay(Settings, Settings.Day)}
{\bf  addDay}\\
\begin{lstlisting}[frame=none]
void addDay(it.matlice.ingsw.model.data.Settings db,it.matlice.ingsw.model.data.Settings.Day d) throws java.sql.SQLException\end{lstlisting} %end signature
\begin{itemize}
\item{
{\bf  Description copied from it.matlice.ingsw.model.data.factories.SettingsFactory{\small \refdefined{it.matlice.ingsw.model.data.factories.SettingsFactory}} }

Permette di aggiungere un nuovo giorno per lo scambio
}
\item{
{\bf  Parameters}
  \begin{itemize}
   \item{
\texttt{db} -- istanza di settings}
   \item{
\texttt{d} -- giorno da aggiungere}
  \end{itemize}
}%end item
\item{{\bf  Throws}
  \begin{itemize}
   \item{\vskip -.6ex \texttt{java.sql.SQLException} -- errore di database}
  \end{itemize}
}%end item
\end{itemize}
}%end item
\item{ 
\index{addInterval(Settings, Interval)}
{\bf  addInterval}\\
\begin{lstlisting}[frame=none]
void addInterval(it.matlice.ingsw.model.data.Settings db,it.matlice.ingsw.model.data.Interval i) throws java.sql.SQLException\end{lstlisting} %end signature
\begin{itemize}
\item{
{\bf  Description copied from it.matlice.ingsw.model.data.factories.SettingsFactory{\small \refdefined{it.matlice.ingsw.model.data.factories.SettingsFactory}} }

Permette di aggiungere un nuovo intervallo orario per lo scambio
}
\item{
{\bf  Parameters}
  \begin{itemize}
   \item{
\texttt{db} -- istanza di settings}
   \item{
\texttt{i} -- intervallo da aggiungere}
  \end{itemize}
}%end item
\item{{\bf  Throws}
  \begin{itemize}
   \item{\vskip -.6ex \texttt{java.sql.SQLException} -- errore di database}
  \end{itemize}
}%end item
\end{itemize}
}%end item
\item{ 
\index{addLocation(Settings, String)}
{\bf  addLocation}\\
\begin{lstlisting}[frame=none]
void addLocation(it.matlice.ingsw.model.data.Settings db,java.lang.String l) throws java.sql.SQLException\end{lstlisting} %end signature
\begin{itemize}
\item{
{\bf  Description copied from it.matlice.ingsw.model.data.factories.SettingsFactory{\small \refdefined{it.matlice.ingsw.model.data.factories.SettingsFactory}} }

Permette di aggiungere un nuovo luogo per lo scambio
}
\item{
{\bf  Parameters}
  \begin{itemize}
   \item{
\texttt{db} -- istanza di settings}
   \item{
\texttt{l} -- luogo da aggiungere}
  \end{itemize}
}%end item
\item{{\bf  Throws}
  \begin{itemize}
   \item{\vskip -.6ex \texttt{java.sql.SQLException} -- errore di database}
  \end{itemize}
}%end item
\end{itemize}
}%end item
\item{ 
\index{makeSettings(String, int, List, List, List)}
{\bf  makeSettings}\\
\begin{lstlisting}[frame=none]
it.matlice.ingsw.model.data.Settings makeSettings(java.lang.String city,int due,java.util.List locations,java.util.List days,java.util.List intervals) throws java.sql.SQLException\end{lstlisting} %end signature
\begin{itemize}
\item{
{\bf  Description copied from it.matlice.ingsw.model.data.factories.SettingsFactory{\small \refdefined{it.matlice.ingsw.model.data.factories.SettingsFactory}} }

Crea una istanza di impostazioni
}
\item{
{\bf  Parameters}
  \begin{itemize}
   \item{
\texttt{city} -- Piazza di scambio}
   \item{
\texttt{due} -- numero di giorni di scadenza}
   \item{
\texttt{locations} -- luoghi di scambio}
   \item{
\texttt{days} -- Giorni di scambio}
   \item{
\texttt{intervals} -- intervalli di scambio}
  \end{itemize}
}%end item
\item{{\bf  Returns} -- 
l'istanza creata 
}%end item
\item{{\bf  Throws}
  \begin{itemize}
   \item{\vskip -.6ex \texttt{java.sql.SQLException} -- .}
  \end{itemize}
}%end item
\end{itemize}
}%end item
\item{ 
\index{readSettings()}
{\bf  readSettings}\\
\begin{lstlisting}[frame=none]
it.matlice.ingsw.model.data.Settings readSettings() throws java.sql.SQLException\end{lstlisting} %end signature
\begin{itemize}
\item{
{\bf  Description copied from it.matlice.ingsw.model.data.factories.SettingsFactory{\small \refdefined{it.matlice.ingsw.model.data.factories.SettingsFactory}} }

Legge le impostazioni e le ritorna
}
\item{{\bf  Returns} -- 
impostazioni ottenute dalla base di dati. Se fossero presenti piú impostazioni, il comportamento non é definito e ne verrá ritornata solo una. 
}%end item
\item{{\bf  Throws}
  \begin{itemize}
   \item{\vskip -.6ex \texttt{java.sql.SQLException} -- .}
  \end{itemize}
}%end item
\end{itemize}
}%end item
\item{ 
\index{removeDays(Settings)}
{\bf  removeDays}\\
\begin{lstlisting}[frame=none]
void removeDays(it.matlice.ingsw.model.data.Settings db) throws java.sql.SQLException\end{lstlisting} %end signature
\begin{itemize}
\item{
{\bf  Description copied from it.matlice.ingsw.model.data.factories.SettingsFactory{\small \refdefined{it.matlice.ingsw.model.data.factories.SettingsFactory}} }

Permette di rimuovere tutti i giorni per lo scambio
}
\item{
{\bf  Parameters}
  \begin{itemize}
   \item{
\texttt{db} -- istanza di settings}
  \end{itemize}
}%end item
\item{{\bf  Throws}
  \begin{itemize}
   \item{\vskip -.6ex \texttt{java.sql.SQLException} -- errore di database}
  \end{itemize}
}%end item
\end{itemize}
}%end item
\item{ 
\index{removeIntervals(Settings)}
{\bf  removeIntervals}\\
\begin{lstlisting}[frame=none]
void removeIntervals(it.matlice.ingsw.model.data.Settings db) throws java.sql.SQLException\end{lstlisting} %end signature
\begin{itemize}
\item{
{\bf  Description copied from it.matlice.ingsw.model.data.factories.SettingsFactory{\small \refdefined{it.matlice.ingsw.model.data.factories.SettingsFactory}} }

Permette di rimuovere tutti gli intervalli per lo scambio
}
\item{
{\bf  Parameters}
  \begin{itemize}
   \item{
\texttt{db} -- istanza di settings}
  \end{itemize}
}%end item
\item{{\bf  Throws}
  \begin{itemize}
   \item{\vskip -.6ex \texttt{java.sql.SQLException} -- errore di database}
  \end{itemize}
}%end item
\end{itemize}
}%end item
\item{ 
\index{removeLocations(Settings)}
{\bf  removeLocations}\\
\begin{lstlisting}[frame=none]
void removeLocations(it.matlice.ingsw.model.data.Settings db) throws java.sql.SQLException\end{lstlisting} %end signature
\begin{itemize}
\item{
{\bf  Description copied from it.matlice.ingsw.model.data.factories.SettingsFactory{\small \refdefined{it.matlice.ingsw.model.data.factories.SettingsFactory}} }

Permette di rimuovere tutti i luoghi per lo scambio
}
\item{
{\bf  Parameters}
  \begin{itemize}
   \item{
\texttt{db} -- istanza di settings}
  \end{itemize}
}%end item
\item{{\bf  Throws}
  \begin{itemize}
   \item{\vskip -.6ex \texttt{java.sql.SQLException} -- errore di database}
  \end{itemize}
}%end item
\end{itemize}
}%end item
\item{ 
\index{setDue(Settings, int)}
{\bf  setDue}\\
\begin{lstlisting}[frame=none]
void setDue(it.matlice.ingsw.model.data.Settings db,int due) throws java.sql.SQLException\end{lstlisting} %end signature
\begin{itemize}
\item{
{\bf  Description copied from it.matlice.ingsw.model.data.factories.SettingsFactory{\small \refdefined{it.matlice.ingsw.model.data.factories.SettingsFactory}} }

Permette di impostare i giorni di scadenza
}
\item{
{\bf  Parameters}
  \begin{itemize}
   \item{
\texttt{db} -- istanza di settings}
   \item{
\texttt{due} -- numero di giorni per la scadenza}
  \end{itemize}
}%end item
\item{{\bf  Throws}
  \begin{itemize}
   \item{\vskip -.6ex \texttt{java.sql.SQLException} -- errore di database}
  \end{itemize}
}%end item
\end{itemize}
}%end item
\end{itemize}
}
}
\section{\label{it.matlice.ingsw.model.data.impl.jdbc.UserFactoryImpl}\index{UserFactoryImpl}Class UserFactoryImpl}{
\vskip .1in 
classe in grado di istanziare User nella giusta declinazione a partire da una base di dati Jdbc\vskip .1in 
\subsection{Declaration}{
\begin{lstlisting}[frame=none]
public class UserFactoryImpl
 extends java.lang.Object implements it.matlice.ingsw.model.data.factories.UserFactory\end{lstlisting}
\subsection{Constructor summary}{
\begin{verse}
{\bf UserFactoryImpl()} \\
\end{verse}
}
\subsection{Method summary}{
\begin{verse}
{\bf createUser(String, User.UserTypes)} \\
{\bf doesUserExist(String)} \\
{\bf getUser(String)} \\
{\bf getUsers()} \\
{\bf saveUser(User)} \\
\end{verse}
}
\subsection{Constructors}{
\vskip -2em
\begin{itemize}
\item{ 
\index{UserFactoryImpl()}
{\bf  UserFactoryImpl}\\
\begin{lstlisting}[frame=none]
public UserFactoryImpl() throws java.sql.SQLException\end{lstlisting} %end signature
}%end item
\end{itemize}
}
\subsection{Methods}{
\vskip -2em
\begin{itemize}
\item{ 
\index{createUser(String, User.UserTypes)}
{\bf  createUser}\\
\begin{lstlisting}[frame=none]
it.matlice.ingsw.model.data.User createUser(java.lang.String username,it.matlice.ingsw.model.data.User.UserTypes userType) throws java.sql.SQLException, it.matlice.ingsw.model.exceptions.InvalidUserTypeException\end{lstlisting} %end signature
\begin{itemize}
\item{
{\bf  Description copied from it.matlice.ingsw.model.data.factories.UserFactory{\small \refdefined{it.matlice.ingsw.model.data.factories.UserFactory}} }

crea un utente e lo salva nella bs
}
\item{
{\bf  Parameters}
  \begin{itemize}
   \item{
\texttt{username} -- username del nuovo utente}
   \item{
\texttt{userType} -- tipo di utente}
  \end{itemize}
}%end item
\item{{\bf  Returns} -- 
l'utente creato 
}%end item
\item{{\bf  Throws}
  \begin{itemize}
   \item{\vskip -.6ex \texttt{java.sql.SQLException} -- }
   \item{\vskip -.6ex \texttt{it.matlice.ingsw.model.exceptions.InvalidUserTypeException} -- }
  \end{itemize}
}%end item
\end{itemize}
}%end item
\item{ 
\index{doesUserExist(String)}
{\bf  doesUserExist}\\
\begin{lstlisting}[frame=none]
boolean doesUserExist(java.lang.String username) throws java.sql.SQLException\end{lstlisting} %end signature
}%end item
\item{ 
\index{getUser(String)}
{\bf  getUser}\\
\begin{lstlisting}[frame=none]
it.matlice.ingsw.model.data.User getUser(java.lang.String username) throws java.sql.SQLException, it.matlice.ingsw.model.exceptions.InvalidUserException\end{lstlisting} %end signature
\begin{itemize}
\item{
{\bf  Parameters}
  \begin{itemize}
   \item{
\texttt{username} -- username dell'utente voluto (univoco)}
  \end{itemize}
}%end item
\item{{\bf  Returns} -- 
l'utente tratto dalla base di dati 
}%end item
\item{{\bf  Throws}
  \begin{itemize}
   \item{\vskip -.6ex \texttt{java.sql.SQLException} -- }
   \item{\vskip -.6ex \texttt{it.matlice.ingsw.model.exceptions.InvalidUserException} -- }
  \end{itemize}
}%end item
\end{itemize}
}%end item
\item{ 
\index{getUsers()}
{\bf  getUsers}\\
\begin{lstlisting}[frame=none]
java.util.List getUsers() throws java.sql.SQLException\end{lstlisting} %end signature
\begin{itemize}
\item{
{\bf  Description copied from it.matlice.ingsw.model.data.factories.UserFactory{\small \refdefined{it.matlice.ingsw.model.data.factories.UserFactory}} }

Ritorna la lista degli utente a database
}
\item{{\bf  Returns} -- 
lista di utenti 
}%end item
\item{{\bf  Throws}
  \begin{itemize}
   \item{\vskip -.6ex \texttt{java.sql.SQLException} -- errore di database}
  \end{itemize}
}%end item
\end{itemize}
}%end item
\item{ 
\index{saveUser(User)}
{\bf  saveUser}\\
\begin{lstlisting}[frame=none]
it.matlice.ingsw.model.data.User saveUser(it.matlice.ingsw.model.data.User u) throws java.sql.SQLException\end{lstlisting} %end signature
\begin{itemize}
\item{
{\bf  Description copied from it.matlice.ingsw.model.data.factories.UserFactory{\small \refdefined{it.matlice.ingsw.model.data.factories.UserFactory}} }

Permette di salvare un utente a database
}
\item{
{\bf  Parameters}
  \begin{itemize}
   \item{
\texttt{u} -- utente da salvare}
  \end{itemize}
}%end item
\item{{\bf  Returns} -- 
utente salvato 
}%end item
\item{{\bf  Throws}
  \begin{itemize}
   \item{\vskip -.6ex \texttt{java.sql.SQLException} -- errore di database}
  \end{itemize}
}%end item
\end{itemize}
}%end item
\end{itemize}
}
}
\section{\label{it.matlice.ingsw.model.data.impl.jdbc.XMLImport}\index{XMLImport}Class XMLImport}{
\vskip .1in 
\subsection{Declaration}{
\begin{lstlisting}[frame=none]
public class XMLImport
 extends java.lang.Object\end{lstlisting}
\subsection{Constructor summary}{
\begin{verse}
{\bf XMLImport(FileInputStream)} \\
\end{verse}
}
\subsection{Method summary}{
\begin{verse}
{\bf parse()} \\
\end{verse}
}
\subsection{Constructors}{
\vskip -2em
\begin{itemize}
\item{ 
\index{XMLImport(FileInputStream)}
{\bf  XMLImport}\\
\begin{lstlisting}[frame=none]
public XMLImport(java.io.FileInputStream file)\end{lstlisting} %end signature
}%end item
\end{itemize}
}
\subsection{Methods}{
\vskip -2em
\begin{itemize}
\item{ 
\index{parse()}
{\bf  parse}\\
\begin{lstlisting}[frame=none]
public XMLImport.ConfigurationXML parse() throws javax.xml.stream.XMLStreamException\end{lstlisting} %end signature
}%end item
\end{itemize}
}
}
\section{\label{it.matlice.ingsw.model.data.impl.jdbc.XMLImport.CategoryXML}\index{XMLImport.CategoryXML}Class XMLImport.CategoryXML}{
\vskip .1in 
\subsection{Declaration}{
\begin{lstlisting}[frame=none]
public static class XMLImport.CategoryXML
 extends java.lang.Object\end{lstlisting}
\subsection{Field summary}{
\begin{verse}
{\bf categories} \\
{\bf description} \\
{\bf fields} \\
{\bf name} \\
\end{verse}
}
\subsection{Constructor summary}{
\begin{verse}
{\bf CategoryXML(String, String, List, List)} \\
\end{verse}
}
\subsection{Fields}{
\begin{itemize}
\item{
\index{name}
\label{it.matlice.ingsw.model.data.impl.jdbc.XMLImport.CategoryXML.name}\texttt{public java.lang.String\ {\bf  name}}
}
\item{
\index{description}
\label{it.matlice.ingsw.model.data.impl.jdbc.XMLImport.CategoryXML.description}\texttt{public java.lang.String\ {\bf  description}}
}
\item{
\index{fields}
\label{it.matlice.ingsw.model.data.impl.jdbc.XMLImport.CategoryXML.fields}\texttt{public java.util.List\ {\bf  fields}}
}
\item{
\index{categories}
\label{it.matlice.ingsw.model.data.impl.jdbc.XMLImport.CategoryXML.categories}\texttt{public java.util.List\ {\bf  categories}}
}
\end{itemize}
}
\subsection{Constructors}{
\vskip -2em
\begin{itemize}
\item{ 
\index{CategoryXML(String, String, List, List)}
{\bf  CategoryXML}\\
\begin{lstlisting}[frame=none]
public CategoryXML(java.lang.String name,java.lang.String description,java.util.List fields,java.util.List categories)\end{lstlisting} %end signature
}%end item
\end{itemize}
}
}
\section{\label{it.matlice.ingsw.model.data.impl.jdbc.XMLImport.ConfigurationXML}\index{XMLImport.ConfigurationXML}Class XMLImport.ConfigurationXML}{
\vskip .1in 
\subsection{Declaration}{
\begin{lstlisting}[frame=none]
public static class XMLImport.ConfigurationXML
 extends java.lang.Object\end{lstlisting}
\subsection{Field summary}{
\begin{verse}
{\bf hierarchies} \\
{\bf settings} \\
\end{verse}
}
\subsection{Constructor summary}{
\begin{verse}
{\bf ConfigurationXML(XMLImport.SettingsXML, List)} \\
\end{verse}
}
\subsection{Fields}{
\begin{itemize}
\item{
\index{settings}
\label{it.matlice.ingsw.model.data.impl.jdbc.XMLImport.ConfigurationXML.settings}\texttt{public XMLImport.SettingsXML\ {\bf  settings}}
}
\item{
\index{hierarchies}
\label{it.matlice.ingsw.model.data.impl.jdbc.XMLImport.ConfigurationXML.hierarchies}\texttt{public java.util.List\ {\bf  hierarchies}}
}
\end{itemize}
}
\subsection{Constructors}{
\vskip -2em
\begin{itemize}
\item{ 
\index{ConfigurationXML(XMLImport.SettingsXML, List)}
{\bf  ConfigurationXML}\\
\begin{lstlisting}[frame=none]
public ConfigurationXML(XMLImport.SettingsXML settings,java.util.List hierarchies)\end{lstlisting} %end signature
}%end item
\end{itemize}
}
}
\section{\label{it.matlice.ingsw.model.data.impl.jdbc.XMLImport.FieldXML}\index{XMLImport.FieldXML}Class XMLImport.FieldXML}{
\vskip .1in 
\subsection{Declaration}{
\begin{lstlisting}[frame=none]
public static class XMLImport.FieldXML
 extends java.lang.Object\end{lstlisting}
\subsection{Field summary}{
\begin{verse}
{\bf name} \\
{\bf required} \\
{\bf type} \\
\end{verse}
}
\subsection{Constructor summary}{
\begin{verse}
{\bf FieldXML(String, boolean)} \\
{\bf FieldXML(String, boolean, TypeDefinition.TypeAssociation)} \\
\end{verse}
}
\subsection{Fields}{
\begin{itemize}
\item{
\index{name}
\label{it.matlice.ingsw.model.data.impl.jdbc.XMLImport.FieldXML.name}\texttt{public java.lang.String\ {\bf  name}}
}
\item{
\index{required}
\label{it.matlice.ingsw.model.data.impl.jdbc.XMLImport.FieldXML.required}\texttt{public boolean\ {\bf  required}}
}
\item{
\index{type}
\label{it.matlice.ingsw.model.data.impl.jdbc.XMLImport.FieldXML.type}\texttt{public it.matlice.ingsw.model.data.TypeDefinition.TypeAssociation\ {\bf  type}}
}
\end{itemize}
}
\subsection{Constructors}{
\vskip -2em
\begin{itemize}
\item{ 
\index{FieldXML(String, boolean)}
{\bf  FieldXML}\\
\begin{lstlisting}[frame=none]
public FieldXML(java.lang.String name,boolean required)\end{lstlisting} %end signature
}%end item
\item{ 
\index{FieldXML(String, boolean, TypeDefinition.TypeAssociation)}
{\bf  FieldXML}\\
\begin{lstlisting}[frame=none]
public FieldXML(java.lang.String name,boolean required,it.matlice.ingsw.model.data.TypeDefinition.TypeAssociation type)\end{lstlisting} %end signature
}%end item
\end{itemize}
}
}
\section{\label{it.matlice.ingsw.model.data.impl.jdbc.XMLImport.HierarchyXML}\index{XMLImport.HierarchyXML}Class XMLImport.HierarchyXML}{
\vskip .1in 
\subsection{Declaration}{
\begin{lstlisting}[frame=none]
public static class XMLImport.HierarchyXML
 extends java.lang.Object\end{lstlisting}
\subsection{Field summary}{
\begin{verse}
{\bf root} \\
\end{verse}
}
\subsection{Constructor summary}{
\begin{verse}
{\bf HierarchyXML(XMLImport.CategoryXML)} \\
\end{verse}
}
\subsection{Fields}{
\begin{itemize}
\item{
\index{root}
\label{it.matlice.ingsw.model.data.impl.jdbc.XMLImport.HierarchyXML.root}\texttt{public XMLImport.CategoryXML\ {\bf  root}}
}
\end{itemize}
}
\subsection{Constructors}{
\vskip -2em
\begin{itemize}
\item{ 
\index{HierarchyXML(XMLImport.CategoryXML)}
{\bf  HierarchyXML}\\
\begin{lstlisting}[frame=none]
public HierarchyXML(XMLImport.CategoryXML root)\end{lstlisting} %end signature
}%end item
\end{itemize}
}
}
\section{\label{it.matlice.ingsw.model.data.impl.jdbc.XMLImport.SettingsXML}\index{XMLImport.SettingsXML}Class XMLImport.SettingsXML}{
\vskip .1in 
\subsection{Declaration}{
\begin{lstlisting}[frame=none]
public static class XMLImport.SettingsXML
 extends java.lang.Object\end{lstlisting}
\subsection{Field summary}{
\begin{verse}
{\bf city} \\
{\bf days} \\
{\bf expiration} \\
{\bf intervals} \\
{\bf locations} \\
\end{verse}
}
\subsection{Constructor summary}{
\begin{verse}
{\bf SettingsXML(String, List, List, List, int)} \\
\end{verse}
}
\subsection{Fields}{
\begin{itemize}
\item{
\index{city}
\label{it.matlice.ingsw.model.data.impl.jdbc.XMLImport.SettingsXML.city}\texttt{public java.lang.String\ {\bf  city}}
}
\item{
\index{locations}
\label{it.matlice.ingsw.model.data.impl.jdbc.XMLImport.SettingsXML.locations}\texttt{public java.util.List\ {\bf  locations}}
}
\item{
\index{days}
\label{it.matlice.ingsw.model.data.impl.jdbc.XMLImport.SettingsXML.days}\texttt{public java.util.List\ {\bf  days}}
}
\item{
\index{intervals}
\label{it.matlice.ingsw.model.data.impl.jdbc.XMLImport.SettingsXML.intervals}\texttt{public java.util.List\ {\bf  intervals}}
}
\item{
\index{expiration}
\label{it.matlice.ingsw.model.data.impl.jdbc.XMLImport.SettingsXML.expiration}\texttt{public int\ {\bf  expiration}}
}
\end{itemize}
}
\subsection{Constructors}{
\vskip -2em
\begin{itemize}
\item{ 
\index{SettingsXML(String, List, List, List, int)}
{\bf  SettingsXML}\\
\begin{lstlisting}[frame=none]
public SettingsXML(java.lang.String city,java.util.List locations,java.util.List days,java.util.List intervals,int expiration)\end{lstlisting} %end signature
}%end item
\end{itemize}
}
}
}
\chapter{Package it.matlice.ingsw.model.data.impl.jdbc.types}{
\label{it.matlice.ingsw.model.data.impl.jdbc.types}\hskip -.05in
\hbox to \hsize{\textit{ Package Contents\hfil Page}}
\vskip .13in
\hbox{{\bf  Interfaces}}
\entityintro{CategoryImpl}{it.matlice.ingsw.model.data.impl.jdbc.types.CategoryImpl}{}
\entityintro{UserImpl}{it.matlice.ingsw.model.data.impl.jdbc.types.UserImpl}{}
\vskip .13in
\hbox{{\bf  Classes}}
\entityintro{ConfiguratorUserImpl}{it.matlice.ingsw.model.data.impl.jdbc.types.ConfiguratorUserImpl}{}
\entityintro{CustomerUserImpl}{it.matlice.ingsw.model.data.impl.jdbc.types.CustomerUserImpl}{}
\entityintro{HierarchyImpl}{it.matlice.ingsw.model.data.impl.jdbc.types.HierarchyImpl}{}
\entityintro{LeafCategoryImpl}{it.matlice.ingsw.model.data.impl.jdbc.types.LeafCategoryImpl}{}
\entityintro{MessageImpl}{it.matlice.ingsw.model.data.impl.jdbc.types.MessageImpl}{}
\entityintro{NodeCategoryImpl}{it.matlice.ingsw.model.data.impl.jdbc.types.NodeCategoryImpl}{}
\entityintro{OfferImpl}{it.matlice.ingsw.model.data.impl.jdbc.types.OfferImpl}{}
\entityintro{SettingsImpl}{it.matlice.ingsw.model.data.impl.jdbc.types.SettingsImpl}{}
\vskip .1in
\vskip .1in
\section{\label{it.matlice.ingsw.model.data.impl.jdbc.types.CategoryImpl}\index{CategoryImpl@\textit{ CategoryImpl}}Interface CategoryImpl}{
\vskip .1in 
\subsection{Declaration}{
\begin{lstlisting}[frame=none]
public interface CategoryImpl
\end{lstlisting}
\subsection{All known subinterfaces}{NodeCategoryImpl\small{\refdefined{it.matlice.ingsw.model.data.impl.jdbc.types.NodeCategoryImpl}}, LeafCategoryImpl\small{\refdefined{it.matlice.ingsw.model.data.impl.jdbc.types.LeafCategoryImpl}}}
\subsection{All classes known to implement interface}{NodeCategoryImpl\small{\refdefined{it.matlice.ingsw.model.data.impl.jdbc.types.NodeCategoryImpl}}, LeafCategoryImpl\small{\refdefined{it.matlice.ingsw.model.data.impl.jdbc.types.LeafCategoryImpl}}}
\subsection{Method summary}{
\begin{verse}
{\bf getDbData()} \\
\end{verse}
}
\subsection{Methods}{
\vskip -2em
\begin{itemize}
\item{ 
\index{getDbData()}
{\bf  getDbData}\\
\begin{lstlisting}[frame=none]
it.matlice.ingsw.model.data.impl.jdbc.db.CategoryDB getDbData()\end{lstlisting} %end signature
}%end item
\end{itemize}
}
}
\section{\label{it.matlice.ingsw.model.data.impl.jdbc.types.UserImpl}\index{UserImpl@\textit{ UserImpl}}Interface UserImpl}{
\vskip .1in 
\subsection{Declaration}{
\begin{lstlisting}[frame=none]
public interface UserImpl
\end{lstlisting}
\subsection{All known subinterfaces}{CustomerUserImpl\small{\refdefined{it.matlice.ingsw.model.data.impl.jdbc.types.CustomerUserImpl}}, ConfiguratorUserImpl\small{\refdefined{it.matlice.ingsw.model.data.impl.jdbc.types.ConfiguratorUserImpl}}}
\subsection{All classes known to implement interface}{CustomerUserImpl\small{\refdefined{it.matlice.ingsw.model.data.impl.jdbc.types.CustomerUserImpl}}, ConfiguratorUserImpl\small{\refdefined{it.matlice.ingsw.model.data.impl.jdbc.types.ConfiguratorUserImpl}}}
\subsection{Method summary}{
\begin{verse}
{\bf getDbData()} \\
\end{verse}
}
\subsection{Methods}{
\vskip -2em
\begin{itemize}
\item{ 
\index{getDbData()}
{\bf  getDbData}\\
\begin{lstlisting}[frame=none]
it.matlice.ingsw.model.data.impl.jdbc.db.UserDB getDbData()\end{lstlisting} %end signature
}%end item
\end{itemize}
}
}
\section{\label{it.matlice.ingsw.model.data.impl.jdbc.types.ConfiguratorUserImpl}\index{ConfiguratorUserImpl}Class ConfiguratorUserImpl}{
\vskip .1in 
\subsection{Declaration}{
\begin{lstlisting}[frame=none]
public class ConfiguratorUserImpl
 extends it.matlice.ingsw.model.data.ConfiguratorUser implements it.matlice.ingsw.model.auth.password.PasswordAuthenticable, UserImpl\end{lstlisting}
\subsection{Constructor summary}{
\begin{verse}
{\bf ConfiguratorUserImpl(String)} \\
{\bf ConfiguratorUserImpl(UserDB)} \\
\end{verse}
}
\subsection{Method summary}{
\begin{verse}
{\bf equals(Object)} \\
{\bf getAuthMethods()} \\
{\bf getDbData()} \\
{\bf getLastLoginTime()} \\
{\bf getPasswordHash()} \\
{\bf getPasswordSalt()} \\
{\bf getUsername()} \\
{\bf setLastLoginTime(long)} \\
{\bf setPassword(byte\lbrack \rbrack )} \\
{\bf setSalt(byte\lbrack \rbrack )} \\
\end{verse}
}
\subsection{Constructors}{
\vskip -2em
\begin{itemize}
\item{ 
\index{ConfiguratorUserImpl(String)}
{\bf  ConfiguratorUserImpl}\\
\begin{lstlisting}[frame=none]
public ConfiguratorUserImpl(java.lang.String username)\end{lstlisting} %end signature
}%end item
\item{ 
\index{ConfiguratorUserImpl(UserDB)}
{\bf  ConfiguratorUserImpl}\\
\begin{lstlisting}[frame=none]
public ConfiguratorUserImpl(it.matlice.ingsw.model.data.impl.jdbc.db.UserDB from)\end{lstlisting} %end signature
}%end item
\end{itemize}
}
\subsection{Methods}{
\vskip -2em
\begin{itemize}
\item{ 
\index{equals(Object)}
{\bf  equals}\\
\begin{lstlisting}[frame=none]
public boolean equals(java.lang.Object arg0)\end{lstlisting} %end signature
}%end item
\item{ 
\index{getAuthMethods()}
{\bf  getAuthMethods}\\
\begin{lstlisting}[frame=none]
java.util.List getAuthMethods()\end{lstlisting} %end signature
\begin{itemize}
\item{
{\bf  Description copied from it.matlice.ingsw.model.auth.Authenticable{\small \refdefined{it.matlice.ingsw.model.auth.Authenticable}} }

Fornisce una lista di metodi di autenticazione in grado di autenticare la classe.
}
\item{{\bf  Returns} -- 
una lista di istanze di metodi di autenticazione. 
}%end item
\end{itemize}
}%end item
\item{ 
\index{getDbData()}
{\bf  getDbData}\\
\begin{lstlisting}[frame=none]
it.matlice.ingsw.model.data.impl.jdbc.db.UserDB getDbData()\end{lstlisting} %end signature
}%end item
\item{ 
\index{getLastLoginTime()}
{\bf  getLastLoginTime}\\
\begin{lstlisting}[frame=none]
public abstract java.lang.Long getLastLoginTime()\end{lstlisting} %end signature
}%end item
\item{ 
\index{getPasswordHash()}
{\bf  getPasswordHash}\\
\begin{lstlisting}[frame=none]
byte[] getPasswordHash()\end{lstlisting} %end signature
\begin{itemize}
\item{
{\bf  Description copied from it.matlice.ingsw.model.auth.password.PasswordAuthenticable{\small \refdefined{it.matlice.ingsw.model.auth.password.PasswordAuthenticable}} }

Il metodo ritorna l'hash della password salvato precedentemente. l'hash è computato secondo la seguente espressione: \textbackslash \lbrack\ passwordHash = base64enc(hmac\_\{sha256\}(key=password, data=salt)) \textbackslash \rbrack 
}
\item{{\bf  Returns} -- 
l'hash della password salvato in precedenza 
}%end item
\end{itemize}
}%end item
\item{ 
\index{getPasswordSalt()}
{\bf  getPasswordSalt}\\
\begin{lstlisting}[frame=none]
byte[] getPasswordSalt()\end{lstlisting} %end signature
\begin{itemize}
\item{{\bf  Returns} -- 
ritorna il salt, generato casualmente, utilizzato per "salare la password" 
}%end item
\end{itemize}
}%end item
\item{ 
\index{getUsername()}
{\bf  getUsername}\\
\begin{lstlisting}[frame=none]
public abstract java.lang.String getUsername()\end{lstlisting} %end signature
}%end item
\item{ 
\index{setLastLoginTime(long)}
{\bf  setLastLoginTime}\\
\begin{lstlisting}[frame=none]
public abstract void setLastLoginTime(long time)\end{lstlisting} %end signature
}%end item
\item{ 
\index{setPassword(byte\lbrack \rbrack )}
{\bf  setPassword}\\
\begin{lstlisting}[frame=none]
void setPassword(byte[] password) throws it.matlice.ingsw.model.auth.exceptions.InvalidPasswordException\end{lstlisting} %end signature
\begin{itemize}
\item{
{\bf  Description copied from it.matlice.ingsw.model.auth.password.PasswordAuthenticable{\small \refdefined{it.matlice.ingsw.model.auth.password.PasswordAuthenticable}} }

Aggiorna la password associata all'utente \_\_rigenerando\_\_\ il salt
}
\item{
{\bf  Parameters}
  \begin{itemize}
   \item{
\texttt{password} -- nuova password}
  \end{itemize}
}%end item
\end{itemize}
}%end item
\item{ 
\index{setSalt(byte\lbrack \rbrack )}
{\bf  setSalt}\\
\begin{lstlisting}[frame=none]
void setSalt(byte[] salt) throws it.matlice.ingsw.model.auth.exceptions.InvalidPasswordException\end{lstlisting} %end signature
}%end item
\end{itemize}
}
\subsection{Members inherited from class User }{
\texttt{it.matlice.ingsw.model.data.User} {\small 
\refdefined{it.matlice.ingsw.model.data.User}}
{\small 

\vskip -2em
\begin{itemize}
\item{\vskip -1.5ex 
\texttt{public abstract List {\bf  getAuthMethods}()
}%end signature
}%end item
\item{\vskip -1.5ex 
\texttt{public abstract Long {\bf  getLastLoginTime}()
}%end signature
}%end item
\item{\vskip -1.5ex 
\texttt{public abstract String {\bf  getUsername}()
}%end signature
}%end item
\item{\vskip -1.5ex 
\texttt{public abstract void {\bf  setLastLoginTime}(\texttt{long} {\bf  time})
}%end signature
}%end item
\end{itemize}
}
}
\section{\label{it.matlice.ingsw.model.data.impl.jdbc.types.CustomerUserImpl}\index{CustomerUserImpl}Class CustomerUserImpl}{
\vskip .1in 
\subsection{Declaration}{
\begin{lstlisting}[frame=none]
public class CustomerUserImpl
 extends it.matlice.ingsw.model.data.CustomerUser implements it.matlice.ingsw.model.auth.password.PasswordAuthenticable, UserImpl\end{lstlisting}
\subsection{Constructor summary}{
\begin{verse}
{\bf CustomerUserImpl(String)} \\
{\bf CustomerUserImpl(UserDB)} \\
\end{verse}
}
\subsection{Method summary}{
\begin{verse}
{\bf equals(Object)} \\
{\bf getAuthMethods()} \\
{\bf getDbData()} \\
{\bf getLastLoginTime()} \\
{\bf getPasswordHash()} \\
{\bf getPasswordSalt()} \\
{\bf getUsername()} \\
{\bf setLastLoginTime(long)} \\
{\bf setPassword(byte\lbrack \rbrack )} \\
{\bf setSalt(byte\lbrack \rbrack )} \\
\end{verse}
}
\subsection{Constructors}{
\vskip -2em
\begin{itemize}
\item{ 
\index{CustomerUserImpl(String)}
{\bf  CustomerUserImpl}\\
\begin{lstlisting}[frame=none]
public CustomerUserImpl(java.lang.String username)\end{lstlisting} %end signature
}%end item
\item{ 
\index{CustomerUserImpl(UserDB)}
{\bf  CustomerUserImpl}\\
\begin{lstlisting}[frame=none]
public CustomerUserImpl(it.matlice.ingsw.model.data.impl.jdbc.db.UserDB from)\end{lstlisting} %end signature
}%end item
\end{itemize}
}
\subsection{Methods}{
\vskip -2em
\begin{itemize}
\item{ 
\index{equals(Object)}
{\bf  equals}\\
\begin{lstlisting}[frame=none]
public boolean equals(java.lang.Object arg0)\end{lstlisting} %end signature
}%end item
\item{ 
\index{getAuthMethods()}
{\bf  getAuthMethods}\\
\begin{lstlisting}[frame=none]
java.util.List getAuthMethods()\end{lstlisting} %end signature
\begin{itemize}
\item{
{\bf  Description copied from it.matlice.ingsw.model.auth.Authenticable{\small \refdefined{it.matlice.ingsw.model.auth.Authenticable}} }

Fornisce una lista di metodi di autenticazione in grado di autenticare la classe.
}
\item{{\bf  Returns} -- 
una lista di istanze di metodi di autenticazione. 
}%end item
\end{itemize}
}%end item
\item{ 
\index{getDbData()}
{\bf  getDbData}\\
\begin{lstlisting}[frame=none]
it.matlice.ingsw.model.data.impl.jdbc.db.UserDB getDbData()\end{lstlisting} %end signature
}%end item
\item{ 
\index{getLastLoginTime()}
{\bf  getLastLoginTime}\\
\begin{lstlisting}[frame=none]
public abstract java.lang.Long getLastLoginTime()\end{lstlisting} %end signature
}%end item
\item{ 
\index{getPasswordHash()}
{\bf  getPasswordHash}\\
\begin{lstlisting}[frame=none]
byte[] getPasswordHash()\end{lstlisting} %end signature
\begin{itemize}
\item{
{\bf  Description copied from it.matlice.ingsw.model.auth.password.PasswordAuthenticable{\small \refdefined{it.matlice.ingsw.model.auth.password.PasswordAuthenticable}} }

Il metodo ritorna l'hash della password salvato precedentemente. l'hash è computato secondo la seguente espressione: \textbackslash \lbrack\ passwordHash = base64enc(hmac\_\{sha256\}(key=password, data=salt)) \textbackslash \rbrack 
}
\item{{\bf  Returns} -- 
l'hash della password salvato in precedenza 
}%end item
\end{itemize}
}%end item
\item{ 
\index{getPasswordSalt()}
{\bf  getPasswordSalt}\\
\begin{lstlisting}[frame=none]
byte[] getPasswordSalt()\end{lstlisting} %end signature
\begin{itemize}
\item{{\bf  Returns} -- 
ritorna il salt, generato casualmente, utilizzato per "salare la password" 
}%end item
\end{itemize}
}%end item
\item{ 
\index{getUsername()}
{\bf  getUsername}\\
\begin{lstlisting}[frame=none]
public abstract java.lang.String getUsername()\end{lstlisting} %end signature
}%end item
\item{ 
\index{setLastLoginTime(long)}
{\bf  setLastLoginTime}\\
\begin{lstlisting}[frame=none]
public abstract void setLastLoginTime(long time)\end{lstlisting} %end signature
}%end item
\item{ 
\index{setPassword(byte\lbrack \rbrack )}
{\bf  setPassword}\\
\begin{lstlisting}[frame=none]
void setPassword(byte[] password) throws it.matlice.ingsw.model.auth.exceptions.InvalidPasswordException\end{lstlisting} %end signature
\begin{itemize}
\item{
{\bf  Description copied from it.matlice.ingsw.model.auth.password.PasswordAuthenticable{\small \refdefined{it.matlice.ingsw.model.auth.password.PasswordAuthenticable}} }

Aggiorna la password associata all'utente \_\_rigenerando\_\_\ il salt
}
\item{
{\bf  Parameters}
  \begin{itemize}
   \item{
\texttt{password} -- nuova password}
  \end{itemize}
}%end item
\end{itemize}
}%end item
\item{ 
\index{setSalt(byte\lbrack \rbrack )}
{\bf  setSalt}\\
\begin{lstlisting}[frame=none]
void setSalt(byte[] salt) throws it.matlice.ingsw.model.auth.exceptions.InvalidPasswordException\end{lstlisting} %end signature
}%end item
\end{itemize}
}
\subsection{Members inherited from class User }{
\texttt{it.matlice.ingsw.model.data.User} {\small 
\refdefined{it.matlice.ingsw.model.data.User}}
{\small 

\vskip -2em
\begin{itemize}
\item{\vskip -1.5ex 
\texttt{public abstract List {\bf  getAuthMethods}()
}%end signature
}%end item
\item{\vskip -1.5ex 
\texttt{public abstract Long {\bf  getLastLoginTime}()
}%end signature
}%end item
\item{\vskip -1.5ex 
\texttt{public abstract String {\bf  getUsername}()
}%end signature
}%end item
\item{\vskip -1.5ex 
\texttt{public abstract void {\bf  setLastLoginTime}(\texttt{long} {\bf  time})
}%end signature
}%end item
\end{itemize}
}
}
\section{\label{it.matlice.ingsw.model.data.impl.jdbc.types.HierarchyImpl}\index{HierarchyImpl}Class HierarchyImpl}{
\vskip .1in 
\subsection{Declaration}{
\begin{lstlisting}[frame=none]
public class HierarchyImpl
 extends it.matlice.ingsw.model.data.Hierarchy\end{lstlisting}
\subsection{Constructor summary}{
\begin{verse}
{\bf HierarchyImpl(HierarchyDB, Category)} \\
\end{verse}
}
\subsection{Method summary}{
\begin{verse}
{\bf getDbData()} \\
\end{verse}
}
\subsection{Constructors}{
\vskip -2em
\begin{itemize}
\item{ 
\index{HierarchyImpl(HierarchyDB, Category)}
{\bf  HierarchyImpl}\\
\begin{lstlisting}[frame=none]
public HierarchyImpl(it.matlice.ingsw.model.data.impl.jdbc.db.HierarchyDB from,it.matlice.ingsw.model.data.Category root_category)\end{lstlisting} %end signature
}%end item
\end{itemize}
}
\subsection{Methods}{
\vskip -2em
\begin{itemize}
\item{ 
\index{getDbData()}
{\bf  getDbData}\\
\begin{lstlisting}[frame=none]
public it.matlice.ingsw.model.data.impl.jdbc.db.HierarchyDB getDbData()\end{lstlisting} %end signature
}%end item
\end{itemize}
}
\subsection{Members inherited from class Hierarchy }{
\texttt{it.matlice.ingsw.model.data.Hierarchy} {\small 
\refdefined{it.matlice.ingsw.model.data.Hierarchy}}
{\small 

\vskip -2em
\begin{itemize}
\item{\vskip -1.5ex 
\texttt{public Category {\bf  getRootCategory}()
}%end signature
}%end item
\end{itemize}
}
}
\section{\label{it.matlice.ingsw.model.data.impl.jdbc.types.LeafCategoryImpl}\index{LeafCategoryImpl}Class LeafCategoryImpl}{
\vskip .1in 
\subsection{Declaration}{
\begin{lstlisting}[frame=none]
public class LeafCategoryImpl
 extends it.matlice.ingsw.model.data.LeafCategory implements CategoryImpl\end{lstlisting}
\subsection{Constructor summary}{
\begin{verse}
{\bf LeafCategoryImpl(CategoryDB)} \\
\end{verse}
}
\subsection{Method summary}{
\begin{verse}
{\bf convertToNode()} \\
{\bf equals(Object)} \\
{\bf getChildLeafs()} \\
{\bf getDbData()} \\
{\bf getDescription()} \\
{\bf getName()} \\
{\bf isValidChildCategoryName(String)} \\
\end{verse}
}
\subsection{Constructors}{
\vskip -2em
\begin{itemize}
\item{ 
\index{LeafCategoryImpl(CategoryDB)}
{\bf  LeafCategoryImpl}\\
\begin{lstlisting}[frame=none]
public LeafCategoryImpl(it.matlice.ingsw.model.data.impl.jdbc.db.CategoryDB from)\end{lstlisting} %end signature
}%end item
\end{itemize}
}
\subsection{Methods}{
\vskip -2em
\begin{itemize}
\item{ 
\index{convertToNode()}
{\bf  convertToNode}\\
\begin{lstlisting}[frame=none]
public abstract it.matlice.ingsw.model.data.NodeCategory convertToNode()\end{lstlisting} %end signature
}%end item
\item{ 
\index{equals(Object)}
{\bf  equals}\\
\begin{lstlisting}[frame=none]
public boolean equals(java.lang.Object arg0)\end{lstlisting} %end signature
}%end item
\item{ 
\index{getChildLeafs()}
{\bf  getChildLeafs}\\
\begin{lstlisting}[frame=none]
public abstract java.util.List getChildLeafs()\end{lstlisting} %end signature
\begin{itemize}
\item{{\bf  Returns} -- 
tutte le categorie foglia discendenti (o se stessa se è foglia) 
}%end item
\end{itemize}
}%end item
\item{ 
\index{getDbData()}
{\bf  getDbData}\\
\begin{lstlisting}[frame=none]
it.matlice.ingsw.model.data.impl.jdbc.db.CategoryDB getDbData()\end{lstlisting} %end signature
}%end item
\item{ 
\index{getDescription()}
{\bf  getDescription}\\
\begin{lstlisting}[frame=none]
public abstract java.lang.String getDescription()\end{lstlisting} %end signature
\begin{itemize}
\item{{\bf  Returns} -- 
ritorna la descrizione della categoria 
}%end item
\end{itemize}
}%end item
\item{ 
\index{getName()}
{\bf  getName}\\
\begin{lstlisting}[frame=none]
public abstract java.lang.String getName()\end{lstlisting} %end signature
\begin{itemize}
\item{{\bf  Returns} -- 
ritorna il nome della categoria 
}%end item
\end{itemize}
}%end item
\item{ 
\index{isValidChildCategoryName(String)}
{\bf  isValidChildCategoryName}\\
\begin{lstlisting}[frame=none]
public abstract boolean isValidChildCategoryName(java.lang.String name)\end{lstlisting} %end signature
}%end item
\end{itemize}
}
\subsection{Members inherited from class LeafCategory }{
\texttt{it.matlice.ingsw.model.data.LeafCategory} {\small 
\refdefined{it.matlice.ingsw.model.data.LeafCategory}}
{\small 

\vskip -2em
\begin{itemize}
\item{\vskip -1.5ex 
\texttt{public abstract NodeCategory {\bf  convertToNode}()
}%end signature
}%end item
\end{itemize}
}
\subsection{Members inherited from class Category }{
\texttt{it.matlice.ingsw.model.data.Category} {\small 
\refdefined{it.matlice.ingsw.model.data.Category}}
{\small 

\vskip -2em
\begin{itemize}
\item{\vskip -1.5ex 
\texttt{public void {\bf  clear}()
}%end signature
}%end item
\item{\vskip -1.5ex 
\texttt{public boolean {\bf  containsKey}(\texttt{java.lang.Object} {\bf  key})
}%end signature
}%end item
\item{\vskip -1.5ex 
\texttt{public boolean {\bf  containsValue}(\texttt{java.lang.Object} {\bf  value})
}%end signature
}%end item
\item{\vskip -1.5ex 
\texttt{public Set {\bf  fullEntrySet}()
}%end signature
}%end item
\item{\vskip -1.5ex 
\texttt{public String {\bf  fullToString}()
}%end signature
}%end item
\item{\vskip -1.5ex 
\texttt{public TypeDefinition {\bf  get}(\texttt{java.lang.Object} {\bf  key})
}%end signature
}%end item
\item{\vskip -1.5ex 
\texttt{public abstract List {\bf  getChildLeafs}()
}%end signature
}%end item
\item{\vskip -1.5ex 
\texttt{public abstract String {\bf  getDescription}()
}%end signature
}%end item
\item{\vskip -1.5ex 
\texttt{public NodeCategory {\bf  getFather}()
}%end signature
}%end item
\item{\vskip -1.5ex 
\texttt{public abstract String {\bf  getName}()
}%end signature
}%end item
\item{\vskip -1.5ex 
\texttt{public boolean {\bf  isCategoryValid}()
}%end signature
}%end item
\item{\vskip -1.5ex 
\texttt{public boolean {\bf  isEmpty}()
}%end signature
}%end item
\item{\vskip -1.5ex 
\texttt{public boolean {\bf  isRequired}(\texttt{java.lang.String} {\bf  name})
}%end signature
}%end item
\item{\vskip -1.5ex 
\texttt{public boolean {\bf  isRoot}()
}%end signature
}%end item
\item{\vskip -1.5ex 
\texttt{public abstract boolean {\bf  isValidChildCategoryName}(\texttt{java.lang.String} {\bf  name})
}%end signature
}%end item
\item{\vskip -1.5ex 
\texttt{public TypeDefinition {\bf  put}(\texttt{java.lang.String} {\bf  key},
\texttt{TypeDefinition} {\bf  value})
}%end signature
}%end item
\item{\vskip -1.5ex 
\texttt{public TypeDefinition {\bf  remove}(\texttt{java.lang.Object} {\bf  key})
}%end signature
}%end item
\item{\vskip -1.5ex 
\texttt{public void {\bf  setFather}(\texttt{NodeCategory} {\bf  father})
}%end signature
}%end item
\item{\vskip -1.5ex 
\texttt{public int {\bf  size}()
}%end signature
}%end item
\item{\vskip -1.5ex 
\texttt{public String {\bf  toString}()
}%end signature
}%end item
\end{itemize}
}
\subsection{Members inherited from class HashMap }{
\texttt{java.util.HashMap} {\small 
\refdefined{java.util.HashMap}}
{\small 

\vskip -2em
\begin{itemize}
\item{\vskip -1.5ex 
\texttt{public void {\bf  clear}()
}%end signature
}%end item
\item{\vskip -1.5ex 
\texttt{public Object {\bf  clone}()
}%end signature
}%end item
\item{\vskip -1.5ex 
\texttt{public Object {\bf  compute}(\texttt{java.lang.Object} {\bf  arg0},
\texttt{function.BiFunction} {\bf  arg1})
}%end signature
}%end item
\item{\vskip -1.5ex 
\texttt{public Object {\bf  computeIfAbsent}(\texttt{java.lang.Object} {\bf  arg0},
\texttt{function.Function} {\bf  arg1})
}%end signature
}%end item
\item{\vskip -1.5ex 
\texttt{public Object {\bf  computeIfPresent}(\texttt{java.lang.Object} {\bf  arg0},
\texttt{function.BiFunction} {\bf  arg1})
}%end signature
}%end item
\item{\vskip -1.5ex 
\texttt{public boolean {\bf  containsKey}(\texttt{java.lang.Object} {\bf  arg0})
}%end signature
}%end item
\item{\vskip -1.5ex 
\texttt{public boolean {\bf  containsValue}(\texttt{java.lang.Object} {\bf  arg0})
}%end signature
}%end item
\item{\vskip -1.5ex 
\texttt{public Set {\bf  entrySet}()
}%end signature
}%end item
\item{\vskip -1.5ex 
\texttt{public void {\bf  forEach}(\texttt{function.BiConsumer} {\bf  arg0})
}%end signature
}%end item
\item{\vskip -1.5ex 
\texttt{public Object {\bf  get}(\texttt{java.lang.Object} {\bf  arg0})
}%end signature
}%end item
\item{\vskip -1.5ex 
\texttt{public Object {\bf  getOrDefault}(\texttt{java.lang.Object} {\bf  arg0},
\texttt{java.lang.Object} {\bf  arg1})
}%end signature
}%end item
\item{\vskip -1.5ex 
\texttt{public boolean {\bf  isEmpty}()
}%end signature
}%end item
\item{\vskip -1.5ex 
\texttt{public Set {\bf  keySet}()
}%end signature
}%end item
\item{\vskip -1.5ex 
\texttt{public Object {\bf  merge}(\texttt{java.lang.Object} {\bf  arg0},
\texttt{java.lang.Object} {\bf  arg1},
\texttt{function.BiFunction} {\bf  arg2})
}%end signature
}%end item
\item{\vskip -1.5ex 
\texttt{public Object {\bf  put}(\texttt{java.lang.Object} {\bf  arg0},
\texttt{java.lang.Object} {\bf  arg1})
}%end signature
}%end item
\item{\vskip -1.5ex 
\texttt{public void {\bf  putAll}(\texttt{Map} {\bf  arg0})
}%end signature
}%end item
\item{\vskip -1.5ex 
\texttt{public Object {\bf  putIfAbsent}(\texttt{java.lang.Object} {\bf  arg0},
\texttt{java.lang.Object} {\bf  arg1})
}%end signature
}%end item
\item{\vskip -1.5ex 
\texttt{public Object {\bf  remove}(\texttt{java.lang.Object} {\bf  arg0})
}%end signature
}%end item
\item{\vskip -1.5ex 
\texttt{public boolean {\bf  remove}(\texttt{java.lang.Object} {\bf  arg0},
\texttt{java.lang.Object} {\bf  arg1})
}%end signature
}%end item
\item{\vskip -1.5ex 
\texttt{public Object {\bf  replace}(\texttt{java.lang.Object} {\bf  arg0},
\texttt{java.lang.Object} {\bf  arg1})
}%end signature
}%end item
\item{\vskip -1.5ex 
\texttt{public boolean {\bf  replace}(\texttt{java.lang.Object} {\bf  arg0},
\texttt{java.lang.Object} {\bf  arg1},
\texttt{java.lang.Object} {\bf  arg2})
}%end signature
}%end item
\item{\vskip -1.5ex 
\texttt{public void {\bf  replaceAll}(\texttt{function.BiFunction} {\bf  arg0})
}%end signature
}%end item
\item{\vskip -1.5ex 
\texttt{public int {\bf  size}()
}%end signature
}%end item
\item{\vskip -1.5ex 
\texttt{public Collection {\bf  values}()
}%end signature
}%end item
\end{itemize}
}
\subsection{Members inherited from class AbstractMap }{
\texttt{java.util.AbstractMap} {\small 
\refdefined{java.util.AbstractMap}}
{\small 

\vskip -2em
\begin{itemize}
\item{\vskip -1.5ex 
\texttt{public void {\bf  clear}()
}%end signature
}%end item
\item{\vskip -1.5ex 
\texttt{protected Object {\bf  clone}() throws java.lang.CloneNotSupportedException
}%end signature
}%end item
\item{\vskip -1.5ex 
\texttt{public boolean {\bf  containsKey}(\texttt{java.lang.Object} {\bf  arg0})
}%end signature
}%end item
\item{\vskip -1.5ex 
\texttt{public boolean {\bf  containsValue}(\texttt{java.lang.Object} {\bf  arg0})
}%end signature
}%end item
\item{\vskip -1.5ex 
\texttt{public abstract Set {\bf  entrySet}()
}%end signature
}%end item
\item{\vskip -1.5ex 
\texttt{public boolean {\bf  equals}(\texttt{java.lang.Object} {\bf  arg0})
}%end signature
}%end item
\item{\vskip -1.5ex 
\texttt{public Object {\bf  get}(\texttt{java.lang.Object} {\bf  arg0})
}%end signature
}%end item
\item{\vskip -1.5ex 
\texttt{public int {\bf  hashCode}()
}%end signature
}%end item
\item{\vskip -1.5ex 
\texttt{public boolean {\bf  isEmpty}()
}%end signature
}%end item
\item{\vskip -1.5ex 
\texttt{public Set {\bf  keySet}()
}%end signature
}%end item
\item{\vskip -1.5ex 
\texttt{public Object {\bf  put}(\texttt{java.lang.Object} {\bf  arg0},
\texttt{java.lang.Object} {\bf  arg1})
}%end signature
}%end item
\item{\vskip -1.5ex 
\texttt{public void {\bf  putAll}(\texttt{Map} {\bf  arg0})
}%end signature
}%end item
\item{\vskip -1.5ex 
\texttt{public Object {\bf  remove}(\texttt{java.lang.Object} {\bf  arg0})
}%end signature
}%end item
\item{\vskip -1.5ex 
\texttt{public int {\bf  size}()
}%end signature
}%end item
\item{\vskip -1.5ex 
\texttt{public String {\bf  toString}()
}%end signature
}%end item
\item{\vskip -1.5ex 
\texttt{public Collection {\bf  values}()
}%end signature
}%end item
\end{itemize}
}
}
\section{\label{it.matlice.ingsw.model.data.impl.jdbc.types.MessageImpl}\index{MessageImpl}Class MessageImpl}{
\vskip .1in 
\subsection{Declaration}{
\begin{lstlisting}[frame=none]
public class MessageImpl
 extends it.matlice.ingsw.model.data.Message\end{lstlisting}
\subsection{Constructor summary}{
\begin{verse}
{\bf MessageImpl(MessageDB, Offer)} \\
\end{verse}
}
\subsection{Method summary}{
\begin{verse}
{\bf getDate()} \\
{\bf getDbData()} \\
{\bf getLocation()} \\
{\bf getReferencedOffer()} \\
{\bf getTime()} \\
\end{verse}
}
\subsection{Constructors}{
\vskip -2em
\begin{itemize}
\item{ 
\index{MessageImpl(MessageDB, Offer)}
{\bf  MessageImpl}\\
\begin{lstlisting}[frame=none]
public MessageImpl(it.matlice.ingsw.model.data.impl.jdbc.db.MessageDB dbData,it.matlice.ingsw.model.data.Offer referenced_offer)\end{lstlisting} %end signature
}%end item
\end{itemize}
}
\subsection{Methods}{
\vskip -2em
\begin{itemize}
\item{ 
\index{getDate()}
{\bf  getDate}\\
\begin{lstlisting}[frame=none]
public abstract java.util.Calendar getDate()\end{lstlisting} %end signature
}%end item
\item{ 
\index{getDbData()}
{\bf  getDbData}\\
\begin{lstlisting}[frame=none]
public it.matlice.ingsw.model.data.impl.jdbc.db.MessageDB getDbData()\end{lstlisting} %end signature
}%end item
\item{ 
\index{getLocation()}
{\bf  getLocation}\\
\begin{lstlisting}[frame=none]
public abstract java.lang.String getLocation()\end{lstlisting} %end signature
}%end item
\item{ 
\index{getReferencedOffer()}
{\bf  getReferencedOffer}\\
\begin{lstlisting}[frame=none]
public abstract it.matlice.ingsw.model.data.Offer getReferencedOffer()\end{lstlisting} %end signature
}%end item
\item{ 
\index{getTime()}
{\bf  getTime}\\
\begin{lstlisting}[frame=none]
public abstract java.lang.Long getTime()\end{lstlisting} %end signature
}%end item
\end{itemize}
}
\subsection{Members inherited from class Message }{
\texttt{it.matlice.ingsw.model.data.Message} {\small 
\refdefined{it.matlice.ingsw.model.data.Message}}
{\small 

\vskip -2em
\begin{itemize}
\item{\vskip -1.5ex 
\texttt{public abstract Calendar {\bf  getDate}()
}%end signature
}%end item
\item{\vskip -1.5ex 
\texttt{public abstract String {\bf  getLocation}()
}%end signature
}%end item
\item{\vskip -1.5ex 
\texttt{public abstract Offer {\bf  getReferencedOffer}()
}%end signature
}%end item
\item{\vskip -1.5ex 
\texttt{public abstract Long {\bf  getTime}()
}%end signature
}%end item
\item{\vskip -1.5ex 
\texttt{public String {\bf  toString}()
}%end signature
}%end item
\end{itemize}
}
}
\section{\label{it.matlice.ingsw.model.data.impl.jdbc.types.NodeCategoryImpl}\index{NodeCategoryImpl}Class NodeCategoryImpl}{
\vskip .1in 
\subsection{Declaration}{
\begin{lstlisting}[frame=none]
public class NodeCategoryImpl
 extends it.matlice.ingsw.model.data.NodeCategory implements CategoryImpl\end{lstlisting}
\subsection{Constructor summary}{
\begin{verse}
{\bf NodeCategoryImpl(CategoryDB)} \\
\end{verse}
}
\subsection{Method summary}{
\begin{verse}
{\bf addChild(Category)} \\
{\bf clone()} Returns a shallow copy of this \texttt{\small HashMap} instance: the keys and values themselves are not cloned.\\
{\bf equals(Object)} \\
{\bf getChildLeafs()} \\
{\bf getDbData()} \\
{\bf getDescription()} \\
{\bf getName()} \\
{\bf isValidChildCategoryName(String)} \\
{\bf removeChild(Category)} \\
\end{verse}
}
\subsection{Constructors}{
\vskip -2em
\begin{itemize}
\item{ 
\index{NodeCategoryImpl(CategoryDB)}
{\bf  NodeCategoryImpl}\\
\begin{lstlisting}[frame=none]
public NodeCategoryImpl(it.matlice.ingsw.model.data.impl.jdbc.db.CategoryDB from)\end{lstlisting} %end signature
}%end item
\end{itemize}
}
\subsection{Methods}{
\vskip -2em
\begin{itemize}
\item{ 
\index{addChild(Category)}
{\bf  addChild}\\
\begin{lstlisting}[frame=none]
public it.matlice.ingsw.model.data.Category addChild(it.matlice.ingsw.model.data.Category child)\end{lstlisting} %end signature
\begin{itemize}
\item{
{\bf  Description copied from it.matlice.ingsw.model.data.NodeCategory{\small \refdefined{it.matlice.ingsw.model.data.NodeCategory}} }

aggiunge una categoria tra i propri figli.
}
\item{
{\bf  Parameters}
  \begin{itemize}
   \item{
\texttt{child} -- categoria figlio}
  \end{itemize}
}%end item
\item{{\bf  Returns} -- 
la categoria aggiunta 
}%end item
\end{itemize}
}%end item
\item{ 
\index{clone()}
{\bf  clone}\\
\begin{lstlisting}[frame=none]
public java.lang.Object clone()\end{lstlisting} %end signature
\begin{itemize}
\item{
{\bf  Description}

Returns a shallow copy of this \texttt{\small HashMap} instance: the keys and values themselves are not cloned.
}
\item{{\bf  Returns} -- 
a shallow copy of this map 
}%end item
\end{itemize}
}%end item
\item{ 
\index{equals(Object)}
{\bf  equals}\\
\begin{lstlisting}[frame=none]
public boolean equals(java.lang.Object arg0)\end{lstlisting} %end signature
}%end item
\item{ 
\index{getChildLeafs()}
{\bf  getChildLeafs}\\
\begin{lstlisting}[frame=none]
public abstract java.util.List getChildLeafs()\end{lstlisting} %end signature
\begin{itemize}
\item{{\bf  Returns} -- 
tutte le categorie foglia discendenti (o se stessa se è foglia) 
}%end item
\end{itemize}
}%end item
\item{ 
\index{getDbData()}
{\bf  getDbData}\\
\begin{lstlisting}[frame=none]
it.matlice.ingsw.model.data.impl.jdbc.db.CategoryDB getDbData()\end{lstlisting} %end signature
}%end item
\item{ 
\index{getDescription()}
{\bf  getDescription}\\
\begin{lstlisting}[frame=none]
public abstract java.lang.String getDescription()\end{lstlisting} %end signature
\begin{itemize}
\item{{\bf  Returns} -- 
ritorna la descrizione della categoria 
}%end item
\end{itemize}
}%end item
\item{ 
\index{getName()}
{\bf  getName}\\
\begin{lstlisting}[frame=none]
public abstract java.lang.String getName()\end{lstlisting} %end signature
\begin{itemize}
\item{{\bf  Returns} -- 
ritorna il nome della categoria 
}%end item
\end{itemize}
}%end item
\item{ 
\index{isValidChildCategoryName(String)}
{\bf  isValidChildCategoryName}\\
\begin{lstlisting}[frame=none]
public abstract boolean isValidChildCategoryName(java.lang.String name)\end{lstlisting} %end signature
}%end item
\item{ 
\index{removeChild(Category)}
{\bf  removeChild}\\
\begin{lstlisting}[frame=none]
public it.matlice.ingsw.model.data.Category removeChild(it.matlice.ingsw.model.data.Category child)\end{lstlisting} %end signature
\begin{itemize}
\item{
{\bf  Description copied from it.matlice.ingsw.model.data.NodeCategory{\small \refdefined{it.matlice.ingsw.model.data.NodeCategory}} }

rimuove un figlio
}
\item{
{\bf  Parameters}
  \begin{itemize}
   \item{
\texttt{child} -- figlio da rimuovere}
  \end{itemize}
}%end item
\item{{\bf  Returns} -- 
il figlio rimosso 
}%end item
\end{itemize}
}%end item
\end{itemize}
}
\subsection{Members inherited from class NodeCategory }{
\texttt{it.matlice.ingsw.model.data.NodeCategory} {\small 
\refdefined{it.matlice.ingsw.model.data.NodeCategory}}
{\small 

\vskip -2em
\begin{itemize}
\item{\vskip -1.5ex 
\texttt{public Category {\bf  addChild}(\texttt{Category} {\bf  child})
}%end signature
}%end item
\item{\vskip -1.5ex 
\texttt{public abstract Object {\bf  clone}()
}%end signature
}%end item
\item{\vskip -1.5ex 
\texttt{public Category {\bf  getChildren}()
}%end signature
}%end item
\item{\vskip -1.5ex 
\texttt{public Category {\bf  removeChild}(\texttt{Category} {\bf  child})
}%end signature
}%end item
\end{itemize}
}
\subsection{Members inherited from class Category }{
\texttt{it.matlice.ingsw.model.data.Category} {\small 
\refdefined{it.matlice.ingsw.model.data.Category}}
{\small 

\vskip -2em
\begin{itemize}
\item{\vskip -1.5ex 
\texttt{public void {\bf  clear}()
}%end signature
}%end item
\item{\vskip -1.5ex 
\texttt{public boolean {\bf  containsKey}(\texttt{java.lang.Object} {\bf  key})
}%end signature
}%end item
\item{\vskip -1.5ex 
\texttt{public boolean {\bf  containsValue}(\texttt{java.lang.Object} {\bf  value})
}%end signature
}%end item
\item{\vskip -1.5ex 
\texttt{public Set {\bf  fullEntrySet}()
}%end signature
}%end item
\item{\vskip -1.5ex 
\texttt{public String {\bf  fullToString}()
}%end signature
}%end item
\item{\vskip -1.5ex 
\texttt{public TypeDefinition {\bf  get}(\texttt{java.lang.Object} {\bf  key})
}%end signature
}%end item
\item{\vskip -1.5ex 
\texttt{public abstract List {\bf  getChildLeafs}()
}%end signature
}%end item
\item{\vskip -1.5ex 
\texttt{public abstract String {\bf  getDescription}()
}%end signature
}%end item
\item{\vskip -1.5ex 
\texttt{public NodeCategory {\bf  getFather}()
}%end signature
}%end item
\item{\vskip -1.5ex 
\texttt{public abstract String {\bf  getName}()
}%end signature
}%end item
\item{\vskip -1.5ex 
\texttt{public boolean {\bf  isCategoryValid}()
}%end signature
}%end item
\item{\vskip -1.5ex 
\texttt{public boolean {\bf  isEmpty}()
}%end signature
}%end item
\item{\vskip -1.5ex 
\texttt{public boolean {\bf  isRequired}(\texttt{java.lang.String} {\bf  name})
}%end signature
}%end item
\item{\vskip -1.5ex 
\texttt{public boolean {\bf  isRoot}()
}%end signature
}%end item
\item{\vskip -1.5ex 
\texttt{public abstract boolean {\bf  isValidChildCategoryName}(\texttt{java.lang.String} {\bf  name})
}%end signature
}%end item
\item{\vskip -1.5ex 
\texttt{public TypeDefinition {\bf  put}(\texttt{java.lang.String} {\bf  key},
\texttt{TypeDefinition} {\bf  value})
}%end signature
}%end item
\item{\vskip -1.5ex 
\texttt{public TypeDefinition {\bf  remove}(\texttt{java.lang.Object} {\bf  key})
}%end signature
}%end item
\item{\vskip -1.5ex 
\texttt{public void {\bf  setFather}(\texttt{NodeCategory} {\bf  father})
}%end signature
}%end item
\item{\vskip -1.5ex 
\texttt{public int {\bf  size}()
}%end signature
}%end item
\item{\vskip -1.5ex 
\texttt{public String {\bf  toString}()
}%end signature
}%end item
\end{itemize}
}
\subsection{Members inherited from class HashMap }{
\texttt{java.util.HashMap} {\small 
\refdefined{java.util.HashMap}}
{\small 

\vskip -2em
\begin{itemize}
\item{\vskip -1.5ex 
\texttt{public void {\bf  clear}()
}%end signature
}%end item
\item{\vskip -1.5ex 
\texttt{public Object {\bf  clone}()
}%end signature
}%end item
\item{\vskip -1.5ex 
\texttt{public Object {\bf  compute}(\texttt{java.lang.Object} {\bf  arg0},
\texttt{function.BiFunction} {\bf  arg1})
}%end signature
}%end item
\item{\vskip -1.5ex 
\texttt{public Object {\bf  computeIfAbsent}(\texttt{java.lang.Object} {\bf  arg0},
\texttt{function.Function} {\bf  arg1})
}%end signature
}%end item
\item{\vskip -1.5ex 
\texttt{public Object {\bf  computeIfPresent}(\texttt{java.lang.Object} {\bf  arg0},
\texttt{function.BiFunction} {\bf  arg1})
}%end signature
}%end item
\item{\vskip -1.5ex 
\texttt{public boolean {\bf  containsKey}(\texttt{java.lang.Object} {\bf  arg0})
}%end signature
}%end item
\item{\vskip -1.5ex 
\texttt{public boolean {\bf  containsValue}(\texttt{java.lang.Object} {\bf  arg0})
}%end signature
}%end item
\item{\vskip -1.5ex 
\texttt{public Set {\bf  entrySet}()
}%end signature
}%end item
\item{\vskip -1.5ex 
\texttt{public void {\bf  forEach}(\texttt{function.BiConsumer} {\bf  arg0})
}%end signature
}%end item
\item{\vskip -1.5ex 
\texttt{public Object {\bf  get}(\texttt{java.lang.Object} {\bf  arg0})
}%end signature
}%end item
\item{\vskip -1.5ex 
\texttt{public Object {\bf  getOrDefault}(\texttt{java.lang.Object} {\bf  arg0},
\texttt{java.lang.Object} {\bf  arg1})
}%end signature
}%end item
\item{\vskip -1.5ex 
\texttt{public boolean {\bf  isEmpty}()
}%end signature
}%end item
\item{\vskip -1.5ex 
\texttt{public Set {\bf  keySet}()
}%end signature
}%end item
\item{\vskip -1.5ex 
\texttt{public Object {\bf  merge}(\texttt{java.lang.Object} {\bf  arg0},
\texttt{java.lang.Object} {\bf  arg1},
\texttt{function.BiFunction} {\bf  arg2})
}%end signature
}%end item
\item{\vskip -1.5ex 
\texttt{public Object {\bf  put}(\texttt{java.lang.Object} {\bf  arg0},
\texttt{java.lang.Object} {\bf  arg1})
}%end signature
}%end item
\item{\vskip -1.5ex 
\texttt{public void {\bf  putAll}(\texttt{Map} {\bf  arg0})
}%end signature
}%end item
\item{\vskip -1.5ex 
\texttt{public Object {\bf  putIfAbsent}(\texttt{java.lang.Object} {\bf  arg0},
\texttt{java.lang.Object} {\bf  arg1})
}%end signature
}%end item
\item{\vskip -1.5ex 
\texttt{public Object {\bf  remove}(\texttt{java.lang.Object} {\bf  arg0})
}%end signature
}%end item
\item{\vskip -1.5ex 
\texttt{public boolean {\bf  remove}(\texttt{java.lang.Object} {\bf  arg0},
\texttt{java.lang.Object} {\bf  arg1})
}%end signature
}%end item
\item{\vskip -1.5ex 
\texttt{public Object {\bf  replace}(\texttt{java.lang.Object} {\bf  arg0},
\texttt{java.lang.Object} {\bf  arg1})
}%end signature
}%end item
\item{\vskip -1.5ex 
\texttt{public boolean {\bf  replace}(\texttt{java.lang.Object} {\bf  arg0},
\texttt{java.lang.Object} {\bf  arg1},
\texttt{java.lang.Object} {\bf  arg2})
}%end signature
}%end item
\item{\vskip -1.5ex 
\texttt{public void {\bf  replaceAll}(\texttt{function.BiFunction} {\bf  arg0})
}%end signature
}%end item
\item{\vskip -1.5ex 
\texttt{public int {\bf  size}()
}%end signature
}%end item
\item{\vskip -1.5ex 
\texttt{public Collection {\bf  values}()
}%end signature
}%end item
\end{itemize}
}
\subsection{Members inherited from class AbstractMap }{
\texttt{java.util.AbstractMap} {\small 
\refdefined{java.util.AbstractMap}}
{\small 

\vskip -2em
\begin{itemize}
\item{\vskip -1.5ex 
\texttt{public void {\bf  clear}()
}%end signature
}%end item
\item{\vskip -1.5ex 
\texttt{protected Object {\bf  clone}() throws java.lang.CloneNotSupportedException
}%end signature
}%end item
\item{\vskip -1.5ex 
\texttt{public boolean {\bf  containsKey}(\texttt{java.lang.Object} {\bf  arg0})
}%end signature
}%end item
\item{\vskip -1.5ex 
\texttt{public boolean {\bf  containsValue}(\texttt{java.lang.Object} {\bf  arg0})
}%end signature
}%end item
\item{\vskip -1.5ex 
\texttt{public abstract Set {\bf  entrySet}()
}%end signature
}%end item
\item{\vskip -1.5ex 
\texttt{public boolean {\bf  equals}(\texttt{java.lang.Object} {\bf  arg0})
}%end signature
}%end item
\item{\vskip -1.5ex 
\texttt{public Object {\bf  get}(\texttt{java.lang.Object} {\bf  arg0})
}%end signature
}%end item
\item{\vskip -1.5ex 
\texttt{public int {\bf  hashCode}()
}%end signature
}%end item
\item{\vskip -1.5ex 
\texttt{public boolean {\bf  isEmpty}()
}%end signature
}%end item
\item{\vskip -1.5ex 
\texttt{public Set {\bf  keySet}()
}%end signature
}%end item
\item{\vskip -1.5ex 
\texttt{public Object {\bf  put}(\texttt{java.lang.Object} {\bf  arg0},
\texttt{java.lang.Object} {\bf  arg1})
}%end signature
}%end item
\item{\vskip -1.5ex 
\texttt{public void {\bf  putAll}(\texttt{Map} {\bf  arg0})
}%end signature
}%end item
\item{\vskip -1.5ex 
\texttt{public Object {\bf  remove}(\texttt{java.lang.Object} {\bf  arg0})
}%end signature
}%end item
\item{\vskip -1.5ex 
\texttt{public int {\bf  size}()
}%end signature
}%end item
\item{\vskip -1.5ex 
\texttt{public String {\bf  toString}()
}%end signature
}%end item
\item{\vskip -1.5ex 
\texttt{public Collection {\bf  values}()
}%end signature
}%end item
\end{itemize}
}
}
\section{\label{it.matlice.ingsw.model.data.impl.jdbc.types.OfferImpl}\index{OfferImpl}Class OfferImpl}{
\vskip .1in 
\subsection{Declaration}{
\begin{lstlisting}[frame=none]
public class OfferImpl
 extends it.matlice.ingsw.model.data.Offer\end{lstlisting}
\subsection{Constructor summary}{
\begin{verse}
{\bf OfferImpl(OfferDB, LeafCategory, User, Offer)} \\
\end{verse}
}
\subsection{Method summary}{
\begin{verse}
{\bf getCategory()} \\
{\bf getDbData()} \\
{\bf getLinkedOffer()} \\
{\bf getName()} \\
{\bf getOwner()} \\
{\bf getProposedTime()} \\
{\bf getStatus()} \\
{\bf setLinkedOffer(OfferImpl)} \\
\end{verse}
}
\subsection{Constructors}{
\vskip -2em
\begin{itemize}
\item{ 
\index{OfferImpl(OfferDB, LeafCategory, User, Offer)}
{\bf  OfferImpl}\\
\begin{lstlisting}[frame=none]
public OfferImpl(it.matlice.ingsw.model.data.impl.jdbc.db.OfferDB dbData,it.matlice.ingsw.model.data.LeafCategory category,it.matlice.ingsw.model.data.User owner,it.matlice.ingsw.model.data.Offer linked_offer)\end{lstlisting} %end signature
}%end item
\end{itemize}
}
\subsection{Methods}{
\vskip -2em
\begin{itemize}
\item{ 
\index{getCategory()}
{\bf  getCategory}\\
\begin{lstlisting}[frame=none]
public abstract it.matlice.ingsw.model.data.LeafCategory getCategory()\end{lstlisting} %end signature
}%end item
\item{ 
\index{getDbData()}
{\bf  getDbData}\\
\begin{lstlisting}[frame=none]
public it.matlice.ingsw.model.data.impl.jdbc.db.OfferDB getDbData()\end{lstlisting} %end signature
}%end item
\item{ 
\index{getLinkedOffer()}
{\bf  getLinkedOffer}\\
\begin{lstlisting}[frame=none]
public abstract it.matlice.ingsw.model.data.Offer getLinkedOffer()\end{lstlisting} %end signature
}%end item
\item{ 
\index{getName()}
{\bf  getName}\\
\begin{lstlisting}[frame=none]
public abstract java.lang.String getName()\end{lstlisting} %end signature
}%end item
\item{ 
\index{getOwner()}
{\bf  getOwner}\\
\begin{lstlisting}[frame=none]
public abstract it.matlice.ingsw.model.data.User getOwner()\end{lstlisting} %end signature
}%end item
\item{ 
\index{getProposedTime()}
{\bf  getProposedTime}\\
\begin{lstlisting}[frame=none]
public abstract java.lang.Long getProposedTime()\end{lstlisting} %end signature
}%end item
\item{ 
\index{getStatus()}
{\bf  getStatus}\\
\begin{lstlisting}[frame=none]
public abstract it.matlice.ingsw.model.data.Offer.OfferStatus getStatus()\end{lstlisting} %end signature
}%end item
\item{ 
\index{setLinkedOffer(OfferImpl)}
{\bf  setLinkedOffer}\\
\begin{lstlisting}[frame=none]
public void setLinkedOffer(OfferImpl linked)\end{lstlisting} %end signature
}%end item
\end{itemize}
}
\subsection{Members inherited from class Offer }{
\texttt{it.matlice.ingsw.model.data.Offer} {\small 
\refdefined{it.matlice.ingsw.model.data.Offer}}
{\small 

\vskip -2em
\begin{itemize}
\item{\vskip -1.5ex 
\texttt{public abstract LeafCategory {\bf  getCategory}()
}%end signature
}%end item
\item{\vskip -1.5ex 
\texttt{public abstract Offer {\bf  getLinkedOffer}()
}%end signature
}%end item
\item{\vskip -1.5ex 
\texttt{public abstract String {\bf  getName}()
}%end signature
}%end item
\item{\vskip -1.5ex 
\texttt{public abstract User {\bf  getOwner}()
}%end signature
}%end item
\item{\vskip -1.5ex 
\texttt{public abstract Long {\bf  getProposedTime}()
}%end signature
}%end item
\item{\vskip -1.5ex 
\texttt{public abstract Offer.OfferStatus {\bf  getStatus}()
}%end signature
}%end item
\item{\vskip -1.5ex 
\texttt{public String {\bf  toString}()
}%end signature
}%end item
\end{itemize}
}
\subsection{Members inherited from class HashMap }{
\texttt{java.util.HashMap} {\small 
\refdefined{java.util.HashMap}}
{\small 

\vskip -2em
\begin{itemize}
\item{\vskip -1.5ex 
\texttt{public void {\bf  clear}()
}%end signature
}%end item
\item{\vskip -1.5ex 
\texttt{public Object {\bf  clone}()
}%end signature
}%end item
\item{\vskip -1.5ex 
\texttt{public Object {\bf  compute}(\texttt{java.lang.Object} {\bf  arg0},
\texttt{function.BiFunction} {\bf  arg1})
}%end signature
}%end item
\item{\vskip -1.5ex 
\texttt{public Object {\bf  computeIfAbsent}(\texttt{java.lang.Object} {\bf  arg0},
\texttt{function.Function} {\bf  arg1})
}%end signature
}%end item
\item{\vskip -1.5ex 
\texttt{public Object {\bf  computeIfPresent}(\texttt{java.lang.Object} {\bf  arg0},
\texttt{function.BiFunction} {\bf  arg1})
}%end signature
}%end item
\item{\vskip -1.5ex 
\texttt{public boolean {\bf  containsKey}(\texttt{java.lang.Object} {\bf  arg0})
}%end signature
}%end item
\item{\vskip -1.5ex 
\texttt{public boolean {\bf  containsValue}(\texttt{java.lang.Object} {\bf  arg0})
}%end signature
}%end item
\item{\vskip -1.5ex 
\texttt{public Set {\bf  entrySet}()
}%end signature
}%end item
\item{\vskip -1.5ex 
\texttt{public void {\bf  forEach}(\texttt{function.BiConsumer} {\bf  arg0})
}%end signature
}%end item
\item{\vskip -1.5ex 
\texttt{public Object {\bf  get}(\texttt{java.lang.Object} {\bf  arg0})
}%end signature
}%end item
\item{\vskip -1.5ex 
\texttt{public Object {\bf  getOrDefault}(\texttt{java.lang.Object} {\bf  arg0},
\texttt{java.lang.Object} {\bf  arg1})
}%end signature
}%end item
\item{\vskip -1.5ex 
\texttt{public boolean {\bf  isEmpty}()
}%end signature
}%end item
\item{\vskip -1.5ex 
\texttt{public Set {\bf  keySet}()
}%end signature
}%end item
\item{\vskip -1.5ex 
\texttt{public Object {\bf  merge}(\texttt{java.lang.Object} {\bf  arg0},
\texttt{java.lang.Object} {\bf  arg1},
\texttt{function.BiFunction} {\bf  arg2})
}%end signature
}%end item
\item{\vskip -1.5ex 
\texttt{public Object {\bf  put}(\texttt{java.lang.Object} {\bf  arg0},
\texttt{java.lang.Object} {\bf  arg1})
}%end signature
}%end item
\item{\vskip -1.5ex 
\texttt{public void {\bf  putAll}(\texttt{Map} {\bf  arg0})
}%end signature
}%end item
\item{\vskip -1.5ex 
\texttt{public Object {\bf  putIfAbsent}(\texttt{java.lang.Object} {\bf  arg0},
\texttt{java.lang.Object} {\bf  arg1})
}%end signature
}%end item
\item{\vskip -1.5ex 
\texttt{public Object {\bf  remove}(\texttt{java.lang.Object} {\bf  arg0})
}%end signature
}%end item
\item{\vskip -1.5ex 
\texttt{public boolean {\bf  remove}(\texttt{java.lang.Object} {\bf  arg0},
\texttt{java.lang.Object} {\bf  arg1})
}%end signature
}%end item
\item{\vskip -1.5ex 
\texttt{public Object {\bf  replace}(\texttt{java.lang.Object} {\bf  arg0},
\texttt{java.lang.Object} {\bf  arg1})
}%end signature
}%end item
\item{\vskip -1.5ex 
\texttt{public boolean {\bf  replace}(\texttt{java.lang.Object} {\bf  arg0},
\texttt{java.lang.Object} {\bf  arg1},
\texttt{java.lang.Object} {\bf  arg2})
}%end signature
}%end item
\item{\vskip -1.5ex 
\texttt{public void {\bf  replaceAll}(\texttt{function.BiFunction} {\bf  arg0})
}%end signature
}%end item
\item{\vskip -1.5ex 
\texttt{public int {\bf  size}()
}%end signature
}%end item
\item{\vskip -1.5ex 
\texttt{public Collection {\bf  values}()
}%end signature
}%end item
\end{itemize}
}
\subsection{Members inherited from class AbstractMap }{
\texttt{java.util.AbstractMap} {\small 
\refdefined{java.util.AbstractMap}}
{\small 

\vskip -2em
\begin{itemize}
\item{\vskip -1.5ex 
\texttt{public void {\bf  clear}()
}%end signature
}%end item
\item{\vskip -1.5ex 
\texttt{protected Object {\bf  clone}() throws java.lang.CloneNotSupportedException
}%end signature
}%end item
\item{\vskip -1.5ex 
\texttt{public boolean {\bf  containsKey}(\texttt{java.lang.Object} {\bf  arg0})
}%end signature
}%end item
\item{\vskip -1.5ex 
\texttt{public boolean {\bf  containsValue}(\texttt{java.lang.Object} {\bf  arg0})
}%end signature
}%end item
\item{\vskip -1.5ex 
\texttt{public abstract Set {\bf  entrySet}()
}%end signature
}%end item
\item{\vskip -1.5ex 
\texttt{public boolean {\bf  equals}(\texttt{java.lang.Object} {\bf  arg0})
}%end signature
}%end item
\item{\vskip -1.5ex 
\texttt{public Object {\bf  get}(\texttt{java.lang.Object} {\bf  arg0})
}%end signature
}%end item
\item{\vskip -1.5ex 
\texttt{public int {\bf  hashCode}()
}%end signature
}%end item
\item{\vskip -1.5ex 
\texttt{public boolean {\bf  isEmpty}()
}%end signature
}%end item
\item{\vskip -1.5ex 
\texttt{public Set {\bf  keySet}()
}%end signature
}%end item
\item{\vskip -1.5ex 
\texttt{public Object {\bf  put}(\texttt{java.lang.Object} {\bf  arg0},
\texttt{java.lang.Object} {\bf  arg1})
}%end signature
}%end item
\item{\vskip -1.5ex 
\texttt{public void {\bf  putAll}(\texttt{Map} {\bf  arg0})
}%end signature
}%end item
\item{\vskip -1.5ex 
\texttt{public Object {\bf  remove}(\texttt{java.lang.Object} {\bf  arg0})
}%end signature
}%end item
\item{\vskip -1.5ex 
\texttt{public int {\bf  size}()
}%end signature
}%end item
\item{\vskip -1.5ex 
\texttt{public String {\bf  toString}()
}%end signature
}%end item
\item{\vskip -1.5ex 
\texttt{public Collection {\bf  values}()
}%end signature
}%end item
\end{itemize}
}
}
\section{\label{it.matlice.ingsw.model.data.impl.jdbc.types.SettingsImpl}\index{SettingsImpl}Class SettingsImpl}{
\vskip .1in 
\subsection{Declaration}{
\begin{lstlisting}[frame=none]
public class SettingsImpl
 extends it.matlice.ingsw.model.data.Settings\end{lstlisting}
\subsection{Constructor summary}{
\begin{verse}
{\bf SettingsImpl(SettingsDB, List, List, List)} \\
\end{verse}
}
\subsection{Method summary}{
\begin{verse}
{\bf getCity()} \\
{\bf getDays()} \\
{\bf getDbData()} \\
{\bf getDue()} \\
{\bf getIntervals()} \\
{\bf getLocations()} \\
\end{verse}
}
\subsection{Constructors}{
\vskip -2em
\begin{itemize}
\item{ 
\index{SettingsImpl(SettingsDB, List, List, List)}
{\bf  SettingsImpl}\\
\begin{lstlisting}[frame=none]
public SettingsImpl(it.matlice.ingsw.model.data.impl.jdbc.db.SettingsDB dbData,java.util.List dbLocations,java.util.List dbIntervals,java.util.List dbDays)\end{lstlisting} %end signature
}%end item
\end{itemize}
}
\subsection{Methods}{
\vskip -2em
\begin{itemize}
\item{ 
\index{getCity()}
{\bf  getCity}\\
\begin{lstlisting}[frame=none]
public abstract java.lang.String getCity()\end{lstlisting} %end signature
}%end item
\item{ 
\index{getDays()}
{\bf  getDays}\\
\begin{lstlisting}[frame=none]
public abstract java.util.List getDays()\end{lstlisting} %end signature
}%end item
\item{ 
\index{getDbData()}
{\bf  getDbData}\\
\begin{lstlisting}[frame=none]
public it.matlice.ingsw.model.data.impl.jdbc.db.SettingsDB getDbData()\end{lstlisting} %end signature
}%end item
\item{ 
\index{getDue()}
{\bf  getDue}\\
\begin{lstlisting}[frame=none]
public abstract int getDue()\end{lstlisting} %end signature
}%end item
\item{ 
\index{getIntervals()}
{\bf  getIntervals}\\
\begin{lstlisting}[frame=none]
public abstract java.util.List getIntervals()\end{lstlisting} %end signature
}%end item
\item{ 
\index{getLocations()}
{\bf  getLocations}\\
\begin{lstlisting}[frame=none]
public abstract java.util.List getLocations()\end{lstlisting} %end signature
}%end item
\end{itemize}
}
\subsection{Members inherited from class Settings }{
\texttt{it.matlice.ingsw.model.data.Settings} {\small 
\refdefined{it.matlice.ingsw.model.data.Settings}}
{\small 

\vskip -2em
\begin{itemize}
\item{\vskip -1.5ex 
\texttt{public abstract String {\bf  getCity}()
}%end signature
}%end item
\item{\vskip -1.5ex 
\texttt{public abstract List {\bf  getDays}()
}%end signature
}%end item
\item{\vskip -1.5ex 
\texttt{public abstract int {\bf  getDue}()
}%end signature
}%end item
\item{\vskip -1.5ex 
\texttt{public abstract List {\bf  getIntervals}()
}%end signature
}%end item
\item{\vskip -1.5ex 
\texttt{public abstract List {\bf  getLocations}()
}%end signature
}%end item
\end{itemize}
}
}
}
\chapter{Package it.matlice.ingsw.model.data.impl.jdbc.db}{
\label{it.matlice.ingsw.model.data.impl.jdbc.db}\hskip -.05in
\hbox to \hsize{\textit{ Package Contents\hfil Page}}
\vskip .13in
\hbox{{\bf  Classes}}
\entityintro{CategoryDB}{it.matlice.ingsw.model.data.impl.jdbc.db.CategoryDB}{Classe che rappresenta a database la tabella `categories`}
\entityintro{CategoryFieldDB}{it.matlice.ingsw.model.data.impl.jdbc.db.CategoryFieldDB}{Classe che rappresenta a database la tabella `fields` in cui sono salvati i campi relativi alle categorie}
\entityintro{DaysDB}{it.matlice.ingsw.model.data.impl.jdbc.db.DaysDB}{}
\entityintro{HierarchyDB}{it.matlice.ingsw.model.data.impl.jdbc.db.HierarchyDB}{Classe che rappresenta a database la tabella `hierarchies`}
\entityintro{IntervalsDB}{it.matlice.ingsw.model.data.impl.jdbc.db.IntervalsDB}{}
\entityintro{LocationsDB}{it.matlice.ingsw.model.data.impl.jdbc.db.LocationsDB}{}
\entityintro{MessageDB}{it.matlice.ingsw.model.data.impl.jdbc.db.MessageDB}{}
\entityintro{OfferDB}{it.matlice.ingsw.model.data.impl.jdbc.db.OfferDB}{}
\entityintro{OfferFieldDB}{it.matlice.ingsw.model.data.impl.jdbc.db.OfferFieldDB}{}
\entityintro{SettingsDB}{it.matlice.ingsw.model.data.impl.jdbc.db.SettingsDB}{}
\entityintro{UserDB}{it.matlice.ingsw.model.data.impl.jdbc.db.UserDB}{Classe che rappresenta a database la tabella `users`}
\vskip .1in
\vskip .1in
\section{\label{it.matlice.ingsw.model.data.impl.jdbc.db.CategoryDB}\index{CategoryDB}Class CategoryDB}{
\vskip .1in 
Classe che rappresenta a database la tabella `categories`\vskip .1in 
\subsection{Declaration}{
\begin{lstlisting}[frame=none]
public class CategoryDB
 extends java.lang.Object\end{lstlisting}
\subsection{Constructor summary}{
\begin{verse}
{\bf CategoryDB(String, String, CategoryDB)} \\
\end{verse}
}
\subsection{Method summary}{
\begin{verse}
{\bf equals(Object)} \\
{\bf getCategoryDescription()} \\
{\bf getCategoryId()} \\
{\bf getCategoryName()} \\
{\bf getFather()} \\
{\bf hashCode()} \\
{\bf setFather(CategoryDB)} \\
\end{verse}
}
\subsection{Constructors}{
\vskip -2em
\begin{itemize}
\item{ 
\index{CategoryDB(String, String, CategoryDB)}
{\bf  CategoryDB}\\
\begin{lstlisting}[frame=none]
public CategoryDB(java.lang.String category_name,java.lang.String category_description,CategoryDB father)\end{lstlisting} %end signature
}%end item
\end{itemize}
}
\subsection{Methods}{
\vskip -2em
\begin{itemize}
\item{ 
\index{equals(Object)}
{\bf  equals}\\
\begin{lstlisting}[frame=none]
public boolean equals(java.lang.Object arg0)\end{lstlisting} %end signature
}%end item
\item{ 
\index{getCategoryDescription()}
{\bf  getCategoryDescription}\\
\begin{lstlisting}[frame=none]
public java.lang.String getCategoryDescription()\end{lstlisting} %end signature
}%end item
\item{ 
\index{getCategoryId()}
{\bf  getCategoryId}\\
\begin{lstlisting}[frame=none]
public int getCategoryId()\end{lstlisting} %end signature
}%end item
\item{ 
\index{getCategoryName()}
{\bf  getCategoryName}\\
\begin{lstlisting}[frame=none]
public java.lang.String getCategoryName()\end{lstlisting} %end signature
}%end item
\item{ 
\index{getFather()}
{\bf  getFather}\\
\begin{lstlisting}[frame=none]
public CategoryDB getFather()\end{lstlisting} %end signature
}%end item
\item{ 
\index{hashCode()}
{\bf  hashCode}\\
\begin{lstlisting}[frame=none]
public native int hashCode()\end{lstlisting} %end signature
}%end item
\item{ 
\index{setFather(CategoryDB)}
{\bf  setFather}\\
\begin{lstlisting}[frame=none]
public void setFather(CategoryDB father)\end{lstlisting} %end signature
}%end item
\end{itemize}
}
}
\section{\label{it.matlice.ingsw.model.data.impl.jdbc.db.CategoryFieldDB}\index{CategoryFieldDB}Class CategoryFieldDB}{
\vskip .1in 
Classe che rappresenta a database la tabella `fields` in cui sono salvati i campi relativi alle categorie\vskip .1in 
\subsection{Declaration}{
\begin{lstlisting}[frame=none]
public class CategoryFieldDB
 extends java.lang.Object\end{lstlisting}
\subsection{Constructor summary}{
\begin{verse}
{\bf CategoryFieldDB(String, CategoryDB, String, boolean)} \\
\end{verse}
}
\subsection{Method summary}{
\begin{verse}
{\bf getCategory()} \\
{\bf getFieldId()} \\
{\bf getFieldName()} \\
{\bf getType()} \\
{\bf isRequired()} \\
{\bf setCategory(CategoryDB)} \\
{\bf setFieldId(int)} \\
{\bf setFieldName(String)} \\
{\bf setRequired(boolean)} \\
{\bf setType(String)} \\
\end{verse}
}
\subsection{Constructors}{
\vskip -2em
\begin{itemize}
\item{ 
\index{CategoryFieldDB(String, CategoryDB, String, boolean)}
{\bf  CategoryFieldDB}\\
\begin{lstlisting}[frame=none]
public CategoryFieldDB(java.lang.String fieldName,CategoryDB category,java.lang.String type,boolean required)\end{lstlisting} %end signature
}%end item
\end{itemize}
}
\subsection{Methods}{
\vskip -2em
\begin{itemize}
\item{ 
\index{getCategory()}
{\bf  getCategory}\\
\begin{lstlisting}[frame=none]
public CategoryDB getCategory()\end{lstlisting} %end signature
}%end item
\item{ 
\index{getFieldId()}
{\bf  getFieldId}\\
\begin{lstlisting}[frame=none]
public int getFieldId()\end{lstlisting} %end signature
}%end item
\item{ 
\index{getFieldName()}
{\bf  getFieldName}\\
\begin{lstlisting}[frame=none]
public java.lang.String getFieldName()\end{lstlisting} %end signature
}%end item
\item{ 
\index{getType()}
{\bf  getType}\\
\begin{lstlisting}[frame=none]
public java.lang.String getType()\end{lstlisting} %end signature
}%end item
\item{ 
\index{isRequired()}
{\bf  isRequired}\\
\begin{lstlisting}[frame=none]
public boolean isRequired()\end{lstlisting} %end signature
}%end item
\item{ 
\index{setCategory(CategoryDB)}
{\bf  setCategory}\\
\begin{lstlisting}[frame=none]
public void setCategory(CategoryDB category)\end{lstlisting} %end signature
}%end item
\item{ 
\index{setFieldId(int)}
{\bf  setFieldId}\\
\begin{lstlisting}[frame=none]
public void setFieldId(int fieldId)\end{lstlisting} %end signature
}%end item
\item{ 
\index{setFieldName(String)}
{\bf  setFieldName}\\
\begin{lstlisting}[frame=none]
public void setFieldName(java.lang.String fieldName)\end{lstlisting} %end signature
}%end item
\item{ 
\index{setRequired(boolean)}
{\bf  setRequired}\\
\begin{lstlisting}[frame=none]
public void setRequired(boolean required)\end{lstlisting} %end signature
}%end item
\item{ 
\index{setType(String)}
{\bf  setType}\\
\begin{lstlisting}[frame=none]
public void setType(java.lang.String type)\end{lstlisting} %end signature
}%end item
\end{itemize}
}
}
\section{\label{it.matlice.ingsw.model.data.impl.jdbc.db.DaysDB}\index{DaysDB}Class DaysDB}{
\vskip .1in 
\subsection{Declaration}{
\begin{lstlisting}[frame=none]
public class DaysDB
 extends java.lang.Object\end{lstlisting}
\subsection{Constructor summary}{
\begin{verse}
{\bf DaysDB(SettingsDB, Settings.Day)} \\
\end{verse}
}
\subsection{Method summary}{
\begin{verse}
{\bf getDay()} \\
{\bf getRef()} \\
{\bf setLocation(Settings.Day)} \\
{\bf setRef(SettingsDB)} \\
\end{verse}
}
\subsection{Constructors}{
\vskip -2em
\begin{itemize}
\item{ 
\index{DaysDB(SettingsDB, Settings.Day)}
{\bf  DaysDB}\\
\begin{lstlisting}[frame=none]
public DaysDB(SettingsDB ref,it.matlice.ingsw.model.data.Settings.Day day)\end{lstlisting} %end signature
}%end item
\end{itemize}
}
\subsection{Methods}{
\vskip -2em
\begin{itemize}
\item{ 
\index{getDay()}
{\bf  getDay}\\
\begin{lstlisting}[frame=none]
public it.matlice.ingsw.model.data.Settings.Day getDay()\end{lstlisting} %end signature
}%end item
\item{ 
\index{getRef()}
{\bf  getRef}\\
\begin{lstlisting}[frame=none]
public SettingsDB getRef()\end{lstlisting} %end signature
}%end item
\item{ 
\index{setLocation(Settings.Day)}
{\bf  setLocation}\\
\begin{lstlisting}[frame=none]
public void setLocation(it.matlice.ingsw.model.data.Settings.Day day)\end{lstlisting} %end signature
}%end item
\item{ 
\index{setRef(SettingsDB)}
{\bf  setRef}\\
\begin{lstlisting}[frame=none]
public void setRef(SettingsDB ref)\end{lstlisting} %end signature
}%end item
\end{itemize}
}
}
\section{\label{it.matlice.ingsw.model.data.impl.jdbc.db.HierarchyDB}\index{HierarchyDB}Class HierarchyDB}{
\vskip .1in 
Classe che rappresenta a database la tabella `hierarchies`\vskip .1in 
\subsection{Declaration}{
\begin{lstlisting}[frame=none]
public class HierarchyDB
 extends java.lang.Object\end{lstlisting}
\subsection{Constructor summary}{
\begin{verse}
{\bf HierarchyDB(CategoryDB)} \\
\end{verse}
}
\subsection{Method summary}{
\begin{verse}
{\bf getHierarchyId()} \\
{\bf getRoot()} \\
\end{verse}
}
\subsection{Constructors}{
\vskip -2em
\begin{itemize}
\item{ 
\index{HierarchyDB(CategoryDB)}
{\bf  HierarchyDB}\\
\begin{lstlisting}[frame=none]
public HierarchyDB(CategoryDB root)\end{lstlisting} %end signature
}%end item
\end{itemize}
}
\subsection{Methods}{
\vskip -2em
\begin{itemize}
\item{ 
\index{getHierarchyId()}
{\bf  getHierarchyId}\\
\begin{lstlisting}[frame=none]
public int getHierarchyId()\end{lstlisting} %end signature
}%end item
\item{ 
\index{getRoot()}
{\bf  getRoot}\\
\begin{lstlisting}[frame=none]
public CategoryDB getRoot()\end{lstlisting} %end signature
}%end item
\end{itemize}
}
}
\section{\label{it.matlice.ingsw.model.data.impl.jdbc.db.IntervalsDB}\index{IntervalsDB}Class IntervalsDB}{
\vskip .1in 
\subsection{Declaration}{
\begin{lstlisting}[frame=none]
public class IntervalsDB
 extends java.lang.Object\end{lstlisting}
\subsection{Constructor summary}{
\begin{verse}
{\bf IntervalsDB(SettingsDB, Interval)} \\
\end{verse}
}
\subsection{Method summary}{
\begin{verse}
{\bf getInterval()} \\
{\bf getRef()} \\
{\bf setInterval(Interval)} \\
{\bf setRef(SettingsDB)} \\
\end{verse}
}
\subsection{Constructors}{
\vskip -2em
\begin{itemize}
\item{ 
\index{IntervalsDB(SettingsDB, Interval)}
{\bf  IntervalsDB}\\
\begin{lstlisting}[frame=none]
public IntervalsDB(SettingsDB ref,it.matlice.ingsw.model.data.Interval interval)\end{lstlisting} %end signature
}%end item
\end{itemize}
}
\subsection{Methods}{
\vskip -2em
\begin{itemize}
\item{ 
\index{getInterval()}
{\bf  getInterval}\\
\begin{lstlisting}[frame=none]
public it.matlice.ingsw.model.data.Interval getInterval()\end{lstlisting} %end signature
}%end item
\item{ 
\index{getRef()}
{\bf  getRef}\\
\begin{lstlisting}[frame=none]
public SettingsDB getRef()\end{lstlisting} %end signature
}%end item
\item{ 
\index{setInterval(Interval)}
{\bf  setInterval}\\
\begin{lstlisting}[frame=none]
public void setInterval(it.matlice.ingsw.model.data.Interval interval)\end{lstlisting} %end signature
}%end item
\item{ 
\index{setRef(SettingsDB)}
{\bf  setRef}\\
\begin{lstlisting}[frame=none]
public void setRef(SettingsDB ref)\end{lstlisting} %end signature
}%end item
\end{itemize}
}
}
\section{\label{it.matlice.ingsw.model.data.impl.jdbc.db.LocationsDB}\index{LocationsDB}Class LocationsDB}{
\vskip .1in 
\subsection{Declaration}{
\begin{lstlisting}[frame=none]
public class LocationsDB
 extends java.lang.Object\end{lstlisting}
\subsection{Constructor summary}{
\begin{verse}
{\bf LocationsDB(SettingsDB, String)} \\
\end{verse}
}
\subsection{Method summary}{
\begin{verse}
{\bf getLocation()} \\
{\bf getRef()} \\
{\bf setLocation(String)} \\
{\bf setRef(SettingsDB)} \\
\end{verse}
}
\subsection{Constructors}{
\vskip -2em
\begin{itemize}
\item{ 
\index{LocationsDB(SettingsDB, String)}
{\bf  LocationsDB}\\
\begin{lstlisting}[frame=none]
public LocationsDB(SettingsDB ref,java.lang.String location)\end{lstlisting} %end signature
}%end item
\end{itemize}
}
\subsection{Methods}{
\vskip -2em
\begin{itemize}
\item{ 
\index{getLocation()}
{\bf  getLocation}\\
\begin{lstlisting}[frame=none]
public java.lang.String getLocation()\end{lstlisting} %end signature
}%end item
\item{ 
\index{getRef()}
{\bf  getRef}\\
\begin{lstlisting}[frame=none]
public SettingsDB getRef()\end{lstlisting} %end signature
}%end item
\item{ 
\index{setLocation(String)}
{\bf  setLocation}\\
\begin{lstlisting}[frame=none]
public void setLocation(java.lang.String location)\end{lstlisting} %end signature
}%end item
\item{ 
\index{setRef(SettingsDB)}
{\bf  setRef}\\
\begin{lstlisting}[frame=none]
public void setRef(SettingsDB ref)\end{lstlisting} %end signature
}%end item
\end{itemize}
}
}
\section{\label{it.matlice.ingsw.model.data.impl.jdbc.db.MessageDB}\index{MessageDB}Class MessageDB}{
\vskip .1in 
\subsection{Declaration}{
\begin{lstlisting}[frame=none]
public class MessageDB
 extends java.lang.Object\end{lstlisting}
\subsection{Constructor summary}{
\begin{verse}
{\bf MessageDB(Long, String, OfferDB)} \\
\end{verse}
}
\subsection{Method summary}{
\begin{verse}
{\bf getAnswer()} \\
{\bf getId()} \\
{\bf getProposedDate()} \\
{\bf getProposedLocation()} \\
{\bf getRelative\_offer()} \\
{\bf setAnswer(MessageDB)} \\
\end{verse}
}
\subsection{Constructors}{
\vskip -2em
\begin{itemize}
\item{ 
\index{MessageDB(Long, String, OfferDB)}
{\bf  MessageDB}\\
\begin{lstlisting}[frame=none]
public MessageDB(java.lang.Long proposedDate,java.lang.String proposedLocation,OfferDB relative_offer)\end{lstlisting} %end signature
}%end item
\end{itemize}
}
\subsection{Methods}{
\vskip -2em
\begin{itemize}
\item{ 
\index{getAnswer()}
{\bf  getAnswer}\\
\begin{lstlisting}[frame=none]
public MessageDB getAnswer()\end{lstlisting} %end signature
}%end item
\item{ 
\index{getId()}
{\bf  getId}\\
\begin{lstlisting}[frame=none]
public java.lang.Integer getId()\end{lstlisting} %end signature
}%end item
\item{ 
\index{getProposedDate()}
{\bf  getProposedDate}\\
\begin{lstlisting}[frame=none]
public java.lang.Long getProposedDate()\end{lstlisting} %end signature
}%end item
\item{ 
\index{getProposedLocation()}
{\bf  getProposedLocation}\\
\begin{lstlisting}[frame=none]
public java.lang.String getProposedLocation()\end{lstlisting} %end signature
}%end item
\item{ 
\index{getRelative\_offer()}
{\bf  getRelative\_offer}\\
\begin{lstlisting}[frame=none]
public OfferDB getRelative_offer()\end{lstlisting} %end signature
}%end item
\item{ 
\index{setAnswer(MessageDB)}
{\bf  setAnswer}\\
\begin{lstlisting}[frame=none]
public void setAnswer(MessageDB answer)\end{lstlisting} %end signature
}%end item
\end{itemize}
}
}
\section{\label{it.matlice.ingsw.model.data.impl.jdbc.db.OfferDB}\index{OfferDB}Class OfferDB}{
\vskip .1in 
\subsection{Declaration}{
\begin{lstlisting}[frame=none]
public class OfferDB
 extends java.lang.Object\end{lstlisting}
\subsection{Constructor summary}{
\begin{verse}
{\bf OfferDB(String, UserDB, CategoryDB, Offer.OfferStatus, OfferDB, Long)} \\
\end{verse}
}
\subsection{Method summary}{
\begin{verse}
{\bf equals(Object)} \\
{\bf getCategory()} \\
{\bf getId()} \\
{\bf getLinkedOffer()} \\
{\bf getName()} \\
{\bf getOwner()} \\
{\bf getProposedTime()} \\
{\bf getStatus()} \\
{\bf setLinkedOffer(OfferDB)} \\
{\bf setProposedTime(long)} \\
{\bf setStatus(Offer.OfferStatus)} \\
\end{verse}
}
\subsection{Constructors}{
\vskip -2em
\begin{itemize}
\item{ 
\index{OfferDB(String, UserDB, CategoryDB, Offer.OfferStatus, OfferDB, Long)}
{\bf  OfferDB}\\
\begin{lstlisting}[frame=none]
public OfferDB(java.lang.String name,UserDB owner,CategoryDB category,it.matlice.ingsw.model.data.Offer.OfferStatus status,OfferDB linked_offer,java.lang.Long proposed_time)\end{lstlisting} %end signature
}%end item
\end{itemize}
}
\subsection{Methods}{
\vskip -2em
\begin{itemize}
\item{ 
\index{equals(Object)}
{\bf  equals}\\
\begin{lstlisting}[frame=none]
public boolean equals(java.lang.Object arg0)\end{lstlisting} %end signature
}%end item
\item{ 
\index{getCategory()}
{\bf  getCategory}\\
\begin{lstlisting}[frame=none]
public CategoryDB getCategory()\end{lstlisting} %end signature
}%end item
\item{ 
\index{getId()}
{\bf  getId}\\
\begin{lstlisting}[frame=none]
public java.lang.Integer getId()\end{lstlisting} %end signature
}%end item
\item{ 
\index{getLinkedOffer()}
{\bf  getLinkedOffer}\\
\begin{lstlisting}[frame=none]
public OfferDB getLinkedOffer()\end{lstlisting} %end signature
}%end item
\item{ 
\index{getName()}
{\bf  getName}\\
\begin{lstlisting}[frame=none]
public java.lang.String getName()\end{lstlisting} %end signature
}%end item
\item{ 
\index{getOwner()}
{\bf  getOwner}\\
\begin{lstlisting}[frame=none]
public UserDB getOwner()\end{lstlisting} %end signature
}%end item
\item{ 
\index{getProposedTime()}
{\bf  getProposedTime}\\
\begin{lstlisting}[frame=none]
public java.lang.Long getProposedTime()\end{lstlisting} %end signature
}%end item
\item{ 
\index{getStatus()}
{\bf  getStatus}\\
\begin{lstlisting}[frame=none]
public it.matlice.ingsw.model.data.Offer.OfferStatus getStatus()\end{lstlisting} %end signature
}%end item
\item{ 
\index{setLinkedOffer(OfferDB)}
{\bf  setLinkedOffer}\\
\begin{lstlisting}[frame=none]
public void setLinkedOffer(OfferDB dbData)\end{lstlisting} %end signature
}%end item
\item{ 
\index{setProposedTime(long)}
{\bf  setProposedTime}\\
\begin{lstlisting}[frame=none]
public void setProposedTime(long time)\end{lstlisting} %end signature
}%end item
\item{ 
\index{setStatus(Offer.OfferStatus)}
{\bf  setStatus}\\
\begin{lstlisting}[frame=none]
public void setStatus(it.matlice.ingsw.model.data.Offer.OfferStatus status)\end{lstlisting} %end signature
}%end item
\end{itemize}
}
}
\section{\label{it.matlice.ingsw.model.data.impl.jdbc.db.OfferFieldDB}\index{OfferFieldDB}Class OfferFieldDB}{
\vskip .1in 
\subsection{Declaration}{
\begin{lstlisting}[frame=none]
public class OfferFieldDB
 extends java.lang.Object\end{lstlisting}
\subsection{Constructor summary}{
\begin{verse}
{\bf OfferFieldDB(CategoryFieldDB, OfferDB, String)} \\
\end{verse}
}
\subsection{Method summary}{
\begin{verse}
{\bf getId()} \\
{\bf getRef()} \\
{\bf getValue()} \\
\end{verse}
}
\subsection{Constructors}{
\vskip -2em
\begin{itemize}
\item{ 
\index{OfferFieldDB(CategoryFieldDB, OfferDB, String)}
{\bf  OfferFieldDB}\\
\begin{lstlisting}[frame=none]
public OfferFieldDB(CategoryFieldDB ref,OfferDB aref,java.lang.String value)\end{lstlisting} %end signature
}%end item
\end{itemize}
}
\subsection{Methods}{
\vskip -2em
\begin{itemize}
\item{ 
\index{getId()}
{\bf  getId}\\
\begin{lstlisting}[frame=none]
public java.lang.Integer getId()\end{lstlisting} %end signature
}%end item
\item{ 
\index{getRef()}
{\bf  getRef}\\
\begin{lstlisting}[frame=none]
public CategoryFieldDB getRef()\end{lstlisting} %end signature
}%end item
\item{ 
\index{getValue()}
{\bf  getValue}\\
\begin{lstlisting}[frame=none]
public java.lang.String getValue()\end{lstlisting} %end signature
}%end item
\end{itemize}
}
}
\section{\label{it.matlice.ingsw.model.data.impl.jdbc.db.SettingsDB}\index{SettingsDB}Class SettingsDB}{
\vskip .1in 
\subsection{Declaration}{
\begin{lstlisting}[frame=none]
public class SettingsDB
 extends java.lang.Object\end{lstlisting}
\subsection{Constructor summary}{
\begin{verse}
{\bf SettingsDB(String, Integer)} \\
\end{verse}
}
\subsection{Method summary}{
\begin{verse}
{\bf getCity()} \\
{\bf getDue()} \\
{\bf getId()} \\
{\bf setDue(Integer)} \\
\end{verse}
}
\subsection{Constructors}{
\vskip -2em
\begin{itemize}
\item{ 
\index{SettingsDB(String, Integer)}
{\bf  SettingsDB}\\
\begin{lstlisting}[frame=none]
public SettingsDB(java.lang.String city,java.lang.Integer due)\end{lstlisting} %end signature
}%end item
\end{itemize}
}
\subsection{Methods}{
\vskip -2em
\begin{itemize}
\item{ 
\index{getCity()}
{\bf  getCity}\\
\begin{lstlisting}[frame=none]
public java.lang.String getCity()\end{lstlisting} %end signature
}%end item
\item{ 
\index{getDue()}
{\bf  getDue}\\
\begin{lstlisting}[frame=none]
public java.lang.Integer getDue()\end{lstlisting} %end signature
}%end item
\item{ 
\index{getId()}
{\bf  getId}\\
\begin{lstlisting}[frame=none]
public java.lang.Integer getId()\end{lstlisting} %end signature
}%end item
\item{ 
\index{setDue(Integer)}
{\bf  setDue}\\
\begin{lstlisting}[frame=none]
public void setDue(java.lang.Integer due)\end{lstlisting} %end signature
}%end item
\end{itemize}
}
}
\section{\label{it.matlice.ingsw.model.data.impl.jdbc.db.UserDB}\index{UserDB}Class UserDB}{
\vskip .1in 
Classe che rappresenta a database la tabella `users`\vskip .1in 
\subsection{Declaration}{
\begin{lstlisting}[frame=none]
public class UserDB
 extends java.lang.Object\end{lstlisting}
\subsection{Constructor summary}{
\begin{verse}
{\bf UserDB()} \\
\end{verse}
}
\subsection{Method summary}{
\begin{verse}
{\bf equals(Object)} \\
{\bf getLastAccess()} \\
{\bf getPasswordHash()} \\
{\bf getPasswordSalt()} \\
{\bf getType()} \\
{\bf getUsername()} \\
{\bf setLastAccess(Long)} \\
{\bf setPasswordHash(String)} \\
{\bf setPasswordSalt(String)} \\
{\bf setType(String)} \\
{\bf setUsername(String)} \\
\end{verse}
}
\subsection{Constructors}{
\vskip -2em
\begin{itemize}
\item{ 
\index{UserDB()}
{\bf  UserDB}\\
\begin{lstlisting}[frame=none]
public UserDB()\end{lstlisting} %end signature
}%end item
\end{itemize}
}
\subsection{Methods}{
\vskip -2em
\begin{itemize}
\item{ 
\index{equals(Object)}
{\bf  equals}\\
\begin{lstlisting}[frame=none]
public boolean equals(java.lang.Object arg0)\end{lstlisting} %end signature
}%end item
\item{ 
\index{getLastAccess()}
{\bf  getLastAccess}\\
\begin{lstlisting}[frame=none]
public java.lang.Long getLastAccess()\end{lstlisting} %end signature
}%end item
\item{ 
\index{getPasswordHash()}
{\bf  getPasswordHash}\\
\begin{lstlisting}[frame=none]
public java.lang.String getPasswordHash()\end{lstlisting} %end signature
}%end item
\item{ 
\index{getPasswordSalt()}
{\bf  getPasswordSalt}\\
\begin{lstlisting}[frame=none]
public java.lang.String getPasswordSalt()\end{lstlisting} %end signature
}%end item
\item{ 
\index{getType()}
{\bf  getType}\\
\begin{lstlisting}[frame=none]
public java.lang.String getType()\end{lstlisting} %end signature
}%end item
\item{ 
\index{getUsername()}
{\bf  getUsername}\\
\begin{lstlisting}[frame=none]
public java.lang.String getUsername()\end{lstlisting} %end signature
}%end item
\item{ 
\index{setLastAccess(Long)}
{\bf  setLastAccess}\\
\begin{lstlisting}[frame=none]
public void setLastAccess(java.lang.Long lastAccess)\end{lstlisting} %end signature
}%end item
\item{ 
\index{setPasswordHash(String)}
{\bf  setPasswordHash}\\
\begin{lstlisting}[frame=none]
public void setPasswordHash(java.lang.String password_hash)\end{lstlisting} %end signature
}%end item
\item{ 
\index{setPasswordSalt(String)}
{\bf  setPasswordSalt}\\
\begin{lstlisting}[frame=none]
public void setPasswordSalt(java.lang.String password_salt)\end{lstlisting} %end signature
}%end item
\item{ 
\index{setType(String)}
{\bf  setType}\\
\begin{lstlisting}[frame=none]
public void setType(java.lang.String type)\end{lstlisting} %end signature
}%end item
\item{ 
\index{setUsername(String)}
{\bf  setUsername}\\
\begin{lstlisting}[frame=none]
public void setUsername(java.lang.String username)\end{lstlisting} %end signature
}%end item
\end{itemize}
}
}
}
\chapter{Package it.matlice.ingsw.model.data.factories}{
\label{it.matlice.ingsw.model.data.factories}\hskip -.05in
\hbox to \hsize{\textit{ Package Contents\hfil Page}}
\vskip .13in
\hbox{{\bf  Interfaces}}
\entityintro{CategoryFactory}{it.matlice.ingsw.model.data.factories.CategoryFactory}{rappresenta una classe che si occuperà di istanziare implementazioni di categorie, correttamente identificate da NodeCategory o LeafCategory, complete di struttura di (eventuali) figli}
\entityintro{HierarchyFactory}{it.matlice.ingsw.model.data.factories.HierarchyFactory}{Rappresenta una classe che si occupa di istanziare elementi di tipo Hierarchy una volta caricati da una base di dati esterna.}
\entityintro{MessageFactory}{it.matlice.ingsw.model.data.factories.MessageFactory}{Rappresenta una classe che si occupa di istanziare elementi di tipo Message una volta caricati da una base di dati esterna.}
\entityintro{OfferFactory}{it.matlice.ingsw.model.data.factories.OfferFactory}{rappresenta una classe che si occuperà di istanziare implementazioni di articoli, con i campi compilati opportunamente e associati al proprietario}
\entityintro{SettingsFactory}{it.matlice.ingsw.model.data.factories.SettingsFactory}{}
\entityintro{UserFactory}{it.matlice.ingsw.model.data.factories.UserFactory}{Interfaccia che rappresenta una classe in grado di istanziare User nella giusta declinazione a partire da una base di dati}
\vskip .1in
\vskip .1in
\section{\label{it.matlice.ingsw.model.data.factories.CategoryFactory}\index{CategoryFactory@\textit{ CategoryFactory}}Interface CategoryFactory}{
\vskip .1in 
rappresenta una classe che si occuperà di istanziare implementazioni di categorie, correttamente identificate da NodeCategory o LeafCategory, complete di struttura di (eventuali) figli\vskip .1in 
\subsection{Declaration}{
\begin{lstlisting}[frame=none]
public interface CategoryFactory
\end{lstlisting}
\subsection{All known subinterfaces}{CategoryFactoryImpl\small{\refdefined{it.matlice.ingsw.model.data.impl.jdbc.CategoryFactoryImpl}}}
\subsection{All classes known to implement interface}{CategoryFactoryImpl\small{\refdefined{it.matlice.ingsw.model.data.impl.jdbc.CategoryFactoryImpl}}}
\subsection{Method summary}{
\begin{verse}
{\bf createCategory(String, String, Category, boolean)} crea e salva una nuova categoria\\
{\bf getCategory(int)} Ottiene una categoria tramite id.\\
{\bf saveCategory(Category)} salva la categoria nel database aggiornandola, inoltre salva i campi se non esistono.\\
\end{verse}
}
\subsection{Methods}{
\vskip -2em
\begin{itemize}
\item{ 
\index{createCategory(String, String, Category, boolean)}
{\bf  createCategory}\\
\begin{lstlisting}[frame=none]
it.matlice.ingsw.model.data.Category createCategory(java.lang.String nome,java.lang.String description,it.matlice.ingsw.model.data.Category father,boolean isLeaf)\end{lstlisting} %end signature
\begin{itemize}
\item{
{\bf  Description}

crea e salva una nuova categoria
}
\item{
{\bf  Parameters}
  \begin{itemize}
   \item{
\texttt{nome} -- nome della categoria}
   \item{
\texttt{father} -- categoria padre (null se si vuole creare una root category}
   \item{
\texttt{isLeaf} -- indica se la categoria creata potrà avere figli o se è eldiana nel finale alternativo di aot}
  \end{itemize}
}%end item
\item{{\bf  Returns} -- 
la categoria creata 
}%end item
\end{itemize}
}%end item
\item{ 
\index{getCategory(int)}
{\bf  getCategory}\\
\begin{lstlisting}[frame=none]
it.matlice.ingsw.model.data.Category getCategory(int id) throws java.sql.SQLException\end{lstlisting} %end signature
\begin{itemize}
\item{
{\bf  Description}

Ottiene una categoria tramite id. notare che questo metodo non dovrebbe essere usato direttamente dal controller, ma deve essere utilizzato per ottenere la categoria a partire da una gerarchia. la gestione dell'id è lasciata all'implementazione e non deve incidere nello sviluppo del controller.
}
\item{
{\bf  Parameters}
  \begin{itemize}
   \item{
\texttt{id} -- numero incrementale identificativo univoco della categoria}
  \end{itemize}
}%end item
\item{{\bf  Returns} -- 
una categoria se esiste 
}%end item
\item{{\bf  Throws}
  \begin{itemize}
   \item{\vskip -.6ex \texttt{java.sql.SQLException} -- }
  \end{itemize}
}%end item
\end{itemize}
}%end item
\item{ 
\index{saveCategory(Category)}
{\bf  saveCategory}\\
\begin{lstlisting}[frame=none]
it.matlice.ingsw.model.data.Category saveCategory(it.matlice.ingsw.model.data.Category category) throws java.sql.SQLException\end{lstlisting} %end signature
\begin{itemize}
\item{
{\bf  Description}

salva la categoria nel database aggiornandola, inoltre salva i campi se non esistono. Si noti che non è necessario cancellare i campi rimossi dato che da specifica, le categorie sono immutabili. È necessario quindi solo aggiungere i nuovi campi al momento della creazione.
}
\item{
{\bf  Parameters}
  \begin{itemize}
   \item{
\texttt{category} -- categoria da salvare}
  \end{itemize}
}%end item
\item{{\bf  Returns} -- 
la categoria aggiornata 
}%end item
\item{{\bf  Throws}
  \begin{itemize}
   \item{\vskip -.6ex \texttt{java.sql.SQLException} -- }
  \end{itemize}
}%end item
\end{itemize}
}%end item
\end{itemize}
}
}
\section{\label{it.matlice.ingsw.model.data.factories.HierarchyFactory}\index{HierarchyFactory@\textit{ HierarchyFactory}}Interface HierarchyFactory}{
\vskip .1in 
Rappresenta una classe che si occupa di istanziare elementi di tipo Hierarchy una volta caricati da una base di dati esterna.\vskip .1in 
\subsection{Declaration}{
\begin{lstlisting}[frame=none]
public interface HierarchyFactory
\end{lstlisting}
\subsection{All known subinterfaces}{HierarchyFactoryImpl\small{\refdefined{it.matlice.ingsw.model.data.impl.jdbc.HierarchyFactoryImpl}}}
\subsection{All classes known to implement interface}{HierarchyFactoryImpl\small{\refdefined{it.matlice.ingsw.model.data.impl.jdbc.HierarchyFactoryImpl}}}
\subsection{Method summary}{
\begin{verse}
{\bf createHierarchy(Category)} crea una nuova gerarchia e la salva sulla base di dati\\
{\bf deleteHierarchy(Hierarchy)} rimuove una gerarchia dalla bd\\
{\bf getHierarchies()} \\
\end{verse}
}
\subsection{Methods}{
\vskip -2em
\begin{itemize}
\item{ 
\index{createHierarchy(Category)}
{\bf  createHierarchy}\\
\begin{lstlisting}[frame=none]
it.matlice.ingsw.model.data.Hierarchy createHierarchy(it.matlice.ingsw.model.data.Category rootCategory) throws java.sql.SQLException\end{lstlisting} %end signature
\begin{itemize}
\item{
{\bf  Description}

crea una nuova gerarchia e la salva sulla base di dati
}
\item{
{\bf  Parameters}
  \begin{itemize}
   \item{
\texttt{rootCategory} -- la categoria root}
  \end{itemize}
}%end item
\item{{\bf  Returns} -- 
la nuova gerarchia 
}%end item
\item{{\bf  Throws}
  \begin{itemize}
   \item{\vskip -.6ex \texttt{java.sql.SQLException} -- }
  \end{itemize}
}%end item
\end{itemize}
}%end item
\item{ 
\index{deleteHierarchy(Hierarchy)}
{\bf  deleteHierarchy}\\
\begin{lstlisting}[frame=none]
void deleteHierarchy(it.matlice.ingsw.model.data.Hierarchy h) throws java.sql.SQLException\end{lstlisting} %end signature
\begin{itemize}
\item{
{\bf  Description}

rimuove una gerarchia dalla bd
}
\item{
{\bf  Parameters}
  \begin{itemize}
   \item{
\texttt{h} -- gerarchia da rimuovere}
  \end{itemize}
}%end item
\item{{\bf  Throws}
  \begin{itemize}
   \item{\vskip -.6ex \texttt{java.sql.SQLException} -- }
  \end{itemize}
}%end item
\end{itemize}
}%end item
\item{ 
\index{getHierarchies()}
{\bf  getHierarchies}\\
\begin{lstlisting}[frame=none]
java.util.List getHierarchies() throws java.sql.SQLException\end{lstlisting} %end signature
\begin{itemize}
\item{{\bf  Returns} -- 
Ottiene la lista di getrarchie presenti nel programma 
}%end item
\item{{\bf  Throws}
  \begin{itemize}
   \item{\vskip -.6ex \texttt{java.sql.SQLException} -- }
  \end{itemize}
}%end item
\end{itemize}
}%end item
\end{itemize}
}
}
\section{\label{it.matlice.ingsw.model.data.factories.MessageFactory}\index{MessageFactory@\textit{ MessageFactory}}Interface MessageFactory}{
\vskip .1in 
Rappresenta una classe che si occupa di istanziare elementi di tipo Message una volta caricati da una base di dati esterna.\vskip .1in 
\subsection{Declaration}{
\begin{lstlisting}[frame=none]
public interface MessageFactory
\end{lstlisting}
\subsection{All known subinterfaces}{MessageFactoryImpl\small{\refdefined{it.matlice.ingsw.model.data.impl.jdbc.MessageFactoryImpl}}}
\subsection{All classes known to implement interface}{MessageFactoryImpl\small{\refdefined{it.matlice.ingsw.model.data.impl.jdbc.MessageFactoryImpl}}}
\subsection{Method summary}{
\begin{verse}
{\bf answer(Message, Offer, String, Calendar)} Permette di aggiungere un nuovo messaggio in risposta ad una proposta di uno scambio\\
{\bf getUserMessages(User)} Ottiene tutti i messaggi a cui l'utente deve fornire una risposta\\
{\bf send(Offer, String, Calendar, long)} Permette di aggiungere un nuovo messaggio durante la fase di proposta di uno scambio\\
\end{verse}
}
\subsection{Methods}{
\vskip -2em
\begin{itemize}
\item{ 
\index{answer(Message, Offer, String, Calendar)}
{\bf  answer}\\
\begin{lstlisting}[frame=none]
it.matlice.ingsw.model.data.Message answer(it.matlice.ingsw.model.data.Message msg,it.matlice.ingsw.model.data.Offer offer,java.lang.String location,java.util.Calendar date) throws java.sql.SQLException\end{lstlisting} %end signature
\begin{itemize}
\item{
{\bf  Description}

Permette di aggiungere un nuovo messaggio in risposta ad una proposta di uno scambio
}
\item{
{\bf  Parameters}
  \begin{itemize}
   \item{
\texttt{msg} -- messaggio a cui rispondere}
   \item{
\texttt{offer} -- offerta a cui si riferisce il messaggio (del ricevitore)}
   \item{
\texttt{location} -- luogo proposto}
   \item{
\texttt{date} -- data proposta}
  \end{itemize}
}%end item
\item{{\bf  Returns} -- 
messaggio inviato 
}%end item
\item{{\bf  Throws}
  \begin{itemize}
   \item{\vskip -.6ex \texttt{java.sql.SQLException} -- errore di database durante la creazione del messaggio}
  \end{itemize}
}%end item
\end{itemize}
}%end item
\item{ 
\index{getUserMessages(User)}
{\bf  getUserMessages}\\
\begin{lstlisting}[frame=none]
java.util.List getUserMessages(it.matlice.ingsw.model.data.User u) throws java.sql.SQLException\end{lstlisting} %end signature
\begin{itemize}
\item{
{\bf  Description}

Ottiene tutti i messaggi a cui l'utente deve fornire una risposta
}
\item{
{\bf  Parameters}
  \begin{itemize}
   \item{
\texttt{u} -- utente ricevitore dei messaggi}
  \end{itemize}
}%end item
\item{{\bf  Returns} -- 
lista di messaggi per l'utente 
}%end item
\item{{\bf  Throws}
  \begin{itemize}
   \item{\vskip -.6ex \texttt{java.sql.SQLException} -- errore di database durante l'ottenimento dei messaggi}
  \end{itemize}
}%end item
\end{itemize}
}%end item
\item{ 
\index{send(Offer, String, Calendar, long)}
{\bf  send}\\
\begin{lstlisting}[frame=none]
it.matlice.ingsw.model.data.Message send(it.matlice.ingsw.model.data.Offer offer,java.lang.String location,java.util.Calendar date,long timestamp) throws java.sql.SQLException\end{lstlisting} %end signature
\begin{itemize}
\item{
{\bf  Description}

Permette di aggiungere un nuovo messaggio durante la fase di proposta di uno scambio
}
\item{
{\bf  Parameters}
  \begin{itemize}
   \item{
\texttt{offer} -- offerta a cui si riferisce il messaggio (del ricevitore)}
   \item{
\texttt{location} -- luogo proposto}
   \item{
\texttt{date} -- data proposta}
   \item{
\texttt{timestamp} -- momento di creazione del messaggio}
  \end{itemize}
}%end item
\item{{\bf  Returns} -- 
messaggio inviato 
}%end item
\item{{\bf  Throws}
  \begin{itemize}
   \item{\vskip -.6ex \texttt{java.sql.SQLException} -- errore di database durante la creazione del messaggio}
  \end{itemize}
}%end item
\end{itemize}
}%end item
\end{itemize}
}
}
\section{\label{it.matlice.ingsw.model.data.factories.OfferFactory}\index{OfferFactory@\textit{ OfferFactory}}Interface OfferFactory}{
\vskip .1in 
rappresenta una classe che si occuperà di istanziare implementazioni di articoli, con i campi compilati opportunamente e associati al proprietario\vskip .1in 
\subsection{Declaration}{
\begin{lstlisting}[frame=none]
public interface OfferFactory
\end{lstlisting}
\subsection{All known subinterfaces}{OfferFactoryImpl\small{\refdefined{it.matlice.ingsw.model.data.impl.jdbc.OfferFactoryImpl}}}
\subsection{All classes known to implement interface}{OfferFactoryImpl\small{\refdefined{it.matlice.ingsw.model.data.impl.jdbc.OfferFactoryImpl}}}
\subsection{Method summary}{
\begin{verse}
{\bf acceptTradeOffer(Offer, MessageFactory, String, Calendar)} \\
{\bf checkForDueDate()} \\
{\bf closeTradeOffer(Message)} \\
{\bf createTradeOffer(Offer, Offer)} Permette di accoppiare due offerte, la prima è dell'utente che propone lo scambio\\
{\bf getOffers(LeafCategory)} Ottiene la lista di offerte associate a una categoria.\\
{\bf getOffers(User)} \\
{\bf getSelectedOffers(User)} \\
{\bf makeOffer(String, User, LeafCategory, Map)} Crea un'offerta per l'utente nella categoria.\\
{\bf setOfferStatus(Offer, Offer.OfferStatus)} Imposta lo stato di un'offerta\\
{\bf updateDate(Offer, Calendar)} \\
\end{verse}
}
\subsection{Methods}{
\vskip -2em
\begin{itemize}
\item{ 
\index{acceptTradeOffer(Offer, MessageFactory, String, Calendar)}
{\bf  acceptTradeOffer}\\
\begin{lstlisting}[frame=none]
void acceptTradeOffer(it.matlice.ingsw.model.data.Offer offer,MessageFactory mf,java.lang.String location,java.util.Calendar date) throws java.sql.SQLException\end{lstlisting} %end signature
}%end item
\item{ 
\index{checkForDueDate()}
{\bf  checkForDueDate}\\
\begin{lstlisting}[frame=none]
void checkForDueDate() throws java.sql.SQLException\end{lstlisting} %end signature
}%end item
\item{ 
\index{closeTradeOffer(Message)}
{\bf  closeTradeOffer}\\
\begin{lstlisting}[frame=none]
void closeTradeOffer(it.matlice.ingsw.model.data.Message m) throws java.sql.SQLException\end{lstlisting} %end signature
}%end item
\item{ 
\index{createTradeOffer(Offer, Offer)}
{\bf  createTradeOffer}\\
\begin{lstlisting}[frame=none]
void createTradeOffer(it.matlice.ingsw.model.data.Offer offerToTrade,it.matlice.ingsw.model.data.Offer offerToAccept) throws java.sql.SQLException\end{lstlisting} %end signature
\begin{itemize}
\item{
{\bf  Description}

Permette di accoppiare due offerte, la prima è dell'utente che propone lo scambio
}
\item{
{\bf  Parameters}
  \begin{itemize}
   \item{
\texttt{offerToTrade} -- offerta accoppiata}
   \item{
\texttt{offerToAccept} -- offerta selezionata}
  \end{itemize}
}%end item
\item{{\bf  Throws}
  \begin{itemize}
   \item{\vskip -.6ex \texttt{java.sql.SQLException} -- }
  \end{itemize}
}%end item
\end{itemize}
}%end item
\item{ 
\index{getOffers(LeafCategory)}
{\bf  getOffers}\\
\begin{lstlisting}[frame=none]
java.util.List getOffers(it.matlice.ingsw.model.data.LeafCategory cat) throws java.sql.SQLException\end{lstlisting} %end signature
\begin{itemize}
\item{
{\bf  Description}

Ottiene la lista di offerte associate a una categoria. Notare che l'istanza di categoria deve essere stata precedentemente creata e deve fare parte dell'albero delle categorie per far si che vengano rilevati tutti i campi delle categorie padre sopra di essa.
}
\item{
{\bf  Parameters}
  \begin{itemize}
   \item{
\texttt{cat} -- categoria}
  \end{itemize}
}%end item
\item{{\bf  Returns} -- 
offerte associate alla categoria 
}%end item
\item{{\bf  Throws}
  \begin{itemize}
   \item{\vskip -.6ex \texttt{java.sql.SQLException} -- .}
  \end{itemize}
}%end item
\end{itemize}
}%end item
\item{ 
\index{getOffers(User)}
{\bf  getOffers}\\
\begin{lstlisting}[frame=none]
java.util.List getOffers(it.matlice.ingsw.model.data.User owner) throws java.sql.SQLException\end{lstlisting} %end signature
\begin{itemize}
\item{
{\bf  Parameters}
  \begin{itemize}
   \item{
\texttt{owner} -- utente}
  \end{itemize}
}%end item
\item{{\bf  Returns} -- 
Ritorna tutte le offerte associate all'utente 
}%end item
\item{{\bf  Throws}
  \begin{itemize}
   \item{\vskip -.6ex \texttt{java.sql.SQLException} -- .}
  \end{itemize}
}%end item
\end{itemize}
}%end item
\item{ 
\index{getSelectedOffers(User)}
{\bf  getSelectedOffers}\\
\begin{lstlisting}[frame=none]
java.util.List getSelectedOffers(it.matlice.ingsw.model.data.User owner) throws java.sql.SQLException\end{lstlisting} %end signature
}%end item
\item{ 
\index{makeOffer(String, User, LeafCategory, Map)}
{\bf  makeOffer}\\
\begin{lstlisting}[frame=none]
it.matlice.ingsw.model.data.Offer makeOffer(java.lang.String name,it.matlice.ingsw.model.data.User owner,it.matlice.ingsw.model.data.LeafCategory category,java.util.Map field_values) throws it.matlice.ingsw.model.exceptions.RequiredFieldConstrainException, java.sql.SQLException\end{lstlisting} %end signature
\begin{itemize}
\item{
{\bf  Description}

Crea un'offerta per l'utente nella categoria. Notare che l'istanza di categoria deve essere stata precedentemente creata e deve fare parte dell'albero delle categorie per far si che vengano rilevati tutti i campi delle categorie padre sopra di essa.
}
\item{
{\bf  Parameters}
  \begin{itemize}
   \item{
\texttt{name} -- nome dell'articolo offerto}
   \item{
\texttt{owner} -- istanza dell'utente proprietario}
   \item{
\texttt{category} -- categoria dove posizionare l'offerta}
   \item{
\texttt{field\_values} -- mappa dei valori dei campi}
  \end{itemize}
}%end item
\item{{\bf  Returns} -- 
l'offerta istanziata 
}%end item
\item{{\bf  Throws}
  \begin{itemize}
   \item{\vskip -.6ex \texttt{it.matlice.ingsw.model.exceptions.RequiredFieldConstrainException} -- Se un campo obbligatorio non é stato compilato}
   \item{\vskip -.6ex \texttt{java.sql.SQLException} -- .}
  \end{itemize}
}%end item
\end{itemize}
}%end item
\item{ 
\index{setOfferStatus(Offer, Offer.OfferStatus)}
{\bf  setOfferStatus}\\
\begin{lstlisting}[frame=none]
void setOfferStatus(it.matlice.ingsw.model.data.Offer offer,it.matlice.ingsw.model.data.Offer.OfferStatus status) throws java.sql.SQLException\end{lstlisting} %end signature
\begin{itemize}
\item{
{\bf  Description}

Imposta lo stato di un'offerta
}
\item{
{\bf  Parameters}
  \begin{itemize}
   \item{
\texttt{offer} -- istanza dell'offerta da creare}
   \item{
\texttt{status} -- stato dell'offerta}
  \end{itemize}
}%end item
\item{{\bf  Throws}
  \begin{itemize}
   \item{\vskip -.6ex \texttt{java.sql.SQLException} -- .}
  \end{itemize}
}%end item
\end{itemize}
}%end item
\item{ 
\index{updateDate(Offer, Calendar)}
{\bf  updateDate}\\
\begin{lstlisting}[frame=none]
void updateDate(it.matlice.ingsw.model.data.Offer offer,java.util.Calendar date) throws java.sql.SQLException\end{lstlisting} %end signature
}%end item
\end{itemize}
}
}
\section{\label{it.matlice.ingsw.model.data.factories.SettingsFactory}\index{SettingsFactory@\textit{ SettingsFactory}}Interface SettingsFactory}{
\vskip .1in 
\subsection{Declaration}{
\begin{lstlisting}[frame=none]
public interface SettingsFactory
\end{lstlisting}
\subsection{All known subinterfaces}{SettingsFactoryImpl\small{\refdefined{it.matlice.ingsw.model.data.impl.jdbc.SettingsFactoryImpl}}}
\subsection{All classes known to implement interface}{SettingsFactoryImpl\small{\refdefined{it.matlice.ingsw.model.data.impl.jdbc.SettingsFactoryImpl}}}
\subsection{Method summary}{
\begin{verse}
{\bf addDay(Settings, Settings.Day)} Permette di aggiungere un nuovo giorno per lo scambio\\
{\bf addInterval(Settings, Interval)} Permette di aggiungere un nuovo intervallo orario per lo scambio\\
{\bf addLocation(Settings, String)} Permette di aggiungere un nuovo luogo per lo scambio\\
{\bf makeSettings(String, int, List, List, List)} Crea una istanza di impostazioni\\
{\bf readSettings()} Legge le impostazioni e le ritorna\\
{\bf removeDays(Settings)} Permette di rimuovere tutti i giorni per lo scambio\\
{\bf removeIntervals(Settings)} Permette di rimuovere tutti gli intervalli per lo scambio\\
{\bf removeLocations(Settings)} Permette di rimuovere tutti i luoghi per lo scambio\\
{\bf setDue(Settings, int)} Permette di impostare i giorni di scadenza\\
\end{verse}
}
\subsection{Methods}{
\vskip -2em
\begin{itemize}
\item{ 
\index{addDay(Settings, Settings.Day)}
{\bf  addDay}\\
\begin{lstlisting}[frame=none]
void addDay(it.matlice.ingsw.model.data.Settings db,it.matlice.ingsw.model.data.Settings.Day d) throws java.sql.SQLException\end{lstlisting} %end signature
\begin{itemize}
\item{
{\bf  Description}

Permette di aggiungere un nuovo giorno per lo scambio
}
\item{
{\bf  Parameters}
  \begin{itemize}
   \item{
\texttt{db} -- istanza di settings}
   \item{
\texttt{d} -- giorno da aggiungere}
  \end{itemize}
}%end item
\item{{\bf  Throws}
  \begin{itemize}
   \item{\vskip -.6ex \texttt{java.sql.SQLException} -- errore di database}
  \end{itemize}
}%end item
\end{itemize}
}%end item
\item{ 
\index{addInterval(Settings, Interval)}
{\bf  addInterval}\\
\begin{lstlisting}[frame=none]
void addInterval(it.matlice.ingsw.model.data.Settings db,it.matlice.ingsw.model.data.Interval i) throws java.sql.SQLException\end{lstlisting} %end signature
\begin{itemize}
\item{
{\bf  Description}

Permette di aggiungere un nuovo intervallo orario per lo scambio
}
\item{
{\bf  Parameters}
  \begin{itemize}
   \item{
\texttt{db} -- istanza di settings}
   \item{
\texttt{i} -- intervallo da aggiungere}
  \end{itemize}
}%end item
\item{{\bf  Throws}
  \begin{itemize}
   \item{\vskip -.6ex \texttt{java.sql.SQLException} -- errore di database}
  \end{itemize}
}%end item
\end{itemize}
}%end item
\item{ 
\index{addLocation(Settings, String)}
{\bf  addLocation}\\
\begin{lstlisting}[frame=none]
void addLocation(it.matlice.ingsw.model.data.Settings db,java.lang.String l) throws java.sql.SQLException\end{lstlisting} %end signature
\begin{itemize}
\item{
{\bf  Description}

Permette di aggiungere un nuovo luogo per lo scambio
}
\item{
{\bf  Parameters}
  \begin{itemize}
   \item{
\texttt{db} -- istanza di settings}
   \item{
\texttt{l} -- luogo da aggiungere}
  \end{itemize}
}%end item
\item{{\bf  Throws}
  \begin{itemize}
   \item{\vskip -.6ex \texttt{java.sql.SQLException} -- errore di database}
  \end{itemize}
}%end item
\end{itemize}
}%end item
\item{ 
\index{makeSettings(String, int, List, List, List)}
{\bf  makeSettings}\\
\begin{lstlisting}[frame=none]
it.matlice.ingsw.model.data.Settings makeSettings(java.lang.String city,int due,java.util.List locations,java.util.List days,java.util.List intervals) throws java.sql.SQLException\end{lstlisting} %end signature
\begin{itemize}
\item{
{\bf  Description}

Crea una istanza di impostazioni
}
\item{
{\bf  Parameters}
  \begin{itemize}
   \item{
\texttt{city} -- Piazza di scambio}
   \item{
\texttt{due} -- numero di giorni di scadenza}
   \item{
\texttt{locations} -- luoghi di scambio}
   \item{
\texttt{days} -- Giorni di scambio}
   \item{
\texttt{intervals} -- intervalli di scambio}
  \end{itemize}
}%end item
\item{{\bf  Returns} -- 
l'istanza creata 
}%end item
\item{{\bf  Throws}
  \begin{itemize}
   \item{\vskip -.6ex \texttt{java.sql.SQLException} -- .}
  \end{itemize}
}%end item
\end{itemize}
}%end item
\item{ 
\index{readSettings()}
{\bf  readSettings}\\
\begin{lstlisting}[frame=none]
it.matlice.ingsw.model.data.Settings readSettings() throws java.sql.SQLException\end{lstlisting} %end signature
\begin{itemize}
\item{
{\bf  Description}

Legge le impostazioni e le ritorna
}
\item{{\bf  Returns} -- 
impostazioni ottenute dalla base di dati. Se fossero presenti piú impostazioni, il comportamento non é definito e ne verrá ritornata solo una. 
}%end item
\item{{\bf  Throws}
  \begin{itemize}
   \item{\vskip -.6ex \texttt{java.sql.SQLException} -- .}
  \end{itemize}
}%end item
\end{itemize}
}%end item
\item{ 
\index{removeDays(Settings)}
{\bf  removeDays}\\
\begin{lstlisting}[frame=none]
void removeDays(it.matlice.ingsw.model.data.Settings db) throws java.sql.SQLException\end{lstlisting} %end signature
\begin{itemize}
\item{
{\bf  Description}

Permette di rimuovere tutti i giorni per lo scambio
}
\item{
{\bf  Parameters}
  \begin{itemize}
   \item{
\texttt{db} -- istanza di settings}
  \end{itemize}
}%end item
\item{{\bf  Throws}
  \begin{itemize}
   \item{\vskip -.6ex \texttt{java.sql.SQLException} -- errore di database}
  \end{itemize}
}%end item
\end{itemize}
}%end item
\item{ 
\index{removeIntervals(Settings)}
{\bf  removeIntervals}\\
\begin{lstlisting}[frame=none]
void removeIntervals(it.matlice.ingsw.model.data.Settings db) throws java.sql.SQLException\end{lstlisting} %end signature
\begin{itemize}
\item{
{\bf  Description}

Permette di rimuovere tutti gli intervalli per lo scambio
}
\item{
{\bf  Parameters}
  \begin{itemize}
   \item{
\texttt{db} -- istanza di settings}
  \end{itemize}
}%end item
\item{{\bf  Throws}
  \begin{itemize}
   \item{\vskip -.6ex \texttt{java.sql.SQLException} -- errore di database}
  \end{itemize}
}%end item
\end{itemize}
}%end item
\item{ 
\index{removeLocations(Settings)}
{\bf  removeLocations}\\
\begin{lstlisting}[frame=none]
void removeLocations(it.matlice.ingsw.model.data.Settings db) throws java.sql.SQLException\end{lstlisting} %end signature
\begin{itemize}
\item{
{\bf  Description}

Permette di rimuovere tutti i luoghi per lo scambio
}
\item{
{\bf  Parameters}
  \begin{itemize}
   \item{
\texttt{db} -- istanza di settings}
  \end{itemize}
}%end item
\item{{\bf  Throws}
  \begin{itemize}
   \item{\vskip -.6ex \texttt{java.sql.SQLException} -- errore di database}
  \end{itemize}
}%end item
\end{itemize}
}%end item
\item{ 
\index{setDue(Settings, int)}
{\bf  setDue}\\
\begin{lstlisting}[frame=none]
void setDue(it.matlice.ingsw.model.data.Settings db,int due) throws java.sql.SQLException\end{lstlisting} %end signature
\begin{itemize}
\item{
{\bf  Description}

Permette di impostare i giorni di scadenza
}
\item{
{\bf  Parameters}
  \begin{itemize}
   \item{
\texttt{db} -- istanza di settings}
   \item{
\texttt{due} -- numero di giorni per la scadenza}
  \end{itemize}
}%end item
\item{{\bf  Throws}
  \begin{itemize}
   \item{\vskip -.6ex \texttt{java.sql.SQLException} -- errore di database}
  \end{itemize}
}%end item
\end{itemize}
}%end item
\end{itemize}
}
}
\section{\label{it.matlice.ingsw.model.data.factories.UserFactory}\index{UserFactory@\textit{ UserFactory}}Interface UserFactory}{
\vskip .1in 
Interfaccia che rappresenta una classe in grado di istanziare User nella giusta declinazione a partire da una base di dati\vskip .1in 
\subsection{Declaration}{
\begin{lstlisting}[frame=none]
public interface UserFactory
\end{lstlisting}
\subsection{All known subinterfaces}{UserFactoryImpl\small{\refdefined{it.matlice.ingsw.model.data.impl.jdbc.UserFactoryImpl}}}
\subsection{All classes known to implement interface}{UserFactoryImpl\small{\refdefined{it.matlice.ingsw.model.data.impl.jdbc.UserFactoryImpl}}}
\subsection{Method summary}{
\begin{verse}
{\bf createUser(String, User.UserTypes)} crea un utente e lo salva nella bs\\
{\bf doesUserExist(String)} \\
{\bf getUser(String)} \\
{\bf getUsers()} Ritorna la lista degli utente a database\\
{\bf saveUser(User)} Permette di salvare un utente a database\\
\end{verse}
}
\subsection{Methods}{
\vskip -2em
\begin{itemize}
\item{ 
\index{createUser(String, User.UserTypes)}
{\bf  createUser}\\
\begin{lstlisting}[frame=none]
it.matlice.ingsw.model.data.User createUser(java.lang.String username,it.matlice.ingsw.model.data.User.UserTypes userType) throws java.sql.SQLException, it.matlice.ingsw.model.exceptions.InvalidUserTypeException\end{lstlisting} %end signature
\begin{itemize}
\item{
{\bf  Description}

crea un utente e lo salva nella bs
}
\item{
{\bf  Parameters}
  \begin{itemize}
   \item{
\texttt{username} -- username del nuovo utente}
   \item{
\texttt{userType} -- tipo di utente}
  \end{itemize}
}%end item
\item{{\bf  Returns} -- 
l'utente creato 
}%end item
\item{{\bf  Throws}
  \begin{itemize}
   \item{\vskip -.6ex \texttt{java.sql.SQLException} -- }
   \item{\vskip -.6ex \texttt{it.matlice.ingsw.model.exceptions.InvalidUserTypeException} -- }
  \end{itemize}
}%end item
\end{itemize}
}%end item
\item{ 
\index{doesUserExist(String)}
{\bf  doesUserExist}\\
\begin{lstlisting}[frame=none]
boolean doesUserExist(java.lang.String username) throws java.sql.SQLException\end{lstlisting} %end signature
}%end item
\item{ 
\index{getUser(String)}
{\bf  getUser}\\
\begin{lstlisting}[frame=none]
it.matlice.ingsw.model.data.User getUser(java.lang.String username) throws java.sql.SQLException, it.matlice.ingsw.model.exceptions.InvalidUserException\end{lstlisting} %end signature
\begin{itemize}
\item{
{\bf  Parameters}
  \begin{itemize}
   \item{
\texttt{username} -- username dell'utente voluto (univoco)}
  \end{itemize}
}%end item
\item{{\bf  Returns} -- 
l'utente tratto dalla base di dati 
}%end item
\item{{\bf  Throws}
  \begin{itemize}
   \item{\vskip -.6ex \texttt{java.sql.SQLException} -- }
   \item{\vskip -.6ex \texttt{it.matlice.ingsw.model.exceptions.InvalidUserException} -- }
  \end{itemize}
}%end item
\end{itemize}
}%end item
\item{ 
\index{getUsers()}
{\bf  getUsers}\\
\begin{lstlisting}[frame=none]
java.util.List getUsers() throws java.sql.SQLException\end{lstlisting} %end signature
\begin{itemize}
\item{
{\bf  Description}

Ritorna la lista degli utente a database
}
\item{{\bf  Returns} -- 
lista di utenti 
}%end item
\item{{\bf  Throws}
  \begin{itemize}
   \item{\vskip -.6ex \texttt{java.sql.SQLException} -- errore di database}
  \end{itemize}
}%end item
\end{itemize}
}%end item
\item{ 
\index{saveUser(User)}
{\bf  saveUser}\\
\begin{lstlisting}[frame=none]
it.matlice.ingsw.model.data.User saveUser(it.matlice.ingsw.model.data.User u) throws java.sql.SQLException\end{lstlisting} %end signature
\begin{itemize}
\item{
{\bf  Description}

Permette di salvare un utente a database
}
\item{
{\bf  Parameters}
  \begin{itemize}
   \item{
\texttt{u} -- utente da salvare}
  \end{itemize}
}%end item
\item{{\bf  Returns} -- 
utente salvato 
}%end item
\item{{\bf  Throws}
  \begin{itemize}
   \item{\vskip -.6ex \texttt{java.sql.SQLException} -- errore di database}
  \end{itemize}
}%end item
\end{itemize}
}%end item
\end{itemize}
}
}
}
\chapter{Package it.matlice.ingsw.view.menu}{
\label{it.matlice.ingsw.view.menu}\hskip -.05in
\hbox to \hsize{\textit{ Package Contents\hfil Page}}
\vskip .13in
\hbox{{\bf  Interfaces}}
\entityintro{MenuAction}{it.matlice.ingsw.view.menu.MenuAction}{}
\vskip .13in
\hbox{{\bf  Classes}}
\entityintro{Menu}{it.matlice.ingsw.view.menu.Menu}{}
\entityintro{MenuEntry}{it.matlice.ingsw.view.menu.MenuEntry}{}
\vskip .1in
\vskip .1in
\section{\label{it.matlice.ingsw.view.menu.MenuAction}\index{MenuAction@\textit{ MenuAction}}Interface MenuAction}{
\vskip .1in 
\subsection{Declaration}{
\begin{lstlisting}[frame=none]
public interface MenuAction
\end{lstlisting}
\subsection{Method summary}{
\begin{verse}
{\bf onUserSelect(Scanner, PrintStream, MenuEntry)} \\
\end{verse}
}
\subsection{Methods}{
\vskip -2em
\begin{itemize}
\item{ 
\index{onUserSelect(Scanner, PrintStream, MenuEntry)}
{\bf  onUserSelect}\\
\begin{lstlisting}[frame=none]
java.lang.Object onUserSelect(java.util.Scanner in,java.io.PrintStream out,MenuEntry ref)\end{lstlisting} %end signature
}%end item
\end{itemize}
}
}
\section{\label{it.matlice.ingsw.view.menu.Menu}\index{Menu}Class Menu}{
\vskip .1in 
\subsection{Declaration}{
\begin{lstlisting}[frame=none]
public class Menu
 extends java.lang.Object\end{lstlisting}
\subsection{Constructor summary}{
\begin{verse}
{\bf Menu()} \\
\end{verse}
}
\subsection{Method summary}{
\begin{verse}
{\bf addEntry(int, String, MenuAction)} adds an entry to the menu\\
{\bf addEntry(int, String, MenuAction, int)} adds an entry to the menu\\
{\bf addEntry(MenuEntry)} adds an entry to the menu\\
{\bf addEntry(String, MenuAction)} adds an entry to the menu.\\
{\bf addEntry(String, MenuAction, int)} adds an entry to the menu\\
{\bf disable(boolean)} Sets the state of the last added entry\\
{\bf display(Scanner, PrintStream)} displays the menu on the output stream until the user inserts a valid answer\\
{\bf displayOnce(Scanner, PrintStream)} displays the menu on the output stream\\
{\bf getEntry(int)} Gets the reference to a MenuEntry\\
{\bf getEntryByPosition(int)} Gets the reference to a MenuEntry\\
{\bf onlyDisplay(PrintStream)} shows the menu without asking for any input.\\
{\bf payload(Object)} Sets the last added entry's payload\\
{\bf selectAfterDisplay(Scanner, PrintStream)} asks for an entry of the menu\\
{\bf setPrompt(String)} sets the ask prompt value\\
\end{verse}
}
\subsection{Constructors}{
\vskip -2em
\begin{itemize}
\item{ 
\index{Menu()}
{\bf  Menu}\\
\begin{lstlisting}[frame=none]
public Menu()\end{lstlisting} %end signature
}%end item
\end{itemize}
}
\subsection{Methods}{
\vskip -2em
\begin{itemize}
\item{ 
\index{addEntry(int, String, MenuAction)}
{\bf  addEntry}\\
\begin{lstlisting}[frame=none]
public Menu addEntry(int call_index,java.lang.String action_str,MenuAction action)\end{lstlisting} %end signature
\begin{itemize}
\item{
{\bf  Description}

adds an entry to the menu
}
\item{
{\bf  Parameters}
  \begin{itemize}
   \item{
\texttt{action\_str} -- the string the entry will be described with}
   \item{
\texttt{action} -- a \texttt{\small MenuAction}{\small 
\refdefined{it.matlice.ingsw.view.menu.MenuAction}} or null for a default action (returns call index)}
   \item{
\texttt{call\_index} -- the index the user will choose with}
  \end{itemize}
}%end item
\item{{\bf  Returns} -- 
this 
}%end item
\end{itemize}
}%end item
\item{ 
\index{addEntry(int, String, MenuAction, int)}
{\bf  addEntry}\\
\begin{lstlisting}[frame=none]
public Menu addEntry(int call_index,java.lang.String action_str,MenuAction action,int position)\end{lstlisting} %end signature
\begin{itemize}
\item{
{\bf  Description}

adds an entry to the menu
}
\item{
{\bf  Parameters}
  \begin{itemize}
   \item{
\texttt{action\_str} -- the string the entry will be described with}
   \item{
\texttt{action} -- a \texttt{\small MenuAction}{\small 
\refdefined{it.matlice.ingsw.view.menu.MenuAction}} or null for a default action (returns call index)}
   \item{
\texttt{position} -- the position in the list the entry will be displayed: negative values start from the end (-1 = last) or null to assign it based on the number of elements in the menu}
   \item{
\texttt{call\_index} -- the index the user will choose with}
  \end{itemize}
}%end item
\item{{\bf  Returns} -- 
this 
}%end item
\end{itemize}
}%end item
\item{ 
\index{addEntry(MenuEntry)}
{\bf  addEntry}\\
\begin{lstlisting}[frame=none]
public Menu addEntry(MenuEntry entry)\end{lstlisting} %end signature
\begin{itemize}
\item{
{\bf  Description}

adds an entry to the menu
}
\item{
{\bf  Parameters}
  \begin{itemize}
   \item{
\texttt{entry} -- the entry}
  \end{itemize}
}%end item
\item{{\bf  Returns} -- 
this 
}%end item
\end{itemize}
}%end item
\item{ 
\index{addEntry(String, MenuAction)}
{\bf  addEntry}\\
\begin{lstlisting}[frame=none]
public Menu addEntry(java.lang.String action_str,MenuAction action)\end{lstlisting} %end signature
\begin{itemize}
\item{
{\bf  Description}

adds an entry to the menu. position will be assigned incrementally based oin the number of elements in the menu
}
\item{
{\bf  Parameters}
  \begin{itemize}
   \item{
\texttt{action\_str} -- the string the entry will be described with}
   \item{
\texttt{action} -- a \texttt{\small MenuAction}{\small 
\refdefined{it.matlice.ingsw.view.menu.MenuAction}} or null for a default action (returns call index)}
  \end{itemize}
}%end item
\item{{\bf  Returns} -- 
this 
}%end item
\end{itemize}
}%end item
\item{ 
\index{addEntry(String, MenuAction, int)}
{\bf  addEntry}\\
\begin{lstlisting}[frame=none]
public Menu addEntry(java.lang.String action_str,MenuAction action,int position)\end{lstlisting} %end signature
\begin{itemize}
\item{
{\bf  Description}

adds an entry to the menu
}
\item{
{\bf  Parameters}
  \begin{itemize}
   \item{
\texttt{action\_str} -- the string the entry will be described with}
   \item{
\texttt{action} -- a \texttt{\small MenuAction}{\small 
\refdefined{it.matlice.ingsw.view.menu.MenuAction}} or null for a default action (returns call index)}
   \item{
\texttt{position} -- the position in the list the entry will be displayed: negative values start from the end (-1 = last)}
  \end{itemize}
}%end item
\item{{\bf  Returns} -- 
this 
}%end item
\end{itemize}
}%end item
\item{ 
\index{disable(boolean)}
{\bf  disable}\\
\begin{lstlisting}[frame=none]
public Menu disable(boolean disable)\end{lstlisting} %end signature
\begin{itemize}
\item{
{\bf  Description}

Sets the state of the last added entry
}
\item{
{\bf  Parameters}
  \begin{itemize}
   \item{
\texttt{disable} -- the state to be assumed}
  \end{itemize}
}%end item
\item{{\bf  Returns} -- 
this 
}%end item
\end{itemize}
}%end item
\item{ 
\index{display(Scanner, PrintStream)}
{\bf  display}\\
\begin{lstlisting}[frame=none]
public java.lang.Object display(java.util.Scanner in,java.io.PrintStream out)\end{lstlisting} %end signature
\begin{itemize}
\item{
{\bf  Description}

displays the menu on the output stream until the user inserts a valid answer
}
\item{
{\bf  Parameters}
  \begin{itemize}
   \item{
\texttt{in} -- input stream where to get user input}
   \item{
\texttt{out} -- output stream to print data}
  \end{itemize}
}%end item
\item{{\bf  Returns} -- 
the result of the call to \texttt{\small MenuAction}{\small 
\refdefined{it.matlice.ingsw.view.menu.MenuAction}} 
}%end item
\end{itemize}
}%end item
\item{ 
\index{displayOnce(Scanner, PrintStream)}
{\bf  displayOnce}\\
\begin{lstlisting}[frame=none]
public java.lang.Object displayOnce(java.util.Scanner in,java.io.PrintStream out)\end{lstlisting} %end signature
\begin{itemize}
\item{
{\bf  Description}

displays the menu on the output stream
}
\item{
{\bf  Parameters}
  \begin{itemize}
   \item{
\texttt{in} -- input stream where to get user input}
   \item{
\texttt{out} -- output stream to print data}
  \end{itemize}
}%end item
\item{{\bf  Returns} -- 
the result of the call to \texttt{\small MenuAction}{\small 
\refdefined{it.matlice.ingsw.view.menu.MenuAction}} 
}%end item
\end{itemize}
}%end item
\item{ 
\index{getEntry(int)}
{\bf  getEntry}\\
\begin{lstlisting}[frame=none]
public MenuEntry getEntry(int call_reference)\end{lstlisting} %end signature
\begin{itemize}
\item{
{\bf  Description}

Gets the reference to a MenuEntry
}
\item{
{\bf  Parameters}
  \begin{itemize}
   \item{
\texttt{call\_reference} -- the integer to be typed to load that entry}
  \end{itemize}
}%end item
\item{{\bf  Returns} -- 
the entry reference or null if inexistent 
}%end item
\end{itemize}
}%end item
\item{ 
\index{getEntryByPosition(int)}
{\bf  getEntryByPosition}\\
\begin{lstlisting}[frame=none]
public MenuEntry getEntryByPosition(int position_reference)\end{lstlisting} %end signature
\begin{itemize}
\item{
{\bf  Description}

Gets the reference to a MenuEntry
}
\item{
{\bf  Parameters}
  \begin{itemize}
   \item{
\texttt{position\_reference} -- the integer to be typed to load that entry}
  \end{itemize}
}%end item
\item{{\bf  Returns} -- 
the entry reference or null if inexistent 
}%end item
\end{itemize}
}%end item
\item{ 
\index{onlyDisplay(PrintStream)}
{\bf  onlyDisplay}\\
\begin{lstlisting}[frame=none]
public void onlyDisplay(java.io.PrintStream out)\end{lstlisting} %end signature
\begin{itemize}
\item{
{\bf  Description}

shows the menu without asking for any input. input should be managed by the user
}
\item{
{\bf  Parameters}
  \begin{itemize}
   \item{
\texttt{out} -- }
  \end{itemize}
}%end item
\end{itemize}
}%end item
\item{ 
\index{payload(Object)}
{\bf  payload}\\
\begin{lstlisting}[frame=none]
public Menu payload(java.lang.Object payload)\end{lstlisting} %end signature
\begin{itemize}
\item{
{\bf  Description}

Sets the last added entry's payload
}
\item{
{\bf  Parameters}
  \begin{itemize}
   \item{
\texttt{payload} -- the object to be passed}
  \end{itemize}
}%end item
\item{{\bf  Returns} -- 
this 
}%end item
\end{itemize}
}%end item
\item{ 
\index{selectAfterDisplay(Scanner, PrintStream)}
{\bf  selectAfterDisplay}\\
\begin{lstlisting}[frame=none]
public java.lang.Object selectAfterDisplay(java.util.Scanner in,java.io.PrintStream out)\end{lstlisting} %end signature
\begin{itemize}
\item{
{\bf  Description}

asks for an entry of the menu
}
\item{
{\bf  Parameters}
  \begin{itemize}
   \item{
\texttt{in} -- }
   \item{
\texttt{out} -- }
  \end{itemize}
}%end item
\item{{\bf  Returns} -- 
the return of the action lambda 
}%end item
\end{itemize}
}%end item
\item{ 
\index{setPrompt(String)}
{\bf  setPrompt}\\
\begin{lstlisting}[frame=none]
public Menu setPrompt(java.lang.String prompt)\end{lstlisting} %end signature
\begin{itemize}
\item{
{\bf  Description}

sets the ask prompt value
}
\item{
{\bf  Parameters}
  \begin{itemize}
   \item{
\texttt{prompt} -- the string to be prompted}
  \end{itemize}
}%end item
\item{{\bf  Returns} -- 
this 
}%end item
\end{itemize}
}%end item
\end{itemize}
}
}
\section{\label{it.matlice.ingsw.view.menu.MenuEntry}\index{MenuEntry}Class MenuEntry}{
\vskip .1in 
\subsection{Declaration}{
\begin{lstlisting}[frame=none]
public class MenuEntry
 extends java.lang.Object implements java.lang.Comparable\end{lstlisting}
\subsection{Constructor summary}{
\begin{verse}
{\bf MenuEntry(String, int, int, MenuAction, Object, boolean)} \\
\end{verse}
}
\subsection{Method summary}{
\begin{verse}
{\bf compareTo(MenuEntry)} \\
{\bf getAction()} \\
{\bf getCall\_index()} \\
{\bf getName()} \\
{\bf getPayload()} \\
{\bf getPos\_index()} \\
{\bf isDisabled()} \\
{\bf setDisabled(boolean)} \\
{\bf setPayload(Object)} \\
{\bf toString()} \\
\end{verse}
}
\subsection{Constructors}{
\vskip -2em
\begin{itemize}
\item{ 
\index{MenuEntry(String, int, int, MenuAction, Object, boolean)}
{\bf  MenuEntry}\\
\begin{lstlisting}[frame=none]
public MenuEntry(java.lang.String name,int pos_index,int call_index,MenuAction action,java.lang.Object payload,boolean disabled)\end{lstlisting} %end signature
}%end item
\end{itemize}
}
\subsection{Methods}{
\vskip -2em
\begin{itemize}
\item{ 
\index{compareTo(MenuEntry)}
{\bf  compareTo}\\
\begin{lstlisting}[frame=none]
public int compareTo(MenuEntry o)\end{lstlisting} %end signature
}%end item
\item{ 
\index{getAction()}
{\bf  getAction}\\
\begin{lstlisting}[frame=none]
public MenuAction getAction()\end{lstlisting} %end signature
}%end item
\item{ 
\index{getCall\_index()}
{\bf  getCall\_index}\\
\begin{lstlisting}[frame=none]
public int getCall_index()\end{lstlisting} %end signature
}%end item
\item{ 
\index{getName()}
{\bf  getName}\\
\begin{lstlisting}[frame=none]
public java.lang.String getName()\end{lstlisting} %end signature
}%end item
\item{ 
\index{getPayload()}
{\bf  getPayload}\\
\begin{lstlisting}[frame=none]
public java.lang.Object getPayload()\end{lstlisting} %end signature
}%end item
\item{ 
\index{getPos\_index()}
{\bf  getPos\_index}\\
\begin{lstlisting}[frame=none]
public int getPos_index()\end{lstlisting} %end signature
}%end item
\item{ 
\index{isDisabled()}
{\bf  isDisabled}\\
\begin{lstlisting}[frame=none]
public boolean isDisabled()\end{lstlisting} %end signature
}%end item
\item{ 
\index{setDisabled(boolean)}
{\bf  setDisabled}\\
\begin{lstlisting}[frame=none]
public void setDisabled(boolean disabled)\end{lstlisting} %end signature
}%end item
\item{ 
\index{setPayload(Object)}
{\bf  setPayload}\\
\begin{lstlisting}[frame=none]
public void setPayload(java.lang.Object payload)\end{lstlisting} %end signature
}%end item
\item{ 
\index{toString()}
{\bf  toString}\\
\begin{lstlisting}[frame=none]
public java.lang.String toString()\end{lstlisting} %end signature
}%end item
\end{itemize}
}
}
}
\chapter{Package it.matlice.ingsw.view.stream}{
\label{it.matlice.ingsw.view.stream}\hskip -.05in
\hbox to \hsize{\textit{ Package Contents\hfil Page}}
\vskip .13in
\hbox{{\bf  Classes}}
\entityintro{StreamView}{it.matlice.ingsw.view.stream.StreamView}{Classe che implementa l'interfaccia View dell'applicazione Si interfaccia all'utente tramite stream di input e di output (ad esempio stdin e stdout)}
\vskip .1in
\vskip .1in
\section{\label{it.matlice.ingsw.view.stream.StreamView}\index{StreamView}Class StreamView}{
\vskip .1in 
Classe che implementa l'interfaccia View dell'applicazione Si interfaccia all'utente tramite stream di input e di output (ad esempio stdin e stdout)\vskip .1in 
\subsection{Declaration}{
\begin{lstlisting}[frame=none]
public class StreamView
 extends java.lang.Object implements it.matlice.ingsw.view.View\end{lstlisting}
\subsection{Constructor summary}{
\begin{verse}
{\bf StreamView(PrintStream, Scanner)} Costruttore per StreamView\\
\end{verse}
}
\subsection{Method summary}{
\begin{verse}
{\bf chooseOption(List, String)} Richiede all'utente la scelta di un'opzione\\
{\bf error(String)} Comunica un errore all'utente\\
{\bf get(String)} Richiede all'utente l'inserimento di una stringa\\
{\bf getGenericList(String, boolean, Function)} Ritorna una lista di oggetti, inseriti dall'utente come stringa dall'utente e convertiti tramite una mappa di conversione\\
{\bf getGenericList(String, boolean, Function, String, String)} Ritorna una lista di oggetti, inseriti dall'utente come stringa dall'utente e convertiti tramite una mappa di conversione\\
{\bf getInt(String)} Richiede all'utente l'inserimento di un valore intero\\
{\bf getInt(String, Function)} Richiede all'utente l'inserimento di un valore intero\\
{\bf getInt(String, Function, String)} Richiede all'utente l'inserimento di un valore intero\\
{\bf getLine(String)} Richiede all'utente l'inserimento di una stringa (compresa di blanks) Ritorna anche stringa vuota\\
{\bf getLine(String, Function)} \\
{\bf getLineWithConversion(String, Function, String)} Richiede all'utente l'inserimento di una stringa, che verrà convertita in un oggetto tramite una funzione di conversione\\
{\bf getPassword()} Richiede all'utente l'inserimento di una password (separato dal metodo `get()` per far sì che l'inserimento della password possa essere offuscato\\
{\bf getPassword(String)} Richiede all'utente l'inserimento di una password (separato dal metodo `get()` per far sì che l'inserimento della password possa essere offuscato\\
{\bf getStringList(String, boolean)} Ritorna una lista di stringhe non vuote inserite dall'utente\\
{\bf getStringList(String, boolean, String, String)} Ritorna una lista di stringhe non vuote inserite dall'utente\\
{\bf getTrimmedLine(String, boolean)} Richiede all'utente l'inserimento di una stringa (a cui sono rimossi i blank iniziali e finali) Permette di specificare se la stringa immessa deve essere non vuota\\
{\bf info(String)} Comunica un messaggio all'utente\\
{\bf info(String, boolean)} Comunica un messaggio all'utente\\
{\bf showList(String, List)} Show a list of objects as strings\\
{\bf warn(String)} Comunica un avvertimento all'utente\\
{\bf warn(String, boolean)} Comunica un avvertimento all'utente\\
\end{verse}
}
\subsection{Constructors}{
\vskip -2em
\begin{itemize}
\item{ 
\index{StreamView(PrintStream, Scanner)}
{\bf  StreamView}\\
\begin{lstlisting}[frame=none]
public StreamView(java.io.PrintStream out,java.util.Scanner in)\end{lstlisting} %end signature
\begin{itemize}
\item{
{\bf  Description}

Costruttore per StreamView
}
\item{
{\bf  Parameters}
  \begin{itemize}
   \item{
\texttt{out} -- stream di output verso l'utente}
   \item{
\texttt{in} -- stream di input dall'utente}
  \end{itemize}
}%end item
\end{itemize}
}%end item
\end{itemize}
}
\subsection{Methods}{
\vskip -2em
\begin{itemize}
\item{ 
\index{chooseOption(List, String)}
{\bf  chooseOption}\\
\begin{lstlisting}[frame=none]
public it.matlice.ingsw.controller.MenuAction chooseOption(java.util.List choices,java.lang.String prompt)\end{lstlisting} %end signature
\begin{itemize}
\item{
{\bf  Description}

Richiede all'utente la scelta di un'opzione
}
\item{
{\bf  Parameters}
  \begin{itemize}
   \item{
\texttt{choices} -- lista di opzioni disponibili tra cui scegliere}
   \item{
\texttt{prompt} -- prompt della richiesta di scelta}
  \end{itemize}
}%end item
\item{{\bf  Returns} -- 
l'azione scelta dall'utente 
}%end item
\end{itemize}
}%end item
\item{ 
\index{error(String)}
{\bf  error}\\
\begin{lstlisting}[frame=none]
public void error(java.lang.String text)\end{lstlisting} %end signature
\begin{itemize}
\item{
{\bf  Description}

Comunica un errore all'utente
}
\item{
{\bf  Parameters}
  \begin{itemize}
   \item{
\texttt{text} -- il testo dell'errore}
  \end{itemize}
}%end item
\end{itemize}
}%end item
\item{ 
\index{get(String)}
{\bf  get}\\
\begin{lstlisting}[frame=none]
public java.lang.String get(java.lang.String prompt)\end{lstlisting} %end signature
\begin{itemize}
\item{
{\bf  Description}

Richiede all'utente l'inserimento di una stringa
}
\item{
{\bf  Parameters}
  \begin{itemize}
   \item{
\texttt{prompt} -- }
  \end{itemize}
}%end item
\item{{\bf  Returns} -- 
la stringa inserita 
}%end item
\end{itemize}
}%end item
\item{ 
\index{getGenericList(String, boolean, Function)}
{\bf  getGenericList}\\
\begin{lstlisting}[frame=none]
public java.util.List getGenericList(java.lang.String prompt,boolean unique,java.util.function.Function conversionMap)\end{lstlisting} %end signature
\begin{itemize}
\item{
{\bf  Description}

Ritorna una lista di oggetti, inseriti dall'utente come stringa dall'utente e convertiti tramite una mappa di conversione
}
\item{
{\bf  Parameters}
  \begin{itemize}
   \item{
\texttt{prompt} -- messaggio per l'inserimento}
   \item{
\texttt{unique} -- true se non ci possono essere ripetizioni}
   \item{
\texttt{conversionMap} -- funzione che mappa i possibili input (stringhe) agli oggetti V deve ritornare null per valori di stringhe non validi}
  \end{itemize}
}%end item
\item{{\bf  Returns} -- 
lista di oggetti inseriti dall'utente 
}%end item
\end{itemize}
}%end item
\item{ 
\index{getGenericList(String, boolean, Function, String, String)}
{\bf  getGenericList}\\
\begin{lstlisting}[frame=none]
public java.util.List getGenericList(java.lang.String prompt,boolean unique,java.util.function.Function conversionMap,java.lang.String duplicateErrorMessage,java.lang.String nonValidErrorMessage)\end{lstlisting} %end signature
\begin{itemize}
\item{
{\bf  Description}

Ritorna una lista di oggetti, inseriti dall'utente come stringa dall'utente e convertiti tramite una mappa di conversione
}
\item{
{\bf  Parameters}
  \begin{itemize}
   \item{
\texttt{prompt} -- messaggio per l'inserimento}
   \item{
\texttt{unique} -- true se non ci possono essere ripetizioni}
   \item{
\texttt{conversionMap} -- funzione che mappa i possibili input (stringhe) agli oggetti V deve ritornare null per valori di stringhe non validi}
   \item{
\texttt{duplicateErrorMessage} -- messaggio di errore per valori già inseriti}
   \item{
\texttt{nonValidErrorMessage} -- messaggio di errore per valori non validi}
  \end{itemize}
}%end item
\item{{\bf  Returns} -- 
lista di oggetti inseriti dall'utente 
}%end item
\end{itemize}
}%end item
\item{ 
\index{getInt(String)}
{\bf  getInt}\\
\begin{lstlisting}[frame=none]
public int getInt(java.lang.String prompt)\end{lstlisting} %end signature
\begin{itemize}
\item{
{\bf  Description}

Richiede all'utente l'inserimento di un valore intero
}
\item{
{\bf  Parameters}
  \begin{itemize}
   \item{
\texttt{prompt} -- messaggio di richiesta all'utente}
  \end{itemize}
}%end item
\item{{\bf  Returns} -- 
l'intero inserito 
}%end item
\end{itemize}
}%end item
\item{ 
\index{getInt(String, Function)}
{\bf  getInt}\\
\begin{lstlisting}[frame=none]
public int getInt(java.lang.String prompt,java.util.function.Function available)\end{lstlisting} %end signature
\begin{itemize}
\item{
{\bf  Description}

Richiede all'utente l'inserimento di un valore intero
}
\item{
{\bf  Parameters}
  \begin{itemize}
   \item{
\texttt{prompt} -- messaggio di richiesta all'utente}
   \item{
\texttt{available} -- returns true se l'intero inserito è valido, false per richiederlo}
  \end{itemize}
}%end item
\item{{\bf  Returns} -- 
l'intero inserito 
}%end item
\end{itemize}
}%end item
\item{ 
\index{getInt(String, Function, String)}
{\bf  getInt}\\
\begin{lstlisting}[frame=none]
public int getInt(java.lang.String prompt,java.util.function.Function available,java.lang.String nonValidErrorMessage)\end{lstlisting} %end signature
\begin{itemize}
\item{
{\bf  Description}

Richiede all'utente l'inserimento di un valore intero
}
\item{
{\bf  Parameters}
  \begin{itemize}
   \item{
\texttt{prompt} -- messaggio di richiesta all'utente}
   \item{
\texttt{available} -- returns true se l'intero inserito è valido, false per richiederlo}
   \item{
\texttt{nonValidErrorMessage} -- messaggio di errore per valori non validi}
  \end{itemize}
}%end item
\item{{\bf  Returns} -- 
l'intero inserito 
}%end item
\end{itemize}
}%end item
\item{ 
\index{getLine(String)}
{\bf  getLine}\\
\begin{lstlisting}[frame=none]
public java.lang.String getLine(java.lang.String prompt)\end{lstlisting} %end signature
\begin{itemize}
\item{
{\bf  Description}

Richiede all'utente l'inserimento di una stringa (compresa di blanks) Ritorna anche stringa vuota
}
\item{
{\bf  Parameters}
  \begin{itemize}
   \item{
\texttt{prompt} -- messaggio di richiesta all'utente}
  \end{itemize}
}%end item
\item{{\bf  Returns} -- 
la stringa inserita 
}%end item
\end{itemize}
}%end item
\item{ 
\index{getLine(String, Function)}
{\bf  getLine}\\
\begin{lstlisting}[frame=none]
java.lang.String getLine(java.lang.String prompt,java.util.function.Function available)\end{lstlisting} %end signature
\begin{itemize}
\item{
{\bf  Description copied from it.matlice.ingsw.view.View{\small \refdefined{it.matlice.ingsw.view.View}} }

Richiede all'utente l'inserimento di una stringa (compresa di blanks), validata secondo una funzione
}
\item{
{\bf  Parameters}
  \begin{itemize}
   \item{
\texttt{prompt} -- messaggio di richiesta all'utente}
   \item{
\texttt{available} -- funzione che ritorna true se la stringa in input è valida}
  \end{itemize}
}%end item
\item{{\bf  Returns} -- 
la stringa inserita 
}%end item
\end{itemize}
}%end item
\item{ 
\index{getLineWithConversion(String, Function, String)}
{\bf  getLineWithConversion}\\
\begin{lstlisting}[frame=none]
public java.lang.Object getLineWithConversion(java.lang.String prompt,java.util.function.Function conversionMap,java.lang.String nonValidErrorMessage)\end{lstlisting} %end signature
\begin{itemize}
\item{
{\bf  Description}

Richiede all'utente l'inserimento di una stringa, che verrà convertita in un oggetto tramite una funzione di conversione
}
\item{
{\bf  Parameters}
  \begin{itemize}
   \item{
\texttt{prompt} -- messaggio di richiesta all'utente}
   \item{
\texttt{conversionMap} -- funzione di conversione}
   \item{
\texttt{nonValidErrorMessage} -- errore durante il parsing}
  \end{itemize}
}%end item
\item{{\bf  Returns} -- 
oggetto creato da stringa 
}%end item
\end{itemize}
}%end item
\item{ 
\index{getPassword()}
{\bf  getPassword}\\
\begin{lstlisting}[frame=none]
public java.lang.String getPassword()\end{lstlisting} %end signature
\begin{itemize}
\item{
{\bf  Description}

Richiede all'utente l'inserimento di una password (separato dal metodo `get()` per far sì che l'inserimento della password possa essere offuscato
}
\item{{\bf  Returns} -- 
la password inserita 
}%end item
\end{itemize}
}%end item
\item{ 
\index{getPassword(String)}
{\bf  getPassword}\\
\begin{lstlisting}[frame=none]
public java.lang.String getPassword(java.lang.String prompt)\end{lstlisting} %end signature
\begin{itemize}
\item{
{\bf  Description}

Richiede all'utente l'inserimento di una password (separato dal metodo `get()` per far sì che l'inserimento della password possa essere offuscato
}
\item{
{\bf  Parameters}
  \begin{itemize}
   \item{
\texttt{prompt} -- prompt della richiesta della password}
  \end{itemize}
}%end item
\item{{\bf  Returns} -- 
la password inserita 
}%end item
\end{itemize}
}%end item
\item{ 
\index{getStringList(String, boolean)}
{\bf  getStringList}\\
\begin{lstlisting}[frame=none]
public java.util.List getStringList(java.lang.String prompt,boolean unique)\end{lstlisting} %end signature
\begin{itemize}
\item{
{\bf  Description}

Ritorna una lista di stringhe non vuote inserite dall'utente
}
\item{
{\bf  Parameters}
  \begin{itemize}
   \item{
\texttt{prompt} -- messaggio per l'inserimento}
   \item{
\texttt{unique} -- true se non ci possono essere ripetizioni}
  \end{itemize}
}%end item
\item{{\bf  Returns} -- 
lista di stringhe inserite dall'utente 
}%end item
\end{itemize}
}%end item
\item{ 
\index{getStringList(String, boolean, String, String)}
{\bf  getStringList}\\
\begin{lstlisting}[frame=none]
public java.util.List getStringList(java.lang.String prompt,boolean unique,java.lang.String duplicateErrorMessage,java.lang.String nonValidErrorMessage)\end{lstlisting} %end signature
\begin{itemize}
\item{
{\bf  Description}

Ritorna una lista di stringhe non vuote inserite dall'utente
}
\item{
{\bf  Parameters}
  \begin{itemize}
   \item{
\texttt{prompt} -- messaggio per l'inserimento}
   \item{
\texttt{unique} -- true se non ci possono essere ripetizioni}
   \item{
\texttt{duplicateErrorMessage} -- messaggio di errore per valori già inseriti}
   \item{
\texttt{nonValidErrorMessage} -- messaggio di errore per valori non validi}
  \end{itemize}
}%end item
\item{{\bf  Returns} -- 
lista di stringhe inserite dall'utente 
}%end item
\end{itemize}
}%end item
\item{ 
\index{getTrimmedLine(String, boolean)}
{\bf  getTrimmedLine}\\
\begin{lstlisting}[frame=none]
public java.lang.String getTrimmedLine(java.lang.String prompt,boolean canBeEmpty)\end{lstlisting} %end signature
\begin{itemize}
\item{
{\bf  Description}

Richiede all'utente l'inserimento di una stringa (a cui sono rimossi i blank iniziali e finali) Permette di specificare se la stringa immessa deve essere non vuota
}
\item{
{\bf  Parameters}
  \begin{itemize}
   \item{
\texttt{prompt} -- messaggio di richiesta all'utente}
   \item{
\texttt{canBeEmpty} -- false if the input string must not be empty}
  \end{itemize}
}%end item
\item{{\bf  Returns} -- 
la stringa inserita 
}%end item
\end{itemize}
}%end item
\item{ 
\index{info(String)}
{\bf  info}\\
\begin{lstlisting}[frame=none]
public void info(java.lang.String text)\end{lstlisting} %end signature
\begin{itemize}
\item{
{\bf  Description}

Comunica un messaggio all'utente
}
\item{
{\bf  Parameters}
  \begin{itemize}
   \item{
\texttt{text} -- il testo del messaggio}
  \end{itemize}
}%end item
\end{itemize}
}%end item
\item{ 
\index{info(String, boolean)}
{\bf  info}\\
\begin{lstlisting}[frame=none]
public void info(java.lang.String text,boolean separated)\end{lstlisting} %end signature
\begin{itemize}
\item{
{\bf  Description}

Comunica un messaggio all'utente
}
\item{
{\bf  Parameters}
  \begin{itemize}
   \item{
\texttt{text} -- il testo del messaggio}
   \item{
\texttt{separated} -- true se si separa dal contesto del precedente messaggio}
  \end{itemize}
}%end item
\end{itemize}
}%end item
\item{ 
\index{showList(String, List)}
{\bf  showList}\\
\begin{lstlisting}[frame=none]
public void showList(java.lang.String message,java.util.List list)\end{lstlisting} %end signature
\begin{itemize}
\item{
{\bf  Description}

Show a list of objects as strings
}
\item{
{\bf  Parameters}
  \begin{itemize}
   \item{
\texttt{message} -- message to show before the list}
   \item{
\texttt{list} -- list of string to show}
  \end{itemize}
}%end item
\end{itemize}
}%end item
\item{ 
\index{warn(String)}
{\bf  warn}\\
\begin{lstlisting}[frame=none]
public void warn(java.lang.String text)\end{lstlisting} %end signature
\begin{itemize}
\item{
{\bf  Description}

Comunica un avvertimento all'utente
}
\item{
{\bf  Parameters}
  \begin{itemize}
   \item{
\texttt{text} -- il testo del messaggio}
  \end{itemize}
}%end item
\end{itemize}
}%end item
\item{ 
\index{warn(String, boolean)}
{\bf  warn}\\
\begin{lstlisting}[frame=none]
public void warn(java.lang.String text,boolean separated)\end{lstlisting} %end signature
\begin{itemize}
\item{
{\bf  Description}

Comunica un avvertimento all'utente
}
\item{
{\bf  Parameters}
  \begin{itemize}
   \item{
\texttt{text} -- il testo del messaggio}
   \item{
\texttt{separated} -- true se si separa dal contesto del precedente messaggio}
  \end{itemize}
}%end item
\end{itemize}
}%end item
\end{itemize}
}
}
}
\chapter{Package it.matlice.ingsw.view}{
\label{it.matlice.ingsw.view}\hskip -.05in
\hbox to \hsize{\textit{ Package Contents\hfil Page}}
\vskip .13in
\hbox{{\bf  Interfaces}}
\entityintro{View}{it.matlice.ingsw.view.View}{Interfaccia per la view dell'applicazione}
\vskip .1in
\vskip .1in
\section{\label{it.matlice.ingsw.view.View}\index{View@\textit{ View}}Interface View}{
\vskip .1in 
Interfaccia per la view dell'applicazione\vskip .1in 
\subsection{Declaration}{
\begin{lstlisting}[frame=none]
public interface View
\end{lstlisting}
\subsection{All known subinterfaces}{StreamView\small{\refdefined{it.matlice.ingsw.view.stream.StreamView}}}
\subsection{All classes known to implement interface}{StreamView\small{\refdefined{it.matlice.ingsw.view.stream.StreamView}}}
\subsection{Method summary}{
\begin{verse}
{\bf chooseOption(List)} Richiede all'utente la scelta di un'opzione\\
{\bf chooseOption(List, String)} Richiede all'utente la scelta di un'opzione\\
{\bf error(String)} Comunica un errore all'utente\\
{\bf get(String)} Richiede all'utente l'inserimento di una stringa\\
{\bf getGenericList(String, boolean, Function)} Ritorna una lista di oggetti, inseriti dall'utente come stringa dall'utente e convertiti tramite una mappa di conversione\\
{\bf getGenericList(String, boolean, Function, String, String)} Ritorna una lista di oggetti, inseriti dall'utente come stringa dall'utente e convertiti tramite una mappa di conversione\\
{\bf getInt(String)} Richiede all'utente l'inserimento di un valore intero\\
{\bf getInt(String, Function)} Richiede all'utente l'inserimento di un valore intero\\
{\bf getInt(String, Function, String)} Richiede all'utente l'inserimento di un valore intero\\
{\bf getLine(String)} Richiede all'utente l'inserimento di una stringa (compresa di blanks)\\
{\bf getLine(String, Function)} Richiede all'utente l'inserimento di una stringa (compresa di blanks), validata secondo una funzione\\
{\bf getLineWithConversion(String, Function, String)} Richiede all'utente l'inserimento di una stringa, che verrà convertita in un oggetto tramite una funzione di conversione\\
{\bf getPassword()} Richiede all'utente l'inserimento di una password (separato dal metodo `get()` per far sì che l'inserimento della password possa essere offuscato\\
{\bf getPassword(String)} Richiede all'utente l'inserimento di una password (separato dal metodo `get()` per far sì che l'inserimento della password possa essere offuscato\\
{\bf getStringList(String, boolean)} Ritorna una lista di stringhe non vuote inserite dall'utente\\
{\bf getStringList(String, boolean, String, String)} Ritorna una lista di stringhe non vuote inserite dall'utente\\
{\bf getTrimmedLine(String, boolean)} Richiede all'utente l'inserimento di una stringa (a cui sono rimossi i blank iniziali e finali) Permette di specificare se la stringa immessa deve essere non vuota\\
{\bf info(String)} Comunica un messaggio all'utente\\
{\bf info(String, boolean)} Comunica un messaggio all'utente\\
{\bf showList(String, List)} Show a list of objects as strings\\
{\bf warn(String)} Comunica un avvertimento all'utente\\
{\bf warn(String, boolean)} Comunica un avvertimento all'utente\\
\end{verse}
}
\subsection{Methods}{
\vskip -2em
\begin{itemize}
\item{ 
\index{chooseOption(List)}
{\bf  chooseOption}\\
\begin{lstlisting}[frame=none]
it.matlice.ingsw.controller.MenuAction chooseOption(java.util.List choices)\end{lstlisting} %end signature
\begin{itemize}
\item{
{\bf  Description}

Richiede all'utente la scelta di un'opzione
}
\item{
{\bf  Parameters}
  \begin{itemize}
   \item{
\texttt{choices} -- lista di opzioni disponibili tra cui scegliere}
  \end{itemize}
}%end item
\item{{\bf  Returns} -- 
l'azione scelta dall'utente 
}%end item
\end{itemize}
}%end item
\item{ 
\index{chooseOption(List, String)}
{\bf  chooseOption}\\
\begin{lstlisting}[frame=none]
it.matlice.ingsw.controller.MenuAction chooseOption(java.util.List choices,java.lang.String prompt)\end{lstlisting} %end signature
\begin{itemize}
\item{
{\bf  Description}

Richiede all'utente la scelta di un'opzione
}
\item{
{\bf  Parameters}
  \begin{itemize}
   \item{
\texttt{choices} -- lista di opzioni disponibili tra cui scegliere}
   \item{
\texttt{prompt} -- prompt della richiesta di scelta}
  \end{itemize}
}%end item
\item{{\bf  Returns} -- 
l'azione scelta dall'utente 
}%end item
\end{itemize}
}%end item
\item{ 
\index{error(String)}
{\bf  error}\\
\begin{lstlisting}[frame=none]
void error(java.lang.String text)\end{lstlisting} %end signature
\begin{itemize}
\item{
{\bf  Description}

Comunica un errore all'utente
}
\item{
{\bf  Parameters}
  \begin{itemize}
   \item{
\texttt{text} -- il testo dell'errore}
  \end{itemize}
}%end item
\end{itemize}
}%end item
\item{ 
\index{get(String)}
{\bf  get}\\
\begin{lstlisting}[frame=none]
java.lang.String get(java.lang.String prompt)\end{lstlisting} %end signature
\begin{itemize}
\item{
{\bf  Description}

Richiede all'utente l'inserimento di una stringa
}
\item{{\bf  Returns} -- 
la stringa inserita 
}%end item
\end{itemize}
}%end item
\item{ 
\index{getGenericList(String, boolean, Function)}
{\bf  getGenericList}\\
\begin{lstlisting}[frame=none]
java.util.List getGenericList(java.lang.String prompt,boolean unique,java.util.function.Function conversionMap)\end{lstlisting} %end signature
\begin{itemize}
\item{
{\bf  Description}

Ritorna una lista di oggetti, inseriti dall'utente come stringa dall'utente e convertiti tramite una mappa di conversione
}
\item{
{\bf  Parameters}
  \begin{itemize}
   \item{
\texttt{prompt} -- messaggio per l'inserimento}
   \item{
\texttt{unique} -- true se non ci possono essere ripetizioni}
   \item{
\texttt{conversionMap} -- funzione che mappa i possibili input (stringhe) agli oggetti V deve ritornare null per valori di stringhe non validi}
  \end{itemize}
}%end item
\item{{\bf  Returns} -- 
lista di oggetti inseriti dall'utente 
}%end item
\end{itemize}
}%end item
\item{ 
\index{getGenericList(String, boolean, Function, String, String)}
{\bf  getGenericList}\\
\begin{lstlisting}[frame=none]
java.util.List getGenericList(java.lang.String prompt,boolean unique,java.util.function.Function conversionMap,java.lang.String duplicateErrorMessage,java.lang.String nonValidErrorMessage)\end{lstlisting} %end signature
\begin{itemize}
\item{
{\bf  Description}

Ritorna una lista di oggetti, inseriti dall'utente come stringa dall'utente e convertiti tramite una mappa di conversione
}
\item{
{\bf  Parameters}
  \begin{itemize}
   \item{
\texttt{prompt} -- messaggio per l'inserimento}
   \item{
\texttt{unique} -- true se non ci possono essere ripetizioni}
   \item{
\texttt{conversionMap} -- funzione che mappa i possibili input (stringhe) agli oggetti V deve ritornare null per valori di stringhe non validi}
   \item{
\texttt{duplicateErrorMessage} -- messaggio di errore per valori già inseriti}
   \item{
\texttt{nonValidErrorMessage} -- messaggio di errore per valori non validi}
  \end{itemize}
}%end item
\item{{\bf  Returns} -- 
lista di oggetti inseriti dall'utente 
}%end item
\end{itemize}
}%end item
\item{ 
\index{getInt(String)}
{\bf  getInt}\\
\begin{lstlisting}[frame=none]
int getInt(java.lang.String prompt)\end{lstlisting} %end signature
\begin{itemize}
\item{
{\bf  Description}

Richiede all'utente l'inserimento di un valore intero
}
\item{
{\bf  Parameters}
  \begin{itemize}
   \item{
\texttt{prompt} -- messaggio di richiesta all'utente}
  \end{itemize}
}%end item
\item{{\bf  Returns} -- 
l'intero inserito 
}%end item
\end{itemize}
}%end item
\item{ 
\index{getInt(String, Function)}
{\bf  getInt}\\
\begin{lstlisting}[frame=none]
int getInt(java.lang.String prompt,java.util.function.Function available)\end{lstlisting} %end signature
\begin{itemize}
\item{
{\bf  Description}

Richiede all'utente l'inserimento di un valore intero
}
\item{
{\bf  Parameters}
  \begin{itemize}
   \item{
\texttt{prompt} -- messaggio di richiesta all'utente}
   \item{
\texttt{available} -- returns true se l'intero inserito è valido, false per richiederlo}
  \end{itemize}
}%end item
\item{{\bf  Returns} -- 
l'intero inserito 
}%end item
\end{itemize}
}%end item
\item{ 
\index{getInt(String, Function, String)}
{\bf  getInt}\\
\begin{lstlisting}[frame=none]
int getInt(java.lang.String prompt,java.util.function.Function available,java.lang.String nonValidErrorMessage)\end{lstlisting} %end signature
\begin{itemize}
\item{
{\bf  Description}

Richiede all'utente l'inserimento di un valore intero
}
\item{
{\bf  Parameters}
  \begin{itemize}
   \item{
\texttt{prompt} -- messaggio di richiesta all'utente}
   \item{
\texttt{available} -- returns true se l'intero inserito è valido, false per richiederlo}
   \item{
\texttt{nonValidErrorMessage} -- messaggio di errore per valori non validi}
  \end{itemize}
}%end item
\item{{\bf  Returns} -- 
l'intero inserito 
}%end item
\end{itemize}
}%end item
\item{ 
\index{getLine(String)}
{\bf  getLine}\\
\begin{lstlisting}[frame=none]
java.lang.String getLine(java.lang.String prompt)\end{lstlisting} %end signature
\begin{itemize}
\item{
{\bf  Description}

Richiede all'utente l'inserimento di una stringa (compresa di blanks)
}
\item{
{\bf  Parameters}
  \begin{itemize}
   \item{
\texttt{prompt} -- messaggio di richiesta all'utente}
  \end{itemize}
}%end item
\item{{\bf  Returns} -- 
la stringa inserita 
}%end item
\end{itemize}
}%end item
\item{ 
\index{getLine(String, Function)}
{\bf  getLine}\\
\begin{lstlisting}[frame=none]
java.lang.String getLine(java.lang.String prompt,java.util.function.Function available)\end{lstlisting} %end signature
\begin{itemize}
\item{
{\bf  Description}

Richiede all'utente l'inserimento di una stringa (compresa di blanks), validata secondo una funzione
}
\item{
{\bf  Parameters}
  \begin{itemize}
   \item{
\texttt{prompt} -- messaggio di richiesta all'utente}
   \item{
\texttt{available} -- funzione che ritorna true se la stringa in input è valida}
  \end{itemize}
}%end item
\item{{\bf  Returns} -- 
la stringa inserita 
}%end item
\end{itemize}
}%end item
\item{ 
\index{getLineWithConversion(String, Function, String)}
{\bf  getLineWithConversion}\\
\begin{lstlisting}[frame=none]
java.lang.Object getLineWithConversion(java.lang.String prompt,java.util.function.Function conversionMap,java.lang.String nonValidErrorMessage)\end{lstlisting} %end signature
\begin{itemize}
\item{
{\bf  Description}

Richiede all'utente l'inserimento di una stringa, che verrà convertita in un oggetto tramite una funzione di conversione
}
\item{
{\bf  Parameters}
  \begin{itemize}
   \item{
\texttt{prompt} -- messaggio di richiesta all'utente}
   \item{
\texttt{conversionMap} -- funzione di conversione}
   \item{
\texttt{nonValidErrorMessage} -- errore durante il parsing}
  \end{itemize}
}%end item
\item{{\bf  Returns} -- 
oggetto creato da stringa 
}%end item
\end{itemize}
}%end item
\item{ 
\index{getPassword()}
{\bf  getPassword}\\
\begin{lstlisting}[frame=none]
java.lang.String getPassword()\end{lstlisting} %end signature
\begin{itemize}
\item{
{\bf  Description}

Richiede all'utente l'inserimento di una password (separato dal metodo `get()` per far sì che l'inserimento della password possa essere offuscato
}
\item{{\bf  Returns} -- 
la password inserita 
}%end item
\end{itemize}
}%end item
\item{ 
\index{getPassword(String)}
{\bf  getPassword}\\
\begin{lstlisting}[frame=none]
java.lang.String getPassword(java.lang.String prompt)\end{lstlisting} %end signature
\begin{itemize}
\item{
{\bf  Description}

Richiede all'utente l'inserimento di una password (separato dal metodo `get()` per far sì che l'inserimento della password possa essere offuscato
}
\item{
{\bf  Parameters}
  \begin{itemize}
   \item{
\texttt{prompt} -- prompt della richiesta della password}
  \end{itemize}
}%end item
\item{{\bf  Returns} -- 
la password inserita 
}%end item
\end{itemize}
}%end item
\item{ 
\index{getStringList(String, boolean)}
{\bf  getStringList}\\
\begin{lstlisting}[frame=none]
java.util.List getStringList(java.lang.String prompt,boolean unique)\end{lstlisting} %end signature
\begin{itemize}
\item{
{\bf  Description}

Ritorna una lista di stringhe non vuote inserite dall'utente
}
\item{
{\bf  Parameters}
  \begin{itemize}
   \item{
\texttt{prompt} -- messaggio per l'inserimento}
   \item{
\texttt{unique} -- true se non ci possono essere ripetizioni}
  \end{itemize}
}%end item
\item{{\bf  Returns} -- 
lista di stringhe inserite dall'utente 
}%end item
\end{itemize}
}%end item
\item{ 
\index{getStringList(String, boolean, String, String)}
{\bf  getStringList}\\
\begin{lstlisting}[frame=none]
java.util.List getStringList(java.lang.String prompt,boolean unique,java.lang.String duplicateErrorMessage,java.lang.String nonValidErrorMessage)\end{lstlisting} %end signature
\begin{itemize}
\item{
{\bf  Description}

Ritorna una lista di stringhe non vuote inserite dall'utente
}
\item{
{\bf  Parameters}
  \begin{itemize}
   \item{
\texttt{prompt} -- messaggio per l'inserimento}
   \item{
\texttt{unique} -- true se non ci possono essere ripetizioni}
   \item{
\texttt{duplicateErrorMessage} -- messaggio di errore per valori già inseriti}
   \item{
\texttt{nonValidErrorMessage} -- messaggio di errore per valori non validi}
  \end{itemize}
}%end item
\item{{\bf  Returns} -- 
lista di stringhe inserite dall'utente 
}%end item
\end{itemize}
}%end item
\item{ 
\index{getTrimmedLine(String, boolean)}
{\bf  getTrimmedLine}\\
\begin{lstlisting}[frame=none]
java.lang.String getTrimmedLine(java.lang.String prompt,boolean canBeEmpty)\end{lstlisting} %end signature
\begin{itemize}
\item{
{\bf  Description}

Richiede all'utente l'inserimento di una stringa (a cui sono rimossi i blank iniziali e finali) Permette di specificare se la stringa immessa deve essere non vuota
}
\item{
{\bf  Parameters}
  \begin{itemize}
   \item{
\texttt{prompt} -- messaggio di richiesta all'utente}
   \item{
\texttt{canBeEmpty} -- false if the input string must not be empty}
  \end{itemize}
}%end item
\item{{\bf  Returns} -- 
la stringa inserita 
}%end item
\end{itemize}
}%end item
\item{ 
\index{info(String)}
{\bf  info}\\
\begin{lstlisting}[frame=none]
void info(java.lang.String text)\end{lstlisting} %end signature
\begin{itemize}
\item{
{\bf  Description}

Comunica un messaggio all'utente
}
\item{
{\bf  Parameters}
  \begin{itemize}
   \item{
\texttt{text} -- il testo del messaggio}
  \end{itemize}
}%end item
\end{itemize}
}%end item
\item{ 
\index{info(String, boolean)}
{\bf  info}\\
\begin{lstlisting}[frame=none]
void info(java.lang.String text,boolean separated)\end{lstlisting} %end signature
\begin{itemize}
\item{
{\bf  Description}

Comunica un messaggio all'utente
}
\item{
{\bf  Parameters}
  \begin{itemize}
   \item{
\texttt{text} -- il testo del messaggio}
   \item{
\texttt{separated} -- true se si separa dal contesto del precedente messaggio}
  \end{itemize}
}%end item
\end{itemize}
}%end item
\item{ 
\index{showList(String, List)}
{\bf  showList}\\
\begin{lstlisting}[frame=none]
void showList(java.lang.String message,java.util.List list)\end{lstlisting} %end signature
\begin{itemize}
\item{
{\bf  Description}

Show a list of objects as strings
}
\item{
{\bf  Parameters}
  \begin{itemize}
   \item{
\texttt{message} -- message to show before the list}
   \item{
\texttt{list} -- list of string to show}
  \end{itemize}
}%end item
\end{itemize}
}%end item
\item{ 
\index{warn(String)}
{\bf  warn}\\
\begin{lstlisting}[frame=none]
void warn(java.lang.String text)\end{lstlisting} %end signature
\begin{itemize}
\item{
{\bf  Description}

Comunica un avvertimento all'utente
}
\item{
{\bf  Parameters}
  \begin{itemize}
   \item{
\texttt{text} -- il testo del messaggio}
  \end{itemize}
}%end item
\end{itemize}
}%end item
\item{ 
\index{warn(String, boolean)}
{\bf  warn}\\
\begin{lstlisting}[frame=none]
void warn(java.lang.String text,boolean separated)\end{lstlisting} %end signature
\begin{itemize}
\item{
{\bf  Description}

Comunica un avvertimento all'utente
}
\item{
{\bf  Parameters}
  \begin{itemize}
   \item{
\texttt{text} -- il testo del messaggio}
   \item{
\texttt{separated} -- true se si separa dal contesto del precedente messaggio}
  \end{itemize}
}%end item
\end{itemize}
}%end item
\end{itemize}
}
}
}
\chapter{Package it.matlice.ingsw.tree}{
\label{it.matlice.ingsw.tree}\hskip -.05in
\hbox to \hsize{\textit{ Package Contents\hfil Page}}
\vskip .13in
\hbox{{\bf  Interfaces}}
\entityintro{Node}{it.matlice.ingsw.tree.Node}{}
\entityintro{TreeAction}{it.matlice.ingsw.tree.TreeAction}{}
\vskip .13in
\hbox{{\bf  Classes}}
\entityintro{MapNode}{it.matlice.ingsw.tree.MapNode}{}
\entityintro{TreeNode}{it.matlice.ingsw.tree.TreeNode}{}
\vskip .1in
\vskip .1in
\section{\label{it.matlice.ingsw.tree.Node}\index{Node@\textit{ Node}}Interface Node}{
\vskip .1in 
\subsection{Declaration}{
\begin{lstlisting}[frame=none]
public interface Node
\end{lstlisting}
\subsection{All known subinterfaces}{TreeNode\small{\refdefined{it.matlice.ingsw.tree.TreeNode}}, MapNode\small{\refdefined{it.matlice.ingsw.tree.MapNode}}}
\subsection{All classes known to implement interface}{TreeNode\small{\refdefined{it.matlice.ingsw.tree.TreeNode}}, MapNode\small{\refdefined{it.matlice.ingsw.tree.MapNode}}}
\subsection{Method summary}{
\begin{verse}
{\bf addChild(T)} Adds a child to the node\\
{\bf BFT(TreeAction)} Branch First Traverse Executes an action on each node traversing the tree prioritizing childs first, then childs of childes...\\
{\bf DFT(TreeAction)} Depth First Traverse Executes an action on each node traversing the tree prioritizing branches\\
{\bf getChildren()} \\
{\bf getChildrenData()} Returns all node's children data\\
{\bf getData()} \\
{\bf getParent()} \\
{\bf reverse\_BFT(TreeAction)} Breadth First Traverse reversed order Executes an action on each node traversing the tree prioritizing childs first, then childs of childes starting from leafes...\\
{\bf setData(T)} \\
{\bf traverse\_linear(TreeAction)} traverses the tree layer by layer\\
{\bf traverse\_reverse\_linear(TreeAction)} traverses the tree layer by layer, bottom first\\
{\bf traverse(TreeAction)} traverses the tree recursively\\
\end{verse}
}
\subsection{Methods}{
\vskip -2em
\begin{itemize}
\item{ 
\index{addChild(T)}
{\bf  addChild}\\
\begin{lstlisting}[frame=none]
Node addChild(java.lang.Object child)\end{lstlisting} %end signature
\begin{itemize}
\item{
{\bf  Description}

Adds a child to the node
}
\item{
{\bf  Parameters}
  \begin{itemize}
   \item{
\texttt{child} -- the data of the new child}
  \end{itemize}
}%end item
\item{{\bf  Returns} -- 
this 
}%end item
\end{itemize}
}%end item
\item{ 
\index{BFT(TreeAction)}
{\bf  BFT}\\
\begin{lstlisting}[frame=none]
void BFT(TreeAction action)\end{lstlisting} %end signature
\begin{itemize}
\item{
{\bf  Description}

Branch First Traverse Executes an action on each node traversing the tree prioritizing childs first, then childs of childes...
}
\item{
{\bf  Parameters}
  \begin{itemize}
   \item{
\texttt{action} -- the action to be performed}
  \end{itemize}
}%end item
\end{itemize}
}%end item
\item{ 
\index{DFT(TreeAction)}
{\bf  DFT}\\
\begin{lstlisting}[frame=none]
void DFT(TreeAction action)\end{lstlisting} %end signature
\begin{itemize}
\item{
{\bf  Description}

Depth First Traverse Executes an action on each node traversing the tree prioritizing branches
}
\item{
{\bf  Parameters}
  \begin{itemize}
   \item{
\texttt{action} -- the action to be performed}
  \end{itemize}
}%end item
\end{itemize}
}%end item
\item{ 
\index{getChildren()}
{\bf  getChildren}\\
\begin{lstlisting}[frame=none]
Node[] getChildren()\end{lstlisting} %end signature
}%end item
\item{ 
\index{getChildrenData()}
{\bf  getChildrenData}\\
\begin{lstlisting}[frame=none]
java.lang.Object[] getChildrenData()\end{lstlisting} %end signature
\begin{itemize}
\item{
{\bf  Description}

Returns all node's children data
}
\item{{\bf  Returns} -- 
all node's children data 
}%end item
\end{itemize}
}%end item
\item{ 
\index{getData()}
{\bf  getData}\\
\begin{lstlisting}[frame=none]
java.lang.Object getData()\end{lstlisting} %end signature
}%end item
\item{ 
\index{getParent()}
{\bf  getParent}\\
\begin{lstlisting}[frame=none]
Node getParent()\end{lstlisting} %end signature
}%end item
\item{ 
\index{reverse\_BFT(TreeAction)}
{\bf  reverse\_BFT}\\
\begin{lstlisting}[frame=none]
void reverse_BFT(TreeAction action)\end{lstlisting} %end signature
\begin{itemize}
\item{
{\bf  Description}

Breadth First Traverse reversed order Executes an action on each node traversing the tree prioritizing childs first, then childs of childes starting from leafes...
}
\item{
{\bf  Parameters}
  \begin{itemize}
   \item{
\texttt{action} -- the action to be performed}
  \end{itemize}
}%end item
\end{itemize}
}%end item
\item{ 
\index{setData(T)}
{\bf  setData}\\
\begin{lstlisting}[frame=none]
Node setData(java.lang.Object data)\end{lstlisting} %end signature
}%end item
\item{ 
\index{traverse\_linear(TreeAction)}
{\bf  traverse\_linear}\\
\begin{lstlisting}[frame=none]
void traverse_linear(TreeAction action)\end{lstlisting} %end signature
\begin{itemize}
\item{
{\bf  Description}

traverses the tree layer by layer
}
\item{
{\bf  Parameters}
  \begin{itemize}
   \item{
\texttt{action} -- the action to be performed on each node}
  \end{itemize}
}%end item
\item{{\bf  See also}
  \begin{itemize}
\item{ \texttt{Node.BFT(TreeAction)} {\small 
\refdefined{it.matlice.ingsw.tree.Node.BFT(it.matlice.ingsw.tree.TreeAction)}}%end
}
  \end{itemize}
}%end item
\end{itemize}
}%end item
\item{ 
\index{traverse\_reverse\_linear(TreeAction)}
{\bf  traverse\_reverse\_linear}\\
\begin{lstlisting}[frame=none]
void traverse_reverse_linear(TreeAction action)\end{lstlisting} %end signature
\begin{itemize}
\item{
{\bf  Description}

traverses the tree layer by layer, bottom first
}
\item{
{\bf  Parameters}
  \begin{itemize}
   \item{
\texttt{action} -- the action to be performed on each node}
  \end{itemize}
}%end item
\item{{\bf  See also}
  \begin{itemize}
\item{ \texttt{Node.BFT(TreeAction)} {\small 
\refdefined{it.matlice.ingsw.tree.Node.BFT(it.matlice.ingsw.tree.TreeAction)}}%end
}
  \end{itemize}
}%end item
\end{itemize}
}%end item
\item{ 
\index{traverse(TreeAction)}
{\bf  traverse}\\
\begin{lstlisting}[frame=none]
void traverse(TreeAction action)\end{lstlisting} %end signature
\begin{itemize}
\item{
{\bf  Description}

traverses the tree recursively
}
\item{
{\bf  Parameters}
  \begin{itemize}
   \item{
\texttt{action} -- the action to be performed on each node}
  \end{itemize}
}%end item
\item{{\bf  See also}
  \begin{itemize}
\item{ \texttt{Node.DFT(TreeAction)} {\small 
\refdefined{it.matlice.ingsw.tree.Node.DFT(it.matlice.ingsw.tree.TreeAction)}}%end
}
  \end{itemize}
}%end item
\end{itemize}
}%end item
\end{itemize}
}
}
\section{\label{it.matlice.ingsw.tree.TreeAction}\index{TreeAction@\textit{ TreeAction}}Interface TreeAction}{
\vskip .1in 
\subsection{Declaration}{
\begin{lstlisting}[frame=none]
public interface TreeAction
\end{lstlisting}
\subsection{Method summary}{
\begin{verse}
{\bf nodeAction(Node)} \\
\end{verse}
}
\subsection{Methods}{
\vskip -2em
\begin{itemize}
\item{ 
\index{nodeAction(Node)}
{\bf  nodeAction}\\
\begin{lstlisting}[frame=none]
void nodeAction(Node ref)\end{lstlisting} %end signature
}%end item
\end{itemize}
}
}
\section{\label{it.matlice.ingsw.tree.MapNode}\index{MapNode}Class MapNode}{
\vskip .1in 
\subsection{Declaration}{
\begin{lstlisting}[frame=none]
public class MapNode
 extends java.lang.Object implements Node\end{lstlisting}
\subsection{Constructor summary}{
\begin{verse}
{\bf MapNode(T)} \\
\end{verse}
}
\subsection{Method summary}{
\begin{verse}
{\bf addChild(String, T)} Adds a child to the node\\
{\bf addChild(T)} Adds a child to the node\\
{\bf BFT(TreeAction)} Branch First Traverse Executes an action on each node traversing the tree prioritizing children first, then children of childes...\\
{\bf childrenSet()} \\
{\bf DFT(TreeAction)} Depth First Traverse Executes an action on each node traversing the tree prioritizing branches\\
{\bf getChildren()} \\
{\bf getChildren(String)} \\
{\bf getChildrenData()} Returns all node's children data\\
{\bf getChildrenMap()} \\
{\bf getData()} \\
{\bf getParent()} \\
{\bf reverse\_BFT(TreeAction)} Branch First Traverse reversed order Executes an action on each node traversing the tree prioritizing childs first, then childs of childes starting from leafs...\\
{\bf setData(T)} \\
{\bf toString()} \\
{\bf traverse\_linear(TreeAction)} traverses the tree layer by layer\\
{\bf traverse\_reverse\_linear(TreeAction)} traverses the tree layer by layer, bottom first\\
{\bf traverse(TreeAction)} traverses the tree recursively\\
\end{verse}
}
\subsection{Constructors}{
\vskip -2em
\begin{itemize}
\item{ 
\index{MapNode(T)}
{\bf  MapNode}\\
\begin{lstlisting}[frame=none]
public MapNode(java.lang.Object data)\end{lstlisting} %end signature
}%end item
\end{itemize}
}
\subsection{Methods}{
\vskip -2em
\begin{itemize}
\item{ 
\index{addChild(String, T)}
{\bf  addChild}\\
\begin{lstlisting}[frame=none]
public MapNode addChild(java.lang.String key,java.lang.Object child)\end{lstlisting} %end signature
\begin{itemize}
\item{
{\bf  Description}

Adds a child to the node
}
\item{
{\bf  Parameters}
  \begin{itemize}
   \item{
\texttt{key} -- the name of the child}
   \item{
\texttt{child} -- the data of the new child}
  \end{itemize}
}%end item
\item{{\bf  Returns} -- 
this 
}%end item
\end{itemize}
}%end item
\item{ 
\index{addChild(T)}
{\bf  addChild}\\
\begin{lstlisting}[frame=none]
public Node addChild(java.lang.Object child)\end{lstlisting} %end signature
\begin{itemize}
\item{
{\bf  Description}

Adds a child to the node
}
\item{
{\bf  Parameters}
  \begin{itemize}
   \item{
\texttt{child} -- the data of the new child}
  \end{itemize}
}%end item
\item{{\bf  Returns} -- 
this 
}%end item
\end{itemize}
}%end item
\item{ 
\index{BFT(TreeAction)}
{\bf  BFT}\\
\begin{lstlisting}[frame=none]
public void BFT(TreeAction action)\end{lstlisting} %end signature
\begin{itemize}
\item{
{\bf  Description}

Branch First Traverse Executes an action on each node traversing the tree prioritizing children first, then children of childes...
}
\item{
{\bf  Parameters}
  \begin{itemize}
   \item{
\texttt{action} -- the action to be performed}
  \end{itemize}
}%end item
\end{itemize}
}%end item
\item{ 
\index{childrenSet()}
{\bf  childrenSet}\\
\begin{lstlisting}[frame=none]
public java.util.Set childrenSet()\end{lstlisting} %end signature
}%end item
\item{ 
\index{DFT(TreeAction)}
{\bf  DFT}\\
\begin{lstlisting}[frame=none]
public void DFT(TreeAction action)\end{lstlisting} %end signature
\begin{itemize}
\item{
{\bf  Description}

Depth First Traverse Executes an action on each node traversing the tree prioritizing branches
}
\item{
{\bf  Parameters}
  \begin{itemize}
   \item{
\texttt{action} -- the action to be performed}
  \end{itemize}
}%end item
\end{itemize}
}%end item
\item{ 
\index{getChildren()}
{\bf  getChildren}\\
\begin{lstlisting}[frame=none]
Node[] getChildren()\end{lstlisting} %end signature
}%end item
\item{ 
\index{getChildren(String)}
{\bf  getChildren}\\
\begin{lstlisting}[frame=none]
public MapNode getChildren(java.lang.String key)\end{lstlisting} %end signature
}%end item
\item{ 
\index{getChildrenData()}
{\bf  getChildrenData}\\
\begin{lstlisting}[frame=none]
public java.lang.Object[] getChildrenData()\end{lstlisting} %end signature
\begin{itemize}
\item{
{\bf  Description}

Returns all node's children data
}
\item{{\bf  Returns} -- 
 
}%end item
\end{itemize}
}%end item
\item{ 
\index{getChildrenMap()}
{\bf  getChildrenMap}\\
\begin{lstlisting}[frame=none]
public java.util.TreeMap getChildrenMap()\end{lstlisting} %end signature
}%end item
\item{ 
\index{getData()}
{\bf  getData}\\
\begin{lstlisting}[frame=none]
java.lang.Object getData()\end{lstlisting} %end signature
}%end item
\item{ 
\index{getParent()}
{\bf  getParent}\\
\begin{lstlisting}[frame=none]
Node getParent()\end{lstlisting} %end signature
}%end item
\item{ 
\index{reverse\_BFT(TreeAction)}
{\bf  reverse\_BFT}\\
\begin{lstlisting}[frame=none]
public void reverse_BFT(TreeAction action)\end{lstlisting} %end signature
\begin{itemize}
\item{
{\bf  Description}

Branch First Traverse reversed order Executes an action on each node traversing the tree prioritizing childs first, then childs of childes starting from leafs...
}
\item{
{\bf  Parameters}
  \begin{itemize}
   \item{
\texttt{action} -- the action to be performed}
  \end{itemize}
}%end item
\end{itemize}
}%end item
\item{ 
\index{setData(T)}
{\bf  setData}\\
\begin{lstlisting}[frame=none]
Node setData(java.lang.Object data)\end{lstlisting} %end signature
}%end item
\item{ 
\index{toString()}
{\bf  toString}\\
\begin{lstlisting}[frame=none]
public java.lang.String toString()\end{lstlisting} %end signature
}%end item
\item{ 
\index{traverse\_linear(TreeAction)}
{\bf  traverse\_linear}\\
\begin{lstlisting}[frame=none]
public void traverse_linear(TreeAction action)\end{lstlisting} %end signature
\begin{itemize}
\item{
{\bf  Description}

traverses the tree layer by layer
}
\item{
{\bf  Parameters}
  \begin{itemize}
   \item{
\texttt{action} -- the action to be performed on each node}
  \end{itemize}
}%end item
\item{{\bf  See also}
  \begin{itemize}
\item{ \texttt{Node.BFT(TreeAction)} {\small 
\refdefined{it.matlice.ingsw.tree.Node.BFT(it.matlice.ingsw.tree.TreeAction)}}%end
}
  \end{itemize}
}%end item
\end{itemize}
}%end item
\item{ 
\index{traverse\_reverse\_linear(TreeAction)}
{\bf  traverse\_reverse\_linear}\\
\begin{lstlisting}[frame=none]
public void traverse_reverse_linear(TreeAction action)\end{lstlisting} %end signature
\begin{itemize}
\item{
{\bf  Description}

traverses the tree layer by layer, bottom first
}
\item{
{\bf  Parameters}
  \begin{itemize}
   \item{
\texttt{action} -- the action to be performed on each node}
  \end{itemize}
}%end item
\item{{\bf  See also}
  \begin{itemize}
\item{ \texttt{Node.BFT(TreeAction)} {\small 
\refdefined{it.matlice.ingsw.tree.Node.BFT(it.matlice.ingsw.tree.TreeAction)}}%end
}
  \end{itemize}
}%end item
\end{itemize}
}%end item
\item{ 
\index{traverse(TreeAction)}
{\bf  traverse}\\
\begin{lstlisting}[frame=none]
public void traverse(TreeAction action)\end{lstlisting} %end signature
\begin{itemize}
\item{
{\bf  Description}

traverses the tree recursively
}
\item{
{\bf  Parameters}
  \begin{itemize}
   \item{
\texttt{action} -- the action to be performed on each node}
  \end{itemize}
}%end item
\item{{\bf  See also}
  \begin{itemize}
\item{ \texttt{Node.DFT(TreeAction)} {\small 
\refdefined{it.matlice.ingsw.tree.Node.DFT(it.matlice.ingsw.tree.TreeAction)}}%end
}
  \end{itemize}
}%end item
\end{itemize}
}%end item
\end{itemize}
}
}
\section{\label{it.matlice.ingsw.tree.TreeNode}\index{TreeNode}Class TreeNode}{
\vskip .1in 
\subsection{Declaration}{
\begin{lstlisting}[frame=none]
public class TreeNode
 extends java.lang.Object implements Node\end{lstlisting}
\subsection{Constructor summary}{
\begin{verse}
{\bf TreeNode(T)} \\
\end{verse}
}
\subsection{Method summary}{
\begin{verse}
{\bf addChild(T)} Adds a child to the node\\
{\bf BFT(TreeAction)} Breadth First Traverse Executes an action on each node traversing the tree prioritizing childs first, then childs of childes...\\
{\bf DFT(TreeAction)} Depth First Traverse Executes an action on each node traversing the tree prioritizing branches\\
{\bf getChildren()} \\
{\bf getChildrenData()} Returns all node's children data\\
{\bf getData()} \\
{\bf getParent()} \\
{\bf reverse\_BFT(TreeAction)} Branch First Traverse reversed order Executes an action on each node traversing the tree prioritizing childs first, then childs of childes starting from leafes...\\
{\bf setData(T)} \\
{\bf traverse\_linear(TreeAction)} traverses the tree layer by layer\\
{\bf traverse\_reverse\_linear(TreeAction)} traverses the tree layer by layer, bottom first\\
{\bf traverse(TreeAction)} traverses the tree recursively\\
\end{verse}
}
\subsection{Constructors}{
\vskip -2em
\begin{itemize}
\item{ 
\index{TreeNode(T)}
{\bf  TreeNode}\\
\begin{lstlisting}[frame=none]
public TreeNode(java.lang.Object data)\end{lstlisting} %end signature
}%end item
\end{itemize}
}
\subsection{Methods}{
\vskip -2em
\begin{itemize}
\item{ 
\index{addChild(T)}
{\bf  addChild}\\
\begin{lstlisting}[frame=none]
public TreeNode addChild(java.lang.Object child)\end{lstlisting} %end signature
\begin{itemize}
\item{
{\bf  Description}

Adds a child to the node
}
\item{
{\bf  Parameters}
  \begin{itemize}
   \item{
\texttt{child} -- the data of the new child}
  \end{itemize}
}%end item
\item{{\bf  Returns} -- 
this 
}%end item
\end{itemize}
}%end item
\item{ 
\index{BFT(TreeAction)}
{\bf  BFT}\\
\begin{lstlisting}[frame=none]
public void BFT(TreeAction action)\end{lstlisting} %end signature
\begin{itemize}
\item{
{\bf  Description}

Breadth First Traverse Executes an action on each node traversing the tree prioritizing childs first, then childs of childes...
}
\item{
{\bf  Parameters}
  \begin{itemize}
   \item{
\texttt{action} -- the action to be performed}
  \end{itemize}
}%end item
\end{itemize}
}%end item
\item{ 
\index{DFT(TreeAction)}
{\bf  DFT}\\
\begin{lstlisting}[frame=none]
public void DFT(TreeAction action)\end{lstlisting} %end signature
\begin{itemize}
\item{
{\bf  Description}

Depth First Traverse Executes an action on each node traversing the tree prioritizing branches
}
\item{
{\bf  Parameters}
  \begin{itemize}
   \item{
\texttt{action} -- the action to be performed}
  \end{itemize}
}%end item
\end{itemize}
}%end item
\item{ 
\index{getChildren()}
{\bf  getChildren}\\
\begin{lstlisting}[frame=none]
Node[] getChildren()\end{lstlisting} %end signature
}%end item
\item{ 
\index{getChildrenData()}
{\bf  getChildrenData}\\
\begin{lstlisting}[frame=none]
public java.lang.Object[] getChildrenData()\end{lstlisting} %end signature
\begin{itemize}
\item{
{\bf  Description}

Returns all node's children data
}
\item{{\bf  Returns} -- 
 
}%end item
\end{itemize}
}%end item
\item{ 
\index{getData()}
{\bf  getData}\\
\begin{lstlisting}[frame=none]
java.lang.Object getData()\end{lstlisting} %end signature
}%end item
\item{ 
\index{getParent()}
{\bf  getParent}\\
\begin{lstlisting}[frame=none]
Node getParent()\end{lstlisting} %end signature
}%end item
\item{ 
\index{reverse\_BFT(TreeAction)}
{\bf  reverse\_BFT}\\
\begin{lstlisting}[frame=none]
public void reverse_BFT(TreeAction action)\end{lstlisting} %end signature
\begin{itemize}
\item{
{\bf  Description}

Branch First Traverse reversed order Executes an action on each node traversing the tree prioritizing childs first, then childs of childes starting from leafes...
}
\item{
{\bf  Parameters}
  \begin{itemize}
   \item{
\texttt{action} -- the action to be performed}
  \end{itemize}
}%end item
\end{itemize}
}%end item
\item{ 
\index{setData(T)}
{\bf  setData}\\
\begin{lstlisting}[frame=none]
Node setData(java.lang.Object data)\end{lstlisting} %end signature
}%end item
\item{ 
\index{traverse\_linear(TreeAction)}
{\bf  traverse\_linear}\\
\begin{lstlisting}[frame=none]
public void traverse_linear(TreeAction action)\end{lstlisting} %end signature
\begin{itemize}
\item{
{\bf  Description}

traverses the tree layer by layer
}
\item{
{\bf  Parameters}
  \begin{itemize}
   \item{
\texttt{action} -- the action to be performed on each node}
  \end{itemize}
}%end item
\item{{\bf  See also}
  \begin{itemize}
\item{ \texttt{Node.BFT(TreeAction)} {\small 
\refdefined{it.matlice.ingsw.tree.Node.BFT(it.matlice.ingsw.tree.TreeAction)}}%end
}
  \end{itemize}
}%end item
\end{itemize}
}%end item
\item{ 
\index{traverse\_reverse\_linear(TreeAction)}
{\bf  traverse\_reverse\_linear}\\
\begin{lstlisting}[frame=none]
public void traverse_reverse_linear(TreeAction action)\end{lstlisting} %end signature
\begin{itemize}
\item{
{\bf  Description}

traverses the tree layer by layer, bottom first
}
\item{
{\bf  Parameters}
  \begin{itemize}
   \item{
\texttt{action} -- the action to be performed on each node}
  \end{itemize}
}%end item
\item{{\bf  See also}
  \begin{itemize}
\item{ \texttt{Node.BFT(TreeAction)} {\small 
\refdefined{it.matlice.ingsw.tree.Node.BFT(it.matlice.ingsw.tree.TreeAction)}}%end
}
  \end{itemize}
}%end item
\end{itemize}
}%end item
\item{ 
\index{traverse(TreeAction)}
{\bf  traverse}\\
\begin{lstlisting}[frame=none]
public void traverse(TreeAction action)\end{lstlisting} %end signature
\begin{itemize}
\item{
{\bf  Description}

traverses the tree recursively
}
\item{
{\bf  Parameters}
  \begin{itemize}
   \item{
\texttt{action} -- the action to be performed on each node}
  \end{itemize}
}%end item
\item{{\bf  See also}
  \begin{itemize}
\item{ \texttt{Node.DFT(TreeAction)} {\small 
\refdefined{it.matlice.ingsw.tree.Node.DFT(it.matlice.ingsw.tree.TreeAction)}}%end
}
  \end{itemize}
}%end item
\end{itemize}
}%end item
\end{itemize}
}
}
}
\chapter{Package it.matlice.ingsw.xml}{
\label{it.matlice.ingsw.xml}\hskip -.05in
\hbox to \hsize{\textit{ Package Contents\hfil Page}}
\vskip .13in
\hbox{{\bf  Interfaces}}
\entityintro{XMLConversion}{it.matlice.ingsw.xml.XMLConversion}{}
\entityintro{XMLWritable}{it.matlice.ingsw.xml.XMLWritable}{}
\vskip .13in
\hbox{{\bf  Classes}}
\entityintro{PrettyXMLStreamWriter}{it.matlice.ingsw.xml.PrettyXMLStreamWriter}{implementation with custom options, such as having newlines after tags with the option to choose which ones shouldn't have the newline}
\entityintro{Utils}{it.matlice.ingsw.xml.Utils}{}
\entityintro{XMLNode}{it.matlice.ingsw.xml.XMLNode}{}
\entityintro{XMLParser}{it.matlice.ingsw.xml.XMLParser}{}
\vskip .1in
\vskip .1in
\section{\label{it.matlice.ingsw.xml.XMLConversion}\index{XMLConversion@\textit{ XMLConversion}}Interface XMLConversion}{
\vskip .1in 
\subsection{Declaration}{
\begin{lstlisting}[frame=none]
public interface XMLConversion
\end{lstlisting}
\subsection{Method summary}{
\begin{verse}
{\bf convert(XMLNode, TreeMap, XMLNode, MapNode)} converts a given node to a java object representation\\
\end{verse}
}
\subsection{Methods}{
\vskip -2em
\begin{itemize}
\item{ 
\index{convert(XMLNode, TreeMap, XMLNode, MapNode)}
{\bf  convert}\\
\begin{lstlisting}[frame=none]
java.lang.Object convert(XMLNode node,java.util.TreeMap children,XMLNode parent,it.matlice.ingsw.tree.MapNode dinasty) throws javax.xml.stream.XMLStreamException\end{lstlisting} %end signature
\begin{itemize}
\item{
{\bf  Description}

converts a given node to a java object representation
}
\item{
{\bf  Parameters}
  \begin{itemize}
   \item{
\texttt{node} -- the current node}
   \item{
\texttt{children} -- a TreeMap of already converted node children}
  \end{itemize}
}%end item
\item{{\bf  Returns} -- 
the converted object 
}%end item
\end{itemize}
}%end item
\end{itemize}
}
}
\section{\label{it.matlice.ingsw.xml.XMLWritable}\index{XMLWritable@\textit{ XMLWritable}}Interface XMLWritable}{
\vskip .1in 
\subsection{Declaration}{
\begin{lstlisting}[frame=none]
public interface XMLWritable
\end{lstlisting}
\subsection{Method summary}{
\begin{verse}
{\bf toXML(XMLStreamWriter)} \\
{\bf toXML(XMLStreamWriter, Object)} \\
\end{verse}
}
\subsection{Methods}{
\vskip -2em
\begin{itemize}
\item{ 
\index{toXML(XMLStreamWriter)}
{\bf  toXML}\\
\begin{lstlisting}[frame=none]
void toXML(javax.xml.stream.XMLStreamWriter xmlw) throws javax.xml.stream.XMLStreamException\end{lstlisting} %end signature
}%end item
\item{ 
\index{toXML(XMLStreamWriter, Object)}
{\bf  toXML}\\
\begin{lstlisting}[frame=none]
void toXML(javax.xml.stream.XMLStreamWriter xmlw,java.lang.Object data) throws javax.xml.stream.XMLStreamException\end{lstlisting} %end signature
}%end item
\end{itemize}
}
}
\section{\label{it.matlice.ingsw.xml.PrettyXMLStreamWriter}\index{PrettyXMLStreamWriter}Class PrettyXMLStreamWriter}{
\vskip .1in 
implementation with custom options, such as having newlines after tags with the option to choose which ones shouldn't have the newline\vskip .1in 
\subsection{Declaration}{
\begin{lstlisting}[frame=none]
public class PrettyXMLStreamWriter
 extends java.lang.Object implements javax.xml.stream.XMLStreamWriter, java.lang.AutoCloseable\end{lstlisting}
\subsection{Field summary}{
\begin{verse}
{\bf ENCODING\_UTF\_16} \\
{\bf ENCODING\_UTF\_8} \\
{\bf XML\_VERSION\_1\_0} \\
\end{verse}
}
\subsection{Constructor summary}{
\begin{verse}
{\bf PrettyXMLStreamWriter(String)} Constructor for \texttt{\small PrettyXMLStreamWriter}{\small 
\refdefined{it.matlice.ingsw.xml.PrettyXMLStreamWriter}}, by default all tags will have the newline after them\\
{\bf PrettyXMLStreamWriter(String, String\lbrack \rbrack )} \\
{\bf PrettyXMLStreamWriter(String, String\lbrack \rbrack , String, String)} Constructor for \texttt{\small PrettyXMLStreamWriter}{\small 
\refdefined{it.matlice.ingsw.xml.PrettyXMLStreamWriter}}, it gives the possibility yo choose which tags shouldn't have the newline after them with the noNewLine param\\
\end{verse}
}
\subsection{Method summary}{
\begin{verse}
{\bf close()} Close this writer and free any resources associated with the writer.\\
{\bf flush()} Write any cached data to the underlying output mechanism.\\
{\bf getNamespaceContext()} Returns the current namespace context.\\
{\bf getPrefix(String)} Gets the prefix the uri is bound to\\
{\bf getProperty(String)} Get the value of a feature/property from the underlying implementation\\
{\bf setDefaultNamespace(String)} Binds a URI to the default namespace This URI is bound in the scope of the current START\_ELEMENT / END\_ELEMENT pair.\\
{\bf setNamespaceContext(NamespaceContext)} Sets the current namespace context for prefix and uri bindings.\\
{\bf setPrefix(String, String)} Sets the prefix the uri is bound to.\\
{\bf writeAttribute(String, String)} Writes an attribute to the output stream without a prefix.\\
{\bf writeAttribute(String, String, String)} Writes an attribute to the output stream\\
{\bf writeAttribute(String, String, String, String)} Writes an attribute to the output stream\\
{\bf writeCData(String)} Writes a CData section\\
{\bf writeCharacters(char\lbrack \rbrack , int, int)} Write text to the output\\
{\bf writeCharacters(String)} Write text to the output\\
{\bf writeComment(String)} Writes an xml comment with the data enclosed\\
{\bf writeDefaultNamespace(String)} Writes the default namespace to the stream\\
{\bf writeDTD(String)} Write a DTD section.\\
{\bf writeEmptyElement(String)} Writes an empty element tag to the output\\
{\bf writeEmptyElement(String, String)} Writes an empty element tag to the output\\
{\bf writeEmptyElement(String, String, String)} Writes an empty element tag to the output\\
{\bf writeEndDocument()} Closes any start tags and writes corresponding end tags.\\
{\bf writeEndElement()} Writes an end tag to the output relying on the internal state of the writer to determine the prefix and local name of the event.\\
{\bf writeEntityRef(String)} Writes an entity reference\\
{\bf writeNamespace(String, String)} Writes a namespace to the output stream If the prefix argument to this method is the empty string, "xmlns", or null this method will delegate to writeDefaultNamespace\\
{\bf writeProcessingInstruction(String)} Writes a processing instruction\\
{\bf writeProcessingInstruction(String, String)} Writes a processing instruction\\
{\bf writeStartDocument()} Write the XML Declaration.\\
{\bf writeStartDocument(String)} Write the XML Declaration.\\
{\bf writeStartDocument(String, String)} Write the XML Declaration.\\
{\bf writeStartElement(String)} Writes a start tag to the output.\\
{\bf writeStartElement(String, String)} Writes a start tag to the output\\
{\bf writeStartElement(String, String, String)} Writes a start tag to the output\\
\end{verse}
}
\subsection{Fields}{
\begin{itemize}
\item{
\index{ENCODING\_UTF\_8}
\label{it.matlice.ingsw.xml.PrettyXMLStreamWriter.ENCODING_UTF_8}\texttt{public static final java.lang.String\ {\bf  ENCODING\_UTF\_8}}
}
\item{
\index{ENCODING\_UTF\_16}
\label{it.matlice.ingsw.xml.PrettyXMLStreamWriter.ENCODING_UTF_16}\texttt{public static final java.lang.String\ {\bf  ENCODING\_UTF\_16}}
}
\item{
\index{XML\_VERSION\_1\_0}
\label{it.matlice.ingsw.xml.PrettyXMLStreamWriter.XML_VERSION_1_0}\texttt{public static final java.lang.String\ {\bf  XML\_VERSION\_1\_0}}
}
\end{itemize}
}
\subsection{Constructors}{
\vskip -2em
\begin{itemize}
\item{ 
\index{PrettyXMLStreamWriter(String)}
{\bf  PrettyXMLStreamWriter}\\
\begin{lstlisting}[frame=none]
public PrettyXMLStreamWriter(java.lang.String filename) throws java.io.FileNotFoundException, javax.xml.stream.XMLStreamException\end{lstlisting} %end signature
\begin{itemize}
\item{
{\bf  Description}

Constructor for \texttt{\small PrettyXMLStreamWriter}{\small 
\refdefined{it.matlice.ingsw.xml.PrettyXMLStreamWriter}}, by default all tags will have the newline after them
}
\item{
{\bf  Parameters}
  \begin{itemize}
   \item{
\texttt{filename} -- filename of the file to write to}
  \end{itemize}
}%end item
\item{{\bf  Throws}
  \begin{itemize}
   \item{\vskip -.6ex \texttt{java.io.FileNotFoundException} -- }
   \item{\vskip -.6ex \texttt{javax.xml.stream.XMLStreamException} -- }
  \end{itemize}
}%end item
\end{itemize}
}%end item
\item{ 
\index{PrettyXMLStreamWriter(String, String\lbrack \rbrack )}
{\bf  PrettyXMLStreamWriter}\\
\begin{lstlisting}[frame=none]
public PrettyXMLStreamWriter(java.lang.String filename,java.lang.String[] noNewLine) throws java.io.FileNotFoundException, javax.xml.stream.XMLStreamException\end{lstlisting} %end signature
}%end item
\item{ 
\index{PrettyXMLStreamWriter(String, String\lbrack \rbrack , String, String)}
{\bf  PrettyXMLStreamWriter}\\
\begin{lstlisting}[frame=none]
public PrettyXMLStreamWriter(java.lang.String filename,java.lang.String[] noNewLine,java.lang.String encoding,java.lang.String version) throws java.io.FileNotFoundException, javax.xml.stream.XMLStreamException\end{lstlisting} %end signature
\begin{itemize}
\item{
{\bf  Description}

Constructor for \texttt{\small PrettyXMLStreamWriter}{\small 
\refdefined{it.matlice.ingsw.xml.PrettyXMLStreamWriter}}, it gives the possibility yo choose which tags shouldn't have the newline after them with the noNewLine param
}
\item{
{\bf  Parameters}
  \begin{itemize}
   \item{
\texttt{filename} -- filename of the file to write to}
   \item{
\texttt{noNewLine} -- String\lbrack \rbrack\ of tags, these tags won't have the newline after them (their CHARS and their end-tags too)}
  \end{itemize}
}%end item
\item{{\bf  Throws}
  \begin{itemize}
   \item{\vskip -.6ex \texttt{java.io.FileNotFoundException} -- }
   \item{\vskip -.6ex \texttt{javax.xml.stream.XMLStreamException} -- }
  \end{itemize}
}%end item
\end{itemize}
}%end item
\end{itemize}
}
\subsection{Methods}{
\vskip -2em
\begin{itemize}
\item{ 
\index{close()}
{\bf  close}\\
\begin{lstlisting}[frame=none]
public void close() throws javax.xml.stream.XMLStreamException\end{lstlisting} %end signature
\begin{itemize}
\item{
{\bf  Description}

Close this writer and free any resources associated with the writer. This must not close the underlying output stream.
}
\item{{\bf  Throws}
  \begin{itemize}
   \item{\vskip -.6ex \texttt{javax.xml.stream.XMLStreamException} -- }
  \end{itemize}
}%end item
\end{itemize}
}%end item
\item{ 
\index{flush()}
{\bf  flush}\\
\begin{lstlisting}[frame=none]
public void flush() throws javax.xml.stream.XMLStreamException\end{lstlisting} %end signature
\begin{itemize}
\item{
{\bf  Description}

Write any cached data to the underlying output mechanism.
}
\item{{\bf  Throws}
  \begin{itemize}
   \item{\vskip -.6ex \texttt{javax.xml.stream.XMLStreamException} -- }
  \end{itemize}
}%end item
\end{itemize}
}%end item
\item{ 
\index{getNamespaceContext()}
{\bf  getNamespaceContext}\\
\begin{lstlisting}[frame=none]
public javax.xml.namespace.NamespaceContext getNamespaceContext()\end{lstlisting} %end signature
\begin{itemize}
\item{
{\bf  Description}

Returns the current namespace context.
}
\item{{\bf  Returns} -- 
the current NamespaceContext 
}%end item
\end{itemize}
}%end item
\item{ 
\index{getPrefix(String)}
{\bf  getPrefix}\\
\begin{lstlisting}[frame=none]
public java.lang.String getPrefix(java.lang.String uri) throws javax.xml.stream.XMLStreamException\end{lstlisting} %end signature
\begin{itemize}
\item{
{\bf  Description}

Gets the prefix the uri is bound to
}
\item{{\bf  Returns} -- 
the prefix or null 
}%end item
\item{{\bf  Throws}
  \begin{itemize}
   \item{\vskip -.6ex \texttt{javax.xml.stream.XMLStreamException} -- }
  \end{itemize}
}%end item
\end{itemize}
}%end item
\item{ 
\index{getProperty(String)}
{\bf  getProperty}\\
\begin{lstlisting}[frame=none]
public java.lang.Object getProperty(java.lang.String name) throws java.lang.IllegalArgumentException\end{lstlisting} %end signature
\begin{itemize}
\item{
{\bf  Description}

Get the value of a feature/property from the underlying implementation
}
\item{
{\bf  Parameters}
  \begin{itemize}
   \item{
\texttt{name} -- The name of the property, may not be null}
  \end{itemize}
}%end item
\item{{\bf  Returns} -- 
The value of the property 
}%end item
\item{{\bf  Throws}
  \begin{itemize}
   \item{\vskip -.6ex \texttt{java.lang.IllegalArgumentException} -- if the property is not supported}
   \item{\vskip -.6ex \texttt{java.lang.NullPointerException} -- if the name is null}
  \end{itemize}
}%end item
\end{itemize}
}%end item
\item{ 
\index{setDefaultNamespace(String)}
{\bf  setDefaultNamespace}\\
\begin{lstlisting}[frame=none]
public void setDefaultNamespace(java.lang.String uri) throws javax.xml.stream.XMLStreamException\end{lstlisting} %end signature
\begin{itemize}
\item{
{\bf  Description}

Binds a URI to the default namespace This URI is bound in the scope of the current START\_ELEMENT / END\_ELEMENT pair. If this method is called before a START\_ELEMENT has been written the uri is bound in the root scope.
}
\item{
{\bf  Parameters}
  \begin{itemize}
   \item{
\texttt{uri} -- the uri to bind to the default namespace, may be null}
  \end{itemize}
}%end item
\item{{\bf  Throws}
  \begin{itemize}
   \item{\vskip -.6ex \texttt{javax.xml.stream.XMLStreamException} -- }
  \end{itemize}
}%end item
\end{itemize}
}%end item
\item{ 
\index{setNamespaceContext(NamespaceContext)}
{\bf  setNamespaceContext}\\
\begin{lstlisting}[frame=none]
public void setNamespaceContext(javax.xml.namespace.NamespaceContext context) throws javax.xml.stream.XMLStreamException\end{lstlisting} %end signature
\begin{itemize}
\item{
{\bf  Description}

Sets the current namespace context for prefix and uri bindings. This context becomes the root namespace context for writing and will replace the current root namespace context. Subsequent calls to setPrefix and setDefaultNamespace will bind namespaces using the context passed to the method as the root context for resolving namespaces. This method may only be called once at the start of the document. It does not cause the namespaces to be declared. If a namespace URI to prefix mapping is found in the namespace context it is treated as declared and the prefix may be used by the StreamWriter.
}
\item{
{\bf  Parameters}
  \begin{itemize}
   \item{
\texttt{context} -- the namespace context to use for this writer, may not be null}
  \end{itemize}
}%end item
\item{{\bf  Throws}
  \begin{itemize}
   \item{\vskip -.6ex \texttt{javax.xml.stream.XMLStreamException} -- }
  \end{itemize}
}%end item
\end{itemize}
}%end item
\item{ 
\index{setPrefix(String, String)}
{\bf  setPrefix}\\
\begin{lstlisting}[frame=none]
public void setPrefix(java.lang.String prefix,java.lang.String uri) throws javax.xml.stream.XMLStreamException\end{lstlisting} %end signature
\begin{itemize}
\item{
{\bf  Description}

Sets the prefix the uri is bound to. This prefix is bound in the scope of the current START\_ELEMENT / END\_ELEMENT pair. If this method is called before a START\_ELEMENT has been written the prefix is bound in the root scope.
}
\item{
{\bf  Parameters}
  \begin{itemize}
   \item{
\texttt{prefix} -- the prefix to bind to the uri, may not be null}
   \item{
\texttt{uri} -- the uri to bind to the prefix, may be null}
  \end{itemize}
}%end item
\item{{\bf  Throws}
  \begin{itemize}
   \item{\vskip -.6ex \texttt{javax.xml.stream.XMLStreamException} -- }
  \end{itemize}
}%end item
\end{itemize}
}%end item
\item{ 
\index{writeAttribute(String, String)}
{\bf  writeAttribute}\\
\begin{lstlisting}[frame=none]
public void writeAttribute(java.lang.String localName,java.lang.String value) throws javax.xml.stream.XMLStreamException\end{lstlisting} %end signature
\begin{itemize}
\item{
{\bf  Description}

Writes an attribute to the output stream without a prefix.
}
\item{
{\bf  Parameters}
  \begin{itemize}
   \item{
\texttt{localName} -- the local name of the attribute}
   \item{
\texttt{value} -- the value of the attribute}
  \end{itemize}
}%end item
\item{{\bf  Throws}
  \begin{itemize}
   \item{\vskip -.6ex \texttt{java.lang.IllegalStateException} -- if the current state does not allow Attribute writing}
   \item{\vskip -.6ex \texttt{javax.xml.stream.XMLStreamException} -- }
  \end{itemize}
}%end item
\end{itemize}
}%end item
\item{ 
\index{writeAttribute(String, String, String)}
{\bf  writeAttribute}\\
\begin{lstlisting}[frame=none]
public void writeAttribute(java.lang.String namespaceURI,java.lang.String localName,java.lang.String value) throws javax.xml.stream.XMLStreamException\end{lstlisting} %end signature
\begin{itemize}
\item{
{\bf  Description}

Writes an attribute to the output stream
}
\item{
{\bf  Parameters}
  \begin{itemize}
   \item{
\texttt{namespaceURI} -- the uri of the prefix for this attribute}
   \item{
\texttt{localName} -- the local name of the attribute}
   \item{
\texttt{value} -- the value of the attribute}
  \end{itemize}
}%end item
\item{{\bf  Throws}
  \begin{itemize}
   \item{\vskip -.6ex \texttt{java.lang.IllegalStateException} -- if the current state does not allow Attribute writing}
   \item{\vskip -.6ex \texttt{javax.xml.stream.XMLStreamException} -- if the namespace URI has not been bound to a prefix and javax.xml.stream.isRepairingNamespaces has not been set to true}
  \end{itemize}
}%end item
\end{itemize}
}%end item
\item{ 
\index{writeAttribute(String, String, String, String)}
{\bf  writeAttribute}\\
\begin{lstlisting}[frame=none]
public void writeAttribute(java.lang.String prefix,java.lang.String namespaceURI,java.lang.String localName,java.lang.String value) throws javax.xml.stream.XMLStreamException\end{lstlisting} %end signature
\begin{itemize}
\item{
{\bf  Description}

Writes an attribute to the output stream
}
\item{
{\bf  Parameters}
  \begin{itemize}
   \item{
\texttt{prefix} -- the prefix for this attribute}
   \item{
\texttt{namespaceURI} -- the uri of the prefix for this attribute}
   \item{
\texttt{localName} -- the local name of the attribute}
   \item{
\texttt{value} -- the value of the attribute}
  \end{itemize}
}%end item
\item{{\bf  Throws}
  \begin{itemize}
   \item{\vskip -.6ex \texttt{java.lang.IllegalStateException} -- if the current state does not allow Attribute writing}
   \item{\vskip -.6ex \texttt{javax.xml.stream.XMLStreamException} -- if the namespace URI has not been bound to a prefix and javax.xml.stream.isRepairingNamespaces has not been set to true}
  \end{itemize}
}%end item
\end{itemize}
}%end item
\item{ 
\index{writeCData(String)}
{\bf  writeCData}\\
\begin{lstlisting}[frame=none]
public void writeCData(java.lang.String data) throws javax.xml.stream.XMLStreamException\end{lstlisting} %end signature
\begin{itemize}
\item{
{\bf  Description}

Writes a CData section
}
\item{
{\bf  Parameters}
  \begin{itemize}
   \item{
\texttt{data} -- the data contained in the CData Section, may not be null}
  \end{itemize}
}%end item
\item{{\bf  Throws}
  \begin{itemize}
   \item{\vskip -.6ex \texttt{javax.xml.stream.XMLStreamException} -- }
  \end{itemize}
}%end item
\end{itemize}
}%end item
\item{ 
\index{writeCharacters(char\lbrack \rbrack , int, int)}
{\bf  writeCharacters}\\
\begin{lstlisting}[frame=none]
public void writeCharacters(char[] text,int start,int len) throws javax.xml.stream.XMLStreamException\end{lstlisting} %end signature
\begin{itemize}
\item{
{\bf  Description}

Write text to the output
}
\item{
{\bf  Parameters}
  \begin{itemize}
   \item{
\texttt{text} -- the value to write}
   \item{
\texttt{start} -- the starting position in the array}
   \item{
\texttt{len} -- the number of characters to write}
  \end{itemize}
}%end item
\item{{\bf  Throws}
  \begin{itemize}
   \item{\vskip -.6ex \texttt{javax.xml.stream.XMLStreamException} -- }
  \end{itemize}
}%end item
\end{itemize}
}%end item
\item{ 
\index{writeCharacters(String)}
{\bf  writeCharacters}\\
\begin{lstlisting}[frame=none]
public void writeCharacters(java.lang.String text) throws javax.xml.stream.XMLStreamException\end{lstlisting} %end signature
\begin{itemize}
\item{
{\bf  Description}

Write text to the output
}
\item{
{\bf  Parameters}
  \begin{itemize}
   \item{
\texttt{text} -- the value to write}
  \end{itemize}
}%end item
\item{{\bf  Throws}
  \begin{itemize}
   \item{\vskip -.6ex \texttt{javax.xml.stream.XMLStreamException} -- }
  \end{itemize}
}%end item
\end{itemize}
}%end item
\item{ 
\index{writeComment(String)}
{\bf  writeComment}\\
\begin{lstlisting}[frame=none]
public void writeComment(java.lang.String data) throws javax.xml.stream.XMLStreamException\end{lstlisting} %end signature
\begin{itemize}
\item{
{\bf  Description}

Writes an xml comment with the data enclosed
}
\item{
{\bf  Parameters}
  \begin{itemize}
   \item{
\texttt{data} -- the data contained in the comment, may be null}
  \end{itemize}
}%end item
\item{{\bf  Throws}
  \begin{itemize}
   \item{\vskip -.6ex \texttt{javax.xml.stream.XMLStreamException} -- }
  \end{itemize}
}%end item
\end{itemize}
}%end item
\item{ 
\index{writeDefaultNamespace(String)}
{\bf  writeDefaultNamespace}\\
\begin{lstlisting}[frame=none]
public void writeDefaultNamespace(java.lang.String namespaceURI) throws javax.xml.stream.XMLStreamException\end{lstlisting} %end signature
\begin{itemize}
\item{
{\bf  Description}

Writes the default namespace to the stream
}
\item{
{\bf  Parameters}
  \begin{itemize}
   \item{
\texttt{namespaceURI} -- the uri to bind the default namespace to}
  \end{itemize}
}%end item
\item{{\bf  Throws}
  \begin{itemize}
   \item{\vskip -.6ex \texttt{java.lang.IllegalStateException} -- if the current state does not allow Namespace writing}
   \item{\vskip -.6ex \texttt{javax.xml.stream.XMLStreamException} -- }
  \end{itemize}
}%end item
\end{itemize}
}%end item
\item{ 
\index{writeDTD(String)}
{\bf  writeDTD}\\
\begin{lstlisting}[frame=none]
public void writeDTD(java.lang.String dtd) throws javax.xml.stream.XMLStreamException\end{lstlisting} %end signature
\begin{itemize}
\item{
{\bf  Description}

Write a DTD section. This string represents the entire doctypedecl production from the XML 1.0 specification.
}
\item{
{\bf  Parameters}
  \begin{itemize}
   \item{
\texttt{dtd} -- the DTD to be written}
  \end{itemize}
}%end item
\item{{\bf  Throws}
  \begin{itemize}
   \item{\vskip -.6ex \texttt{javax.xml.stream.XMLStreamException} -- }
  \end{itemize}
}%end item
\end{itemize}
}%end item
\item{ 
\index{writeEmptyElement(String)}
{\bf  writeEmptyElement}\\
\begin{lstlisting}[frame=none]
public void writeEmptyElement(java.lang.String localName) throws javax.xml.stream.XMLStreamException\end{lstlisting} %end signature
\begin{itemize}
\item{
{\bf  Description}

Writes an empty element tag to the output
}
\item{
{\bf  Parameters}
  \begin{itemize}
   \item{
\texttt{localName} -- local name of the tag, may not be null}
  \end{itemize}
}%end item
\item{{\bf  Throws}
  \begin{itemize}
   \item{\vskip -.6ex \texttt{javax.xml.stream.XMLStreamException} -- }
  \end{itemize}
}%end item
\end{itemize}
}%end item
\item{ 
\index{writeEmptyElement(String, String)}
{\bf  writeEmptyElement}\\
\begin{lstlisting}[frame=none]
public void writeEmptyElement(java.lang.String namespaceURI,java.lang.String localName) throws javax.xml.stream.XMLStreamException\end{lstlisting} %end signature
\begin{itemize}
\item{
{\bf  Description}

Writes an empty element tag to the output
}
\item{
{\bf  Parameters}
  \begin{itemize}
   \item{
\texttt{namespaceURI} -- the uri to bind the tag to, may not be null}
   \item{
\texttt{localName} -- local name of the tag, may not be null}
  \end{itemize}
}%end item
\item{{\bf  Throws}
  \begin{itemize}
   \item{\vskip -.6ex \texttt{javax.xml.stream.XMLStreamException} -- if the namespace URI has not been bound to a prefix and javax.xml.stream.isRepairingNamespaces has not been set to true}
  \end{itemize}
}%end item
\end{itemize}
}%end item
\item{ 
\index{writeEmptyElement(String, String, String)}
{\bf  writeEmptyElement}\\
\begin{lstlisting}[frame=none]
public void writeEmptyElement(java.lang.String prefix,java.lang.String localName,java.lang.String namespaceURI) throws javax.xml.stream.XMLStreamException\end{lstlisting} %end signature
\begin{itemize}
\item{
{\bf  Description}

Writes an empty element tag to the output
}
\item{
{\bf  Parameters}
  \begin{itemize}
   \item{
\texttt{prefix} -- the prefix of the tag, may not be null}
   \item{
\texttt{localName} -- local name of the tag, may not be null}
   \item{
\texttt{namespaceURI} -- the uri to bind the tag to, may not be null}
  \end{itemize}
}%end item
\item{{\bf  Throws}
  \begin{itemize}
   \item{\vskip -.6ex \texttt{javax.xml.stream.XMLStreamException} -- }
  \end{itemize}
}%end item
\end{itemize}
}%end item
\item{ 
\index{writeEndDocument()}
{\bf  writeEndDocument}\\
\begin{lstlisting}[frame=none]
public void writeEndDocument() throws javax.xml.stream.XMLStreamException\end{lstlisting} %end signature
\begin{itemize}
\item{
{\bf  Description}

Closes any start tags and writes corresponding end tags.
}
\item{{\bf  Throws}
  \begin{itemize}
   \item{\vskip -.6ex \texttt{javax.xml.stream.XMLStreamException} -- }
  \end{itemize}
}%end item
\end{itemize}
}%end item
\item{ 
\index{writeEndElement()}
{\bf  writeEndElement}\\
\begin{lstlisting}[frame=none]
public void writeEndElement() throws javax.xml.stream.XMLStreamException\end{lstlisting} %end signature
\begin{itemize}
\item{
{\bf  Description}

Writes an end tag to the output relying on the internal state of the writer to determine the prefix and local name of the event.
}
\item{{\bf  Throws}
  \begin{itemize}
   \item{\vskip -.6ex \texttt{javax.xml.stream.XMLStreamException} -- }
  \end{itemize}
}%end item
\end{itemize}
}%end item
\item{ 
\index{writeEntityRef(String)}
{\bf  writeEntityRef}\\
\begin{lstlisting}[frame=none]
public void writeEntityRef(java.lang.String name) throws javax.xml.stream.XMLStreamException\end{lstlisting} %end signature
\begin{itemize}
\item{
{\bf  Description}

Writes an entity reference
}
\item{
{\bf  Parameters}
  \begin{itemize}
   \item{
\texttt{name} -- the name of the entity}
  \end{itemize}
}%end item
\item{{\bf  Throws}
  \begin{itemize}
   \item{\vskip -.6ex \texttt{javax.xml.stream.XMLStreamException} -- }
  \end{itemize}
}%end item
\end{itemize}
}%end item
\item{ 
\index{writeNamespace(String, String)}
{\bf  writeNamespace}\\
\begin{lstlisting}[frame=none]
public void writeNamespace(java.lang.String prefix,java.lang.String namespaceURI) throws javax.xml.stream.XMLStreamException\end{lstlisting} %end signature
\begin{itemize}
\item{
{\bf  Description}

Writes a namespace to the output stream If the prefix argument to this method is the empty string, "xmlns", or null this method will delegate to writeDefaultNamespace
}
\item{
{\bf  Parameters}
  \begin{itemize}
   \item{
\texttt{prefix} -- the prefix to bind this namespace to}
   \item{
\texttt{namespaceURI} -- the uri to bind the prefix to}
  \end{itemize}
}%end item
\item{{\bf  Throws}
  \begin{itemize}
   \item{\vskip -.6ex \texttt{java.lang.IllegalStateException} -- if the current state does not allow Namespace writing}
   \item{\vskip -.6ex \texttt{javax.xml.stream.XMLStreamException} -- }
  \end{itemize}
}%end item
\end{itemize}
}%end item
\item{ 
\index{writeProcessingInstruction(String)}
{\bf  writeProcessingInstruction}\\
\begin{lstlisting}[frame=none]
public void writeProcessingInstruction(java.lang.String target) throws javax.xml.stream.XMLStreamException\end{lstlisting} %end signature
\begin{itemize}
\item{
{\bf  Description}

Writes a processing instruction
}
\item{
{\bf  Parameters}
  \begin{itemize}
   \item{
\texttt{target} -- the target of the processing instruction, may not be null}
  \end{itemize}
}%end item
\item{{\bf  Throws}
  \begin{itemize}
   \item{\vskip -.6ex \texttt{javax.xml.stream.XMLStreamException} -- }
  \end{itemize}
}%end item
\end{itemize}
}%end item
\item{ 
\index{writeProcessingInstruction(String, String)}
{\bf  writeProcessingInstruction}\\
\begin{lstlisting}[frame=none]
public void writeProcessingInstruction(java.lang.String target,java.lang.String data) throws javax.xml.stream.XMLStreamException\end{lstlisting} %end signature
\begin{itemize}
\item{
{\bf  Description}

Writes a processing instruction
}
\item{
{\bf  Parameters}
  \begin{itemize}
   \item{
\texttt{target} -- the target of the processing instruction, may not be null}
   \item{
\texttt{data} -- the data contained in the processing instruction, may not be null}
  \end{itemize}
}%end item
\item{{\bf  Throws}
  \begin{itemize}
   \item{\vskip -.6ex \texttt{javax.xml.stream.XMLStreamException} -- }
  \end{itemize}
}%end item
\end{itemize}
}%end item
\item{ 
\index{writeStartDocument()}
{\bf  writeStartDocument}\\
\begin{lstlisting}[frame=none]
public void writeStartDocument() throws javax.xml.stream.XMLStreamException\end{lstlisting} %end signature
\begin{itemize}
\item{
{\bf  Description}

Write the XML Declaration. Defaults the XML version to 1.0, and the encoding to utf-8
}
\item{{\bf  Throws}
  \begin{itemize}
   \item{\vskip -.6ex \texttt{javax.xml.stream.XMLStreamException} -- }
  \end{itemize}
}%end item
\end{itemize}
}%end item
\item{ 
\index{writeStartDocument(String)}
{\bf  writeStartDocument}\\
\begin{lstlisting}[frame=none]
public void writeStartDocument(java.lang.String version) throws javax.xml.stream.XMLStreamException\end{lstlisting} %end signature
\begin{itemize}
\item{
{\bf  Description}

Write the XML Declaration. Defaults the XML version to 1.0
}
\item{
{\bf  Parameters}
  \begin{itemize}
   \item{
\texttt{version} -- version of the xml document}
  \end{itemize}
}%end item
\item{{\bf  Throws}
  \begin{itemize}
   \item{\vskip -.6ex \texttt{javax.xml.stream.XMLStreamException} -- }
  \end{itemize}
}%end item
\end{itemize}
}%end item
\item{ 
\index{writeStartDocument(String, String)}
{\bf  writeStartDocument}\\
\begin{lstlisting}[frame=none]
public void writeStartDocument(java.lang.String encoding,java.lang.String version) throws javax.xml.stream.XMLStreamException\end{lstlisting} %end signature
\begin{itemize}
\item{
{\bf  Description}

Write the XML Declaration. Note that the encoding parameter does not set the actual encoding of the underlying output. That must be set when the instance of the XMLStreamWriter is created using the XMLOutputFactory
}
\item{
{\bf  Parameters}
  \begin{itemize}
   \item{
\texttt{encoding} -- encoding of the xml declaration}
   \item{
\texttt{version} -- version of the xml document}
  \end{itemize}
}%end item
\item{{\bf  Throws}
  \begin{itemize}
   \item{\vskip -.6ex \texttt{javax.xml.stream.XMLStreamException} -- If given encoding does not match encoding of the underlying stream}
  \end{itemize}
}%end item
\end{itemize}
}%end item
\item{ 
\index{writeStartElement(String)}
{\bf  writeStartElement}\\
\begin{lstlisting}[frame=none]
public void writeStartElement(java.lang.String localName) throws javax.xml.stream.XMLStreamException\end{lstlisting} %end signature
\begin{itemize}
\item{
{\bf  Description}

Writes a start tag to the output. All writeStartElement methods open a new scope in the internal namespace context. Writing the corresponding EndElement causes the scope to be closed.
}
\item{
{\bf  Parameters}
  \begin{itemize}
   \item{
\texttt{localName} -- local name of the tag, may not be null}
  \end{itemize}
}%end item
\item{{\bf  Throws}
  \begin{itemize}
   \item{\vskip -.6ex \texttt{javax.xml.stream.XMLStreamException} -- }
  \end{itemize}
}%end item
\end{itemize}
}%end item
\item{ 
\index{writeStartElement(String, String)}
{\bf  writeStartElement}\\
\begin{lstlisting}[frame=none]
public void writeStartElement(java.lang.String namespaceURI,java.lang.String localName) throws javax.xml.stream.XMLStreamException\end{lstlisting} %end signature
\begin{itemize}
\item{
{\bf  Description}

Writes a start tag to the output
}
\item{
{\bf  Parameters}
  \begin{itemize}
   \item{
\texttt{namespaceURI} -- the namespaceURI of the prefix to use, may not be null}
   \item{
\texttt{localName} -- local name of the tag, may not be null}
  \end{itemize}
}%end item
\item{{\bf  Throws}
  \begin{itemize}
   \item{\vskip -.6ex \texttt{javax.xml.stream.XMLStreamException} -- if the namespace URI has not been bound to a prefix and javax.xml.stream.isRepairingNamespaces has not been set to true}
  \end{itemize}
}%end item
\end{itemize}
}%end item
\item{ 
\index{writeStartElement(String, String, String)}
{\bf  writeStartElement}\\
\begin{lstlisting}[frame=none]
public void writeStartElement(java.lang.String prefix,java.lang.String localName,java.lang.String namespaceURI) throws javax.xml.stream.XMLStreamException\end{lstlisting} %end signature
\begin{itemize}
\item{
{\bf  Description}

Writes a start tag to the output
}
\item{
{\bf  Parameters}
  \begin{itemize}
   \item{
\texttt{localName} -- local name of the tag, may not be null}
   \item{
\texttt{prefix} -- the prefix of the tag, may not be null}
   \item{
\texttt{namespaceURI} -- the uri to bind the prefix to, may not be null}
  \end{itemize}
}%end item
\item{{\bf  Throws}
  \begin{itemize}
   \item{\vskip -.6ex \texttt{javax.xml.stream.XMLStreamException} -- }
  \end{itemize}
}%end item
\end{itemize}
}%end item
\end{itemize}
}
}
\section{\label{it.matlice.ingsw.xml.Utils}\index{Utils}Class Utils}{
\vskip .1in 
\subsection{Declaration}{
\begin{lstlisting}[frame=none]
public class Utils
 extends java.lang.Object\end{lstlisting}
\subsection{Field summary}{
\begin{verse}
{\bf ArrayListConverter} \\
{\bf BooleanConverter} \\
{\bf DoubleConverter} \\
{\bf IntegerConverter} \\
{\bf LinkedListConverter} \\
{\bf MapConverter} \\
{\bf NullConverter} \\
{\bf ShortConverter} \\
{\bf StringConverter} \\
\end{verse}
}
\subsection{Constructor summary}{
\begin{verse}
{\bf Utils()} \\
\end{verse}
}
\subsection{Method summary}{
\begin{verse}
{\bf AdvBooleanConverter(List)} \\
{\bf XMLWriteList(XMLStreamWriter, List, String, Map)} Method that writes in an XML file a List\\
{\bf XMLWriteString(XMLStreamWriter, String, String, Map)} Method that writes in an XML file a String\\
\end{verse}
}
\subsection{Fields}{
\begin{itemize}
\item{
\index{StringConverter}
\label{it.matlice.ingsw.xml.Utils.StringConverter}\texttt{public static final XMLConversion\ {\bf  StringConverter}}
}
\item{
\index{ArrayListConverter}
\label{it.matlice.ingsw.xml.Utils.ArrayListConverter}\texttt{public static final XMLConversion\ {\bf  ArrayListConverter}}
}
\item{
\index{LinkedListConverter}
\label{it.matlice.ingsw.xml.Utils.LinkedListConverter}\texttt{public static final XMLConversion\ {\bf  LinkedListConverter}}
}
\item{
\index{MapConverter}
\label{it.matlice.ingsw.xml.Utils.MapConverter}\texttt{public static final XMLConversion\ {\bf  MapConverter}}
}
\item{
\index{NullConverter}
\label{it.matlice.ingsw.xml.Utils.NullConverter}\texttt{public static final XMLConversion\ {\bf  NullConverter}}
}
\item{
\index{IntegerConverter}
\label{it.matlice.ingsw.xml.Utils.IntegerConverter}\texttt{public static final XMLConversion\ {\bf  IntegerConverter}}
}
\item{
\index{ShortConverter}
\label{it.matlice.ingsw.xml.Utils.ShortConverter}\texttt{public static final XMLConversion\ {\bf  ShortConverter}}
}
\item{
\index{DoubleConverter}
\label{it.matlice.ingsw.xml.Utils.DoubleConverter}\texttt{public static final XMLConversion\ {\bf  DoubleConverter}}
}
\item{
\index{BooleanConverter}
\label{it.matlice.ingsw.xml.Utils.BooleanConverter}\texttt{public static final XMLConversion\ {\bf  BooleanConverter}}
}
\end{itemize}
}
\subsection{Constructors}{
\vskip -2em
\begin{itemize}
\item{ 
\index{Utils()}
{\bf  Utils}\\
\begin{lstlisting}[frame=none]
public Utils()\end{lstlisting} %end signature
}%end item
\end{itemize}
}
\subsection{Methods}{
\vskip -2em
\begin{itemize}
\item{ 
\index{AdvBooleanConverter(List)}
{\bf  AdvBooleanConverter}\\
\begin{lstlisting}[frame=none]
public static XMLConversion AdvBooleanConverter(java.util.List trueValues)\end{lstlisting} %end signature
}%end item
\item{ 
\index{XMLWriteList(XMLStreamWriter, List, String, Map)}
{\bf  XMLWriteList}\\
\begin{lstlisting}[frame=none]
public static void XMLWriteList(javax.xml.stream.XMLStreamWriter w,java.util.List array,java.lang.String name,java.util.Map attributes) throws javax.xml.stream.XMLStreamException\end{lstlisting} %end signature
\begin{itemize}
\item{
{\bf  Description}

Method that writes in an XML file a List
}
\item{
{\bf  Parameters}
  \begin{itemize}
   \item{
\texttt{w} -- XMLStreamWriter}
   \item{
\texttt{array} -- the list which implements XMLWritable}
   \item{
\texttt{name} -- the name of the string}
   \item{
\texttt{attributes} -- the attributes of the string}
  \end{itemize}
}%end item
\item{{\bf  Throws}
  \begin{itemize}
   \item{\vskip -.6ex \texttt{javax.xml.stream.XMLStreamException} -- }
  \end{itemize}
}%end item
\end{itemize}
}%end item
\item{ 
\index{XMLWriteString(XMLStreamWriter, String, String, Map)}
{\bf  XMLWriteString}\\
\begin{lstlisting}[frame=none]
public static void XMLWriteString(javax.xml.stream.XMLStreamWriter w,java.lang.String text,java.lang.String name,java.util.Map attributes) throws javax.xml.stream.XMLStreamException\end{lstlisting} %end signature
\begin{itemize}
\item{
{\bf  Description}

Method that writes in an XML file a String
}
\item{
{\bf  Parameters}
  \begin{itemize}
   \item{
\texttt{w} -- XMLStreamWriter}
   \item{
\texttt{text} -- the text in the string}
   \item{
\texttt{name} -- the name of the string}
   \item{
\texttt{attributes} -- the attributes of the string}
  \end{itemize}
}%end item
\item{{\bf  Throws}
  \begin{itemize}
   \item{\vskip -.6ex \texttt{javax.xml.stream.XMLStreamException} -- }
  \end{itemize}
}%end item
\end{itemize}
}%end item
\end{itemize}
}
}
\section{\label{it.matlice.ingsw.xml.XMLNode}\index{XMLNode}Class XMLNode}{
\vskip .1in 
\subsection{Declaration}{
\begin{lstlisting}[frame=none]
public class XMLNode
 extends java.lang.Object\end{lstlisting}
\subsection{Constructor summary}{
\begin{verse}
{\bf XMLNode(String)} Constructor of XMLNode\\
\end{verse}
}
\subsection{Method summary}{
\begin{verse}
{\bf add\_attribute(String, String)} Adds an attribute to the HashMap of the class\\
{\bf get\_attribute(String)} \\
{\bf getName()} \\
{\bf getValue()} \\
{\bf setValue(String)} \\
{\bf toString()} \\
\end{verse}
}
\subsection{Constructors}{
\vskip -2em
\begin{itemize}
\item{ 
\index{XMLNode(String)}
{\bf  XMLNode}\\
\begin{lstlisting}[frame=none]
public XMLNode(java.lang.String name)\end{lstlisting} %end signature
\begin{itemize}
\item{
{\bf  Description}

Constructor of XMLNode
}
\item{
{\bf  Parameters}
  \begin{itemize}
   \item{
\texttt{name} -- the node name}
  \end{itemize}
}%end item
\end{itemize}
}%end item
\end{itemize}
}
\subsection{Methods}{
\vskip -2em
\begin{itemize}
\item{ 
\index{add\_attribute(String, String)}
{\bf  add\_attribute}\\
\begin{lstlisting}[frame=none]
public void add_attribute(java.lang.String name,java.lang.String value)\end{lstlisting} %end signature
\begin{itemize}
\item{
{\bf  Description}

Adds an attribute to the HashMap of the class
}
\item{
{\bf  Parameters}
  \begin{itemize}
   \item{
\texttt{name} -- }
   \item{
\texttt{value} -- }
  \end{itemize}
}%end item
\end{itemize}
}%end item
\item{ 
\index{get\_attribute(String)}
{\bf  get\_attribute}\\
\begin{lstlisting}[frame=none]
public java.lang.String get_attribute(java.lang.String name)\end{lstlisting} %end signature
}%end item
\item{ 
\index{getName()}
{\bf  getName}\\
\begin{lstlisting}[frame=none]
public java.lang.String getName()\end{lstlisting} %end signature
}%end item
\item{ 
\index{getValue()}
{\bf  getValue}\\
\begin{lstlisting}[frame=none]
public java.lang.String getValue()\end{lstlisting} %end signature
}%end item
\item{ 
\index{setValue(String)}
{\bf  setValue}\\
\begin{lstlisting}[frame=none]
public void setValue(java.lang.String value)\end{lstlisting} %end signature
}%end item
\item{ 
\index{toString()}
{\bf  toString}\\
\begin{lstlisting}[frame=none]
public java.lang.String toString()\end{lstlisting} %end signature
}%end item
\end{itemize}
}
}
\section{\label{it.matlice.ingsw.xml.XMLParser}\index{XMLParser}Class XMLParser}{
\vskip .1in 
\subsection{Declaration}{
\begin{lstlisting}[frame=none]
public class XMLParser
 extends java.lang.Object\end{lstlisting}
\subsection{Constructor summary}{
\begin{verse}
{\bf XMLParser(File)} Constructor of the class\\
{\bf XMLParser(FileInputStream)} Constructor of the class\\
{\bf XMLParser(String)} Constructor of the class\\
\end{verse}
}
\subsection{Method summary}{
\begin{verse}
{\bf add\_conversion(String, XMLConversion)} adds an XMLConversion to conversion-map attribute\\
{\bf parse()} Parser of an XML file\\
\end{verse}
}
\subsection{Constructors}{
\vskip -2em
\begin{itemize}
\item{ 
\index{XMLParser(File)}
{\bf  XMLParser}\\
\begin{lstlisting}[frame=none]
public XMLParser(java.io.File file) throws javax.xml.stream.XMLStreamException, java.io.FileNotFoundException\end{lstlisting} %end signature
\begin{itemize}
\item{
{\bf  Description}

Constructor of the class
}
\item{
{\bf  Parameters}
  \begin{itemize}
   \item{
\texttt{file} -- The file}
  \end{itemize}
}%end item
\item{{\bf  Throws}
  \begin{itemize}
   \item{\vskip -.6ex \texttt{javax.xml.stream.XMLStreamException} -- }
   \item{\vskip -.6ex \texttt{java.io.FileNotFoundException} -- }
  \end{itemize}
}%end item
\end{itemize}
}%end item
\item{ 
\index{XMLParser(FileInputStream)}
{\bf  XMLParser}\\
\begin{lstlisting}[frame=none]
public XMLParser(java.io.FileInputStream file) throws javax.xml.stream.XMLStreamException\end{lstlisting} %end signature
\begin{itemize}
\item{
{\bf  Description}

Constructor of the class
}
\item{
{\bf  Parameters}
  \begin{itemize}
   \item{
\texttt{file} -- a FileInputStream}
  \end{itemize}
}%end item
\item{{\bf  Throws}
  \begin{itemize}
   \item{\vskip -.6ex \texttt{javax.xml.stream.XMLStreamException} -- }
  \end{itemize}
}%end item
\end{itemize}
}%end item
\item{ 
\index{XMLParser(String)}
{\bf  XMLParser}\\
\begin{lstlisting}[frame=none]
public XMLParser(java.lang.String file) throws javax.xml.stream.XMLStreamException, java.io.FileNotFoundException\end{lstlisting} %end signature
\begin{itemize}
\item{
{\bf  Description}

Constructor of the class
}
\item{
{\bf  Parameters}
  \begin{itemize}
   \item{
\texttt{file} -- Name of file}
  \end{itemize}
}%end item
\item{{\bf  Throws}
  \begin{itemize}
   \item{\vskip -.6ex \texttt{javax.xml.stream.XMLStreamException} -- }
   \item{\vskip -.6ex \texttt{java.io.FileNotFoundException} -- }
  \end{itemize}
}%end item
\end{itemize}
}%end item
\end{itemize}
}
\subsection{Methods}{
\vskip -2em
\begin{itemize}
\item{ 
\index{add\_conversion(String, XMLConversion)}
{\bf  add\_conversion}\\
\begin{lstlisting}[frame=none]
public XMLParser add_conversion(java.lang.String tag_name,XMLConversion conv_fnc)\end{lstlisting} %end signature
\begin{itemize}
\item{
{\bf  Description}

adds an XMLConversion to conversion-map attribute
}
\item{
{\bf  Parameters}
  \begin{itemize}
   \item{
\texttt{tag\_name} -- }
   \item{
\texttt{conv\_fnc} -- }
  \end{itemize}
}%end item
\item{{\bf  Returns} -- 
this 
}%end item
\end{itemize}
}%end item
\item{ 
\index{parse()}
{\bf  parse}\\
\begin{lstlisting}[frame=none]
public java.lang.Object parse() throws javax.xml.stream.XMLStreamException\end{lstlisting} %end signature
\begin{itemize}
\item{
{\bf  Description}

Parser of an XML file
}
\item{{\bf  Returns} -- 
data from an \texttt{\small MapNode}{\small 
\refdefined{it.matlice.ingsw.tree.MapNode}} 
}%end item
\item{{\bf  Throws}
  \begin{itemize}
   \item{\vskip -.6ex \texttt{javax.xml.stream.XMLStreamException} -- }
  \end{itemize}
}%end item
\end{itemize}
}%end item
\end{itemize}
}
}
}
\printindex
\end{document}
