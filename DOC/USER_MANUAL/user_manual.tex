
\section{Configurazione del Sistema}
Una volta avviato la prima volta, il programma fornirà le credenziali di accesso per il primo configuratore, perché possa configurare il sistema.
Finché il sistema non sarà correttamente configurato, sarà impossibile per l'utente accedere alla sua sezione personale.

Il configuratore che effettua il primo accesso potrà decidere se importare una configurazione da un file con estensione .xml oppure inserire manualmente i valori dei campi.
Si ricorda che tutti i campi saranno riconfigurabili, meno che il campo Piazza, univoco e immutabile.

\subsection{Importazione da file}
Il file selezionato per l'importazione dovrà seguire una apposita struttura: se il configuratore dovesse sbagliare a scrivere il documento, il programma non permetterà l'importazione dei dati.
Di seguito un esempio del file testuale, da cui poter trarre spunto:

todo

Porre particolare attenzione a limitazioni logiche non strattamente legate ala formattazione del documento XML.
Non sarà infatti possibile inserire internamente ai campi valori che non possono essere riconosciuti dal programma, come giorni della settimana inesistenti o orari in formato erroneo (per indicazioni, consultare la sezione successiva della documentazione).
Prestare importanza anche all'inserimento di categorie innestate, in quanto il programma non prevede che una categoria possa avere una unica categoria figlia. Se si decide volerne dividere una, assicurarsi di inserire almeno due categorie figlie.

\subsection{Inserimento manuale}
Inserire da tastiera i valori dei campi visualizzati a schermo, facendo particolare attenzione a non commettere errori per evitare di dover riscrivere da capo.
I giorni al momento possono essere inseriti unicamente in lingua italiana.
Gli istanti di tempo inseribili sono unicamente quelli all'inizio di un'ora o a metà di essa: saranno solo accettati quindi input del tipo xx:00 o xx:30.

Gli intervalli orari in particolare, devono essere in formato ``hh:mm\-hh:mm'', utilizzando come inizio fine le precisazioni precedentemente indicate.
Una volta deciso di smettere di inserire un elenco di campi, inviare a capo due volte per passare alla voce successiva, fino alla fine della procedura di configurazione.

\subsection{Inserimento di gerarchie/categorie}
Una volta inseriti i parametri relativi alle impostazioni, il sistema sarà utilizzabile anche dai normali utenti, che però non potranno ancora utilizzarlo al massimo delle funzionalità.
Bisognerà infatti ancora inserire le gerarchie e categorie su cui appoggiare gli oggetti che gli utenti vorranno scambiare (a meno di un'importazione da file).

Per completare al 100\% la configurazione, selezionare la voce ``Aggiungi nuova gerarchia'' scrivendo sulla console il numero relativo all'opzione e premendo il tasto Invio.
Procedere quindi all'inserimento dei campi che vengono richiesti.
Ricordarsi inoltre che il programma non prevede che una categoria possa avere una unica categoria figlia. Se si decide volerne dividere una, assicurarsi di inserire almeno due categorie figlie.
Finché questa condizione non sarà rispettata, non sarà possibile al configuratore salvare lo stato del sistema, poichè la voce per il salvataggio sarà disabilitata.

Sarà possibile inoltre, selezionando la voce ``Importa informazioni da file testuale'', utilizzare il file di importazione per inserire più facilmente nuove categorie, senza cancellare le precedenti.

\section{Utilizzo da parte dell'Utente Fruitore}
La persona che vorrà utilizzare il programma non dovrà per forza aspettare l'intervento di un configuratore per muovere i primi passi. L'utente può in autonomia registrarsi al portale, caricare le informazioni
relative ai propri prodotti e aprire conversazioni con altri utenti.

\subsection{Registrazione e Accesso}
Se l'utente non è ancora registrato, selezionare all'avvio la voce ``Registrati''. Successivamente, inserire l'username che si intende utilizzare e la password per proteggere il proprio account.
Per ragioni di sicurezza la password da inserire deve essere lunga almeno 8 caratteri e deve contenere almeno una lettera minuscola, una lettera maiuscola, un numero e un simbolo.

Una volta inserita la password, reinserirla per poter confermare. Se il nome utente non è già selezionato da qualcun altro, l'user verrà creato e si potrà procedere a effettuare l' accesso.
Selezionare la voce ``Login'', inserire le proprie credenziali e premere Invio. Se le informazioni sono corrette, l'utente potrà entrare nel proprio portale personale.

\subsection{Inserimento/Rimozione di un articolo}
Per interagire con gli altri Friutori sulla piattaforma, l'utente dovrà avere almeno un articolo sul proprio account pronto per essere barattato.
Per inserire un nuovo articolo, selezionare la voce ``Aggiungi nuovo articolo'' scrivendo sulla console il numero relativo all'opzione e premendo il tasto Invio.

Sarà a questo punto mostrata all'utente una lista di categorie a cui poter associare il proprio articolo. Se non trova la categoria adatta, si prega l'utente di contattare un configuratore facendo richiesta di inserire una nuova categoria nel programmma.
Una volta scelta la categoria opportuna, confermare la scelta e iniziare la procedura di inserimento dei dati. Alcuni campi di compilazione potrebbero essere obbligatori, in tal caso non sarà
possibile registrare l'articolo fino all'inserimento di tutti i campi che necessitano di essere completati.

Una volta inserito l'oggetto da barattare, sarà possibile visualizzarlo selezionando dal menù principale la voce ``Mostra le mie offerte'' oppure la voce ``Mostra offerte per categoria''.

Se l'utente dovesse decidere di ritirare una propria offerta, gli sarebbe possibile tramite la voce del menu ``Ritira un'offerta aperta''. Una volta scelta questa voce,
il fruitore potrà inserire l'indice relativo alla offerta che decide di rimuovere dalla lista degli oggetti scambiabili. Così facendo, gli altri utenti che dovessero cercare uno scambio nella categoria di quella offerta,
non la troverebbero più nell'elenco dei possibili affari.
L'utente possessore dell'articolo potrà comunque visualizzarlo cercando fra le proprie offerte, verificando lo stato acquisito di Offerta Chiusa.

\subsection{Proposta di scambio, Accettazione e Messaggi fra utenti}
