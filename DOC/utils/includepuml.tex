\NewDocumentCommand{\includeplantuml}{m}{%
  \directlua{
    function exists(file)
      local ok, err, code = os.rename(file, file)
      if not ok then
          if code == 13 then
            return true
          end
      end
      return ok, err
    end

    function isdir(path)
      return exists(path.."/")
    end

    function compile (jobname, plantUmlSourceFilename, plantUmlMode)
      print("compiling... " .. jobname .. " " .. plantUmlSourceFilename)

      infile = io.open(jobname, "r")
      instr = infile:read("*a")
      infile:close()

      outfile = io.open(plantUmlSourceFilename, "w")
      outfile:write(instr)
      outfile:close()

      require("plantuml.lua")
      convertPlantUmlToTikz("compiled-puml/" .. jobname, plantUmlMode)
    end

    local jobname = \luastring{#1}
    local plantUmlMode = \luastring{\PlantUmlMode}
    local plantUmlSourceFilename = "compiled-puml/" .. jobname .. "-plantuml.txt"

    if not isdir(plantUmlSourceFilename) then
      os.execute("mkdir -p $(dirname " .. plantUmlSourceFilename .. ")")
    end

    local old_file = io.open(plantUmlSourceFilename, "r")

    if old_file == nil then 
      print("No old file found at " .. plantUmlSourceFilename)
      compile(jobname, plantUmlSourceFilename, plantUmlMode)
    else
      local old_data = old_file:read("*a")

      local new_file = io.open(jobname, "r")
      local new_data = new_file:read("*a")
      io.close(old_file) 

      if old_data == new_data then
      else
        compile(jobname, plantUmlSourceFilename, plantUmlMode)
      end
    end
  }
  \ifthenelse{\equal{\PlantUmlMode}{latex}}{
    \begin{adjustbox}{max width=\linewidth}
      \input{compiled-puml/#1-plantuml.latex}
    \end{adjustbox}
  }{
    \includegraphics[width=\maxwidth{\textwidth}]{compiled-puml/#1-plantuml.\PlantUmlMode}
  }
}