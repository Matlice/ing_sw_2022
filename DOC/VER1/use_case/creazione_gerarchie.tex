\begin{minipage}{\textwidth}
    \subsubsection{Creazione gerarchia}
    \usecase
        {Creazione gerarchia}
        {Configuratore}
        {
            \begin{tabular}{l}
                1.  Il configuratore inserisce il nome della categoria root\\
                2.  Il configuratore inserisce la descrizione \\
                3.  Viene chiesto se si vogliono aggiungere campi nativi\\
                3.1 Viene richiesto il nome del campo\\
                3.2 Viene richiesto il tipo del campo\\
                3.3 Viene richiesta l'obbligatorietà del campo\\
                     \textit{Postcondizione: il campo viene aggiunto alla categoria root}\\
                3.4 ritorna al punto 3
            \end{tabular}\\
            \auxcase{
                \begin{tabular}{l}
                    3.a \textit{Precondizione: Il configuratore} \\
                        \textit{non vuole aggiungere ulteriori campi}\\
                        \textit{Postcondizione: la categoria root viene creata}\\
                        \textit{Postcondizione: la gerarchia viene creata e la categoria }\\
                        \textit{vi è assegnata}\\
                    Fine.
                \end{tabular}
            } \\
            \auxcase{
                \begin{tabular}{l}
                    3.1.a \textit{Precondizione: il nome del campo è duplicato} \\
                    Ritorna al punto 3
                \end{tabular}
            }
        }
        \vspace{0.5cm}
\end{minipage}