\begin{minipage}{\textwidth}
    \subsubsection{Creazione gerarchia}
    \usecase
        {Creazione gerarchia}
        {Configuratore}
        {
            \begin{tabular}{l}
                \textit{<<include>> Accesso configuratore}\\
                1.  Il configuratore inserisce il nome della categoria radice\\
                2.  Il configuratore inserisce la descrizione della categoria radice \\
                3.  Il sistema chiede se si vogliono aggiungere campi nativi\\
                4.  L'utente conferma la creazione della gerarchia\\
                \textit{Postcondizione: la gerarchia viene creata, e le viene assegnata}\\
                \textit{la categoria radice}\\
                Fine
            \end{tabular}\\
            \auxcase{
                \begin{tabular}{l}
                    3.a \textit{Precondizione: il configuratore vuole aggiungere un nuovo}\\
                    \textit{campo} \\
                    3.1 Viene richiesto il nome del campo\\
                    3.2 Viene richiesto il tipo del campo\\
                    3.3 Viene richiesta l'obbligatorietà del campo\\
                    \textit{Postcondizione: il campo viene aggiunto alla categoria radice}\\
                    Torna al punto 3
                \end{tabular}
            } \\
            \auxcase{
                \begin{tabular}{l}
                    3.1.a \textit{Precondizione: il nome del campo è già esistente} \\
                    3.2 Viene segnalato che il campo è già stato inserito\\
                    Torna al punto 3
                \end{tabular}
            } \\
            \auxcase{
                \begin{tabular}{l}
                    4.a L'utente richiede di aggiungere una nuova categoria alla\\gerarchia che sta creando\\
                    5. L'utente inserisce a quale categoria vuole aggiungere la sot-\\tocategoria\\
                    6. L'utente inserisce il nome della categoria che vuole aggiungere\\
                    7. L'utente inserisce eventuali campi nativi alla categoria (si veda\\il primo scenario alternativo)\\
                    Torna al punto 4
                \end{tabular}
            }
        }
        \vspace{0.5cm}
    \textbf{Nota bene:} le categorie introdotte devono avere nome univoco all'interno della gerarchia\\
    \textbf{Nota bene:} le categorie radice devono avere nome univoco tra tutte le radici delle gerarchie\\\\
    \textbf{Nota:} ogni categoria radice è dotata almeno dei campi \texttt{Stato di conservazione} (obbligatorio) e \texttt{Descrizione libera},
    le categorie figlie ereditano i campi nativi della categoria padre. I campi hanno nome univoco.
\end{minipage}