\begin{minipage}{\textwidth}
    \subsubsection{Creazione gerarchia}
    \usecase
        {Creazione gerarchia}
        {Configuratore}
        {
            \begin{tabular}{l}
                1.  Il configuratore inserisce il nome della categoria radice\\
                2.  Il configuratore inserisce la descrizione della categoria radice \\
                3.  Il sistema chiede se si vogliono aggiungere campi nativi\\
                \textit{Postcondizione: la gerarchia viene creata, e le viene assegnata}\\
                \textit{la categoria radice}\\
                Fine
            \end{tabular}\\
            \auxcase{
                \begin{tabular}{l}
                    3.a \textit{Precondizione: il configuratore vuole aggiungere un nuovo}\\
                    \textit{campo} \\
                    3.1 Viene richiesto il nome del campo\\
                    3.2 Viene richiesto il tipo del campo\\
                    3.3 Viene richiesta l'obbligatorietà del campo\\
                    \textit{Postcondizione: il campo viene aggiunto alla categoria radice}\\
                    Torna al punto 3
                \end{tabular}
            } \\
            \auxcase{
                \begin{tabular}{l}
                    3.1.a \textit{Precondizione: il nome del campo è già esistente} \\
                    3.2 Viene segnalato che il campo è già stato inserito\\
                    Torna al punto 3
                \end{tabular}
            }
        }
        \vspace{0.5cm}
\end{minipage}