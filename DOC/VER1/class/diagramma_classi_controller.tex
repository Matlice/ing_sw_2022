\subsubsection{Diagramma delle classi \texttt{model}}
\vspace{0.5cm}
\begin{figure}[H]
    \centering
    \includeplantuml{VER1/diagrams/class/class_diagram_controller.puml}
    \caption{Versione 1: Diagramma UML delle classi, package \texttt{model}}
    \label{fig:class_model_v_1}
\end{figure}

Il package \texttt{model} è il responsabile della gestione dei dati e dell'attuazione dei comandi.
I dati delle impostazioni sono ottenuti dalla classe \textit{Settings}, la quale - nelle versioni successive -
acquisirà i metodi di serializzazione su file in formato modificabile. 

La gestione dell'autenticazione avviene mediante la classe \textit{AuthImpl}, classe innestata con costruttore privato
che può solamente essere istanziata da \textit{Model}.
Così facendo la classe \textit{Model} sarà l'unica a poter ``rilasciare'' un'istanza di \textit{AuthImpl} 
solamente dopo aver verificato le credenziali ed aver autenticato l'utente.

La classe \textit{AuthImpl} svolge la funzione di ``Token di autenticazione'' all'interno dell'applicativo, 
di conseguenza verrà utilizzata come parametro per tutte le funzioni di \textit{Model} che richiedano autenticazione.