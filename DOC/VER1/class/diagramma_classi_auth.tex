\subsubsection{Diagramma delle classi \texttt{auth}}
\vspace{0.5cm}
\begin{figure}[H]
    \centering
    \includeplantuml{VER1/diagrams/class/class_diagram_auth.puml}
    \caption{Versione 1: Diagramma UML delle classi, package \texttt{auth}}
    \label{fig:class_auth_v_1}
\end{figure}

Il package \texttt{auth} è il responsabile della gestione delle autenticazioni all'interno dell'applicazione.
Al fine di garantire la massima flessibilità nello sviluppo, le classi utente non hanno alcuna concezione di autenticazione;
come è possibile vedere dalla figura~\ref{fig:class_data_v_1}, l'interfaccia \textit{UserFactory} non richiede alcuna password
per la creazione dell'utente e nemmeno la classe \textit{User} presenta alcun getter per l'attributo \texttt{password}.

Il package \texttt{auth} è stato ideato, dunque, per scindere i comportamenti relativi all'autenticazione dalla classe \textit{User}.
Questa soluzione permette in futuro di implementare nuovi metodi di login indipendenti dalla struttura dell'applicazione.

La referenza all'attributo \texttt{password} si ha solamente nella classe \textit{ConfiguratorUserImpl} (figura~\ref{fig:class_db_v_1}),
dove è necessario al fine di salvare i dati di autenticazione a database.

L'interfaccia \textit{PasswordAuthenticable} è un'estensione dell'interfaccia \textit{Authenticable} implementata dalla classe astratta
\textit{User} cosicchè, data un'istanza rappresentante l'utente, sarà possibile ottenere la lista dei metodi di autenticazione supportati
mediante il metodo \texttt{getAuthMethods()}.

Per permettere a questo metodo di ritornare un'istanza di \textit{PasswordAuthMethod}, la classe che si vuole autenticare deve essere un'implementazione
di \textit{PasswordAuthenticable}, di conseguenza non ci possono essere errori di mancata implementazione.

Per autenticare l'utente sarà quindi necessario ottenere un'istanza di \textit{Authenticable} tramite il metodo \texttt{getAuthMethods()}
sulla quale chiamare il metodo \texttt{performAuthentication(data: AuthData)} con l'istanza di \textit{AuthData}
necessaria da quest'ultimo e costruita dall'applicativo quando necessaria.