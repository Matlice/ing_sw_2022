\subsubsection{Diagramma delle classi \texttt{data}}
\vspace{0.5cm}
\begin{figure}[H]
    \centering
    \includeplantuml{VER1/diagrams/class/class_diagram_data.puml}
    \caption{Versione 1: Diagramma UML delle classi, package \texttt{data}}
    \label{fig:class_data_v_1}
\end{figure}

Il package \texttt{data} offre un'interfaccia strutturata per la gestione dei tipi di dato dell'applicativo.
La struttura del package si presenta con le classi rappresentanti gli oggetti (\textit{User}, \textit{Category} e \textit{Hierarchy})
definite astratte, e con le classi derivate comunque definite astratte perchè mancanti di gestione dei dati effettivi.

Per la rappresentazione degli utenti abbiamo una classe User che accomuna le funzioni uguali per i (futuri) tipi di utente, tra le quali
l'autenticazione. Per questo motivo \textit{User} fornisce l'interfaccia \textit{Authenticable} (figura~\ref{fig:class_auth_v_1}).

L'unica tipologia di utenti per questa versione è \textit{ConfiguratorUser}.

La rappresentazione delle categorie viene invece ripartita su due classi:
\begin{itemize}
    \vspace{-0.2cm}\item \textit{LeafCategory}: Questa classe rapresenta le foglie dell'albero delle categorie.
    \vspace{-0.2cm}\item \textit{NodeCategory}: Questa classe rappresenta i nodi delle categorie, permettendo l'aggiunta di categorie figlie.
\end{itemize}

La struttura dati utilizzata per rappresentare l'albero è una doppia referenza padre-figli e figlio-padre.
Questo perchè, seppur comportando problematiche relative alla gestione di due tipologie di referenze,
essa permette di navigare attraverso l'albero in modo efficiente in qualsvoglia verso.
Inoltre, per semplificare la programmazione, il riferimento al padre verrà aggiornato ogni volta che una categoria viene
aggiunta a un padre oppure rimossa; in questo modo è possibile garantire la consistenza dei dati senza complicazioni ulterori.