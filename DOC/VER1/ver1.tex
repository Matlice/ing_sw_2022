\subsection{Casi d'uso}
Di seguito sono elencati i casi d'uso ritenuti necessari a sopperire alle funzionalità
dell'applicativo.
\\\\
\begin{minipage}{\textwidth}
    \subsubsection{Accesso utente}
    \usecase
        {Accesso utente}
        {Utente}
        {
            %Accesso normale
            \begin{tabular}{l}
                1. L'utente inserisce username e password\\
                \textit{Postcondizione: l'utente ha accesso all'applicazione}
            \end{tabular} \\
            \auxcase{
                \begin{tabular}{l}
                    1.a \textit{Precondizione: utente non esistente o password non corretta}\\
                        Torna al punto 1.
                \end{tabular}
            }
        }
        \vspace{0.5cm}

\end{minipage}
\begin{minipage}{\textwidth}

\subsubsection{Accesso configuratore}
\usecase
    {Accesso configuratore}
    {
        Configuratore %\\
        %\auxactor{teto}
    }
    {
        %Accesso normale
        \begin{tabular}{l}
            1. <<include>> "Accesso utente" (verifica credenziali e permessi)\\
            2. L'utente ha già effettuato l'accesso\\
            3. Mostra il menu di configurazione\\
        \end{tabular} \\

        %primo accesso
        \auxcase{
            \begin{tabular}{l}
                2.a L'utente è al primo accesso\\
                2.b L'utente viene forzato a cambiare la password\\
                2.c ritorna al punto 3
            \end{tabular}
        }
    }
    \vspace{0.5cm}


\end{minipage}
\begin{minipage}{\textwidth}
    \subsubsection{Creazione nuovo account configuratore}
    Per questo caso d'uso è stato considerato un terzo tipo di attore, ovvero
    l'amministratore. Esso può accedere direttamente all'applicativo, modificandone
    i file, ma non è utente dell'applicativo.
    \usecase
        {Creazione nuovo account configuratore}
        {Configuratore o Amministratore}
        {
            \begin{tabular}{l}
                1. \textit{Precondizione: l'amministratore ha accesso alla macchina su}\\
                \textit{cui è eseguito l'applicativo, oppure il configuratore è loggato}\\
                2. L'utente inserisce il nome del nuovo account configuratore\\
                \textit{Postcondizione: Un nuovo account configuratore viene creato} \\
                \textit{con password randomica (da modificare al primo accesso)}\\
                Fine
            \end{tabular} \\
            
            \auxcase{
                \begin{tabular}{l}
                    2.a \textit{Precondizione: il nome utente inserito è già esistente}\\
                    Torna al punto 1
                \end{tabular}
            }
        }
        \vspace{0.5cm}
\end{minipage}
\begin{minipage}{\textwidth}
    \subsubsection{Creazione gerarchia}
    \usecase
        {Creazione gerarchia}
        {Configuratore}
        {
            \begin{tabular}{l}
                \textit{<<include>> Accesso configuratore}\\
                1.  Il configuratore inserisce il nome della categoria radice\\
                2.  Il configuratore inserisce la descrizione della categoria radice \\
                3.  Il sistema chiede se si vogliono aggiungere campi nativi\\
                4.  L'utente conferma la creazione della gerarchia\\
                \textit{Postcondizione: la gerarchia viene creata, e le viene assegnata}\\
                \textit{la categoria radice}\\
                Fine
            \end{tabular}\\
            \auxcase{
                \begin{tabular}{l}
                    3.a \textit{Precondizione: il configuratore vuole aggiungere un nuovo}\\
                    \textit{campo} \\
                    3.1 Viene richiesto il nome del campo\\
                    3.2 Viene richiesto il tipo del campo\\
                    3.3 Viene richiesta l'obbligatorietà del campo\\
                    \textit{Postcondizione: il campo viene aggiunto alla categoria radice}\\
                    Torna al punto 3
                \end{tabular}
            } \\
            \auxcase{
                \begin{tabular}{l}
                    3.1.a \textit{Precondizione: il nome del campo è già esistente} \\
                    3.2 Viene segnalato che il campo è già stato inserito\\
                    Torna al punto 3
                \end{tabular}
            } \\
            \auxcase{
                \begin{tabular}{l}
                    4.a L'utente richiede di aggiungere una nuova categoria alla\\gerarchia che sta creando\\
                    5. L'utente inserisce a quale categoria vuole aggiungere la sot-\\tocategoria\\
                    6. L'utente inserisce il nome della categoria che vuole aggiungere\\
                    7. L'utente inserisce eventuali campi nativi alla categoria (si veda\\il primo scenario alternativo)\\
                    Torna al punto 4
                \end{tabular}
            }
        }
        \vspace{0.5cm}
    \textbf{Nota bene:} le categorie introdotte devono avere nome univoco all'interno della gerarchia\\
    \textbf{Nota bene:} le categorie radice devono avere nome univoco tra tutte le radici delle gerarchie\\\\
    \textbf{Nota:} ogni categoria radice è dotata almeno dei campi \texttt{Stato di conservazione} (obbligatorio) e \texttt{Descrizione libera},
    le categorie figlie ereditano i campi nativi della categoria padre. I campi hanno nome univoco.
\end{minipage}

\pagebreak
\subsection{Casi d'uso (UML)}
\vspace{0.5cm}
\begin{figure}[H]
    \centering
    \includeplantuml{uml/use_case/version_1.puml}
    \caption{Diagramma UML dei casi d'uso, versione 1}
    \label{fig:use_case_uml_v1}
\end{figure}

\pagebreak
\subsection{Diagrammi supplementari}
\subsection{Ciclo di vita dell'utente configuratore}
Il seguente diagramma rappresenta il ciclo di vita dell'utente configuratore differenziando i casi "Primo login", dove 
viene forzato al cambio password e "Login ordinario", dove invece procede al menu principale.
\vspace{0.5cm}
\begin{figure}[H]
    \centering
    \includeplantuml{uml/states/ciclo_vita_utente_configuratore.puml}
    \caption{Diagramma UML degli stati, ciclo  vita utente configuratore}
    \label{fig:states_config_user_lifecycle}
\end{figure}

\pagebreak
\subsection{Algoritmo verifica delle credenziali}
Il seguente diagramma rappresenta il procedimento per autenticare un utente nel sistema
\vspace{0.5cm}
\begin{figure}[H]
    \centering
    \includeplantuml{uml/activity/login_utente.puml}
    \caption{Diagramma UML delle attività, login utente}
    \label{fig:activity_user_login}
\end{figure}

\vspace{0.5cm}
\begin{figure}[H]
    \centering
    \includeplantuml{uml/activity/verifica_password.puml}
    \caption{Diagramma UML delle attività, ref: verifica password}
    \label{fig:activity_user_login}
\end{figure}

\pagebreak
\subsection{Diagramma delle classi}
\vspace{0.5cm}
\begin{figure}[H]
    \centering
    \includeplantuml{uml/class/class_diagram.puml}
    \caption{Diagramma UML delle classi, versione 1}
    \label{fig:class_v_1}
\end{figure}

