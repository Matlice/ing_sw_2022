\subsection{Casi d'uso}
Di seguito sono elencati i casi d'uso ritenuti necessari a sopperire alle funzionalità
dell'applicativo.
\\\\
\begin{minipage}{\textwidth}
    \subsubsection{Accesso utente}
    \usecase
        {Accesso utente}
        {Utente}
        {
            %Accesso normale
            \begin{tabular}{l}
                1. L'utente inserisce username e password\\
                \textit{Postcondizione: l'utente ha accesso all'applicazione}
            \end{tabular} \\
            \auxcase{
                \begin{tabular}{l}
                    1.a \textit{Precondizione: utente non esistente o password non corretta}\\
                        Torna al punto 1.
                \end{tabular}
            }
        }
        \vspace{0.5cm}

\end{minipage}
\begin{minipage}{\textwidth}

\subsubsection{Accesso configuratore}
\usecase
    {Accesso configuratore}
    {
        Configuratore %\\
        %\auxactor{teto}
    }
    {
        %Accesso normale
        \begin{tabular}{l}
            1. <<include>> "Accesso utente" (verifica credenziali e permessi)\\
            2. L'utente ha già effettuato l'accesso\\
            3. Mostra il menu di configurazione\\
        \end{tabular} \\

        %primo accesso
        \auxcase{
            \begin{tabular}{l}
                2.a L'utente è al primo accesso\\
                2.b L'utente viene forzato a cambiare la password\\
                2.c ritorna al punto 3
            \end{tabular}
        }
    }
    \vspace{0.5cm}


\end{minipage}
\begin{minipage}{\textwidth}
    \subsubsection{Creazione nuovo account configuratore}
    Per questo caso d'uso è stato considerato un terzo tipo di attore, ovvero
    l'amministratore. Esso può accedere direttamente all'applicativo, modificandone
    i file, ma non è utente dell'applicativo.
    \usecase
        {Creazione nuovo account configuratore}
        {Configuratore o Amministratore}
        {
            \begin{tabular}{l}
                1. \textit{Precondizione: l'amministratore ha accesso alla macchina su}\\
                \textit{cui è eseguito l'applicativo, oppure il configuratore è loggato}\\
                2. L'utente inserisce il nome del nuovo account configuratore\\
                \textit{Postcondizione: Un nuovo account configuratore viene creato} \\
                \textit{con password randomica (da modificare al primo accesso)}\\
                Fine
            \end{tabular} \\
            
            \auxcase{
                \begin{tabular}{l}
                    2.a \textit{Precondizione: il nome utente inserito è già esistente}\\
                    Torna al punto 1
                \end{tabular}
            }
        }
        \vspace{0.5cm}
\end{minipage}
\begin{minipage}{\textwidth}
    \subsubsection{Creazione gerarchia}
    \usecase
        {Creazione gerarchia}
        {Configuratore}
        {
            \begin{tabular}{l}
                \textit{<<include>> Accesso configuratore}\\
                1.  Il configuratore inserisce il nome della categoria radice\\
                2.  Il configuratore inserisce la descrizione della categoria radice \\
                3.  Il sistema chiede se si vogliono aggiungere campi nativi\\
                4.  L'utente conferma la creazione della gerarchia\\
                \textit{Postcondizione: la gerarchia viene creata, e le viene assegnata}\\
                \textit{la categoria radice}\\
                Fine
            \end{tabular}\\
            \auxcase{
                \begin{tabular}{l}
                    3.a \textit{Precondizione: il configuratore vuole aggiungere un nuovo}\\
                    \textit{campo} \\
                    3.1 Viene richiesto il nome del campo\\
                    3.2 Viene richiesto il tipo del campo\\
                    3.3 Viene richiesta l'obbligatorietà del campo\\
                    \textit{Postcondizione: il campo viene aggiunto alla categoria radice}\\
                    Torna al punto 3
                \end{tabular}
            } \\
            \auxcase{
                \begin{tabular}{l}
                    3.1.a \textit{Precondizione: il nome del campo è già esistente} \\
                    3.2 Viene segnalato che il campo è già stato inserito\\
                    Torna al punto 3
                \end{tabular}
            } \\
            \auxcase{
                \begin{tabular}{l}
                    4.a L'utente richiede di aggiungere una nuova categoria alla\\gerarchia che sta creando\\
                    5. L'utente inserisce a quale categoria vuole aggiungere la sot-\\tocategoria\\
                    6. L'utente inserisce il nome della categoria che vuole aggiungere\\
                    7. L'utente inserisce eventuali campi nativi alla categoria (si veda\\il primo scenario alternativo)\\
                    Torna al punto 4
                \end{tabular}
            }
        }
        \vspace{0.5cm}
    \textbf{Nota bene:} le categorie introdotte devono avere nome univoco all'interno della gerarchia\\
    \textbf{Nota bene:} le categorie radice devono avere nome univoco tra tutte le radici delle gerarchie\\\\
    \textbf{Nota:} ogni categoria radice è dotata almeno dei campi \texttt{Stato di conservazione} (obbligatorio) e \texttt{Descrizione libera},
    le categorie figlie ereditano i campi nativi della categoria padre. I campi hanno nome univoco.
\end{minipage}
\begin{minipage}{\textwidth}
    \subsubsection{Visualizzazione gerarchia}
    \usecase
        {Visualizzazione gerarchia}
        {
            Configuratore
        }
        {
            %Visualizzazione gerarchia
            \begin{tabular}{l}
                1. Il configuratore richiede la visualizzazione delle gerarchie\\
                2. Vengono visualizzate tutte le gerarchie presenti a sistema\\
                3. Il configuratore sceglie di quale gerarchia visualizzare le infor-\\mazioni\\
                4. Il sistema mostra tutte le informazioni relative alla gerarchia,\\
                ovvero l'albero delle categorie con i relativi campi nativi\\
                Fine
            \end{tabular}\\

            \auxcase{
                \begin{tabular}{l}
                    3.a \textit{Precondizione: la gerarchia selezionata non esiste}\\
                    4. Viene mostrato un messaggio di errore\\
                    Torna al punto 3
                \end{tabular}
            }

        }
        \vspace{0.5cm}
\end{minipage}

\pagebreak
\subsection{Casi d'uso (UML)}
\vspace{0.5cm}
\begin{figure}[H]
    \centering
    \includeplantuml{VER3/diagrams/use_case/version_3.puml}
    \caption{Versione 3: Diagramma UML dei casi d'uso}
    \label{fig:use_case_uml_v3}
\end{figure}


\pagebreak
\subsection{Diagrammi delle classi}

\textit{Si noti che, per motivi di spazio, il diagramma generale delle classi è stato suddiviso in più sottodiagrammi, uno per ogni package.}
\textit{Le referenze tra i vari diagrammi sono rappresentate con una classe, senza dettagli, al di fuori del package stesso.}

\subsubsection{Diagramma delle classi \texttt{auth}}
\vspace{0.5cm}
\begin{figure}[H]
    \centering
    \includeplantuml{VER1/diagrams/class/class_diagram_auth.puml}
    \caption{Versione 1: Diagramma UML delle classi, package \texttt{auth}}
    \label{fig:class_auth_v_1}
\end{figure}

Il package \texttt{auth} è il responsabile della gestione delle autenticazioni all'interno dell'applicazione.
Al fine di garantire la massima flessibilità nello sviluppo, le classi utente non hanno alcuna concezione di autenticazione;
come è possibile vedere dalla figura~\ref{fig:class_data_v_1}, l'interfaccia \textit{UserFactory} non richiede alcuna password
per la creazione dell'utente e nemmeno la classe \textit{User} presenta alcun getter per l'attributo \texttt{password}.

Il package \texttt{auth} è stato ideato, dunque, per scindere i comportamenti relativi all'autenticazione dalla classe \textit{User}.
Questa soluzione permette in futuro di implementare nuovi metodi di login indipendenti dalla struttura dell'applicazione.

La referenza all'attributo \texttt{password} si ha solamente nella classe \textit{ConfiguratorUserImpl} (figura~\ref{fig:class_db_v_1}),
dove è necessario al fine di salvare i dati di autenticazione a database.

L'interfaccia \textit{PasswordAuthenticable} è un'estensione dell'interfaccia \textit{Authenticable} implementata dalla classe astratta
\textit{User} cosicchè, data un'istanza rappresentante l'utente, sarà possibile ottenere la lista dei metodi di autenticazione supportati
mediante il metodo \texttt{getAuthMethods()}.

Per permettere a questo metodo di ritornare un'istanza di \textit{PasswordAuthMethod}, la classe che si vuole autenticare deve essere un'implementazione
di \textit{PasswordAuthenticable}, di conseguenza non ci possono essere errori di mancata implementazione.

Per autenticare l'utente sarà quindi necessario ottenere un'istanza di \textit{Authenticable} tramite il metodo \texttt{getAuthMethods()}
sulla quale chiamare il metodo \texttt{performAuthentication(data: AuthData)} con l'istanza di \textit{AuthData}
necessaria da quest'ultimo e costruita dall'applicativo quando necessaria.
\subsubsection{Diagramma delle classi \texttt{controller}}
\vspace{0.5cm}
\begin{figure}[H]
    \centering
    \includeplantuml{VER1/diagrams/class/class_diagram_controller.puml}
    \caption{Versione 1: Diagramma UML delle classi, package \texttt{controller}}
    \label{fig:class_controller_v_1}
\end{figure}
\subsubsection{Diagramma delle classi \texttt{data}}
\vspace{0.5cm}
\begin{figure}[H]
    \centering
    \includeplantuml{VER2/diagrams/class/class_diagram_data.puml}
    \caption{Versione 2: Diagramma UML delle classi, package \texttt{data}}
    \label{fig:class_data_v_2}
\end{figure}

Al package \texttt{data} viene aggiunta un'interfaccia per la gestione dei parametri di configurazione \textit{SettingsFactory}.

Le implementazioni di questa si occupanno di istanziare, a partire dai dati in datase, la classe \textit{Settings}.
Questa contiene la piazza di scambio, il numero di giorni di scadenza, una lista di luoghi (\textit{String}) dove è possibile fare gli scambi,
una lista di giorni (\textit{Day} è un'enum) ed una lista di intervalli orari.

Gli intervalli orari sono rappresentati dalla classe \textit{Interval} che a sua volta è definita tramite gli orari di inizio e fine, la rappresentazione
degli orari è delegata alla classe \textit{Time}. È previsto un metodo overlaps() che permette di verificare se un intervallo sia sovrapposto
ad un altro intervallo, e un metodo statico che permetta di riunire diversi intervalli nel minor numero possibile.
\subsubsection{Diagramma delle classi \texttt{impl.sqlite}}
\vspace{0.5cm}
\begin{figure}[H]
    \centering
    \includeplantuml{VER1/diagrams/class/class_diagram_db.puml}
    \caption{Versione 1: Diagramma UML delle classi, package \texttt{impl.sqlite}}
    \label{fig:class_db_v_1}
\end{figure}

Questo package implementa le interfacce fornite dal package padre, sfruttando come motore di dati un database supportato da JDBC
tramite l'utilizzo di una libreria esterna (\textit{ORMLite}), la quale permette la semplificazione della gestione dei dati nel database
e permette l'utilizzo di helper per la scrittura delle query. L'associazione dato-oggetto è stata così progettata:

Ogni rapresentazione di un oggetto a database viene identificata in una classe wrapper: \textit{UserDB}, \textit{CategoryDB}, \textit{HierarchyDB}.
Mediante queste classi sarà possibile accedere direttamente ai dati così come sono salvati nel database.

Il package offre anche un'implementazione delle Factory definite nel package padre. Lo scopo di queste classi
è istanziare le classi di implementazioni delle classi \textit{User}, \textit{Category}, \textit{Hierarchy} (figura~\ref{fig:class_data_v_1}
nella loro declinazione corretta in base allo stato del programma e del dato a database.

In particolare, la classe \textit{UserFactoryImpl} instanzierà una classe \textit{ConfiguratorUser} se a database l'utente avrà
i permessi di configuratore.

Caso più particolare è quello delle categorie:
la classe \textit{CategoryFactoryImpl} otterrà dall'origine dati la struttura delle categorie definita mediante backreference al padre, 
e ricostruirà l'albero delle categorie composto da elementi di tipo \textit{NodeCategoryImpl} o \textit{LeafCategoryImpl}.

\pagebreak
\subsection{Diagrammi dei componenti}
\subsubsection{Diagramma dei componenti}
\vspace{0.5cm}
\begin{figure}[H]
    \centering
    \includeplantuml{VER1/diagrams/component/diagramma_componenti.puml}
    \caption{Versione 1: Diagramma UML dei componenti}
    \label{fig:components_v_1}
\end{figure}

\pagebreak
\subsection{Diagrammi supplementari}
\subsubsection{Ciclo di vita dell'utente configuratore}
Il seguente diagramma rappresenta il ciclo di vita dell'utente configuratore differenziando i casi "Primo login", dove 
viene forzato al cambio password e "Login ordinario", dove invece procede al menu principale.
\vspace{0.5cm}
\begin{figure}[H]
    \centering
    \includeplantuml{VER1/diagrams/states/ciclo_vita_utente_configuratore.puml}
    \caption{Versione 1: Diagramma UML degli stati, ciclo  vita utente configuratore}
    \label{fig:states_config_user_lifecycle}
\end{figure}
\pagebreak
\subsubsection{Algoritmo verifica delle credenziali}
Il seguente diagramma rappresenta il procedimento per autenticare un utente nel sistema
\vspace{0.5cm}
\begin{figure}[H]
    \centering
    \includeplantuml{VER1/diagrams/activity/login_utente.puml}
    \caption{Diagramma UML delle attività, login utente}
    \label{fig:activity_user_login}
\end{figure}
\pagebreak
\subsubsection{Algoritmo verifica della password}
Il seguente diagramma rappresenta il procedimento per verificare la password
\vspace{0.5cm}
\begin{figure}[H]
    \centering
    \includeplantuml{VER1/diagrams/activity/verifica_password.puml}
    \caption{Diagramma UML delle attività, ref: verifica password}
    \label{fig:activity_user_password}
\end{figure}
