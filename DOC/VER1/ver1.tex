\subsection{Casi d'uso}
Di seguito sono elencati i casi d'uso ritenuti necessari a sopperire alle funzionalità
dell'applicativo.
\\\\
\begin{minipage}{\textwidth}
    \subsubsection{Accesso utente}
    \usecase
        {Accesso utente}
        {Utente}
        {
            %Accesso normale
            \begin{tabular}{l}
                1. All'utente viene presentato un form per il login\\
                2. L'utente inserisce username e password\\
                \textit{Postcondizione: l'utente ha accesso all'applicazione}\\
                Fine
            \end{tabular} \\
            \auxcase{
                \begin{tabular}{l}
                    2.a \textit{Precondizione: utente non esistente o password non corretta}\\
                    3. Il sistema segnala che l'utente inserito non esiste oppure la\\
                       password è errata\\
                    Torna al punto 1
                \end{tabular}
            }
        }
        \vspace{0.5cm}

\end{minipage}
\begin{minipage}{\textwidth}

\subsubsection{Accesso configuratore}
\usecase
    {Accesso configuratore}
    {
        Configuratore %\\
        %\auxactor{teto}
    }
    {
        %Accesso normale
        \begin{tabular}{l}
            1. <<include>> "Accesso utente" (verifica credenziali e permessi)\\
            2. L'utente ha già effettuato l'accesso\\
            3. Mostra il menu di configurazione\\
        \end{tabular} \\

        %primo accesso
        \auxcase{
            \begin{tabular}{l}
                2.a L'utente è al primo accesso\\
                2.b L'utente viene forzato a cambiare la password\\
                2.c ritorna al punto 3
            \end{tabular}
        }
    }
    \vspace{0.5cm}


\end{minipage}
\begin{minipage}{\textwidth}
    \subsubsection{Creazione nuovo account configuratore}
    \usecase
        {Creazione nuovo account configuratore}
        {Amministratore}
        {
            \begin{tabular}{l}
                1. \textit{Precondizione: l'utente deve avere accesso fisico alla macchina}\\
                L'amministratore inserisce il nuovo nome utente\\
                2. \textit{Postcondizione: Un nuovo account configuratore viene aggiunto} \\
                \textit{con passwod randomica}
            \end{tabular} \\
            \auxcase{
                \begin{tabular}{l}
                    1.a \textit{Precondizione: il nome utente inserito è duplicato}\\
                    Errore.
                \end{tabular}
            }
        }
        \vspace{0.5cm}
\end{minipage}
\begin{minipage}{\textwidth}
    \subsubsection{Creazione gerarchia}
    \usecase
        {Creazione gerarchia}
        {Configuratore}
        {
            \begin{tabular}{l}
                1.  Il configuratore inserisce il nome della categoria radice\\
                2.  Il configuratore inserisce la descrizione della categoria radice \\
                3.  Il sistema chiede se si vogliono aggiungere campi nativi\\
                \textit{Postcondizione: la gerarchia viene creata, e le viene assegnata}\\
                \textit{la categoria radice}\\
                Fine
            \end{tabular}\\
            \auxcase{
                \begin{tabular}{l}
                    3.a \textit{Precondizione: il configuratore vuole aggiungere un nuovo}\\
                    \textit{campo} \\
                    3.1 Viene richiesto il nome del campo\\
                    3.2 Viene richiesto il tipo del campo\\
                    3.3 Viene richiesta l'obbligatorietà del campo\\
                    \textit{Postcondizione: il campo viene aggiunto alla categoria radice}\\
                    Torna al punto 3
                \end{tabular}
            } \\
            \auxcase{
                \begin{tabular}{l}
                    3.1.a \textit{Precondizione: il nome del campo è già esistente} \\
                    3.2 Viene segnalato che il campo è già stato inserito\\
                    Torna al punto 3
                \end{tabular}
            }
        }
        \vspace{0.5cm}
\end{minipage}

\pagebreak
\subsection{Casi d'uso (UML)}
\vspace{0.5cm}
\begin{figure}[H]
    \centering
    \includeplantuml{uml/use_case/version_1.puml}
    \caption{Diagramma UML dei casi d'uso, versione 1}
    \label{fig:use_case_uml_v1}
\end{figure}

\pagebreak
\subsection{Diagrammi supplementari}
\subsection{Ciclo di vita dell'utente configuratore}
Il seguente diagramma rappresenta il ciclo di vita dell'utente configuratore differenziando i casi "Primo login", dove 
viene forzato al cambio password e "Login ordinario", dove invece procede al menu principale.
\vspace{0.5cm}
\begin{figure}[H]
    \centering
    \includeplantuml{uml/states/ciclo_vita_utente_configuratore.puml}
    \caption{Diagramma UML degli stati, ciclo  vita utente configuratore}
    \label{fig:states_config_user_lifecycle}
\end{figure}

\pagebreak
\subsection{Algoritmo verifica delle credenziali}
Il seguente diagramma rappresenta il procedimento per autenticare un utente nel sistema
\vspace{0.5cm}
\begin{figure}[H]
    \centering
    \includeplantuml{uml/activity/login_utente.puml}
    \caption{Diagramma UML delle attività, login utente}
    \label{fig:activity_user_login}
\end{figure}

\vspace{0.5cm}
\begin{figure}[H]
    \centering
    \includeplantuml{uml/activity/verifica_password.puml}
    \caption{Diagramma UML delle attività, ref: verifica password}
    \label{fig:activity_user_login}
\end{figure}

\pagebreak
\subsection{Diagramma delle classi}
\vspace{0.5cm}
\begin{figure}[H]
    \centering
    \includeplantuml{uml/class/class_diagram.puml}
    \caption{Diagramma UML delle classi, versione 1}
    \label{fig:class_v_1}
\end{figure}

