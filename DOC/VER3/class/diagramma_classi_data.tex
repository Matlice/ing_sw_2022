\subsubsection{Diagramma delle classi \texttt{data}}
\vspace{0.5cm}
\begin{figure}[H]
    \centering
    \includeplantuml{VER3/diagrams/class/class_diagram_data.puml}
    \caption{Versione 3: Diagramma UML delle classi, package \texttt{data}}
    \label{fig:class_data_v_3}
\end{figure}

Al package \texttt{data} viene aggiunta un'interfaccia per la gestione delle offerte \textit{OfferFactory}.

L'implementazione di questa si occupa di istanziare, a partire dai dati in database, la classe \textit{Offer}.
Questa classe contiene il nome, la categoria foglia a cui è associata e l'utente proprietario; inoltre, possiede una mappa che associa i nomi dei campi al valore compilato in fase di inserimento dall'utente.

Gli intervalli orari sono rappresentati dalla classe \textit{Interval} che a sua volta è definita tramite gli orari di inizio e fine, la rappresentazione
degli orari è delegata alla classe \textit{Time}. È previsto un metodo overlaps() che permette di verificare se un intervallo sia sovrapposto
ad un altro intervallo, e un metodo statico che permetta di riunire diversi intervalli nel minor numero possibile.