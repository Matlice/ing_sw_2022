\subsubsection{Diagramma delle classi \texttt{impl.sqlite}}
\vspace{0.5cm}
\begin{figure}[H]
    \centering
    \includeplantuml{VER3/diagrams/class/class_diagram_db.puml}
    \caption{Versione 3: Diagramma UML delle classi, package \texttt{impl.sqlite}}
    \label{fig:class_db_v_3}
\end{figure}

L'implementazione delle offerte a livello di database è effettuato tramite due tabelle, rappresentate da due classi: \textit{OfferDB} e \textit{OfferFieldDB}.
La prima contiene nelle colonne il nome dell'articolo, lo stato, una referenza all'utente proprietario ed una alla categoria di appartenenza; ogni riga della seconda tabella invece,
rappresenta un singolo campo compilato di un'offerta, per questo ha un riferimento al campo a cui si riferisce (\textit{CategoryFieldDB}) e il valore che questo assume per la specifica offerta
(referenziata da \texttt{offer\_ref}).

La lettura e scrittura da database è mediata da \textit{OfferFactoryImpl}, un'implementazione dell'interfaccia
\textit{OfferFactory}. La classe istanziata è \textit{OfferImpl} che estende la classe astratta \textit{Offer}.