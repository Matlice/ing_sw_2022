A seguire i cambiamenti da apportare per la versione 3 dell'applicativo

\subsection{Casi d'uso}
Di seguito sono elencati i casi d'uso ritenuti necessari a sopperire alle funzionalità
dell'applicativo.
\\\\
\begin{minipage}{\textwidth}
    \subsubsection{Pubblicazione articolo}
    \usecase
        {Pubblicazione articolo}
        {
            Fruitore %\\
        }
        {
            %Pubblicazione
            \begin{tabular}{l}
                1. \textit{<<include>> Accesso fruitore}\\
                2. L'utente inserisce il nome dell'articolo che intende pubblicare\\
                3. L'utente sceglie la categoria di appartenenza\\
                4. L'utente compila i campi relativi alla categoria scelta\\
                5. L'utente conferma\\
                \textit{Postcondizione: l'articolo (legato all'utente}\\
                \textit{che l'ha inserito) è pubblicato e disponibile}\\
                \textit{allo scambio (stato }\texttt{Offerta aperta}\textit{)}\\
                Fine
            \end{tabular}\\

            %categoria non esistente
            \auxcase{
                \begin{tabular}{l}
                    3.a \textit{Precondizione: l'utente sceglie una categoria non esistente (op-}\\
                    \textit{pure non categoria foglia)}\\
                    4. Viene mostrato un messaggio di errore\\
                    Torna al punto 3
                \end{tabular}
            }\\

            %campi non compilati
            \auxcase{
                \begin{tabular}{l}
                    5.a \textit{Precondizione: l'utente non ha inserito i valori per tutti}\\
                    \textit{i campi obbligatori}\\
                    6. Viene mostrato un messaggio di errore\\
                    Torna al punto 4
                \end{tabular}
            }\\

            %utente annulla
            \auxcase{
                \begin{tabular}{l}
                    5.b L'utente annulla l'operazione di pubblicazione dell'articolo\\
                    Fine
                \end{tabular}
            }

        }
        \vspace{0.5cm}
\end{minipage}
\begin{minipage}{\textwidth}
    \subsubsection{Ritiro offerta}
    \usecase
        {Ritiro offerta}
        {
            Fruitore %\\
        }
        {
            %Pubblicazione
            \begin{tabular}{l}
                1. \textit{Precondizione: l'utente ha già inserito almeno un articolo}\\
                2. L'utente seleziona un suo articolo pubblicato\\
                3. L'utente richiede di ritirare la sua offerta\\
                \textit{Postcondizione: l'articolo precedentemente inserito viene ritirato}\\
                \textit{(stato }\texttt{Offerta ritirata}\textit{)}\\
                Fine
            \end{tabular}\\

            %articolo non in Offerta aperta
            \auxcase{
                \begin{tabular}{l}
                    3.a \textit{Precondizione: l'articolo selezionato non è in stato \texttt{Offerta}}\\
                    \textit{\texttt{aperta}}\\
                    4. Viene mostrato un messaggio di errore\\
                    Fine
                \end{tabular}
            }

        }
        \vspace{0.5cm}
\end{minipage}
\begin{minipage}{\textwidth}
    \subsubsection{Visualizzazione offerte per categoria}
    \usecase
        {Visualizzazione offerte per categoria}
        {
            Fruitore o Configuratore %\\
        }
        {
            %Pubblicazione
            \begin{tabular}{l}
                1. \textit{<<include>> Accesso utente}\\
                2. L'utente seleziona una categoria\\
                \textit{Postcondizione: viene fornita una lista di articoli disponibili allo}\\
                \textit{scambio appartenenti (stato \texttt{Offerta aperta}) alla categoria}\\
                \textit{selezionata dall'utente}\\
                Fine
            \end{tabular}\\

            %categoria non esistente
            \auxcase{
                \begin{tabular}{l}
                    2.a \textit{Precondizione: l'utente sceglie una categoria non esistente}\\
                    3. Viene mostrato un messaggio di errore\\
                    Fine
                \end{tabular}
            }\\

            %categoria non foglia
            \auxcase{
                \begin{tabular}{l}
                    2.a \textit{Precondizione: l'utente sceglie una categoria non foglia}\\
                    3. Viene mostrato un messaggio di errore\\
                    Fine
                \end{tabular}
            }

        }
        \vspace{0.5cm}
\end{minipage}
\begin{minipage}{\textwidth}
    \subsubsection{Visualizzazione offerte dell'utente}
    \usecase
        {Visualizzazione offerte dell'utente}
        {
            Fruitore
        }
        {
            %Pubblicazione
            \begin{tabular}{l}
                1. \textit{<<include>> Accesso fruitore}\\
                2. L'utente richiede di visualizzare tutte le sue offerte\\
                \textit{Postcondizione: viene fornita una lista di tutti gli articoli inseriti}\\
                \textit{dall'utente (indipendentemente dallo stato)}\\
                Fine
            \end{tabular}

        }
        \vspace{0.5cm}
\end{minipage}

\pagebreak
\subsection{Casi d'uso (UML)}
\vspace{0.5cm}
\begin{figure}[H]
    \centering
    \includeplantuml{VER3/diagrams/use_case/version_3.puml}
    \caption{Versione 3: Diagramma UML dei casi d'uso}
    \label{fig:use_case_uml_v3}
\end{figure}


\pagebreak
\subsection{Diagrammi delle classi}

\textit{Si noti che, per motivi di spazio, il diagramma generale delle classi è stato suddiviso in più sottodiagrammi, uno per ogni package.}
\textit{Le referenze tra i vari diagrammi sono rappresentate con una classe, senza dettagli, al di fuori del package stesso.}
\textit{Inoltre, le classi già specificate nella versione precedente saranno riportate senza attributi e metodi}

\subsubsection{Diagramma delle classi \texttt{controller}}
\vspace{0.5cm}
\begin{figure}[H]
    \centering
    \includeplantuml{VER1/diagrams/class/class_diagram_controller.puml}
    \caption{Versione 1: Diagramma UML delle classi, package \texttt{controller}}
    \label{fig:class_controller_v_1}
\end{figure}
\subsubsection{Diagramma delle classi \texttt{data}}
\vspace{0.5cm}
\begin{figure}[H]
    \centering
    \includeplantuml{VER2/diagrams/class/class_diagram_data.puml}
    \caption{Versione 2: Diagramma UML delle classi, package \texttt{data}}
    \label{fig:class_data_v_2}
\end{figure}

Al package \texttt{data} viene aggiunta un'interfaccia per la gestione dei parametri di configurazione \textit{SettingsFactory}.

Le implementazioni di questa si occupanno di istanziare, a partire dai dati in datase, la classe \textit{Settings}.
Questa contiene la piazza di scambio, il numero di giorni di scadenza, una lista di luoghi (\textit{String}) dove è possibile fare gli scambi,
una lista di giorni (\textit{Day} è un'enum) ed una lista di intervalli orari.

Gli intervalli orari sono rappresentati dalla classe \textit{Interval} che a sua volta è definita tramite gli orari di inizio e fine, la rappresentazione
degli orari è delegata alla classe \textit{Time}. È previsto un metodo overlaps() che permette di verificare se un intervallo sia sovrapposto
ad un altro intervallo, e un metodo statico che permetta di riunire diversi intervalli nel minor numero possibile.
\subsubsection{Diagramma delle classi \texttt{impl.sqlite}}
\vspace{0.5cm}
\begin{figure}[H]
    \centering
    \includeplantuml{VER1/diagrams/class/class_diagram_db.puml}
    \caption{Versione 1: Diagramma UML delle classi, package \texttt{impl.sqlite}}
    \label{fig:class_db_v_1}
\end{figure}

Questo package implementa le interfacce fornite dal package padre, sfruttando come motore di dati un database supportato da JDBC
tramite l'utilizzo di una libreria esterna (\textit{ORMLite}), la quale permette la semplificazione della gestione dei dati nel database
e permette l'utilizzo di helper per la scrittura delle query. L'associazione dato-oggetto è stata così progettata:

Ogni rapresentazione di un oggetto a database viene identificata in una classe wrapper: \textit{UserDB}, \textit{CategoryDB}, \textit{HierarchyDB}.
Mediante queste classi sarà possibile accedere direttamente ai dati così come sono salvati nel database.

Il package offre anche un'implementazione delle Factory definite nel package padre. Lo scopo di queste classi
è istanziare le classi di implementazioni delle classi \textit{User}, \textit{Category}, \textit{Hierarchy} (figura~\ref{fig:class_data_v_1}
nella loro declinazione corretta in base allo stato del programma e del dato a database.

In particolare, la classe \textit{UserFactoryImpl} instanzierà una classe \textit{ConfiguratorUser} se a database l'utente avrà
i permessi di configuratore.

Caso più particolare è quello delle categorie:
la classe \textit{CategoryFactoryImpl} otterrà dall'origine dati la struttura delle categorie definita mediante backreference al padre, 
e ricostruirà l'albero delle categorie composto da elementi di tipo \textit{NodeCategoryImpl} o \textit{LeafCategoryImpl}.