\subsubsection{Diagramma delle classi \texttt{controller}}
\vspace{0.5cm}
\begin{figure}[H]
    \centering
    \includeplantuml{VER4/diagrams/class/class_diagram_controller.puml}
    \caption{Versione 4: Diagramma UML delle classi, package \texttt{controller}}
    \label{fig:class_controller_v_4}
\end{figure}

Alla classe \textit{Controller} sono stati aggiunti i metodi che permettono all'utente di proporre scambi, accettarli o rifiutarli con una controproposta:
\begin{itemize}
    \item offerTrade(offerToTrade: Offer, offerToAccept: Offer)
    \item acceptTrade(offer: Offer, location: String, day: Day, time: Time)
    \item replyToMessage(replyToMessage(Message replyto, String place, Day day, Time time)
\end{itemize}

La funzione \texttt{replyToMessage()} è così chiamata perchè, come si vedrà successivamente, la procedura che permette a due utenti di accordarsi sul luogo, giorno ed ora dello scambio
è definita tramite uno scambio di messaggi. Essi, tuttavia, non debbono essere confusi col termine quotidiano di messaggio per cui si intende invio e ricezione di testo libero.
All'utente infatti non sarà permesso di inviare un testo a proprio piacimento ma sarà obbligato a scegliere tra una serie di possibilità preimpostate in fase di configurazione.