\subsubsection{Diagramma delle classi \texttt{data}}
\vspace{0.5cm}
\begin{figure}[H]
    \centering
    \includeplantuml{VER4/diagrams/class/class_diagram_data.puml}
    \caption{Versione 4: Diagramma UML delle classi, package \texttt{data}}
    \label{fig:class_data_v_4}
\end{figure}

Al package \texttt{data} viene aggiunta un'interfaccia per la gestione dei messaggi \textit{MessageFactory}.

Un'attezione deve essere posta sul significato dello scambio dei messaggi, infatti questi non debbono essere interpretati come si farebbe
nel linguaggio corrente. L'utente fruitore non può in alcun modo inviare ad un altro utente un messaggio di testo libero, ma è costretto
a rispondere con un luogo, un giorno ed un orario tra quelli prestabiliti in fase di configurazione.

Un primo messaggio viene inviato dall'utente fruitore che per primo propone lo scambio di un suo articolo con un altro articolo,
l'utente fruitore di questo secondo articolo riceverà la comunicazione di aver ricevuto una proposta di scambio al suo prossimo accesso all'applicazione.

Se la proposta di scambio relativa al messaggio non è ancora scaduta, l'utente ricevitore del messaggio può decidere di accettare la proposta concludendo lo scambio,
oppure di rispondere al messaggio stesso con un ulteriore messaggio contente una controproposta.