A seguire i cambiamenti da apportare per la versione 4 dell'applicativo

\subsection{Casi d'uso}
Di seguito sono elencati i casi d'uso ritenuti necessari a sopperire alle funzionalità
dell'applicativo.
\\\\
\begin{minipage}{\textwidth}
    \subsubsection{Proposta di scambio}
    \usecase
        {Proposta di scambio}
        {
            Fruitore A\\
            \auxactor{Fruitore B}
        }
        {
            %proposta di scambio
            \begin{tabular}{l}
                1. \textit{Precondizione: i due fruitori sono distinti ed entrambi hanno}\\
                \textit{almeno un articolo nello stato \texttt{Offerta aperta}}\\
                2. Il fruitore \textit{A} seleziona quale tra i suoi articoli vuole proporre\\
                in scambio\\
                3. Il fruitore \textit{A} seleziona quale tra gli articoli del fruitore \textit{B}\\
                intende accettare in scambio\\
                \textit{Postcondizione: l'articolo del fruitore \textit{A} passa allo stato di \texttt{Of-}}\\
                \textit{\texttt{ferta accoppiata}, mentre l'articolo del fruitore \textit{B} passa allo}\\
                \textit{stato di \texttt{Offerta selezionata}. La coppia degli articoli da scam-}\\
                \textit{biare va a formare una \textbf{proposta di scambio}}\\
                Fine
            \end{tabular}\\

            %articolo appartenenti a due categorie diverse
            \auxcase{
                \begin{tabular}{l}
                    3.a \textit{Precondizione: i due articoli selezionati appartengono a cate-}\\
                    \textit{gorie diverse}\\
                    4. Viene mostrato un messaggio di errore\\
                    Fine
                \end{tabular}
            }\\

            %articolo A non in offerta aperta
            \auxcase{
                \begin{tabular}{l}
                    2.a \textit{Precondizione: l'articolo selezionato non è nello stato di \texttt{Of-}}\\
                    \textit{\texttt{ferta aperta}}\\
                    3. Viene mostrato un messaggio di errore\\
                    Torna al punto 2
                \end{tabular}
            }\\

            %articolo B non in offerta aperta
            \auxcase{
                \begin{tabular}{l}
                    3.a \textit{Precondizione: l'articolo selezionato non è nello stato di \texttt{Of-}}\\
                    \textit{\texttt{ferta aperta}}\\
                    4. Viene mostrato un messaggio di errore\\
                    Torna al punto 2
                \end{tabular}
            }

        }
        \vspace{0.5cm}
\end{minipage}
\begin{minipage}{\textwidth}
    \subsubsection{Accettazione scambio}
    \usecase
        {Accettazione scambio}
        {
            Fruitore A\\
            \auxactor{Fruitore B}
        }
        {
            %proposta di scambio
            \begin{tabular}{l}
                \textit{Precondizione: i due fruitori sono distinti, entrambi hanno un}\\
                \textit{articolo nella medesima \textbf{proposta di scambio}, il fruitore A pos-}\\
                \textit{siede l'articolo nello stato di \texttt{Offerta accoppiata}, mentre il}\\
                \textit{fruitore B possiede l'articolo nello stato di \texttt{Offerta selezionata}}\\
                1. Il fruitore \textit{B} accetta la proposta di scambio\\
                2. Il fruitore \textit{B} indica un luogo, un giorno ed un orario (tra quelli\\
                disponibili) in cui poter effettuare lo scambio\\
                \textit{Postcondizione: entrambi gli articoli passano allo stato di \texttt{Of-}}\\
                \textit{\texttt{ferta in scambio} in attesa della conferma del fruitore A}\\
                Fine
            \end{tabular}\\

            %scambio non accettato, scadenza
            \auxcase{
                \begin{tabular}{l}
                    1.a \textit{Precondizione: il fruitore B non accetta la proposta di scambio}\\
                    \textit{entro la scadenza massima}\\
                    \textit{Postcondizione: entrambi gli articoli ritornano allo stato di \texttt{Of-}}\\
                    \textit{\texttt{ferta aperta}}\\
                    Fine
                \end{tabular}
            }\\

            %luogo, giorno o ora non consentiti
            \auxcase{
                \begin{tabular}{l}
                    2.a \textit{Precondizione: il fruitore B seleziona un luogo, un giorno od}\\
                    \textit{un ora non consentiti}\\
                    3. Visualizza un messaggio di errore\\
                    Torna al punto 2
                \end{tabular}
            }

        }
        \vspace{0.5cm}
    \textbf{Nota bene:} non è contemplata la possibilità per il fruitore \textit{B}
    di rifiutare la proposta, è necessario quindi che lasci scadere il tempo
    a disposizione per poter accettare
\end{minipage}
\begin{minipage}{\textwidth}
    \subsubsection{Accettazione appuntamento}
    Si considera fruitore \textit{B} colui che ha proposto un luogo, un giorno ed un orario per l'appuntamento; mentre è considerato fruitore \textit{A} colui che deve accettare oppure replicare con una controproposta per l'appuntamento.
    \usecase
        {Accettazione appuntamento}
        {
            Fruitore A\\
            \auxactor{Fruitore B}
        }
        {
            %proposta di scambio
            \begin{tabular}{l}
                \textit{Precondizione: i due fruitori sono distinti, entrambi hanno un}\\
                \textit{articolo nella medesima \textbf{proposta di scambio}; entrambi gli ar-}\\
                \textit{ticoli sono nello stato di \texttt{Offerta in scambio}}\\
                1. Il fruitore \textit{A} accetta l'appuntamento\\
                \textit{Postcondizione: entrambi gli articoli passano allo stato di \texttt{Of-}}\\
                \textit{\texttt{ferta chiusa}}\\
                Fine
            \end{tabular}\\

            %appuntamento rifiutato
            \auxcase{
                \begin{tabular}{l}
                    1.a Il fruitore \textit{A} rifiuta la proposta di scambio\\
                    2.a Il fruitore \textit{A} indica un luogo, un giorno ed un orario in cui\\
                    poter effettuare lo scambio al fruitore \textit{B}\\
                    \textit{Postcondizione: la proposta di scambio resta in attesa della}\\
                    \textit{conferma del fruitore B}\\
                    Fine
                \end{tabular}
            }\\

            %luogo, giorno o ora non consentiti
            \auxcase{
                \begin{tabular}{l}
                    2.a.a \textit{Precondizione: il fruitore A seleziona un luogo, un giorno od}\\
                    \textit{un'ora non consentiti}\\
                    3.a.a Visualizza un messaggio di errore\\
                    Torna al punto 2.a
                \end{tabular}
            }\\

            %appuntamento scaduto
            \auxcase{
                \begin{tabular}{l}
                    1.b Il fruitore \textit{A} non accetta entro il numero massimo di giorni\\
                    \textit{Postcondizione: entrambi gli articoli ritornano allo stato di \texttt{Of-}}\\
                    \textit{\texttt{ferta aperta}, concludendo la \textbf{proposta di scambio}}\\
                    Fine
                \end{tabular}
            }\\
        }
        \vspace{0.5cm}
\end{minipage}
\begin{minipage}{\textwidth}
    \subsubsection{Visualizzazione offerta in scambio}
    \usecase
        {Visualizzazione offerta in scambio}
        {
            Fruitore
        }
        {
            %visualizzazione scambio
            \begin{tabular}{l}
                \textit{<<include>> Visualizzazione offerte dell'utente}\\
                1. Il fruitore seleziona un'offerta in scambio\\
                2. Viene visualizzata l'ultima risposta fornita dall'autore dell'of-\\
                ferta ad essa collegata\\
                Fine
            \end{tabular}\\

            %offerta non in scambio
            \auxcase{
                \begin{tabular}{l}
                    1.a \textit{Precondizione: il fruitore selezionata un'offerta non in scambio}\\
                    Fine
                \end{tabular}
            }
        }
        \vspace{0.5cm}
\end{minipage}
\begin{minipage}{\textwidth}
    \subsubsection{Visualizzazione offerte chiuse e in scambio per categoria}
    \usecase
        {Visualizzazione offerte chiuse e in scambio per categoria}
        {
            Configuratore %\\
        }
        {
            %Pubblicazione
            \begin{tabular}{l}
                1. \textit{<<include>> Accesso configuratore}\\
                2. Il configuratore seleziona una categoria\\
                \textit{Postcondizione: viene fornita una lista di articoli disponibili ap-}\\
                \textit{partenenti alla categoria selezionata nello stato di \texttt{Offerta}}\\
                \textit{\texttt{in scambio} o di \texttt{Offerta chiusa}}\\
                Fine
            \end{tabular}\\

            %categoria non esistente
            \auxcase{
                \begin{tabular}{l}
                    2.a \textit{Precondizione: il configuratore sceglie una categoria non}\\
                    \textit{esistente}\\
                    3. Viene mostrato un messaggio di errore\\
                    Fine
                \end{tabular}
            }\\

            %categoria non foglia
            \auxcase{
                \begin{tabular}{l}
                    2.a \textit{Precondizione: il configuratore sceglie una categoria non foglia}\\
                    3. Viene mostrato un messaggio di errore\\
                    Fine
                \end{tabular}
            }

        }
        \vspace{0.5cm}
\end{minipage}

\pagebreak
\subsection{Casi d'uso (UML)}
\vspace{0.5cm}
\begin{figure}[H]
    \centering
    \includeplantuml{VER3/diagrams/use_case/version_3.puml}
    \caption{Versione 3: Diagramma UML dei casi d'uso}
    \label{fig:use_case_uml_v3}
\end{figure}


\pagebreak
\subsection{Diagrammi delle classi}

\textit{Si noti che, per motivi di spazio, il diagramma generale delle classi è stato suddiviso in più sottodiagrammi, uno per ogni package.}
\textit{Le referenze tra i vari diagrammi sono rappresentate con una classe, senza dettagli, al di fuori del package stesso.}
\textit{Inoltre, le classi referenziate ma già specificate nella versione precedente saranno riportate senza attributi e metodi.}

\subsubsection{Diagramma delle classi \texttt{controller}}
\vspace{0.5cm}
\begin{figure}[H]
    \centering
    \includeplantuml{VER1/diagrams/class/class_diagram_controller.puml}
    \caption{Versione 1: Diagramma UML delle classi, package \texttt{controller}}
    \label{fig:class_controller_v_1}
\end{figure}
\subsubsection{Diagramma delle classi \texttt{data}}
\vspace{0.5cm}
\begin{figure}[H]
    \centering
    \includeplantuml{VER2/diagrams/class/class_diagram_data.puml}
    \caption{Versione 2: Diagramma UML delle classi, package \texttt{data}}
    \label{fig:class_data_v_2}
\end{figure}

Al package \texttt{data} viene aggiunta un'interfaccia per la gestione dei parametri di configurazione \textit{SettingsFactory}.

Le implementazioni di questa si occupanno di istanziare, a partire dai dati in datase, la classe \textit{Settings}.
Questa contiene la piazza di scambio, il numero di giorni di scadenza, una lista di luoghi (\textit{String}) dove è possibile fare gli scambi,
una lista di giorni (\textit{Day} è un'enum) ed una lista di intervalli orari.

Gli intervalli orari sono rappresentati dalla classe \textit{Interval} che a sua volta è definita tramite gli orari di inizio e fine, la rappresentazione
degli orari è delegata alla classe \textit{Time}. È previsto un metodo overlaps() che permette di verificare se un intervallo sia sovrapposto
ad un altro intervallo, e un metodo statico che permetta di riunire diversi intervalli nel minor numero possibile.
\subsubsection{Diagramma delle classi \texttt{impl.sqlite}}
\vspace{0.5cm}
\begin{figure}[H]
    \centering
    \includeplantuml{VER1/diagrams/class/class_diagram_db.puml}
    \caption{Versione 1: Diagramma UML delle classi, package \texttt{impl.sqlite}}
    \label{fig:class_db_v_1}
\end{figure}

Questo package implementa le interfacce fornite dal package padre, sfruttando come motore di dati un database supportato da JDBC
tramite l'utilizzo di una libreria esterna (\textit{ORMLite}), la quale permette la semplificazione della gestione dei dati nel database
e permette l'utilizzo di helper per la scrittura delle query. L'associazione dato-oggetto è stata così progettata:

Ogni rapresentazione di un oggetto a database viene identificata in una classe wrapper: \textit{UserDB}, \textit{CategoryDB}, \textit{HierarchyDB}.
Mediante queste classi sarà possibile accedere direttamente ai dati così come sono salvati nel database.

Il package offre anche un'implementazione delle Factory definite nel package padre. Lo scopo di queste classi
è istanziare le classi di implementazioni delle classi \textit{User}, \textit{Category}, \textit{Hierarchy} (figura~\ref{fig:class_data_v_1}
nella loro declinazione corretta in base allo stato del programma e del dato a database.

In particolare, la classe \textit{UserFactoryImpl} instanzierà una classe \textit{ConfiguratorUser} se a database l'utente avrà
i permessi di configuratore.

Caso più particolare è quello delle categorie:
la classe \textit{CategoryFactoryImpl} otterrà dall'origine dati la struttura delle categorie definita mediante backreference al padre, 
e ricostruirà l'albero delle categorie composto da elementi di tipo \textit{NodeCategoryImpl} o \textit{LeafCategoryImpl}.