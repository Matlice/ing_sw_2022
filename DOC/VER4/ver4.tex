A seguire i cambiamenti da apportare per la versione 4 dell'applicativo

\subsection{Casi d'uso}
Di seguito sono elencati i casi d'uso ritenuti necessari a sopperire alle funzionalità
dell'applicativo.
\\\\
\begin{minipage}{\textwidth}
    \subsubsection{Proposta di scambio}
    \usecase
        {Proposta di scambio}
        {
            Fruitore A\\
            \auxactor{Fruitore B}
        }
        {
            %proposta di scambio
            \begin{tabular}{l}
                1. \textit{Precondizione: i due fruitori sono distinti ed entrambi hanno}\\
                \textit{almeno un articolo nello stato \texttt{Offerta aperta}}\\
                2. Il fruitore \textit{A} seleziona quale tra i suoi articoli vuole proporre\\
                in scambio\\
                3. Il fruitore \textit{A} seleziona quale tra gli articoli del fruitore \textit{B}\\
                intende accettare in scambio\\
                \textit{Postcondizione: l'articolo del fruitore \textit{A} passa allo stato di \texttt{Of-}}\\
                \textit{\texttt{ferta accoppiata}, mentre l'articolo del fruitore \textit{B} passa allo}\\
                \textit{stato di \texttt{Offerta selezionata}. La coppia degli articoli da scam-}\\
                \textit{biare va a formare una \textbf{proposta di scambio}}\\
                Fine
            \end{tabular}\\

            %articolo appartenenti a due categorie diverse
            \auxcase{
                \begin{tabular}{l}
                    3.a \textit{Precondizione: i due articoli selezionati appartengono a cate-}\\
                    \textit{gorie diverse}\\
                    4. Viene mostrato un messaggio di errore\\
                    Fine
                \end{tabular}
            }
        }
        \vspace{0.5cm}
\end{minipage}
\begin{minipage}{\textwidth}
    \subsubsection{Accettazione scambio}
    \usecase
        {Accettazione scambio}
        {
            Fruitore A\\
            \auxactor{Fruitore B}
        }
        {
            %proposta di scambio
            \begin{tabular}{l}
                \textit{Precondizione: i due fruitori sono distinti, entrambi hanno un}\\
                \textit{articolo nella medesima \textbf{proposta di scambio}, il fruitore A pos-}\\
                \textit{siede l'articolo nello stato di \texttt{Offerta accoppiata}, mentre il}\\
                \textit{fruitore B possiede l'articolo nello stato di \texttt{Offerta selezionata}}\\
                1. Il fruitore \textit{B} accetta la proposta di scambio\\
                2. Il fruitore \textit{B} indica un luogo, un giorno ed un orario (tra quelli\\
                disponibili) in cui poter effettuare lo scambio\\
                \textit{Postcondizione: entrambi gli articoli passano allo stato di \texttt{Of-}}\\
                \textit{\texttt{ferta in scambio} in attesa della conferma del fruitore A}\\
                Fine
            \end{tabular}\\

            %scambio non accettato, scadenza
            \auxcase{
                \begin{tabular}{l}
                    1.a \textit{Precondizione: il fruitore B non accetta la proposta di scambio}\\
                    \textit{entro la scadenza massima}\\
                    \textit{Postcondizione: entrambi gli articoli ritornano allo stato di \texttt{Of-}}\\
                    \textit{\texttt{ferta aperta}}\\
                    Fine
                \end{tabular}
            }\\

            %luogo, giorno o ora non consentiti
            \auxcase{
                \begin{tabular}{l}
                    2.a \textit{Precondizione: il fruitore B seleziona un luogo, un giorno od}\\
                    \textit{un ora non consentiti}\\
                    3. Visualizza un messaggio di errore\\
                    Torna al punto 2
                \end{tabular}
            }

        }
        \vspace{0.5cm}
    \textbf{Nota bene:} non è contemplata la possibilità per il fruitore \textit{B}
    di rifiutare la proposta, è necessario quindi che lasci scadere il tempo
    a disposizione per poter accettare
\end{minipage}
\begin{minipage}{\textwidth}
    \subsubsection{Accettazione appuntamento}
    Si considera fruitore \textit{B} colui che ha proposto un luogo, un giorno ed un orario per l'appuntamento; mentre è considerato fruitore \textit{A} colui che deve accettare oppure replicare con una controproposta per l'appuntamento.
    \usecase
        {Accettazione appuntamento}
        {
            Fruitore A\\
            \auxactor{Fruitore B}
        }
        {
            %proposta di scambio
            \begin{tabular}{l}
                \textit{Precondizione: i due fruitori sono distinti, entrambi hanno un}\\
                \textit{articolo nella medesima \textbf{proposta di scambio}; entrambi gli ar-}\\
                \textit{ticoli sono nello stato di \texttt{Offerta in scambio}}\\
                1. Il fruitore \textit{A} accetta l'appuntamento\\
                \textit{Postcondizione: entrambi gli articoli passano allo stato di \texttt{Of-}}\\
                \textit{\texttt{ferta chiusa}}\\
                Fine
            \end{tabular}\\

            %appuntamento rifiutato
            \auxcase{
                \begin{tabular}{l}
                    1.a Il fruitore \textit{A} rifiuta la proposta di scambio\\
                    2.a Il fruitore \textit{A} indica un luogo, un giorno ed un orario in cui\\
                    poter effettuare lo scambio al fruitore \textit{B}\\
                    \textit{Postcondizione: la proposta di scambio resta in attesa della}\\
                    \textit{conferma del fruitore B}\\
                    Fine
                \end{tabular}
            }\\

            %luogo, giorno o ora non consentiti
            \auxcase{
                \begin{tabular}{l}
                    2.a.a \textit{Precondizione: il fruitore A seleziona un luogo, un giorno od}\\
                    \textit{un'ora non consentiti}\\
                    3.a.a Visualizza un messaggio di errore\\
                    Torna al punto 2.a
                \end{tabular}
            }\\

            %appuntamento scaduto
            \auxcase{
                \begin{tabular}{l}
                    1.b Il fruitore \textit{A} non accetta entro il numero massimo di giorni\\
                    \textit{Postcondizione: entrambi gli articoli ritornano allo stato di \texttt{Of-}}\\
                    \textit{\texttt{ferta aperta}, concludendo la \textbf{proposta di scambio}}\\
                    Fine
                \end{tabular}
            }\\
        }
        \vspace{0.5cm}
\end{minipage}
\begin{minipage}{\textwidth}
    \subsubsection{Visualizzazione offerta in scambio}
    \usecase
        {Visualizzazione offerta in scambio}
        {
            Fruitore
        }
        {
            %visualizzazione scambio
            \begin{tabular}{l}
                \textit{<<include>> Visualizzazione offerte dell'utente}\\
                1. Il fruitore seleziona un'offerta in scambio\\
                2. Viene visualizzata l'ultima risposta fornita dall'autore dell'of-\\
                ferta ad essa collegata\\
                Fine
            \end{tabular}\\

            %offerta non in scambio
            \auxcase{
                \begin{tabular}{l}
                    1.a \textit{Precondizione: il fruitore selezionata un'offerta non in scambio}\\
                    Fine
                \end{tabular}
            }
        }
        \vspace{0.5cm}
\end{minipage}
\begin{minipage}{\textwidth}
    \subsubsection{Visualizzazione offerte chiuse e in scambio per categoria}
    \usecase
        {Visualizzazione offerte chiuse e in scambio per categoria}
        {
            Configuratore %\\
        }
        {
            %Pubblicazione
            \begin{tabular}{l}
                1. \textit{<<include>> Accesso configuratore}\\
                2. Il configuratore seleziona una categoria\\
                \textit{Postcondizione: viene fornita una lista di articoli disponibili ap-}\\
                \textit{partenenti alla categoria selezionata nello stato di \texttt{Offerta}}\\
                \textit{\texttt{in scambio} o di \texttt{Offerta chiusa}}\\
                Fine
            \end{tabular}\\

            %categoria non esistente
            \auxcase{
                \begin{tabular}{l}
                    2.a \textit{Precondizione: il configuratore sceglie una categoria non}\\
                    \textit{esistente}\\
                    3. Viene mostrato un messaggio di errore\\
                    Fine
                \end{tabular}
            }\\

            %categoria non foglia
            \auxcase{
                \begin{tabular}{l}
                    2.a \textit{Precondizione: il configuratore sceglie una categoria non foglia}\\
                    3. Viene mostrato un messaggio di errore\\
                    Fine
                \end{tabular}
            }

        }
        \vspace{0.5cm}
\end{minipage}

\pagebreak
\subsection{Casi d'uso (UML)}
\vspace{0.5cm}
\begin{figure}[H]
    \centering
    \includeplantuml{VER1/diagrams/use_case/version_1.puml}
    \caption{Versione 1: Diagramma UML dei casi d'uso}
    \label{fig:use_case_uml_v1}
\end{figure}

\pagebreak
\subsection{Diagrammi delle classi}

\textit{Si noti che, per motivi di spazio, il diagramma generale delle classi è stato suddiviso in più sottodiagrammi, uno per ogni package.}
\textit{Le referenze tra i vari diagrammi sono rappresentate con una classe, senza dettagli, al di fuori del package stesso.}
\textit{Inoltre, le classi referenziate ma già specificate nella versione precedente saranno riportate senza attributi e metodi.}

\subsubsection{Diagramma delle classi \texttt{controller}}
\vspace{0.5cm}
\begin{figure}[H]
    \centering
    \includeplantuml{VER3/diagrams/class/class_diagram_controller.puml}
    \caption{Versione 3: Diagramma UML delle classi, package \texttt{controller}}
    \label{fig:class_controller_v_3}
\end{figure}

Alla classe \textit{Controller} sono stati aggiunti i metodi che permettono all'utente di aggiungere, ritirare e visualizzare le offerte (filtrando per utente o per categoria):
\begin{itemize}
    \item createOffer(u: User, name: String, cat: LeafCategory, fields: Map<String, String>)
    \item retractOffer(offer: Offer)
    \item showOffersByUser(user: User)
    \item showOpenOffersByCategory(category: LeafCategory)
\end{itemize}
\subsubsection{Diagramma delle classi \texttt{data}}
\vspace{0.5cm}
\begin{figure}[H]
    \centering
    \includeplantuml{VER3/diagrams/class/class_diagram_data.puml}
    \caption{Versione 3: Diagramma UML delle classi, package \texttt{data}}
    \label{fig:class_data_v_3}
\end{figure}

Al package \texttt{data} viene aggiunta un'interfaccia per la gestione delle offerte \textit{OfferFactory}.

L'implementazione di questa si occupa di istanziare, a partire dai dati in database, la classe \textit{Offer}.
Questa classe contiene il nome, la categoria foglia a cui è associata e l'utente proprietario; inoltre, possiede una mappa che associa i nomi dei campi al valore compilato in fase di inserimento dall'utente.
\subsubsection{Diagramma delle classi \texttt{impl.sqlite}}
\vspace{0.5cm}
\begin{figure}[H]
    \centering
    \includeplantuml{VER3/diagrams/class/class_diagram_db.puml}
    \caption{Versione 3: Diagramma UML delle classi, package \texttt{impl.sqlite}}
    \label{fig:class_db_v_3}
\end{figure}

L'implementazione delle offerte a livello di database è effettuato tramite due tabelle, rappresentate da due classi: \textit{OfferDB} e \textit{OfferFieldDB}.
La prima contiene nelle colonne il nome dell'articolo, lo stato, una referenza all'utente proprietario ed una alla categoria di appartenenza; ogni riga della seconda tabella invece,
rappresenta un singolo campo compilato di un'offerta, per questo ha un riferimento al campo a cui si riferisce (\textit{CategoryFieldDB}) e il valore che questo assume per la specifica offerta
(referenziata da \texttt{offer\_ref}).

La lettura e scrittura da database è mediata da \textit{OfferFactoryImpl}, un'implementazione dell'interfaccia
\textit{OfferFactory}. La classe istanziata è \textit{OfferImpl} che estende la classe astratta \textit{Offer}.