\begin{minipage}{\textwidth}
    \subsubsection{Accettazione scambio}
    \usecase
        {Accettazione scambio}
        {
            Fruitore A\\
            \auxactor{Fruitore B}
        }
        {
            %proposta di scambio
            \begin{tabular}{l}
                \textit{Precondizione: i due fruitori sono distinti, entrambi hanno un}\\
                \textit{articolo nella medesima \textbf{proposta di scambio}, il fruitore A pos-}\\
                \textit{siede l'articolo nello stato di \texttt{Offerta accoppiata}, mentre il}\\
                \textit{fruitore B possiede l'articolo nello stato di \texttt{Offerta selezionata}}\\
                1. Il fruitore \textit{B} accetta la proposta di scambio, perciò indica un\\
                luogo, un giorno ed un orario in cui poter effettuare lo scambio\\
                \textit{Postcondizione: entrambi gli articoli passano allo stato di \texttt{Of-}}\\
                \textit{\texttt{ferta in scambio} in attesa della conferma del fruitore A}\\
                Fine
            \end{tabular}\\

            %scambio non accettato, scadenza
            \auxcase{
                \begin{tabular}{l}
                    1.a \textit{Precondizione: il fruitore B non accetta la proposta di scambio}\\
                    \textit{entro la scadenza massima}\\
                    \textit{Postcondizione: entrambi gli articoli ritornano allo stato di \texttt{Of-}}\\
                    \textit{\texttt{ferta aperta}}\\
                    Fine
                \end{tabular}
            }
        }
        \vspace{0.5cm}
    \textbf{Nota bene:} non è contemplata la possibilità per il fruitore \textit{B} di rifiutare la proposta, è necessario quindi che lasci scadere il tempo a disposizione per poter accettare
\end{minipage}