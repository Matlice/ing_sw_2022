\begin{minipage}{\textwidth}
    \subsubsection{Proposta di scambio}
    \usecase
        {Proposta di scambio}
        {
            Fruitore A\\
            \auxactor{Fruitore B}
        }
        {
            %proposta di scambio
            \begin{tabular}{l}
                1. \textit{Precondizione: i due fruitori sono distinti ed entrambi hanno}\\
                \textit{almeno un articolo nello stato \texttt{Offerta aperta}}\\
                2. Il fruitore \textit{A} seleziona quale tra i suoi articoli vuole proporre\\
                in scambio\\
                3. Il fruitore \textit{A} seleziona quale tra gli articoli del fruitore \textit{B}\\
                intende accettare in scambio\\
                \textit{Postcondizione: l'articolo del fruitore \textit{A} passa allo stato di \texttt{Of-}}\\
                \textit{\texttt{ferta accoppiata}, mentre l'articolo del fruitore \textit{B} passa allo}\\
                \textit{stato di \texttt{Offerta selezionata}. La coppia degli articoli da scam-}\\
                \textit{biare va a formare una \textbf{proposta di scambio}}\\
                Fine
            \end{tabular}\\

            %articolo appartenenti a due categorie diverse
            \auxcase{
                \begin{tabular}{l}
                    3.a \textit{Precondizione: i due articoli selezionati appartengono a cate-}\\
                    \textit{gorie diverse}\\
                    4. Viene mostrato un messaggio di errore\\
                    Fine
                \end{tabular}
            }
        }
        \vspace{0.5cm}
\end{minipage}