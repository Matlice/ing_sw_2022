\begin{minipage}{\textwidth}
    \subsubsection{Accettazione appuntamento}
    Si considera fruitore \textit{B} colui che ha proposto un luogo, un giorno ed un orario per l'appuntamento; mentre è considerato fruitore \textit{A} colui che deve accettare oppure replicare con una controproposta per l'appuntamento.
    \usecase
        {Accettazione appuntamento}
        {
            Fruitore A\\
            \auxactor{Fruitore B}
        }
        {
            %proposta di scambio
            \begin{tabular}{l}
                \textit{Precondizione: i due fruitori sono distinti, entrambi hanno un}\\
                \textit{articolo nella medesima \textbf{proposta di scambio}; entrambi gli ar-}\\
                \textit{ticoli sono nello stato di \texttt{Offerta in scambio}}\\
                1. Il fruitore \textit{A} accetta l'appuntamento\\
                \textit{Postcondizione: entrambi gli articoli passano allo stato di \texttt{Of-}}\\
                \textit{\texttt{ferta chiusa}}\\
                Fine
            \end{tabular}\\

            %appuntamento rifiutato
            \auxcase{
                \begin{tabular}{l}
                    1.a Il fruitore \textit{A} rifiuta la proposta di scambio\\
                    2.a Il fruitore \textit{A} indica un luogo, un giorno ed un orario in cui\\
                    poter effettuare lo scambio al fruitore \textit{B}\\
                    \textit{Postcondizione: la proposta di scambio resta in attesa della}\\
                    \textit{conferma del fruitore B}\\
                    Fine
                \end{tabular}
            }\\

            %luogo, giorno o ora non consentiti
            \auxcase{
                \begin{tabular}{l}
                    2.a.a \textit{Precondizione: il fruitore A seleziona un luogo, un giorno od}\\
                    \textit{un'ora non consentiti}\\
                    3.a.a Visualizza un messaggio di errore\\
                    Torna al punto 2.a
                \end{tabular}
            }\\

            %appuntamento scaduto
            \auxcase{
                \begin{tabular}{l}
                    1.b Il fruitore \textit{A} non accetta entro il numero massimo di giorni\\
                    \textit{Postcondizione: entrambi gli articoli ritornano allo stato di \texttt{Of-}}\\
                    \textit{\texttt{ferta aperta}, concludendo la \textbf{proposta di scambio}}\\
                    Fine
                \end{tabular}
            }\\
        }
        \vspace{0.5cm}
\end{minipage}