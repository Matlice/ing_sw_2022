\begin{minipage}{\textwidth}
    \subsubsection{Configurazione parametri di sistema}
    \usecase
        {Configurazione parametri di sistema}
        {
            Configuratore
        }
        {
            \begin{tabular}{l}
                1. L'utente sceglie la voce di menu ``modifica parametri \\applicativo''\\
                2. Viene mostrata la lista dei parametri configurabili\\
                3. L'utente sceglie il parametro da modificare.\\
                4. L'utente inserisce il nuovo valore.\\
                \textit{Postcondizione: il valore viene modificato runtime}\\
                Fine
            \end{tabular}\\

            \auxcase{
                \begin{tabular}{l}
                    \textit{Precondizione: l'utente sceglie un parametro non esistente } \\
                    3.a Viene mostrato un messaggio di errore\\
                    Torna al punto 2
                \end{tabular}
            }\\

            \auxcase{
                \begin{tabular}{l}
                    \textit{Precondizione: l'utente inserisce un valore non valido } \\
                    4.a Viene mostrato un messaggio di errore\\
                    Torna al punto 2
                \end{tabular}
            }
        }
        \vspace{0.5cm}
\end{minipage}
