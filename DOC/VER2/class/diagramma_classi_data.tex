\subsubsection{Diagramma delle classi \texttt{data}}
\vspace{0.5cm}
\begin{figure}[H]
    \centering
    \includeplantuml{VER2/diagrams/class/class_diagram_data.puml}
    \caption{Versione 2: Diagramma UML delle classi, package \texttt{data}}
    \label{fig:class_data_v_2}
\end{figure}

Al package \texttt{data} viene aggiunta un'interfaccia per la gestione dei parametri di configurazione \textit{SettingsFactory}.

Le implementazioni di questa si occupano di istanziare, a partire dai dati in database, la classe \textit{Settings}.
Questa contiene la piazza di scambio, il numero di giorni di scadenza, una lista di luoghi (\textit{String}) dove è possibile fare gli scambi,
una lista di giorni (\textit{Day} è un'enum) ed una lista di intervalli orari.

Gli intervalli orari sono rappresentati dalla classe \textit{Interval} che a sua volta è definita tramite gli orari di inizio e fine, la rappresentazione
degli orari è delegata alla classe \textit{Time}. È previsto un metodo \texttt{overlaps()} che permette di verificare se un intervallo sia sovrapposto
ad un altro intervallo, e un metodo statico che permetta di riunire diversi intervalli nel minor numero possibile.