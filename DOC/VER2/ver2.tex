A seguire i cambiamenti da apportare per la versione 2 dell'applicativo

\subsection{Casi d'uso}
Di seguito sono elencati i casi d'uso ritenuti necessari a sopperire alle funzionalità
dell'applicativo.
\\\\
\begin{minipage}{\textwidth}

\subsubsection{Accesso configuratore}
\usecase
    {Accesso configuratore}
    {
        Configuratore %\\
        %\auxactor{teto}
    }
    {
        %Accesso normale
        \begin{tabular}{l}
            1. <<include>> "Accesso utente" (verifica credenziali e permessi)\\
            2. L'utente ha già effettuato l'accesso\\
            3. Mostra il menu di configurazione\\
        \end{tabular} \\

        %primo accesso
        \auxcase{
            \begin{tabular}{l}
                2.a L'utente è al primo accesso\\
                2.b L'utente viene forzato a cambiare la password\\
                2.c ritorna al punto 3
            \end{tabular}
        }
    }
    \vspace{0.5cm}


\end{minipage}
\begin{minipage}{\textwidth}
    \subsubsection{Configurazione parametri di sistema}
    \usecase
        {Configurazione parametri di sistema}
        {
            Configuratore
        }
        {
            \begin{tabular}{l}
                1. L'utente sceglie la voce di menu \\ ``modifica parametri applicativo''\\
                2. Viene mostrata la lista dei parametri configurabili\\
                3. L'utente sceglie il parametro da modificare.\\
                4. L'utente inserisce il nuovo valore.\\
                \textit{Postcondizione: Il valore viene modificato runtime}\\
                Fine
            \end{tabular}\\

            \auxcase{
                \begin{tabular}{l}
                    \textit{Precondizione: l'utente sceglie un campo non esistente } \\
                    4.a Mostrando un messaggio di errore, ritorna al punto 2\\
                    Fine.
                \end{tabular}
            }
        }
        \vspace{0.5cm}
\end{minipage}
\begin{minipage}{\textwidth}
    \subsubsection{Registrazione Fruitore}
    \usecase
        {Registrazione Fruitore}
        {
            Fruitore %\\
            %\auxactor{teto}
        }
        {
            %Registrazione
            \begin{tabular}{l}
                \textit{Precondizione: non è ancora stato effettuato il login}\\
                1. L'utente, al momento del login, seleziona la voce ``Registrati''\\
                2. Inserisce username e password (due volte mediante il campo \\ ``conferma password'')\\
                3. L'utente conferma.\\
                \textit{Postcondizione: L'utente viene salvato nel database}\\
                \textit{Postcondizione: viene effettuato il login a nome dell'utente}\\
                Fine
            \end{tabular} \\

            %username duplicato
            \auxcase{
                \begin{tabular}{l}
                    \textit{Precondizione: l'utente sceglie un username già esistente} \\
                    3.a Viene mostrato un messaggio di errore\\
                    Ritorna al menu principale.\\
                    Fine
                \end{tabular}
            }\\

            %password non matcha
            \auxcase{
                \begin{tabular}{l}
                    \textit{Precondizione: le password insarite non corrispondono} \\
                    3.a Viene mostrato un messaggio di errore\\
                    Torna al punto 1
                \end{tabular}
            }\\

            %la password non è sicura
            \auxcase{
                \begin{tabular}{l}
                    \textit{Precondizione: l'utente sceglie una password } \\
                    \textit{che non rispetta i requisiti di sicurezza}\\
                    3.a Viene mostrato un messaggio di errore\\
                    Torna al punto 1
                \end{tabular}
            }\\

            \auxcase{
                \begin{tabular}{l}
                    \textit{Precondizione: l'utente non conferma la registrazione} \\
                    2.a Ritorna al menu principale.\\
                    Fine
                \end{tabular}
            }
        }
        \vspace{0.5cm}
\end{minipage}
\begin{minipage}{\textwidth}
    \subsubsection{Visualizzazione informazioni}
    \usecase
        {Visualizzazione informazioni}
        {
            Fruitore
        }
        {
            %Visualizzazione
            \begin{tabular}{l}
                1. Il fruitore richiede la visualizzazione delle informazioni\\
                2. Il sistema visualizza il nome e la descrizione delle categorie\\
                radice di tutte le gerarchie, la piazza, i luoghi, i giorni e gli\\
                intervalli orari in cui sono possibili gli scambi\\
                Fine
            \end{tabular}

        }
        \vspace{0.5cm}
\end{minipage}

\pagebreak
\subsection{Casi d'uso (UML)}
\vspace{0.5cm}
\begin{figure}[H]
    \centering
    \includeplantuml{VER3/diagrams/use_case/version_3.puml}
    \caption{Versione 3: Diagramma UML dei casi d'uso}
    \label{fig:use_case_uml_v3}
\end{figure}


\pagebreak
\subsection{Diagrammi delle classi}

\textit{Si noti che, per motivi di spazio, il diagramma generale delle classi è stato suddiviso in più sottodiagrammi, uno per ogni package.}
\textit{Le referenze tra i vari diagrammi sono rappresentate con una classe, senza dettagli, al di fuori del package stesso.}
\textit{Inoltre, le classi già specificate nella versione precedente saranno riportate senza attributi e metodi}

\subsubsection{Diagramma delle classi \texttt{controller}}
\vspace{0.5cm}
\begin{figure}[H]
    \centering
    \includeplantuml{VER1/diagrams/class/class_diagram_controller.puml}
    \caption{Versione 1: Diagramma UML delle classi, package \texttt{controller}}
    \label{fig:class_controller_v_1}
\end{figure}
\subsubsection{Diagramma delle classi \texttt{data}}
\vspace{0.5cm}
\begin{figure}[H]
    \centering
    \includeplantuml{VER2/diagrams/class/class_diagram_data.puml}
    \caption{Versione 2: Diagramma UML delle classi, package \texttt{data}}
    \label{fig:class_data_v_2}
\end{figure}

Al package \texttt{data} viene aggiunta un'interfaccia per la gestione dei parametri di configurazione \textit{SettingsFactory}.

Le implementazioni di questa si occupanno di istanziare, a partire dai dati in datase, la classe \textit{Settings}.
Questa contiene la piazza di scambio, il numero di giorni di scadenza, una lista di luoghi (\textit{String}) dove è possibile fare gli scambi,
una lista di giorni (\textit{Day} è un'enum) ed una lista di intervalli orari.

Gli intervalli orari sono rappresentati dalla classe \textit{Interval} che a sua volta è definita tramite gli orari di inizio e fine, la rappresentazione
degli orari è delegata alla classe \textit{Time}. È previsto un metodo overlaps() che permette di verificare se un intervallo sia sovrapposto
ad un altro intervallo, e un metodo statico che permetta di riunire diversi intervalli nel minor numero possibile.
\subsubsection{Diagramma delle classi \texttt{impl.sqlite}}
\vspace{0.5cm}
\begin{figure}[H]
    \centering
    \includeplantuml{VER1/diagrams/class/class_diagram_db.puml}
    \caption{Versione 1: Diagramma UML delle classi, package \texttt{impl.sqlite}}
    \label{fig:class_db_v_1}
\end{figure}

Questo package implementa le interfacce fornite dal package padre, sfruttando come motore di dati un database supportato da JDBC
tramite l'utilizzo di una libreria esterna (\textit{ORMLite}), la quale permette la semplificazione della gestione dei dati nel database
e permette l'utilizzo di helper per la scrittura delle query. L'associazione dato-oggetto è stata così progettata:

Ogni rapresentazione di un oggetto a database viene identificata in una classe wrapper: \textit{UserDB}, \textit{CategoryDB}, \textit{HierarchyDB}.
Mediante queste classi sarà possibile accedere direttamente ai dati così come sono salvati nel database.

Il package offre anche un'implementazione delle Factory definite nel package padre. Lo scopo di queste classi
è istanziare le classi di implementazioni delle classi \textit{User}, \textit{Category}, \textit{Hierarchy} (figura~\ref{fig:class_data_v_1}
nella loro declinazione corretta in base allo stato del programma e del dato a database.

In particolare, la classe \textit{UserFactoryImpl} instanzierà una classe \textit{ConfiguratorUser} se a database l'utente avrà
i permessi di configuratore.

Caso più particolare è quello delle categorie:
la classe \textit{CategoryFactoryImpl} otterrà dall'origine dati la struttura delle categorie definita mediante backreference al padre, 
e ricostruirà l'albero delle categorie composto da elementi di tipo \textit{NodeCategoryImpl} o \textit{LeafCategoryImpl}.