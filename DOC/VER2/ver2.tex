A seguire i cambiamenti da apportare per la versione 2 dell'applicativo

\subsection{Casi d'uso}
Di seguito sono elencati i casi d'uso ritenuti necessari a sopperire alle funzionalità
dell'applicativo.
\\\\
\begin{minipage}{\textwidth}

\subsubsection{Accesso configuratore}
\usecase
    {Accesso configuratore}
    {
        Configuratore %\\
        %\auxactor{teto}
    }
    {
        %Accesso normale
        \begin{tabular}{l}
            1. <<include>> "Accesso utente" (verifica credenziali e permessi)\\
            2. L'utente ha già effettuato l'accesso\\
            3. Mostra il menu di configurazione\\
        \end{tabular} \\

        %primo accesso
        \auxcase{
            \begin{tabular}{l}
                2.a L'utente è al primo accesso\\
                2.b L'utente viene forzato a cambiare la password\\
                2.c ritorna al punto 3
            \end{tabular}
        }
    }
    \vspace{0.5cm}


\end{minipage}
\begin{minipage}{\textwidth}
    \subsubsection{Configurazione parametri di sistema}
    \usecase
        {Configurazione parametri di sistema}
        {
            Configuratore
        }
        {
            \begin{tabular}{l}
                1. L'utente sceglie la voce di menu ``modifica parametri \\applicativo''\\
                2. Viene mostrata la lista dei parametri configurabili\\
                3. L'utente sceglie il parametro da modificare.\\
                4. L'utente inserisce il nuovo valore.\\
                \textit{Postcondizione: il valore viene modificato runtime}\\
                Fine
            \end{tabular}\\

            \auxcase{
                \begin{tabular}{l}
                    \textit{Precondizione: l'utente sceglie un parametro non esistente } \\
                    3.a Viene mostrato un messaggio di errore\\
                    Torna al punto 2
                \end{tabular}
            }\\

            \auxcase{
                \begin{tabular}{l}
                    \textit{Precondizione: l'utente inserisce un valore non valido } \\
                    4.a Viene mostrato un messaggio di errore\\
                    Torna al punto 2
                \end{tabular}
            }
        }
        \vspace{0.5cm}
\end{minipage}

\begin{minipage}{\textwidth}
    \subsubsection{Registrazione Fruitore}
    \usecase
        {Registrazione Fruitore}
        {
            Fruitore %\\
            %\auxactor{teto}
        }
        {
            %Registrazione
            \begin{tabular}{l}
                \textit{Precondizione: non è ancora stato effettuato il login}\\
                1. L'utente, al momento del login, seleziona la voce ``Registrati''\\
                2. Inserisce username e password (due volte mediante il campo \\ ``conferma password'')\\
                3. L'utente conferma.\\
                \textit{Postcondizione: L'utente viene salvato nel database}\\
                \textit{Postcondizione: viene effettuato il login a nome dell'utente}\\
                Fine
            \end{tabular} \\

            %username duplicato
            \auxcase{
                \begin{tabular}{l}
                    \textit{Precondizione: l'utente sceglie un username già esistente} \\
                    3.a Viene mostrato un messaggio di errore\\
                    4. Ritorna al menu principale.\\
                    Fine
                \end{tabular}
            }\\

            %password non matcha
            \auxcase{
                \begin{tabular}{l}
                    \textit{Precondizione: le password insarite non corrispondono} \\
                    3.a Viene mostrato un messaggio di errore\\
                    4. Torna al punto 1
                \end{tabular}
            }\\

            %la password non è sicura
            \auxcase{
                \begin{tabular}{l}
                    \textit{Precondizione: l'utente sceglie una password } \\
                    \textit{che non rispetta i requisiti di sicurezza}\\
                    3.a Viene mostrato un messaggio di errore\\
                    4. Torna al punto 1
                \end{tabular}
            }\\

            \auxcase{
                \begin{tabular}{l}
                    \textit{Precondizione: l'utente non conferma la registrazione} \\
                    2.a Ritorna al menu principale.\\
                    Fine
                \end{tabular}
            }
        }
        \vspace{0.5cm}
\end{minipage}
\begin{minipage}{\textwidth}
    \subsubsection{Visualizzazione informazioni}
    \usecase
        {Visualizzazione informazioni}
        {
            Fruitore
        }
        {
            %Visualizzazione
            \begin{tabular}{l}
                1. Il fruitore richiede la visualizzazione delle informazioni\\
                2. Il sistema visualizza il nome e la descrizione delle categorie\\
                radice di tutte le gerarchie, la piazza, i luoghi, i giorni e gli\\
                intervalli orari in cui sono possibili gli scambi\\
                Fine
            \end{tabular}

        }
        \vspace{0.5cm}
\end{minipage}

\pagebreak
\subsection{Casi d'uso (UML)}
\vspace{0.5cm}
\begin{figure}[H]
    \centering
    \includeplantuml{VER1/diagrams/use_case/version_1.puml}
    \caption{Versione 1: Diagramma UML dei casi d'uso}
    \label{fig:use_case_uml_v1}
\end{figure}

\pagebreak
\subsection{Diagrammi delle classi}

\textit{Si noti che, per motivi di spazio, il diagramma generale delle classi è stato suddiviso in più sottodiagrammi, uno per ogni package.}
\textit{Le referenze tra i vari diagrammi sono rappresentate con una classe, senza dettagli, al di fuori del package stesso.}
\textit{Inoltre, le classi già specificate nella versione precedente saranno riportate senza attributi e metodi}

\subsubsection{Diagramma delle classi \texttt{controller}}
\vspace{0.5cm}
\begin{figure}[H]
    \centering
    \includeplantuml{VER3/diagrams/class/class_diagram_controller.puml}
    \caption{Versione 3: Diagramma UML delle classi, package \texttt{controller}}
    \label{fig:class_controller_v_3}
\end{figure}

Alla classe \textit{Controller} sono stati aggiunti i metodi che permettono all'utente di aggiungere, ritirare e visualizzare le offerte (filtrando per utente o per categoria):
\begin{itemize}
    \item createOffer(u: User, name: String, cat: LeafCategory, fields: Map<String, String>)
    \item retractOffer(offer: Offer)
    \item showOffersByUser(user: User)
    \item showOpenOffersByCategory(category: LeafCategory)
\end{itemize}
\subsubsection{Diagramma delle classi \texttt{data}}
\vspace{0.5cm}
\begin{figure}[H]
    \centering
    \includeplantuml{VER3/diagrams/class/class_diagram_data.puml}
    \caption{Versione 3: Diagramma UML delle classi, package \texttt{data}}
    \label{fig:class_data_v_3}
\end{figure}

Al package \texttt{data} viene aggiunta un'interfaccia per la gestione delle offerte \textit{OfferFactory}.

L'implementazione di questa si occupa di istanziare, a partire dai dati in database, la classe \textit{Offer}.
Questa classe contiene il nome, la categoria foglia a cui è associata e l'utente proprietario; inoltre, possiede una mappa che associa i nomi dei campi al valore compilato in fase di inserimento dall'utente.
\subsubsection{Diagramma delle classi \texttt{impl.sqlite}}
\vspace{0.5cm}
\begin{figure}[H]
    \centering
    \includeplantuml{VER3/diagrams/class/class_diagram_db.puml}
    \caption{Versione 3: Diagramma UML delle classi, package \texttt{impl.sqlite}}
    \label{fig:class_db_v_3}
\end{figure}

L'implementazione delle offerte a livello di database è effettuato tramite due tabelle, rappresentate da due classi: \textit{OfferDB} e \textit{OfferFieldDB}.
La prima contiene nelle colonne il nome dell'articolo, lo stato, una referenza all'utente proprietario ed una alla categoria di appartenenza; ogni riga della seconda tabella invece,
rappresenta un singolo campo compilato di un'offerta, per questo ha un riferimento al campo a cui si riferisce (\textit{CategoryFieldDB}) e il valore che questo assume per la specifica offerta
(referenziata da \texttt{offer\_ref}).

La lettura e scrittura da database è mediata da \textit{OfferFactoryImpl}, un'implementazione dell'interfaccia
\textit{OfferFactory}. La classe istanziata è \textit{OfferImpl} che estende la classe astratta \textit{Offer}.