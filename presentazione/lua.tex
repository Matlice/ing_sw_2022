
\usepackage{shellesc}
\usepackage{luacode}

\begin{luacode*}
    local plantumlJar = "./plantuml.jar"

    function setPlantumlJar(path)
        print("Setting jar path to " .. path)
        plantumlJar = path
    end

    function filename(path)
        return file:match("^.+[/\\](.+)$")
    end

    function makeplantuml(opt, file)
        print("Processing " .. file)

        print("Reading source")
        local infile = io.open(file, "r")
        if infile ~= nil then
            local src_content = infile:read("*a")
            infile:close()
            
            local last_out_file = io.open(file .. ".last", "r")
            local last_src_content = ""
            if last_out_file ~= nil then
                print("Reading last version")
                last_src_content = last_out_file:read("*a")
                last_out_file:close()
            end

            if last_out_file == nil or last_src_content ~= src_content then
                last_out_file = io.open(file .. ".last", "w")
                last_out_file:write(src_content)
                last_out_file:close()
                tex.print("\\immediate\\write18{java -jar " .. plantumlJar .. " -teps " .. file ..  "}")
            else
                print("No rebuild necessary")
            end
            
            tex.print("\\includegraphics[" .. opt .."]{".. file:gsub(".puml", ".eps") .."}")
            return
        end
        print("Inexistent File " .. file)
        tex.print("Cannot file file " .. file)
    end
\end{luacode*}

\newcommand{\plantumlfile}[2][width=.95\textwidth]{
    \directlua{
        makeplantuml("\luaescapestring{#1}", "\luaescapestring{#2}")
    }
}

\newcommand{\setpumljar}[1]{
    \directlua{
        setPlantumlJar("\luaescapestring{#1}")
    }
}