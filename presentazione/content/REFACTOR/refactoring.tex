
\section{Refactoring}

\subsection{Unit test specifico}
\begin{frame}
    \frametitle{Ristrutturazione di createHierarchyFromXML()}
    Si è deciso di ristrutturare il metodo di creazione della gerarchia da file XML, perché esso ricopriva più compiti complessi e quindi nell'insieme risultava difficile da interpretare.\pause

    Al fine di migliorarne la lettura, si è deciso di applicare il metodo di refactoring \textbf{Extract Method}, assegnando nomi autoesplicativi alle porzioni di codice estratte per aiutare la comprensione.\pause

    Prima di eseguire l'operazione di refactor, si è provveduto a mettere opportunamente sotto test il metodo selezionato.
\end{frame}

\begin{frame}[allowframebreaks]
    \frametitle{Casi di test}
    Per ottenere dei casi di test efficaci, si è utilizzato come base la stessa struttura gerarchica descritta nella documentazione relativa all'importazione di configurazione da file testuale.\pause

    Partendo da quella base, sono state effettuate delle modifiche alla struttura gerarchica per poter coprire i casi limite.
\end{frame}

\begin{frame}[fragile]
    \frametitle{Casi di test i}
    \lstset{style=java}
    \begin{itemize}
        \item <1-> Si certifica che venga correttamente lanciata una eccezione in caso di gerarchia vuota
\begin{lstlisting}[language=xml]
<hierarchy/>
\end{lstlisting}
        \item <2-> Si controlla il funzionamento nel caso di una gerarchia senza campi e senza figli, verificando la presenza dei campi obbligatori nella categoria radice
\begin{lstlisting}[language=xml]
<hierarchy>
    <name>Vestiti</name>
    <fields/>
    <children/>
</hierarchy>
\end{lstlisting}
    \end{itemize}
\end{frame}
\note{
    
}

\begin{frame}[fragile]
    \frametitle{Casi di test ii}
    \lstset{style=java}
    \begin{itemize}
        \item <1-> Si controlla il funzionamento nel caso di una gerarchia senza figli
\begin{lstlisting}[language=xml]
<hierarchy>
    <name>Vestiti</name>
    <fields>
        <field required="true"> Taglia </field>
        <field required="false"> Colore </field>
    </fields>
    <children/>
</hierarchy>
\end{lstlisting}
    \end{itemize}
\end{frame}
\note{
    
}

\disablefooter
\begin{frame}[fragile]
    \frametitle{Casi di test iii}
    \lstset{style=tiny_java}
    \begin{itemize}
        \item <1-> Si controlla il funzionamento nel caso di una gerarchia divisa in sole due categorie figlie
\begin{lstlisting}[language=xml]
<hierarchy>
    <name>Vestiti</name>
    <fields>
        <field required="true"> Taglia </field>
        <field required="false"> Colore </field>
    </fields>
    <children>
        <category>
            <name> Pantaloni </name>
            <fields/>
            <description/>
        </category>
        <category>
            <name> Felpe </name>
            <description/>
        </category>
    </children>
</hierarchy>
\end{lstlisting}
        \item <2-> Si controlla il funzionamento nel caso completo
    \end{itemize}
\end{frame}
\note{
    
}
\enablefooter

\subsection{Extract Method}
\begin{frame}[fragile]
    \frametitle{Metodo pre-refactor 1/3}

    \lstset{style=tiny_java}
    \begin{lstlisting}[language=java]
private void createHierarchyFromXML(XMLImport.HierarchyXML hierarchyXML) throws DuplicateCategoryException, InvalidCategoryException, DuplicateFieldException, InvalidFieldException {
        // creazione della categoria root
        Category root = this.createCategoryFromXML(null, hierarchyXML.root);
        if (hierarchyXML.root.fields != null)
            for (var e: hierarchyXML.root.fields) {
                this.addFieldFromXML(root, e);
            }

        // aggiunge i campi default alla categoria radice
        root.put("Stato di conservazione", new TypeDefinition(true));
        root.put("Descrizione libera", new TypeDefinition(false));

        // associa una CategoryXML all'istanza della Category padre già creata
        var categoryStack = new LinkedList<AbstractMap.SimpleEntry<XMLImport.CategoryXML, Category>>();

        if (hierarchyXML.root.categories != null) {
            if (hierarchyXML.root.categories.size() == 1) throw new InvalidCategoryException();
            for (var e: hierarchyXML.root.categories) {
                categoryStack.add(new AbstractMap.SimpleEntry<>(e, root));
            }
        }
    \end{lstlisting}
\end{frame}

\begin{frame}[fragile]
    \frametitle{Metodo pre-refactor 2/3}

    \lstset{style=tiny_java}
    \begin{lstlisting}[language=java, firstnumber=23]
        // creazione delle categorie figlie
        while (categoryStack.size() != 0) {
            var toInsert = categoryStack.pop();
            Category father = toInsert.getValue();

            NodeCategory r = null;
            while (r == null) {
                var newChild = this.createCategoryFromXML(root, toInsert.getKey());
                r = this.appendCategory(father, newChild);

                if (toInsert.getKey().fields != null)
                    for (var f: toInsert.getKey().fields) {
                        this.addFieldFromXML(newChild, f);
                    }

                if (toInsert.getKey().categories != null) {
                    if (toInsert.getKey().categories.size() == 1) throw new InvalidCategoryException();
                    for (var e : toInsert.getKey().categories) {
                        categoryStack.add(new AbstractMap.SimpleEntry<>(e, newChild));
                    }
                }
            }
    \end{lstlisting}
\end{frame}

\begin{frame}[fragile]
    \frametitle{Metodo pre-refactor 3/3}

    \lstset{style=tiny_java}
    \begin{lstlisting}[language=java, caption={Snippet tratto da v5, Controller.java}, firstnumber=46]
            // aggiorna i padri nello stack, serve perchè l'istanza potrebbe essere cambiata
            var newCategoryStack = new LinkedList<AbstractMap.SimpleEntry<XMLImport.CategoryXML, Category>>();
            for(var e: categoryStack) {
                if (e.getValue() == father) {
                    var newElem = new AbstractMap.SimpleEntry<>(e.getKey(), (Category) r);
                    newCategoryStack.add(newElem);
                } else {
                    newCategoryStack.add(e);
                }
            }
            categoryStack = newCategoryStack;

            if (father == root) root = r;
        }

        this.model.createHierarchy(root);
    }
    \end{lstlisting}
\end{frame}

\begin{frame}[fragile]
    \frametitle{Extract Method 1 - Creazione nuovi metodi}
    \begin{lstlisting}[language=tiny_diff]
+ private void XMLCreateRootCategory(){
+ }

private void createHierarchyFromXML(XMLImport.HierarchyXML hierarchyXML) throws DuplicateCategoryException, InvalidCategoryException, DuplicateFieldException, InvalidFieldException, SQLException {
    // creazione della categoria root
    Category root = this.createCategoryFromXML(null, hierarchyXML.root);
    if (hierarchyXML.root.fields != null)
        for (var e: hierarchyXML.root.fields) this.addFieldFromXML(root, e);
    // aggiunge i campi default alla categoria radice
    root.put("Stato di conservazione", new TypeDefinition(true));
    root.put("Descrizione libera", new TypeDefinition(false));
    // associa una CategoryXML all'istanza della Category padre già creata
    var categoryStack = new LinkedList<AbstractMap.SimpleEntry<XMLImport.CategoryXML, Category>>();

    ...
}
\end{lstlisting}
\end{frame}

\begin{frame}[fragile]
    \frametitle{Extract Method 2 - Copia del metodo}
    \begin{lstlisting}[language=tiny_diff]
private void XMLCreateRootCategory(){
+   Category root = this.createCategoryFromXML(null, hierarchyXML.root);
+   if (hierarchyXML.root.fields != null)
+   for (var e: hierarchyXML.root.fields) this.addFieldFromXML(root, e);
+   // aggiunge i campi default alla categoria radice
+   root.put("Stato di conservazione", new TypeDefinition(true));
+   root.put("Descrizione libera", new TypeDefinition(false));
}

private void createHierarchyFromXML(XMLImport.HierarchyXML hierarchyXML){
    // creazione della categoria root
    Category root = this.createCategoryFromXML(null, hierarchyXML.root);
    if (hierarchyXML.root.fields != null)
        for (var e: hierarchyXML.root.fields) this.addFieldFromXML(root, e);
    // aggiunge i campi default alla categoria radice
    root.put("Stato di conservazione", new TypeDefinition(true));
    root.put("Descrizione libera", new TypeDefinition(false));

    // associa una CategoryXML all'istanza della Category padre già creata
    var categoryStack = new LinkedList<AbstractMap.SimpleEntry<XMLImport.CategoryXML, Category>>();

    ...
}
\end{lstlisting}
\end{frame}

\begin{frame}[fragile]
    \frametitle{Extract Method 3 - Aggiunta dei parametri}
    \begin{lstlisting}[language=tiny_diff]
- private void XMLCreateRootCategory(){
+ private void XMLCreateRootCategory(XMLImport.HierarchyXML hierarchyXML){
    Category root = this.createCategoryFromXML(null, hierarchyXML.root);
    if (hierarchyXML.root.fields != null)
    for (var e: hierarchyXML.root.fields) this.addFieldFromXML(root, e);
    // aggiunge i campi default alla categoria radice
    root.put("Stato di conservazione", new TypeDefinition(true));
    root.put("Descrizione libera", new TypeDefinition(false));
}

private void createHierarchyFromXML(XMLImport.HierarchyXML hierarchyXML){
    // creazione della categoria root
    Category root = this.createCategoryFromXML(null, hierarchyXML.root);
    if (hierarchyXML.root.fields != null)
        for (var e: hierarchyXML.root.fields) this.addFieldFromXML(root, e);
    // aggiunge i campi default alla categoria radice
    root.put("Stato di conservazione", new TypeDefinition(true));
    root.put("Descrizione libera", new TypeDefinition(false));

    // associa una CategoryXML all'istanza della Category padre già creata
    var categoryStack = new LinkedList<AbstractMap.SimpleEntry<XMLImport.CategoryXML, Category>>();

    ...
}
\end{lstlisting}
\end{frame}

\disablefooter%
\begin{frame}[fragile]
    \frametitle{Extract Method 5 - Ritorno di variabili locali}
    \begin{lstlisting}[language=tiny_diff]
private void XMLCreateRootCategory(XMLImport.HierarchyXML hierarchyXML){
    Category root = this.createCategoryFromXML(null, hierarchyXML.root);
    if (hierarchyXML.root.fields != null)
    for (var e: hierarchyXML.root.fields) this.addFieldFromXML(root, e);
    // aggiunge i campi default alla categoria radice
    root.put("Stato di conservazione", new TypeDefinition(true));
    root.put("Descrizione libera", new TypeDefinition(false));
+   return root;
}

private void createHierarchyFromXML(XMLImport.HierarchyXML hierarchyXML){
    // creazione della categoria root
    Category root = this.createCategoryFromXML(null, hierarchyXML.root);
    if (hierarchyXML.root.fields != null)
        for (var e: hierarchyXML.root.fields) this.addFieldFromXML(root, e);
    // aggiunge i campi default alla categoria radice
    root.put("Stato di conservazione", new TypeDefinition(true));
    root.put("Descrizione libera", new TypeDefinition(false));

    // associa una CategoryXML all'istanza della Category padre già creata
    var categoryStack = new LinkedList<AbstractMap.SimpleEntry<XMLImport.CategoryXML, Category>>();

    ...
}
\end{lstlisting}
\end{frame}
\note{
    Step 4: ``Se delle variabili locali sono usate solo all'interno del codice estratto, farle diventare variabili locali del metodo, mentre se il codice usa variabili locali, farle diventare parametri del metodo'' non viene proposto in quanto non necessario data l'assenza di tali variabili
}
\enablefooter%

\begin{frame}[fragile]
    \frametitle{Extract Method 6 - Sostituzione con la chiamata al metodo}
    \begin{lstlisting}[language=tiny_diff]
private void XMLCreateRootCategory(XMLImport.HierarchyXML hierarchyXML){
    Category root = this.createCategoryFromXML(null, hierarchyXML.root);
    if (hierarchyXML.root.fields != null)
    for (var e: hierarchyXML.root.fields) this.addFieldFromXML(root, e);
    // aggiunge i campi default alla categoria radice
    root.put("Stato di conservazione", new TypeDefinition(true));
    root.put("Descrizione libera", new TypeDefinition(false));
    return root;
}

private void createHierarchyFromXML(XMLImport.HierarchyXML hierarchyXML){
    // creazione della categoria root
-   Category root = this.createCategoryFromXML(null, hierarchyXML.root);
-   if (hierarchyXML.root.fields != null)
-       for (var e: hierarchyXML.root.fields) this.addFieldFromXML(root, e);
-   // aggiunge i campi default alla categoria radice
-   root.put("Stato di conservazione", new TypeDefinition(true));
-   root.put("Descrizione libera", new TypeDefinition(false));
+   Category root = XMLCreateRootCategory(hierarchyXML);

    // associa una CategoryXML all'istanza della Category padre già creata
    var categoryStack = new LinkedList<AbstractMap.SimpleEntry<XMLImport.CategoryXML, Category>>();

    ...
}
\end{lstlisting}
\end{frame}

\begin{frame}[fragile]
    \frametitle{Extract Method 7 - Compilazione e test}
    Il procedimento sopra descritto è stato applicato per più sezioni del metodo, risultando in quanto segue. I risultati del testing sono stati positivi.
\lstset{style=tiny_java}
\begin{lstlisting}[language=java, caption={Refactor del metodo Controller\#createHierarchyFromXML()}]
private void createHierarchyFromXML(XMLImport.HierarchyXML hierarchyXML){
    Category root = XMLCreateRootCategory(hierarchyXML); // creazione della categoria root
    // associa una CategoryXML all'istanza della Category padre già creata
    var categoryStack = new LinkedList<AbstractMap.SimpleEntry<XMLImport.CategoryXML, Category>>();

    if (hierarchyXML.root.categories != null) {
        if (hierarchyXML.root.categories.size() == 1) throw new InvalidCategoryException();
        for (var e : hierarchyXML.root.categories)
            categoryStack.add(new AbstractMap.SimpleEntry<>(e, root));
    }
    root = XMLCreateChildCategories(root, categoryStack); // creazione delle categorie figlie

    try {
        this.model.createHierarchy(root);
    } catch (CannotRetrieveInformationException e) {
        System.exit(1); //Errore terminale
    }
}
\end{lstlisting}
\end{frame}