\section{SOLID Dependency Inversion}
\begin{frame}[fragile]
    \frametitle{Inversione delle dipendenze}

    Durante lo sviluppo del modello MVC è stato adottato il pattern SOLID Dependency Inversion al fine di garantire l'indipendenza del Controller dall'implementazione di una View.\pause%
    
    In particolare viene mantenuta una logica generica per l'interfacciamento e non vi è dipendenza diretta tra \texttt{StreamView} e \texttt{Controller}.\pause%

    \lstset{style=java}
    \begin{lstlisting}[language=java, caption={Snippet tratto dal costruttore di Controller}]
private View view;
public Controller(View view, Model model) {
    this.view = view;
    this.model = model;
}
    \end{lstlisting}
\end{frame}

\begin{frame}
    \frametitle{Dependency Inversion~-~UML}
    \begin{figure}
        \centering
        \plantumlfile{uml/solid.di.puml}
        \caption{Diagramma dei componenti relativo all'architettura di View e interfacciamento con Controller}
    \end{figure}
\end{frame}

\begin{frame}[fragile]
    \frametitle{Vantaggi}

    È evidente come possa essere facile sostituire il tipo di View per l'applicativo.

    Ipotizzando una classe \texttt{SwingView} la quale implementi \texttt{View}, l'unico cambiamento necessario (oltre all'implementazione della classe) sarebbe quanto segue:

    \lstset{style=java}
    \begin{lstlisting}[language=diff, caption={EntryPoint.java\#main}]
public static void main(String[] args) throws DBException {
    // -- inizializzazione StorageManagers -- (omessa)
    Model model = new Model(...);
-   View view = new StreamView(System.out, new Scanner(System.in));
+   View view = new SwingView(new JFrame());
    Controller controller = new Controller(view, model);
    controller.run()
}
    \end{lstlisting}
\end{frame}