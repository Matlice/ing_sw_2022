\section{SOLID Interface Segregation}
\begin{frame}
    \frametitle{Interface Segregation}

    \begin{minipage}{.48\textwidth}
        Precedentemente al refactor
        \begin{figure}
            \centering
            \plantumlfile{uml/solid.is.pre.puml}
        \end{figure}
    \end{minipage}
    \hfill
    \begin{minipage}{.48\textwidth}
        Successivamente al refactor
        \begin{figure}
            \centering
            \plantumlfile{uml/solid.is.post.puml}
        \end{figure}
    \end{minipage}
\end{frame}

\begin{frame}
    \frametitle{Interface Segregation - Spiegazione}
    L'interfaccia \texttt{OfferFactory} viene suddivisa in funzione del campo di appartenenza delle azioni dei suoi metodi.

    Questo processo è stato effettuato in modo da adattare tutte le factory presenti nel codice, secondo i seguenti criteri:
    \begin{itemize}
        \item<1-> Tutte le azioni relative alla modifica di un'istanza esistente vengono destinate all'interfaccia \texttt{*StorageManager}
        \item<2-> Tutte le azioni di creazione vengono destinate all'interfaccia \texttt{*Factory}
        \item <3-> Tutte le azioni di querying vengono infine destinate all'interfaccia \texttt{*Repository}
    \end{itemize}
\end{frame}
\note{
    Si noti come le funzioni di modifica di una istanza potessero essere delegate alla classe Offer stessa.

    Sebbene possibile e consigliato da Information Expert, questo comporterebbe un incremento delle responsabilità della classe e un aumento dell'accoppiamento, in quanto la classe Offer dovrebbe dipendere dagli oggetti DAO per poter operare sulle istanze di database.

    Di conseguenza si è preferito mantenere distaccata la logica di operazione sulla base di dati e la logica applicativa.
}