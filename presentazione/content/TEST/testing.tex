\section{Testing}

\subsection{Testing End to End}
\begin{frame}
    \frametitle{Strutturazione dei test Black Box}
    Al fine di garantire la continuità e la consistenza dell'applicativo, anche a fronte di modifiche sostanziali alla logica applicativa, è stato deciso un approccio \textbf{End to End} al testing.\pause

    Concettualmente è come se l'interazione utente fosse automatizzata mantenendo l'esecuzione di tutti i componenti del codice.\pause

    Non essendoci isolamento delle componenti, si valuta per lo più l'integrazione tra essi, al fine di validare l'output delle singole esecuzioni.
\end{frame}

\begin{frame}
    \frametitle{Flow di processo 1/3 - Predisposizione}
    \begin{minipage}{.31\textwidth}
        \begin{figure}
            \centering
            \plantumlfile{uml/testing.e2e.flow.init.puml}
        \end{figure}
    \end{minipage}\hfill
    \begin{minipage}{.66\textwidth}
        \begin{itemize}
            \item <1-> Vengono preparati due stream, uno per l'input che andrà a simulare l'utente e uno per l'output che servirà nella fase di validazione.
            \item <2-> Vengono quindi caricati i file contenenti l'input del caso di test e l'output atteso
            \item <3-> Viene inizializzata la struttura e le classi dell'applicativo per il suo funzionamento.
        \end{itemize}
        \pause\pause\pause
        Questo metodo permette di testare sia la View che il Controller in una sola esecuzione.
    \end{minipage}
\end{frame}
\note{
    I test sono stati scritti tramite una JUnit 5 TestFactory che produce una collezione di test dinamica.
}

\begin{frame}[fragile]
    \frametitle{Codice di esempio, Esecuzione test}

    \lstset{style=java}
    \begin{lstlisting}[language=java, caption={Snippet tratto da End2EndTest.java}]
private void testCase(File inFile, File outFile) {
  InputStream in = new BufferedInputStream(new FileInputStream(inFile));
  OutputStream os = new ByteArrayOutputStream();
  PrintStream out = new PrintStream(new TeeOutputStream(os, System.out));
  this.run(in, out); //run the code
  String testFile = Files.readString(outFile.toPath());
  assertEquals(testFile, os.toString());
}       
    \end{lstlisting}
\end{frame}

\begin{frame}
    \frametitle{Flow di processo 2/3 - Esecuzione}
    \begin{minipage}{.31\textwidth}
        \begin{figure}
            \centering
            \plantumlfile{uml/testing.e2e.flow.exec.puml}
        \end{figure}
    \end{minipage}\hfill
    \begin{minipage}{.66\textwidth}
        Il codice viene avviato chiamando \texttt{Controller\#run()} e l'input viene fornito dal test case simulando l'interazione con l'utente.

        Se il test non termina entro un timeout prefissato, questo è considerato fallito.
    \end{minipage}
\end{frame}

\begin{frame}
    \frametitle{Flow di processo 3/3 - Validazione}
    \begin{minipage}{.31\textwidth}
        \begin{figure}
            \centering
            \plantumlfile{uml/testing.e2e.flow.validate.puml}
        \end{figure}
    \end{minipage}\hfill
    \begin{minipage}{.66\textwidth}
        L'output dell'esecuzione viene reperito e confrontato con il caso atteso, e se questi non corrispondono il test è considerato fallito.
    \end{minipage}
\end{frame}

\begin{frame}
    \frametitle{Coverage ottenuto}
    \begin{center}
        \begin{tabular}{ |p{6cm} l l|  }
            \hline
            \multicolumn{3}{|c|}{Coverage Report \textit{it.matlice}} \\
            \hline
            \textbf{Package}& \textbf{Class Coverage} & \textbf{Method Coverage}\\
            \hline
            \textit{ingsw.controller} & 
                \progressbar[linecolor=white, filledcolor=codegreen]{1.0} \footnotesize100\% & 
                \progressbar[linecolor=white, filledcolor=codegreen]{0.939} \footnotesize93\% \\
            \textit{ingsw.model.auth.exceptions} & 
                \progressbar[linecolor=white, filledcolor=codegreen]{1.0} \footnotesize100\% & 
                \progressbar[linecolor=white, filledcolor=codegreen]{1.0} \footnotesize100\% \\
            \textit{ingsw.model.auth.password} & 
                \progressbar[linecolor=white, filledcolor=codegreen]{1.0} \footnotesize100\% & 
                \progressbar[linecolor=white, filledcolor=codegreen]{1.0} \footnotesize100\% \\
            \textit{ingsw.model.data} & 
                \progressbar[linecolor=white, filledcolor=codegreen]{1.0} \footnotesize100\% & 
                \progressbar[linecolor=white, filledcolor=yellow]{0.827} \footnotesize82.7\% \\
            \textit{ingsw.model.data.impl.jdbc} & 
                \progressbar[linecolor=white, filledcolor=codegreen]{1.0} \footnotesize100\% & 
                \progressbar[linecolor=white, filledcolor=yellow]{0.857} \footnotesize85.7\% \\
            \textit{it.matlice.ingsw.view.stream} & 
                \progressbar[linecolor=white, filledcolor=codegreen]{1.0} \footnotesize100\% & 
                \progressbar[linecolor=white, filledcolor=codegreen]{0.911} \footnotesize91.1\% \\
            \textit{ingsw.view.stream.datatypes} & 
                \progressbar[linecolor=white, filledcolor=codegreen]{1.0} \footnotesize100\% & 
                \progressbar[linecolor=white, filledcolor=codegreen]{1.0} \footnotesize100\% \\
            \hline
        \end{tabular}
    \end{center}
\end{frame}