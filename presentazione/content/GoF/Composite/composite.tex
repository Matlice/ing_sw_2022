\section{Gang of Four Composite}
\begin{frame}
    \frametitle{GoF Composite - Prima del refactor}

    \begin{minipage}{.5\textwidth}
        \begin{figure}
            \centering
            \plantumlfile[width=0.83\textwidth]{uml/gof.composite.pre.puml}
        \end{figure}
    \end{minipage}
    \begin{minipage}{.49\textwidth}
        \begin{itemize}
            \item<1-> Si era cercato di applicare il pattern \texttt{Composite} alla classe \texttt{Category} in modo che potesse rappresentare sia le categorie foglia che quelle nodo.
            \item<2-> Tuttavia nel codice c'erano diverse violazioni del pattern
        \end{itemize}
    \end{minipage}

\end{frame}

\begin{frame}[fragile]
    \frametitle{GoF Composite - Violazioni del pattern}

    Inizialmente il codice aveva comportamenti diversi in base alla classe dell'istanza che veniva considera (se \texttt{LeafCategory} oppure \texttt{NodeCategory}), ad esempio:

    \lstset{style=java}
    \begin{lstlisting}[language=java, caption={v5 commit 34c1a67}]
private NodeCategory appendCategory(Category father, Category child) {
    var f = father instanceof LeafCategory ?
            ((LeafCategory) father).convertToNode() : 
            (NodeCategory) father;
    f.addChild(child);
    return f;
}
    \end{lstlisting}

\end{frame}

\begin{frame}[fragile]
    \frametitle{GoF Composite - Violazioni del pattern}

    Allo stesso modo il metodo per determinare la validità di una categoria ha comportamenti diversi in base alla classe (usando \texttt{instanceof}):

    \lstset{style=java}
    \begin{lstlisting}[language=java, caption={v5 commit 34c1a67}]
public boolean isCategoryValid(Category c) {
    if (thcis instanceof LeafCategory)
        return true;
    assert c instanceof NodeCategory;
    return ((NodeCategory) c).getChildren().length >= 2 &&
            Arrays.stream(((NodeCategory) c).getChildren())
                .allMatch(Category::isCategoryValid);
}
    \end{lstlisting}

\end{frame}

\begin{frame}
    \frametitle{GoF Composite - Dopo il refactor}

    \begin{minipage}{.5\textwidth}
        \begin{figure}
            \centering
            \plantumlfile[width=0.83\textwidth]{uml/gof.composite.post.puml}
        \end{figure}
    \end{minipage}
    \begin{minipage}{.49\textwidth}
        \begin{itemize}
            \item<1-> Si era cercato di applicare il pattern \texttt{Composite} alla classe \texttt{Category} in modo che potesse rappresentare sia le categorie foglia che quelle nodo.
            \item<2-> Tuttavia nel codice c'erano diverse violazioni del pattern
        \end{itemize}
    \end{minipage}

\end{frame}