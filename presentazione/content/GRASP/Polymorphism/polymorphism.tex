\section{GRASP Polymorphism}
\begin{frame}[fragile]
    \frametitle{GRASP Polymorphism - Prima del refactor}

    Inizialmente le due tipologie di utente, \texttt{ConfiguratorUser} e \texttt{CustomerUser}, equivalenti ai tipi di utente Configuratore e Fruitore, venivano distinte dal Controller in funzione della loro istanza; come riportato a seguire.

    \lstset{style=java}
    \begin{lstlisting}[language=java, caption={v5 commit 34c1a67}]
if (this.currentUser.getUser() instanceof ConfiguratorUser)
    while (!this.model.hasConfiguredSettings())
        this.configureSettings(true);
else if (!this.model.hasConfiguredSettings())
    this.view.error(Errors.NO_CONF_ERROR);
    \end{lstlisting}

\end{frame}

\begin{frame}[fragile]
    \frametitle{Dopo il refactor}

    \begin{minipage}{.38\textwidth}
        \begin{figure}
            \centering
            \plantumlfile[width=\textwidth]{uml/grasp.poly.post.puml}
            \caption{Diagramma UML delle classi con le estensioni di User}
        \end{figure}
    \end{minipage}
    \hfill
    \begin{minipage}{.58\textwidth}
        \lstset{style=java}
        \begin{lstlisting}[language=java, caption={Codice refactorato in Controller}]
if (this.currentUser.getUser().isAdmin())
   do {
       this.configureSettings(true);
   }while(!model.hasConfiguredSettings());
        \end{lstlisting}

        \begin{lstlisting}[language=java, caption={Utente configuratore è admin}]
@Override
public Boolean isAdmin(){
    return true;
}
        \end{lstlisting}
    \end{minipage}

\end{frame}