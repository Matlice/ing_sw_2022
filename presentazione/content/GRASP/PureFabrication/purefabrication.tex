\section{GRASP Pure Fabrication}

\begin{frame}
    \frametitle{Prima ideazione derivata dal modello di dominio}
    Inizialmente, durante la fase di modellazione del dominio, la classe OfferImpl veniva definita come segue.

    \begin{minipage}{.29\textwidth}
        \begin{figure}
            \centering
            \plantumlfile[width=.8\textwidth]{uml/grasp.pf.pre.puml}
        \end{figure}
    \end{minipage}
    \begin{minipage}{.68\textwidth}
        \begin{itemize}
            \item<1-> La classe si occupa di troppe responsabilità, violando High Cohesion. La classe dovrebbe occuparsi di:
            \begin{itemize}
                \item Gestione dell'offerta a livello di dominio
                \item Gestione della persistenza sul database
                \item Gestione dei valori degli attributi compilati
            \end{itemize}
            \item<2-> La classe inoltre viola Low Coupling essendo dipendente dalle astrazioni di Categoria e Utente oltre che alla connessione al database.
        \end{itemize}
    \end{minipage}
\end{frame}

\begin{frame}
    \frametitle{Soluzione: GRASP Pure Fabrication}
    \begin{minipage}{.48\textwidth}
        \begin{itemize}
            \item<1-> Viene segmentata la classe \texttt{OfferImpl}
            \item <2-> Vengono create le relative classi di convenienza per l'associazione al modello del database
            \item <3-> Vengono creati dei Data Abstraction Object per gestire l'interazione con il database (OrmLite)
            \item Viene delegato alla factory la responsabilità di istanziare le offerte ed effettuare l'associazione dei valori dei campi alle istanze.
        \end{itemize}
    \end{minipage}
    \begin{minipage}{.47\textwidth}
        \begin{figure}
            \centering
            \plantumlfile[width=.8\textwidth]{uml/grasp.pf.post.puml}
        \end{figure}
    \end{minipage}
\end{frame}
